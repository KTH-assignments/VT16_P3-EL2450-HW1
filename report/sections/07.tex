\section{Question 7}

Figure \ref{fig:Q7.rloc_F_G} shows the root locus of the open-loop transfer
function without the use of a zero-order hold, while figure
\ref{fig:Q7.rloc_F_ZOH_G} shows exacly the same, but with the addition of a
zero-order hold between the controller and the process, with sampling time of
$h=1$ sec. It is apparent that without the zero-order hold, the system is stable,
since all poles have negative real values. However, the time delay and the pole
at zero that the zero-order hold introduces deliver a reduction in the degree of
stability.

\begin{figure}[H]\centering
	\centering
	\scalebox{1}{% This file was created by matlab2tikz.
%
%The latest updates can be retrieved from
%  http://www.mathworks.com/matlabcentral/fileexchange/22022-matlab2tikz-matlab2tikz
%where you can also make suggestions and rate matlab2tikz.
%
\definecolor{mycolor1}{rgb}{0.00000,0.44700,0.74100}%
\definecolor{mycolor2}{rgb}{0.00000,0.75000,0.75000}%
%
\begin{tikzpicture}

\begin{axis}[%
width=4.008in,
height=3.052in,
at={(0.818in,0.44in)},
scale only axis,
unbounded coords=jump,
separate axis lines,
every outer x axis line/.append style={white!40!black},
every x tick label/.append style={font=\color{white!40!black}},
xmin=-1.2,
xmax=0.2,
every outer y axis line/.append style={white!40!black},
every y tick label/.append style={font=\color{white!40!black}},
ymin=-1,
ymax=1,
axis background/.style={fill=white}
]
\addplot [color=white,solid,forget plot]
  table[row sep=crcr]{%
0	0\\
};
\addplot [color=mycolor1,only marks,mark=o,mark options={solid},forget plot]
  table[row sep=crcr]{%
-0.134875180971928	0.0760690712223554\\
-0.134875180971928	-0.0760690712223554\\
};
\addplot [color=mycolor1,only marks,mark=x,mark options={solid},forget plot]
  table[row sep=crcr]{%
0	0\\
-1.15912180370184	0\\
-0.0804391007994334	0\\
-0.0804390954987234	0\\
};
\addplot [color=blue,solid,forget plot]
  table[row sep=crcr]{%
0	0\\
-0.00172258141172093	0\\
-0.00212735127812858	0\\
-0.00263131770774314	0\\
-0.00326121735506603	0\\
-0.00405257004497591	0\\
-0.00505372231013738	0\\
-0.00633275328735025	0\\
-0.00799037693414231	0\\
-0.0101872515208663	0\\
-0.0132136773986536	0\\
-0.0177357655010795	0\\
-0.026965149880164	0\\
-0.031861170880117	0\\
-0.0329210964874718	0\\
-0.0329191812447814	0.00106171833401894\\
-0.0326002115036228	0.0144153917668627\\
-0.0322532575366739	0.0222133400181056\\
-0.0319619413767979	0.0287786677028206\\
-0.0317503952262301	0.0349437933283371\\
-0.0316490597307499	0.0410187391041332\\
-0.0316962291519887	0.0471699250742801\\
-0.031940010304282	0.0535065627621177\\
-0.0324408279090809	0.0601100135750878\\
-0.033274662303877	0.0670456186005314\\
-0.0345372857097423	0.0743673458050051\\
-0.0363498872538261	0.0821184817209548\\
-0.0388666735863987	0.0903291485867909\\
-0.0422853521405217	0.099010100663705\\
-0.0468619379890806	0.108140874273028\\
-0.0529322272301958	0.117648003902467\\
-0.0609437701942948	0.127363885294016\\
-0.0715042606053031	0.13694405539213\\
-0.0783478676713363	0.141710946309434\\
-0.0854522168936597	0.145684115756799\\
-0.0945139963054132	0.14948467190357\\
-0.103927241473383	0.152055633108492\\
-0.115887690813282	0.153366226022844\\
-0.121981270592794	0.153161829061482\\
-0.128092091010695	0.152307438983691\\
-0.135502275953594	0.150266973358649\\
-0.142593352928631	0.1470124968458\\
-0.149000509461001	0.142473795143078\\
-0.154225256949533	0.136771887844467\\
-0.158588224713188	0.128207122017414\\
-0.159995472464146	0.120133925111416\\
-0.16	0.12\\
-0.160003931238963	0.11986661611598\\
-0.158775086647782	0.11003569436446\\
-0.155620608994788	0.101890815765771\\
-0.152848362285918	0.0969891014951801\\
-0.148240278725404	0.0904996916159923\\
-0.145010907650277	0.0866083592419506\\
-0.14269208086119	0.0840206196541003\\
-0.140980065689828	0.0821926208248014\\
-0.139687861557493	0.0808507795728236\\
-0.138696039618838	0.0798399007291188\\
-0.137925014775366	0.079064249399346\\
-0.137319748803138	0.0784610606268747\\
-0.136840995561178	0.0779872634890172\\
-0.13646006354647	0.0776122501906733\\
-0.13615554897753	0.0773136698302359\\
-0.135911220601803	0.0770748475944081\\
-0.135714604605612	0.0768831298834395\\
-0.135556010412003	0.0767287831820409\\
-0.135427843048246	0.0766042380585054\\
-0.134876554458581	0.0760703975591964\\
-0.134875180971928	0.0760690712223554\\
};
\addplot [color=mycolor2,solid,forget plot]
  table[row sep=crcr]{%
-0.0804390954987234	0\\
-0.0723951907194905	0\\
-0.0714129432946076	0\\
-0.0702910149341413	0\\
-0.0690026907496458	0\\
-0.0675133206154549	0\\
-0.0657765240062594	0\\
-0.0637275885730999	0\\
-0.061270940685456	0\\
-0.0582532920812395	0\\
-0.0543945004335206	0\\
-0.0490420849527246	0\\
-0.0390023534640328	0\\
-0.0339848604910167	0\\
-0.0329210979667503	0\\
-0.0329191812447814	-0.00106171833401894\\
-0.0326002115036228	-0.0144153917668627\\
-0.0322532575366739	-0.0222133400181056\\
-0.0319619413767979	-0.0287786677028206\\
-0.0317503952262301	-0.0349437933283371\\
-0.0316490597307499	-0.0410187391041332\\
-0.0316962291519887	-0.0471699250742801\\
-0.031940010304282	-0.0535065627621177\\
-0.0324408279090809	-0.0601100135750878\\
-0.033274662303877	-0.0670456186005314\\
-0.0345372857097423	-0.0743673458050051\\
-0.0363498872538261	-0.0821184817209548\\
-0.0388666735863987	-0.0903291485867909\\
-0.0422853521405217	-0.099010100663705\\
-0.0468619379890806	-0.108140874273028\\
-0.0529322272301958	-0.117648003902467\\
-0.0609437701942948	-0.127363885294016\\
-0.0715042606053031	-0.13694405539213\\
-0.0783478676713363	-0.141710946309434\\
-0.0854522168936597	-0.145684115756799\\
-0.0945139963054132	-0.14948467190357\\
-0.103927241473383	-0.152055633108492\\
-0.115887690813282	-0.153366226022844\\
-0.121981270592794	-0.153161829061482\\
-0.128092091010695	-0.152307438983691\\
-0.135502275953594	-0.150266973358649\\
-0.142593352928631	-0.1470124968458\\
-0.149000509461001	-0.142473795143078\\
-0.154225256949533	-0.136771887844467\\
-0.158588224713188	-0.128207122017414\\
-0.159995472464146	-0.120133925111416\\
-0.16	-0.12\\
-0.160003931238963	-0.11986661611598\\
-0.158775086647782	-0.11003569436446\\
-0.155620608994788	-0.101890815765771\\
-0.152848362285918	-0.0969891014951801\\
-0.148240278725404	-0.0904996916159923\\
-0.145010907650277	-0.0866083592419506\\
-0.14269208086119	-0.0840206196541003\\
-0.140980065689828	-0.0821926208248014\\
-0.139687861557493	-0.0808507795728236\\
-0.138696039618838	-0.0798399007291188\\
-0.137925014775366	-0.079064249399346\\
-0.137319748803138	-0.0784610606268747\\
-0.136840995561178	-0.0779872634890172\\
-0.13646006354647	-0.0776122501906733\\
-0.13615554897753	-0.0773136698302359\\
-0.135911220601803	-0.0770748475944081\\
-0.135714604605612	-0.0768831298834395\\
-0.135556010412003	-0.0767287831820409\\
-0.135427843048246	-0.0766042380585054\\
-0.134876554458581	-0.0760703975591964\\
-0.134875180971928	-0.0760690712223554\\
};
\addplot [color=red,solid,forget plot]
  table[row sep=crcr]{%
-0.0804391007994334	0\\
-0.0871714308139377	0\\
-0.0878424493963486	0\\
-0.0885752515881706	0\\
-0.0893746706407535	0\\
-0.0902458011796924	0\\
-0.0911940056346637	0\\
-0.0922249237841842	0\\
-0.0933444873355378	0\\
-0.0945589420922342	0\\
-0.0958748810682332	0\\
-0.0972992929541448	0\\
-0.098839631723794	0\\
-0.0990873538657956	0\\
-0.0990952628791606	0\\
-0.0991031669661326	0\\
-0.100503915040895	0\\
-0.102300861719523	0\\
-0.104240082177677	0\\
-0.106332341190221	0\\
-0.1085899202569	0\\
-0.111027119154807	0\\
-0.113660955484353	0\\
-0.116512152078757	0\\
-0.119606553851946	0\\
-0.122977204738043	0\\
-0.126667474976819	0\\
-0.130735928223321	0\\
-0.135264209614481	0\\
-0.140370482406512	0\\
-0.146233777005586	0\\
-0.153141716997292	0\\
-0.161594175548456	0\\
-0.166949896198557	0\\
-0.172562276926219	0\\
-0.179986616189226	0\\
-0.188281700676067	0\\
-0.200298592453817	0\\
-0.207422830374181	0\\
-0.215606650038428	0\\
-0.227641386353211	0\\
-0.242830362304848	0\\
-0.262768087997266	0\\
-0.28996366787719	0\\
-0.345167432953542	0\\
-0.478882659159308	0\\
-0.5	1.83497017401247e-08\\
-0.499996068761037	0.0211275473195168\\
-0.501224913352219	0.203031956385616\\
-0.504379391005213	0.311290201465268\\
-0.507151637714082	0.389356648767733\\
-0.511759721274597	0.532189945695628\\
-0.514989092349724	0.665058793572132\\
-0.51730791913881	0.797486665596829\\
-0.519019934310172	0.934163041764775\\
-0.520312138442508	1.07815796746604\\
-0.521303960381161	1.23189193181622\\
-0.522074985224634	1.39752275865309\\
-0.522680251196862	1.57713196961159\\
-0.523159004438822	1.77282939436154\\
-0.523539936453529	1.98682063109345\\
-0.523844451022471	2.22145655067634\\
-0.524088779398197	2.47927403302538\\
-0.524285395394388	2.76303275379165\\
-0.524443989587997	3.07575077282271\\
-0.524572156951753	3.42074063839496\\
-0.52512344554142	69.2117540067618\\
inf	0\\
};
\addplot [color=black!50!green,solid,forget plot]
  table[row sep=crcr]{%
-1.15912180370184	0\\
-1.15871079705485	0\\
-1.15861725603092	0\\
-1.15850241576994	0\\
-1.15836142125453	0\\
-1.15818830815988	0\\
-1.15797574804894	0\\
-1.15771473435537	0\\
-1.15739419504487	0\\
-1.15700051430566	0\\
-1.15651694109959	0\\
-1.15592285659205	0\\
-1.15519286493201	0\\
-1.15506661476307	0\\
-1.15506254266662	0\\
-1.1550584705443	0\\
-1.15429566195186	0\\
-1.15319262320713	0\\
-1.15183603506873	0\\
-1.15016686835732	0\\
-1.1481119602816	0\\
-1.14558042254122	0\\
-1.14245902390708	0\\
-1.13860619210308	0\\
-1.1338441215403	0\\
-1.12794822384247	0\\
-1.12063275051553	0\\
-1.11153072460388	0\\
-1.10016508610448	0\\
-1.08590564161533	0\\
-1.06790176853402	0\\
-1.04497074261412	0\\
-1.01539730324094	0\\
-0.996354368458771	0\\
-0.976533289286463	0\\
-0.950985391199949	0\\
-0.923863816377166	0\\
-0.887926025919618	0\\
-0.868614628440231	0\\
-0.848209167940182	0\\
-0.8213540617396	0\\
-0.791982931837891	0\\
-0.759230893080733	0\\
-0.721585818223746	0\\
-0.657656117620083	0\\
-0.521126395912401	0\\
-0.5	-1.83497017401247e-08\\
-0.499996068761037	-0.0211275473195168\\
-0.501224913352219	-0.203031956385616\\
-0.504379391005213	-0.311290201465268\\
-0.507151637714082	-0.389356648767733\\
-0.511759721274597	-0.532189945695628\\
-0.514989092349724	-0.665058793572132\\
-0.51730791913881	-0.797486665596829\\
-0.519019934310172	-0.934163041764775\\
-0.520312138442508	-1.07815796746604\\
-0.521303960381161	-1.23189193181622\\
-0.522074985224634	-1.39752275865309\\
-0.522680251196862	-1.57713196961159\\
-0.523159004438822	-1.77282939436154\\
-0.523539936453529	-1.98682063109345\\
-0.523844451022471	-2.22145655067634\\
-0.524088779398197	-2.47927403302538\\
-0.524285395394388	-2.76303275379165\\
-0.524443989587997	-3.07575077282271\\
-0.524572156951753	-3.42074063839496\\
-0.52512344554142	-69.2117540067618\\
inf	0\\
};
\addplot [color=white!40!black,dotted,forget plot]
  table[row sep=crcr]{%
0	0\\
-0	1.6\\
nan	nan\\
0	0\\
-0.192	1.58843822668683\\
nan	nan\\
0	0\\
-0.384	1.55323662073748\\
nan	nan\\
0	0\\
-0.608	1.47997837822044\\
nan	nan\\
0	0\\
-0.8	1.3856406460551\\
nan	nan\\
0	0\\
-1.024	1.22939985358711\\
nan	nan\\
0	0\\
-1.216	1.0398769157934\\
nan	nan\\
0	0\\
-1.408	0.759957893570427\\
nan	nan\\
0	0\\
-1.552	0.388967864996583\\
nan	nan\\
0	0\\
-1.6	0\\
nan	nan\\
0	-0\\
-0	-1.6\\
nan	nan\\
0	-0\\
-0.192	-1.58843822668683\\
nan	nan\\
0	-0\\
-0.384	-1.55323662073748\\
nan	nan\\
0	-0\\
-0.608	-1.47997837822044\\
nan	nan\\
0	-0\\
-0.8	-1.3856406460551\\
nan	nan\\
0	-0\\
-1.024	-1.22939985358711\\
nan	nan\\
0	-0\\
-1.216	-1.0398769157934\\
nan	nan\\
0	-0\\
-1.408	-0.759957893570427\\
nan	nan\\
0	-0\\
-1.552	-0.388967864996583\\
nan	nan\\
0	-0\\
-1.6	-0\\
nan	nan\\
};
\addplot [color=white!40!black,dotted,forget plot]
  table[row sep=crcr]{%
-0	0\\
-0	0\\
-0	0\\
-0	0\\
-0	0\\
-0	0\\
-0	0\\
-0	0\\
-0	0\\
-0	0\\
-0	0\\
-0	0\\
-0	0\\
-0	0\\
-0	0\\
-0	0\\
-0	0\\
-0	0\\
-0	0\\
-0	0\\
-0	0\\
-0	0\\
-0	0\\
-0	0\\
-0	0\\
-0	0\\
-0	0\\
-0	0\\
-0	0\\
-0	0\\
-0	0\\
-0	0\\
-0	0\\
-0	0\\
-0	0\\
-0	0\\
-0	0\\
-0	0\\
-0	0\\
-0	0\\
-0	0\\
nan	nan\\
-0	0.2\\
-0.0054985357753655	0.199924400972785\\
-0.0110015565231823	0.199697185143074\\
-0.0165135128595992	0.199317093830499\\
-0.0220387854178422	0.198782021162142\\
-0.0275816466568833	0.198089002137158\\
-0.0331462187635179	0.19723419627864\\
-0.0387364262364926	0.196212867270281\\
-0.0443559416523546	0.19501935914194\\
-0.0500081230081741	0.193647069776951\\
-0.0556959409222359	0.192088422776561\\
-0.0614218938590815	0.190334839046255\\
-0.0671879094426594	0.188376709878703\\
-0.0729952298487411	0.186203373812961\\
-0.0788442792497768	0.183803100163145\\
-0.0847345113550247	0.181163082843678\\
-0.0906642352893107	0.17826944897936\\
-0.0966304184403005	0.175107287775388\\
-0.10262846554481	0.171660706219909\\
-0.108651974260549	0.167912919363833\\
-0.114692468872666	0.163846384103808\\
-0.120739115711546	0.159442986478533\\
-0.126778426401034	0.154684293317963\\
-0.132793958282688	0.149551879438594\\
-0.138766025290247	0.144027741165188\\
-0.144671437110985	0.138094805420922\\
-0.150483289467329	0.131737540553527\\
-0.156170833400534	0.12494267003303\\
-0.161699455927761	0.117699982806567\\
-0.167030807509977	0.110003224237133\\
-0.17212311231062	0.101851039309895\\
-0.176931693997889	0.0932479257626533\\
-0.181409741640536	0.0842051402107613\\
-0.185509326192266	0.0747414871118537\\
-0.189182657981152	0.0648839110965614\\
-0.192383550434224	0.054667810659679\\
-0.19506902725479	0.0441369981520023\\
-0.197200983091473	0.0333432492081472\\
-0.198747786053174	0.022345414271427\\
-0.199685699074389	0.0112081035493346\\
-0.2	0\\
nan	nan\\
-0	0.4\\
-0.010997071550731	0.399848801945571\\
-0.0220031130463645	0.399394370286148\\
-0.0330270257191984	0.398634187660998\\
-0.0440775708356844	0.397564042324284\\
-0.0551632933137667	0.396178004274315\\
-0.0662924375270359	0.39446839255728\\
-0.0774728524729852	0.392425734540561\\
-0.0887118833047092	0.39003871828388\\
-0.100016246016348	0.387294139553902\\
-0.111391881844472	0.384176845553122\\
-0.122843787718163	0.38066967809251\\
-0.134375818885319	0.376753419757406\\
-0.145990459697482	0.372406747625923\\
-0.157688558499554	0.367606200326291\\
-0.169469022710049	0.362326165687355\\
-0.181328470578621	0.356538897958719\\
-0.193260836880601	0.350214575550775\\
-0.205256931089619	0.343321412439817\\
-0.217303948521098	0.335825838727666\\
-0.229384937745331	0.327692768207616\\
-0.241478231423093	0.318885972957067\\
-0.253556852802067	0.309368586635927\\
-0.265587916565376	0.299103758877188\\
-0.277532050580494	0.288055482330377\\
-0.28934287422197	0.276189610841844\\
-0.300966578934658	0.263475081107053\\
-0.312341666801068	0.249885340066061\\
-0.323398911855522	0.235399965613134\\
-0.334061615019954	0.220006448474266\\
-0.34424622462124	0.203702078619789\\
-0.353863387995778	0.186495851525307\\
-0.362819483281072	0.168410280421523\\
-0.371018652384532	0.149482974223707\\
-0.378365315962304	0.129767822193123\\
-0.384767100868447	0.109335621319358\\
-0.390138054509581	0.0882739963040046\\
-0.394401966182946	0.0666864984162943\\
-0.397495572106348	0.044690828542854\\
-0.399371398148778	0.0224162070986691\\
-0.4	0\\
nan	nan\\
-0	0.6\\
-0.0164956073260965	0.599773202918356\\
-0.0330046695695468	0.599091555429222\\
-0.0495405385787977	0.597951281491496\\
-0.0661163562535266	0.596346063486426\\
-0.08274493997065	0.594267006411473\\
-0.0994386562905538	0.59170258883592\\
-0.116209278709478	0.588638601810842\\
-0.133067824957064	0.58505807742582\\
-0.150024369024522	0.580941209330853\\
-0.167087822766708	0.576265268329683\\
-0.184265681577245	0.571004517138765\\
-0.201563728327978	0.565130129636109\\
-0.218985689546223	0.558610121438884\\
-0.236532837749331	0.551409300489436\\
-0.254203534065074	0.543489248531033\\
-0.271992705867932	0.534808346938079\\
-0.289891255320901	0.525321863326163\\
-0.307885396634429	0.514982118659726\\
-0.325955922781647	0.503738758091499\\
-0.344077406617997	0.491539152311424\\
-0.362217347134639	0.478328959435601\\
-0.380335279203101	0.46405287995389\\
-0.398381874848064	0.448655638315782\\
-0.416298075870741	0.432083223495565\\
-0.434014311332955	0.414284416262766\\
-0.451449868401987	0.39521262166058\\
-0.468512500201602	0.374828010099091\\
-0.485098367783282	0.353099948419701\\
-0.501092422529931	0.3300096727114\\
-0.51636933693186	0.305553117929684\\
-0.530795081993668	0.27974377728796\\
-0.544229224921608	0.252615420632284\\
-0.556527978576798	0.224224461335561\\
-0.567547973943456	0.194651733289684\\
-0.577150651302671	0.164003431979037\\
-0.585207081764371	0.132410994456007\\
-0.591602949274419	0.100029747624441\\
-0.596243358159523	0.0670362428142811\\
-0.599057097223167	0.0336243106480037\\
-0.6	0\\
nan	nan\\
-0	0.8\\
-0.021994143101462	0.799697603891141\\
-0.044006226092729	0.798788740572297\\
-0.0660540514383969	0.797268375321995\\
-0.0881551416713688	0.795128084648568\\
-0.110326586627533	0.792356008548631\\
-0.132584875054072	0.788936785114559\\
-0.15494570494597	0.784851469081123\\
-0.177423766609418	0.780077436567759\\
-0.200032492032696	0.774588279107804\\
-0.222783763688944	0.768353691106244\\
-0.245687575436326	0.761339356185019\\
-0.268751637770638	0.753506839514812\\
-0.291980919394964	0.744813495251846\\
-0.315377116999107	0.735212400652581\\
-0.338938045420099	0.72465233137471\\
-0.362656941157243	0.713077795917439\\
-0.386521673761202	0.700429151101551\\
-0.410513862179239	0.686642824879635\\
-0.434607897042195	0.671651677455331\\
-0.458769875490663	0.655385536415232\\
-0.482956462846185	0.637771945914134\\
-0.507113705604134	0.618737173271854\\
-0.531175833130752	0.598207517754376\\
-0.555064101160988	0.576110964660754\\
-0.57868574844394	0.552379221683688\\
-0.601933157869316	0.526950162214107\\
-0.624683333602136	0.499770680132121\\
-0.646797823711043	0.470799931226268\\
-0.668123230039908	0.440012896948533\\
-0.68849244924248	0.407404157239578\\
-0.707726775991557	0.372991703050613\\
-0.725638966562144	0.336820560843045\\
-0.742037304769065	0.298965948447415\\
-0.756730631924608	0.259535644386246\\
-0.769534201736895	0.218671242638716\\
-0.780276109019161	0.176547992608009\\
-0.788803932365892	0.133372996832589\\
-0.794991144212697	0.0893816570857081\\
-0.798742796297556	0.0448324141973383\\
-0.8	0\\
nan	nan\\
-0	1\\
-0.0274926788768275	0.999622004863926\\
-0.0550077826159112	0.998485925715371\\
-0.0825675642979961	0.996585469152494\\
-0.110193927089211	0.99391010581071\\
-0.137908233284417	0.990445010685788\\
-0.16573109381759	0.986170981393199\\
-0.193682131182463	0.981064336351403\\
-0.221779708261773	0.975096795709699\\
-0.25004061504087	0.968235348884755\\
-0.27847970461118	0.960442113882805\\
-0.307109469295408	0.951674195231274\\
-0.335939547213297	0.941883549393514\\
-0.364976149243705	0.931016869064807\\
-0.394221396248884	0.919015500815726\\
-0.423672556775123	0.905815414218388\\
-0.453321176446553	0.891347244896798\\
-0.483152092201502	0.875536438876938\\
-0.513142327724049	0.858303531099544\\
-0.543259871302744	0.839564596819164\\
-0.573462344363328	0.81923192051904\\
-0.603695578557732	0.797214932392667\\
-0.633892132005168	0.773421466589817\\
-0.66396979141344	0.74775939719297\\
-0.693830126451235	0.720138705825942\\
-0.723357185554925	0.69047402710461\\
-0.752416447336645	0.658687702767634\\
-0.78085416700267	0.624713350165151\\
-0.808497279638804	0.588499914032835\\
-0.835154037549885	0.550016121185666\\
-0.8606155615531	0.509255196549473\\
-0.884658469989446	0.466239628813266\\
-0.90704870820268	0.421025701053807\\
-0.927546630961331	0.373707435559269\\
-0.94591328990576	0.324419555482807\\
-0.961917752171118	0.273339053298395\\
-0.975345136273952	0.220684990760011\\
-0.986004915457365	0.166716246040736\\
-0.993738930265871	0.111727071357135\\
-0.998428495371944	0.0560405177466728\\
-1	0\\
nan	nan\\
-0	1.2\\
-0.032991214652193	1.19954640583671\\
-0.0660093391390935	1.19818311085844\\
-0.0990810771575953	1.19590256298299\\
-0.132232712507053	1.19269212697285\\
-0.1654898799413	1.18853401282295\\
-0.198877312581108	1.18340517767184\\
-0.232418557418956	1.17727720362168\\
-0.266135649914128	1.17011615485164\\
-0.300048738049045	1.16188241866171\\
-0.334175645533416	1.15253053665937\\
-0.368531363154489	1.14200903427753\\
-0.403127456655956	1.13026025927222\\
-0.437971379092446	1.11722024287777\\
-0.473065675498661	1.10281860097887\\
-0.508407068130148	1.08697849706207\\
-0.543985411735864	1.06961669387616\\
-0.579782510641803	1.05064372665233\\
-0.615770793268858	1.02996423731945\\
-0.651911845563293	1.007477516183\\
-0.688154813235994	0.983078304622849\\
-0.724434694269278	0.956657918871201\\
-0.760670558406201	0.92810575990778\\
-0.796763749696128	0.897311276631564\\
-0.832596151741482	0.864166446991131\\
-0.86802862266591	0.828568832525532\\
-0.902899736803974	0.79042524332116\\
-0.937025000403204	0.749656020198182\\
-0.970196735566565	0.706199896839402\\
-1.00218484505986	0.660019345422799\\
-1.03273867386372	0.611106235859368\\
-1.06159016398734	0.55948755457592\\
-1.08845844984322	0.505230841264568\\
-1.1130559571536	0.448448922671122\\
-1.13509594788691	0.389303466579369\\
-1.15430130260534	0.328006863958074\\
-1.17041416352874	0.264821988912014\\
-1.18320589854884	0.200059495248883\\
-1.19248671631905	0.134072485628562\\
-1.19811419444633	0.0672486212960074\\
-1.2	0\\
nan	nan\\
-0	1.4\\
-0.0384897504275585	1.3994708068095\\
-0.0770108956622757	1.39788029600152\\
-0.115594590017195	1.39521965681349\\
-0.154271497924896	1.39147414813499\\
-0.193071526598183	1.3866230149601\\
-0.232023531344626	1.38063937395048\\
-0.271154983655448	1.37349007089196\\
-0.310491591566482	1.36513551399358\\
-0.350056861057219	1.35552948843866\\
-0.389871586455652	1.34461895943593\\
-0.429953257013571	1.33234387332378\\
-0.470315366098616	1.31863696915092\\
-0.510966608941187	1.30342361669073\\
-0.551909954748438	1.28662170114202\\
-0.593141579485173	1.26814157990574\\
-0.634649647025175	1.24788614285552\\
-0.676412929082103	1.22575101442771\\
-0.718399258813668	1.20162494353936\\
-0.760563819823842	1.17539043554683\\
-0.80284728210866	1.14692468872666\\
-0.845173809980825	1.11610090534973\\
-0.887448984807235	1.08279005322574\\
-0.929557707978816	1.04686315607016\\
-0.97136217703173	1.00819418815632\\
-1.0127000597769	0.966663637946454\\
-1.0533830262713	0.922162783874687\\
-1.09319583380374	0.874598690231212\\
-1.13189619149433	0.823899879645969\\
-1.16921565256984	0.770022569659933\\
-1.20486178617434	0.712957275169262\\
-1.23852185798522	0.652735480338573\\
-1.26986819148375	0.589435981475329\\
-1.29856528334586	0.523190409782976\\
-1.32427860586806	0.45418737767593\\
-1.34668485303957	0.382674674617753\\
-1.36548319078353	0.308958987064016\\
-1.38040688164031	0.23340274445703\\
-1.39123450237222	0.156417899899989\\
-1.39779989352072	0.078456724845342\\
-1.4	0\\
nan	nan\\
-0	1.6\\
-0.043988286202924	1.59939520778228\\
-0.088012452185458	1.59757748114459\\
-0.132108102876794	1.59453675064399\\
-0.176310283342738	1.59025616929714\\
-0.220653173255067	1.58471201709726\\
-0.265169750108143	1.57787357022912\\
-0.309891409891941	1.56970293816225\\
-0.354847533218837	1.56015487313552\\
-0.400064984065393	1.54917655821561\\
-0.445567527377887	1.53670738221249\\
-0.491375150872652	1.52267871237004\\
-0.537503275541275	1.50701367902962\\
-0.583961838789928	1.48962699050369\\
-0.630754233998215	1.47042480130516\\
-0.677876090840198	1.44930466274942\\
-0.725313882314485	1.42615559183488\\
-0.773043347522404	1.4008583022031\\
-0.821027724358478	1.37328564975927\\
-0.869215794084391	1.34330335491066\\
-0.917539750981326	1.31077107283046\\
-0.965912925692371	1.27554389182827\\
-1.01422741120827	1.23747434654371\\
-1.0623516662615	1.19641503550875\\
-1.11012820232198	1.15222192932151\\
-1.15737149688788	1.10475844336738\\
-1.20386631573863	1.05390032442821\\
-1.24936666720427	0.999541360264242\\
-1.29359564742209	0.941599862452536\\
-1.33624646007982	0.880025793897066\\
-1.37698489848496	0.814808314479157\\
-1.41545355198311	0.745983406101226\\
-1.45127793312429	0.673641121686091\\
-1.48407460953813	0.59793189689483\\
-1.51346126384922	0.519071288772491\\
-1.53906840347379	0.437342485277432\\
-1.56055221803832	0.353095985216018\\
-1.57760786473178	0.266745993665177\\
-1.58998228842539	0.178763314171416\\
-1.59748559259511	0.0896648283946765\\
-1.6	0\\
nan	nan\\
-0	-0\\
-0	-0\\
-0	-0\\
-0	-0\\
-0	-0\\
-0	-0\\
-0	-0\\
-0	-0\\
-0	-0\\
-0	-0\\
-0	-0\\
-0	-0\\
-0	-0\\
-0	-0\\
-0	-0\\
-0	-0\\
-0	-0\\
-0	-0\\
-0	-0\\
-0	-0\\
-0	-0\\
-0	-0\\
-0	-0\\
-0	-0\\
-0	-0\\
-0	-0\\
-0	-0\\
-0	-0\\
-0	-0\\
-0	-0\\
-0	-0\\
-0	-0\\
-0	-0\\
-0	-0\\
-0	-0\\
-0	-0\\
-0	-0\\
-0	-0\\
-0	-0\\
-0	-0\\
-0	-0\\
nan	nan\\
-0	-0.2\\
-0.0054985357753655	-0.199924400972785\\
-0.0110015565231823	-0.199697185143074\\
-0.0165135128595992	-0.199317093830499\\
-0.0220387854178422	-0.198782021162142\\
-0.0275816466568833	-0.198089002137158\\
-0.0331462187635179	-0.19723419627864\\
-0.0387364262364926	-0.196212867270281\\
-0.0443559416523546	-0.19501935914194\\
-0.0500081230081741	-0.193647069776951\\
-0.0556959409222359	-0.192088422776561\\
-0.0614218938590815	-0.190334839046255\\
-0.0671879094426594	-0.188376709878703\\
-0.0729952298487411	-0.186203373812961\\
-0.0788442792497768	-0.183803100163145\\
-0.0847345113550247	-0.181163082843678\\
-0.0906642352893107	-0.17826944897936\\
-0.0966304184403005	-0.175107287775388\\
-0.10262846554481	-0.171660706219909\\
-0.108651974260549	-0.167912919363833\\
-0.114692468872666	-0.163846384103808\\
-0.120739115711546	-0.159442986478533\\
-0.126778426401034	-0.154684293317963\\
-0.132793958282688	-0.149551879438594\\
-0.138766025290247	-0.144027741165188\\
-0.144671437110985	-0.138094805420922\\
-0.150483289467329	-0.131737540553527\\
-0.156170833400534	-0.12494267003303\\
-0.161699455927761	-0.117699982806567\\
-0.167030807509977	-0.110003224237133\\
-0.17212311231062	-0.101851039309895\\
-0.176931693997889	-0.0932479257626533\\
-0.181409741640536	-0.0842051402107613\\
-0.185509326192266	-0.0747414871118537\\
-0.189182657981152	-0.0648839110965614\\
-0.192383550434224	-0.054667810659679\\
-0.19506902725479	-0.0441369981520023\\
-0.197200983091473	-0.0333432492081472\\
-0.198747786053174	-0.022345414271427\\
-0.199685699074389	-0.0112081035493346\\
-0.2	-0\\
nan	nan\\
-0	-0.4\\
-0.010997071550731	-0.399848801945571\\
-0.0220031130463645	-0.399394370286148\\
-0.0330270257191984	-0.398634187660998\\
-0.0440775708356844	-0.397564042324284\\
-0.0551632933137667	-0.396178004274315\\
-0.0662924375270359	-0.39446839255728\\
-0.0774728524729852	-0.392425734540561\\
-0.0887118833047092	-0.39003871828388\\
-0.100016246016348	-0.387294139553902\\
-0.111391881844472	-0.384176845553122\\
-0.122843787718163	-0.38066967809251\\
-0.134375818885319	-0.376753419757406\\
-0.145990459697482	-0.372406747625923\\
-0.157688558499554	-0.367606200326291\\
-0.169469022710049	-0.362326165687355\\
-0.181328470578621	-0.356538897958719\\
-0.193260836880601	-0.350214575550775\\
-0.205256931089619	-0.343321412439817\\
-0.217303948521098	-0.335825838727666\\
-0.229384937745331	-0.327692768207616\\
-0.241478231423093	-0.318885972957067\\
-0.253556852802067	-0.309368586635927\\
-0.265587916565376	-0.299103758877188\\
-0.277532050580494	-0.288055482330377\\
-0.28934287422197	-0.276189610841844\\
-0.300966578934658	-0.263475081107053\\
-0.312341666801068	-0.249885340066061\\
-0.323398911855522	-0.235399965613134\\
-0.334061615019954	-0.220006448474266\\
-0.34424622462124	-0.203702078619789\\
-0.353863387995778	-0.186495851525307\\
-0.362819483281072	-0.168410280421523\\
-0.371018652384532	-0.149482974223707\\
-0.378365315962304	-0.129767822193123\\
-0.384767100868447	-0.109335621319358\\
-0.390138054509581	-0.0882739963040046\\
-0.394401966182946	-0.0666864984162943\\
-0.397495572106348	-0.044690828542854\\
-0.399371398148778	-0.0224162070986691\\
-0.4	-0\\
nan	nan\\
-0	-0.6\\
-0.0164956073260965	-0.599773202918356\\
-0.0330046695695468	-0.599091555429222\\
-0.0495405385787977	-0.597951281491496\\
-0.0661163562535266	-0.596346063486426\\
-0.08274493997065	-0.594267006411473\\
-0.0994386562905538	-0.59170258883592\\
-0.116209278709478	-0.588638601810842\\
-0.133067824957064	-0.58505807742582\\
-0.150024369024522	-0.580941209330853\\
-0.167087822766708	-0.576265268329683\\
-0.184265681577245	-0.571004517138765\\
-0.201563728327978	-0.565130129636109\\
-0.218985689546223	-0.558610121438884\\
-0.236532837749331	-0.551409300489436\\
-0.254203534065074	-0.543489248531033\\
-0.271992705867932	-0.534808346938079\\
-0.289891255320901	-0.525321863326163\\
-0.307885396634429	-0.514982118659726\\
-0.325955922781647	-0.503738758091499\\
-0.344077406617997	-0.491539152311424\\
-0.362217347134639	-0.478328959435601\\
-0.380335279203101	-0.46405287995389\\
-0.398381874848064	-0.448655638315782\\
-0.416298075870741	-0.432083223495565\\
-0.434014311332955	-0.414284416262766\\
-0.451449868401987	-0.39521262166058\\
-0.468512500201602	-0.374828010099091\\
-0.485098367783282	-0.353099948419701\\
-0.501092422529931	-0.3300096727114\\
-0.51636933693186	-0.305553117929684\\
-0.530795081993668	-0.27974377728796\\
-0.544229224921608	-0.252615420632284\\
-0.556527978576798	-0.224224461335561\\
-0.567547973943456	-0.194651733289684\\
-0.577150651302671	-0.164003431979037\\
-0.585207081764371	-0.132410994456007\\
-0.591602949274419	-0.100029747624441\\
-0.596243358159523	-0.0670362428142811\\
-0.599057097223167	-0.0336243106480037\\
-0.6	-0\\
nan	nan\\
-0	-0.8\\
-0.021994143101462	-0.799697603891141\\
-0.044006226092729	-0.798788740572297\\
-0.0660540514383969	-0.797268375321995\\
-0.0881551416713688	-0.795128084648568\\
-0.110326586627533	-0.792356008548631\\
-0.132584875054072	-0.788936785114559\\
-0.15494570494597	-0.784851469081123\\
-0.177423766609418	-0.780077436567759\\
-0.200032492032696	-0.774588279107804\\
-0.222783763688944	-0.768353691106244\\
-0.245687575436326	-0.761339356185019\\
-0.268751637770638	-0.753506839514812\\
-0.291980919394964	-0.744813495251846\\
-0.315377116999107	-0.735212400652581\\
-0.338938045420099	-0.72465233137471\\
-0.362656941157243	-0.713077795917439\\
-0.386521673761202	-0.700429151101551\\
-0.410513862179239	-0.686642824879635\\
-0.434607897042195	-0.671651677455331\\
-0.458769875490663	-0.655385536415232\\
-0.482956462846185	-0.637771945914134\\
-0.507113705604134	-0.618737173271854\\
-0.531175833130752	-0.598207517754376\\
-0.555064101160988	-0.576110964660754\\
-0.57868574844394	-0.552379221683688\\
-0.601933157869316	-0.526950162214107\\
-0.624683333602136	-0.499770680132121\\
-0.646797823711043	-0.470799931226268\\
-0.668123230039908	-0.440012896948533\\
-0.68849244924248	-0.407404157239578\\
-0.707726775991557	-0.372991703050613\\
-0.725638966562144	-0.336820560843045\\
-0.742037304769065	-0.298965948447415\\
-0.756730631924608	-0.259535644386246\\
-0.769534201736895	-0.218671242638716\\
-0.780276109019161	-0.176547992608009\\
-0.788803932365892	-0.133372996832589\\
-0.794991144212697	-0.0893816570857081\\
-0.798742796297556	-0.0448324141973383\\
-0.8	-0\\
nan	nan\\
-0	-1\\
-0.0274926788768275	-0.999622004863926\\
-0.0550077826159112	-0.998485925715371\\
-0.0825675642979961	-0.996585469152494\\
-0.110193927089211	-0.99391010581071\\
-0.137908233284417	-0.990445010685788\\
-0.16573109381759	-0.986170981393199\\
-0.193682131182463	-0.981064336351403\\
-0.221779708261773	-0.975096795709699\\
-0.25004061504087	-0.968235348884755\\
-0.27847970461118	-0.960442113882805\\
-0.307109469295408	-0.951674195231274\\
-0.335939547213297	-0.941883549393514\\
-0.364976149243705	-0.931016869064807\\
-0.394221396248884	-0.919015500815726\\
-0.423672556775123	-0.905815414218388\\
-0.453321176446553	-0.891347244896798\\
-0.483152092201502	-0.875536438876938\\
-0.513142327724049	-0.858303531099544\\
-0.543259871302744	-0.839564596819164\\
-0.573462344363328	-0.81923192051904\\
-0.603695578557732	-0.797214932392667\\
-0.633892132005168	-0.773421466589817\\
-0.66396979141344	-0.74775939719297\\
-0.693830126451235	-0.720138705825942\\
-0.723357185554925	-0.69047402710461\\
-0.752416447336645	-0.658687702767634\\
-0.78085416700267	-0.624713350165151\\
-0.808497279638804	-0.588499914032835\\
-0.835154037549885	-0.550016121185666\\
-0.8606155615531	-0.509255196549473\\
-0.884658469989446	-0.466239628813266\\
-0.90704870820268	-0.421025701053807\\
-0.927546630961331	-0.373707435559269\\
-0.94591328990576	-0.324419555482807\\
-0.961917752171118	-0.273339053298395\\
-0.975345136273952	-0.220684990760011\\
-0.986004915457365	-0.166716246040736\\
-0.993738930265871	-0.111727071357135\\
-0.998428495371944	-0.0560405177466728\\
-1	-0\\
nan	nan\\
-0	-1.2\\
-0.032991214652193	-1.19954640583671\\
-0.0660093391390935	-1.19818311085844\\
-0.0990810771575953	-1.19590256298299\\
-0.132232712507053	-1.19269212697285\\
-0.1654898799413	-1.18853401282295\\
-0.198877312581108	-1.18340517767184\\
-0.232418557418956	-1.17727720362168\\
-0.266135649914128	-1.17011615485164\\
-0.300048738049045	-1.16188241866171\\
-0.334175645533416	-1.15253053665937\\
-0.368531363154489	-1.14200903427753\\
-0.403127456655956	-1.13026025927222\\
-0.437971379092446	-1.11722024287777\\
-0.473065675498661	-1.10281860097887\\
-0.508407068130148	-1.08697849706207\\
-0.543985411735864	-1.06961669387616\\
-0.579782510641803	-1.05064372665233\\
-0.615770793268858	-1.02996423731945\\
-0.651911845563293	-1.007477516183\\
-0.688154813235994	-0.983078304622849\\
-0.724434694269278	-0.956657918871201\\
-0.760670558406201	-0.92810575990778\\
-0.796763749696128	-0.897311276631564\\
-0.832596151741482	-0.864166446991131\\
-0.86802862266591	-0.828568832525532\\
-0.902899736803974	-0.79042524332116\\
-0.937025000403204	-0.749656020198182\\
-0.970196735566565	-0.706199896839402\\
-1.00218484505986	-0.660019345422799\\
-1.03273867386372	-0.611106235859368\\
-1.06159016398734	-0.55948755457592\\
-1.08845844984322	-0.505230841264568\\
-1.1130559571536	-0.448448922671122\\
-1.13509594788691	-0.389303466579369\\
-1.15430130260534	-0.328006863958074\\
-1.17041416352874	-0.264821988912014\\
-1.18320589854884	-0.200059495248883\\
-1.19248671631905	-0.134072485628562\\
-1.19811419444633	-0.0672486212960074\\
-1.2	-0\\
nan	nan\\
-0	-1.4\\
-0.0384897504275585	-1.3994708068095\\
-0.0770108956622757	-1.39788029600152\\
-0.115594590017195	-1.39521965681349\\
-0.154271497924896	-1.39147414813499\\
-0.193071526598183	-1.3866230149601\\
-0.232023531344626	-1.38063937395048\\
-0.271154983655448	-1.37349007089196\\
-0.310491591566482	-1.36513551399358\\
-0.350056861057219	-1.35552948843866\\
-0.389871586455652	-1.34461895943593\\
-0.429953257013571	-1.33234387332378\\
-0.470315366098616	-1.31863696915092\\
-0.510966608941187	-1.30342361669073\\
-0.551909954748438	-1.28662170114202\\
-0.593141579485173	-1.26814157990574\\
-0.634649647025175	-1.24788614285552\\
-0.676412929082103	-1.22575101442771\\
-0.718399258813668	-1.20162494353936\\
-0.760563819823842	-1.17539043554683\\
-0.80284728210866	-1.14692468872666\\
-0.845173809980825	-1.11610090534973\\
-0.887448984807235	-1.08279005322574\\
-0.929557707978816	-1.04686315607016\\
-0.97136217703173	-1.00819418815632\\
-1.0127000597769	-0.966663637946454\\
-1.0533830262713	-0.922162783874687\\
-1.09319583380374	-0.874598690231212\\
-1.13189619149433	-0.823899879645969\\
-1.16921565256984	-0.770022569659933\\
-1.20486178617434	-0.712957275169262\\
-1.23852185798522	-0.652735480338573\\
-1.26986819148375	-0.589435981475329\\
-1.29856528334586	-0.523190409782976\\
-1.32427860586806	-0.45418737767593\\
-1.34668485303957	-0.382674674617753\\
-1.36548319078353	-0.308958987064016\\
-1.38040688164031	-0.23340274445703\\
-1.39123450237222	-0.156417899899989\\
-1.39779989352072	-0.078456724845342\\
-1.4	-0\\
nan	nan\\
-0	-1.6\\
-0.043988286202924	-1.59939520778228\\
-0.088012452185458	-1.59757748114459\\
-0.132108102876794	-1.59453675064399\\
-0.176310283342738	-1.59025616929714\\
-0.220653173255067	-1.58471201709726\\
-0.265169750108143	-1.57787357022912\\
-0.309891409891941	-1.56970293816225\\
-0.354847533218837	-1.56015487313552\\
-0.400064984065393	-1.54917655821561\\
-0.445567527377887	-1.53670738221249\\
-0.491375150872652	-1.52267871237004\\
-0.537503275541275	-1.50701367902962\\
-0.583961838789928	-1.48962699050369\\
-0.630754233998215	-1.47042480130516\\
-0.677876090840198	-1.44930466274942\\
-0.725313882314485	-1.42615559183488\\
-0.773043347522404	-1.4008583022031\\
-0.821027724358478	-1.37328564975927\\
-0.869215794084391	-1.34330335491066\\
-0.917539750981326	-1.31077107283046\\
-0.965912925692371	-1.27554389182827\\
-1.01422741120827	-1.23747434654371\\
-1.0623516662615	-1.19641503550875\\
-1.11012820232198	-1.15222192932151\\
-1.15737149688788	-1.10475844336738\\
-1.20386631573863	-1.05390032442821\\
-1.24936666720427	-0.999541360264242\\
-1.29359564742209	-0.941599862452536\\
-1.33624646007982	-0.880025793897066\\
-1.37698489848496	-0.814808314479157\\
-1.41545355198311	-0.745983406101226\\
-1.45127793312429	-0.673641121686091\\
-1.48407460953813	-0.59793189689483\\
-1.51346126384922	-0.519071288772491\\
-1.53906840347379	-0.437342485277432\\
-1.56055221803832	-0.353095985216018\\
-1.57760786473178	-0.266745993665177\\
-1.58998228842539	-0.178763314171416\\
-1.59748559259511	-0.0896648283946765\\
-1.6	-0\\
nan	nan\\
};
\end{axis}
\end{tikzpicture}%}
  \caption{Root locus of the open-loop system without the use of a zero-order hold.}
  \label{fig:Q7.rloc_F_G}
\end{figure}

\begin{figure}[H]\centering
	\centering
	\scalebox{1}{% This file was created by matlab2tikz.
%
%The latest updates can be retrieved from
%  http://www.mathworks.com/matlabcentral/fileexchange/22022-matlab2tikz-matlab2tikz
%where you can also make suggestions and rate matlab2tikz.
%
\definecolor{mycolor1}{rgb}{0.00000,0.44700,0.74100}%
\definecolor{mycolor2}{rgb}{0.75000,0.75000,0.00000}%
\definecolor{mycolor3}{rgb}{0.75000,0.00000,0.75000}%
\definecolor{mycolor4}{rgb}{0.00000,0.75000,0.75000}%
%
\begin{tikzpicture}

\begin{axis}[%
width=4.008in,
height=3.052in,
at={(0.818in,0.44in)},
scale only axis,
unbounded coords=jump,
separate axis lines,
every outer x axis line/.append style={white!40!black},
every x tick label/.append style={font=\color{white!40!black}},
xmin=-10,
xmax=6,
every outer y axis line/.append style={white!40!black},
every y tick label/.append style={font=\color{white!40!black}},
ymin=-8,
ymax=8,
axis background/.style={fill=white}
]
\addplot [color=white,solid,forget plot]
  table[row sep=crcr]{%
0	0\\
};
\addplot [color=mycolor1,only marks,mark=o,mark options={solid},forget plot]
  table[row sep=crcr]{%
-0.134875180971928	0.0760690712223533\\
-0.134875180971928	-0.0760690712223533\\
-2.19779182178189e-16	0\\
};
\addplot [color=mycolor1,only marks,mark=x,mark options={solid},forget plot]
  table[row sep=crcr]{%
-3	1.73205080756888\\
-3	-1.73205080756888\\
-0.0804390944323632	0\\
-0.0804391018657927	0\\
-1.15912180370184	0\\
4.8806657811541e-17	0\\
-1.65289318632125e-18	0\\
};
\addplot [color=blue,solid,forget plot]
  table[row sep=crcr]{%
-3	1.73205080756888\\
-3.00000000000013	1.73205080756892\\
-3.00000000000013	1.73205080756892\\
-3.00000000000013	1.73205080756892\\
-3.00018574345039	1.73210999549279\\
-3.00024952467788	1.73213032886482\\
-3.00033520054779	1.73215764963554\\
-3.00045028154132	1.73219436064315\\
-3.00060485006341	1.73224369235797\\
-3.00081243765992	1.73230998901812\\
-3.00109119859842	1.73239909453141\\
-3.00146547765892	1.73251887357957\\
-3.00186145710671	1.73264577354041\\
-3.00186331722382	1.7326463700815\\
-3.00186517733453	1.73264696662454\\
-3.0019679004465	1.73267991643415\\
-3.00264215298486	1.73289649422645\\
-3.0035466645889	1.73318785704113\\
-3.00475946367187	1.73358000349137\\
-3.00638453721239	1.73410810210982\\
-3.00856008284294	1.73481981859991\\
-3.01146908001185	1.73577990748758\\
-3.01535258878921	1.73707657195051\\
-3.02052605019796	1.73883028872727\\
-3.02739851038195	1.74120602903249\\
-3.03649397681834	1.74443003413943\\
-3.04847286372701	1.74881238821826\\
-3.06414957413294	1.75477626804255\\
-3.08449983889996	1.76289340972457\\
-3.09710955106993	1.76812942496817\\
-3.10942177238854	1.77338999769741\\
-3.10952063090217	1.7734328193247\\
-3.10961947067122	1.77347564204358\\
-3.11064927102963	1.77392235455516\\
-3.11467281437444	1.77567725563282\\
-3.11477591089001	1.77572242111809\\
-3.11487898701366	1.77576758758909\\
-3.12955787385621	1.78230000528241\\
-3.14383451446854	1.78884114691129\\
-3.18533301076605	1.80886061241473\\
-3.2363683952824	1.83540203353982\\
-3.29801300845091	1.87002938698027\\
-3.37111799671745	1.91434308330117\\
-3.45629541263606	1.96986177608149\\
-3.55395606203566	2.03792731170025\\
-3.66438479682309	2.11965788135532\\
-3.7878256102048	2.21595334267563\\
-3.92455452562459	2.32753893105085\\
-4.07493059403701	2.45502726021708\\
-4.23942558458848	2.59898182420575\\
-4.41863789267797	2.75997221768887\\
-4.61329688069298	2.93861752491266\\
-4.82426255606251	3.13561821106619\\
-5.05252378567797	3.35177861306715\\
-5.29919685578166	3.58802254739917\\
-5.56552526159123	3.84540431518027\\
-5.85288107189405	4.12511691924863\\
-6.16276793440798	4.42849883255713\\
-6.49682566295659	4.75704025841028\\
-6.856836306824	5.11238952067469\\
-7.2447316036385	5.49636000692364\\
-7.66260173629013	5.91093794148167\\
-8.11270533960169	6.35829117037366\\
-8.59748072801077	6.84077908118869\\
-9.11955833902618	7.36096374643808\\
-167.50374464114	165.740537665457\\
inf	0\\
};
\addplot [color=darkgray,solid,forget plot]
  table[row sep=crcr]{%
-1.65289318632125e-18	0\\
-2.20017862728807e-16	0\\
-2.20108783189928e-16	0\\
-2.19842449896917e-16	0\\
-2.19779182177895e-16	0\\
-2.19779182177796e-16	0\\
-2.19779182177968e-16	0\\
-2.19779182178119e-16	0\\
-2.19779182177949e-16	0\\
-2.19779182178109e-16	0\\
-2.1977918217816e-16	0\\
-2.19779182178174e-16	0\\
-2.19779182178154e-16	0\\
-2.19779182178129e-16	0\\
-2.19779182178212e-16	0\\
-2.19779182178118e-16	0\\
-2.1977918217807e-16	0\\
-2.19779182178092e-16	0\\
-2.19779182178128e-16	0\\
-2.19779182178124e-16	0\\
-2.19779182178118e-16	0\\
-2.19779182178158e-16	0\\
-2.1977918217815e-16	0\\
-2.19779182178233e-16	0\\
-2.19779182178209e-16	0\\
-2.19779182178189e-16	0\\
-2.19779182178154e-16	0\\
-2.19779182178156e-16	0\\
-2.19779182178181e-16	0\\
-2.19779182178167e-16	0\\
-2.1977918217818e-16	0\\
-2.19779182178181e-16	0\\
-2.19779182178179e-16	0\\
-2.19779182178179e-16	0\\
-2.19779182178186e-16	0\\
-2.19779182178175e-16	0\\
-2.19779182178191e-16	0\\
-2.19779182178185e-16	0\\
-2.1977918217819e-16	0\\
-2.19779182178197e-16	0\\
-2.19779182178192e-16	0\\
-2.19779182178191e-16	0\\
-2.19779182178193e-16	0\\
-2.19779182178186e-16	0\\
-2.19779182178189e-16	0\\
-2.19779182178188e-16	0\\
-2.19779182178191e-16	0\\
-2.1977918217819e-16	0\\
-2.19779182178191e-16	0\\
-2.19779182178189e-16	0\\
-2.19779182178191e-16	0\\
-2.19779182178189e-16	0\\
-2.1977918217819e-16	0\\
-2.1977918217819e-16	0\\
-2.1977918217819e-16	0\\
-2.19779182178192e-16	0\\
-2.19779182178189e-16	0\\
-2.19779182178191e-16	0\\
-2.1977918217819e-16	0\\
-2.1977918217819e-16	0\\
-2.19779182178193e-16	0\\
-2.19779182178191e-16	0\\
-2.19779182178193e-16	0\\
-2.1977918217819e-16	0\\
-2.1977918217819e-16	0\\
-2.1977918217819e-16	0\\
-2.19779182178189e-16	0\\
};
\addplot [color=mycolor2,solid,forget plot]
  table[row sep=crcr]{%
4.8806657811541e-17	0\\
-1.12523922035296e-12	0\\
-1.12590277412615e-12	0\\
-1.12838707433109e-12	0\\
-0.00166082730712354	0\\
-0.00225180971162835	0\\
-0.00306408247812187	0\\
-0.00419138973183814	0\\
-0.00577965690512844	0\\
-0.00807412702544666	0\\
-0.0115480702936874	0\\
-0.0174389045256782	0\\
-0.0315800751047079	0\\
-0.0326381436951767	2.0035159080951e-08\\
-0.0326360143750463	0.00106007432692908\\
-0.0325222010726573	0.00792363802264028\\
-0.031927376276678	0.0212404397374045\\
-0.0314220921049791	0.030652579550166\\
-0.0310745051612175	0.0394437582051263\\
-0.0309807290306922	0.0483272061963577\\
-0.031275969575853	0.0576580061846414\\
-0.0321509516873781	0.0676823977873882\\
-0.0338769893694386	0.0786030766850838\\
-0.0368459721111171	0.0905969461201523\\
-0.0416379398104174	0.10380730055516\\
-0.0491444323196527	0.118301468212372\\
-0.0608186867821561	0.133943512103664\\
-0.0792651060725459	0.149977348430113\\
-0.109961693076483	0.163158310022309\\
-0.134340811494869	0.164275517609474\\
-0.161454212033936	0.152245669411022\\
-0.161651019566681	0.152055912265094\\
-0.161846579877063	0.151864440859541\\
-0.163803722478871	0.149767227711246\\
-0.169449739122379	0.140163783810233\\
-0.169542907387284	0.139901298657482\\
-0.169633301197232	0.139638741840658\\
-0.166873558838923	0.112821005599766\\
-0.160361645034247	0.102269708978243\\
-0.150398223655247	0.0907135830573359\\
-0.145111969835922	0.0854735046907944\\
-0.141920434232436	0.0824661648284569\\
-0.13984760729342	0.0805565715817262\\
-0.138441329560915	0.0792761364164901\\
-0.137460287305012	0.0783888385078409\\
-0.136763009033116	0.0777607526981368\\
-0.136260995524241	0.0773097302627165\\
-0.135896270394896	0.0769826158114825\\
-0.135629562551154	0.0767436917864527\\
-0.135433612638044	0.0765682980073349\\
-0.135289155042162	0.0764390694744112\\
-0.135182391085725	0.0763436003729852\\
-0.135103339569485	0.0762729331033424\\
-0.135044727494537	0.0762205488858227\\
-0.135001226280639	0.0761816761279487\\
-0.134968916076855	0.0761528070676185\\
-0.134944904627433	0.0761313547761771\\
-0.13492705309505	0.0761154068951782\\
-0.134913777157607	0.0761035472397231\\
-0.134903901786412	0.0760947256872687\\
-0.134896554703729	0.0760881627965108\\
-0.134891087932271	0.0760832796164374\\
-0.134887019871364	0.0760796458834948\\
-0.134883992441188	0.0760769417069688\\
-0.134881739326665	0.0760749291846393\\
-0.134875180997266	0.0760690712449868\\
-0.134875180971928	0.0760690712223533\\
};
\addplot [color=mycolor3,solid,forget plot]
  table[row sep=crcr]{%
-1.15912180370184	0\\
-1.15912180370133	0\\
-1.15912180370133	0\\
-1.15912180370133	0\\
-1.15837755310719	0\\
-1.15812187999556	0\\
-1.15777835166931	0\\
-1.15731675994645	0\\
-1.15669649433297	0\\
-1.15586294724616	0\\
-1.15474266645881	0\\
-1.15323680929649	0\\
-1.15164149998898	0\\
-1.15163400078818	0\\
-1.15162650156425	0\\
-1.15121228700693	0\\
-1.1484897526908	0\\
-1.14482724704539	0\\
-1.13989781332376	0\\
-1.13325861905764	0\\
-1.12430787036064	0\\
-1.11222364502067	0\\
-1.09587473390263	0\\
-1.07368527885509	0\\
-1.04341591383222	0\\
-1.00177406642727	0\\
-0.94361079124688	0\\
-0.859847785001545	0\\
-0.729563162458321	0\\
-0.62279283251851	0\\
-0.407945108017524	0\\
-0.388828371233524	0\\
-0.388533949451722	-0.0188523190339899\\
-0.385547006491493	-0.064297528438528\\
-0.375877446503183	-0.142323858148687\\
-0.375681181722702	-0.143859327676446\\
-0.375487711789106	-0.14538209155385\\
-0.36356856730487	-0.294257300730666\\
-0.355803840497215	-0.383853899025846\\
-0.324268765578707	-0.56183489725048\\
-0.27851963488168	-0.720036519658047\\
-0.220066557316655	-0.873726387539305\\
-0.149034395989124	-1.02798551741661\\
-0.0652632578030296	-1.18512067945298\\
0.0314163493406752	-1.34645510295588\\
0.141147805856202	-1.51297533616862\\
0.264086605729042	-1.68560182422243\\
0.400450796019479	-1.86529915451281\\
0.55056015658816	-2.05311242785714\\
0.714859197226519	-2.25017196438932\\
0.893927047720134	-2.45768750018207\\
1.0884792717787	-2.67694155406908\\
1.29936589563199	-2.9092855825112\\
1.52756851317251	-3.15613967944948\\
1.7741980820623	-3.41899545290353\\
2.04049417766808	-3.69942141573024\\
2.32782597652148	-3.99907027181541\\
2.63769498750303	-4.31968762511556\\
2.97173944011421	-4.66312178970166\\
3.33174020861042	-5.03133450231319\\
3.71962815834223	-5.42641243041062\\
4.13749282422239	-5.85057943328586\\
4.58759235947305	-6.30620957833135\\
5.07236472045195	-6.79584094517458\\
5.59444007835284	-7.32219027171108\\
163.978619822137	-165.740461428786\\
inf	0\\
};
\addplot [color=mycolor4,solid,forget plot]
  table[row sep=crcr]{%
-0.0804391018657927	0\\
-0.0804392904054883	0\\
-0.0804392905847375	0\\
-0.0804392904603088	0\\
-0.087194941005693	0\\
-0.0881857638465612	0\\
-0.089310285828975	0\\
-0.0905837184529872	0\\
-0.0920224782264052	0\\
-0.0936442628470515	0\\
-0.0954681844854813	0\\
-0.0975150014520303	0\\
-0.099355034741433	0\\
-0.0993630773738298	0\\
-0.0993711150165938	0\\
-0.0998075099547562	0\\
-0.102371188786128	0\\
-0.105235239566858	0\\
-0.108434249010061	0\\
-0.112010848456188	0\\
-0.116020024801773	0\\
-0.120536291580873	0\\
-0.125666109780082	0\\
-0.13157067652676	0\\
-0.138511185783053	0\\
-0.146949115296747	0\\
-0.157806107734789	0\\
-0.17332285458748	0\\
-0.201513773588797	0\\
-0.2343064423519	0\\
-0.370302923137532	0\\
-0.388828327828783	0\\
-0.388533949451722	0.0188523190339899\\
-0.385547006491493	0.064297528438528\\
-0.375877446503183	0.142323858148687\\
-0.375681181722702	0.143859327676446\\
-0.375487711789106	0.14538209155385\\
-0.36356856730487	0.294257300730666\\
-0.355803840497215	0.383853899025846\\
-0.324268765578707	0.56183489725048\\
-0.27851963488168	0.720036519658047\\
-0.220066557316655	0.873726387539305\\
-0.149034395989124	1.02798551741661\\
-0.0652632578030296	1.18512067945298\\
0.0314163493406752	1.34645510295588\\
0.141147805856202	1.51297533616862\\
0.264086605729042	1.68560182422243\\
0.400450796019479	1.86529915451281\\
0.55056015658816	2.05311242785714\\
0.714859197226519	2.25017196438932\\
0.893927047720134	2.45768750018207\\
1.0884792717787	2.67694155406908\\
1.29936589563199	2.9092855825112\\
1.52756851317251	3.15613967944948\\
1.7741980820623	3.41899545290353\\
2.04049417766808	3.69942141573024\\
2.32782597652148	3.99907027181541\\
2.63769498750303	4.31968762511556\\
2.97173944011421	4.66312178970166\\
3.33174020861042	5.03133450231319\\
3.71962815834223	5.42641243041062\\
4.13749282422239	5.85057943328586\\
4.58759235947305	6.30620957833135\\
5.07236472045195	6.79584094517458\\
5.59444007835284	7.32219027171108\\
163.978619822137	165.740461428786\\
inf	0\\
};
\addplot [color=red,solid,forget plot]
  table[row sep=crcr]{%
-0.0804390944323632	0\\
-0.0804389058918015	0\\
-0.0804389057125516	0\\
-0.0804389058369783	0\\
-0.0723951916792066	0\\
-0.0709414970904845	0\\
-0.0691768789280074	0\\
-0.0670075687860854	0\\
-0.064291670408685	0\\
-0.0607937875615021	0\\
-0.0560586815651727	0\\
-0.0488783294079573	0\\
-0.0337004759514588	0\\
-0.0326381436951767	-2.0035159080951e-08\\
-0.0326360143750463	-0.00106007432692908\\
-0.0325222010726573	-0.00792363802264028\\
-0.031927376276678	-0.0212404397374045\\
-0.0314220921049791	-0.030652579550166\\
-0.0310745051612175	-0.0394437582051263\\
-0.0309807290306922	-0.0483272061963577\\
-0.031275969575853	-0.0576580061846414\\
-0.0321509516873781	-0.0676823977873882\\
-0.0338769893694386	-0.0786030766850838\\
-0.0368459721111171	-0.0905969461201523\\
-0.0416379398104174	-0.10380730055516\\
-0.0491444323196527	-0.118301468212372\\
-0.0608186867821561	-0.133943512103664\\
-0.0792651060725459	-0.149977348430113\\
-0.109961693076483	-0.163158310022309\\
-0.134340811494869	-0.164275517609474\\
-0.161454212033936	-0.152245669411022\\
-0.161651019566681	-0.152055912265094\\
-0.161846579877063	-0.151864440859541\\
-0.163803722478871	-0.149767227711246\\
-0.169449739122379	-0.140163783810233\\
-0.169542907387284	-0.139901298657482\\
-0.169633301197232	-0.139638741840658\\
-0.166873558838923	-0.112821005599766\\
-0.160361645034247	-0.102269708978243\\
-0.150398223655247	-0.0907135830573359\\
-0.145111969835922	-0.0854735046907944\\
-0.141920434232436	-0.0824661648284569\\
-0.13984760729342	-0.0805565715817262\\
-0.138441329560915	-0.0792761364164901\\
-0.137460287305012	-0.0783888385078409\\
-0.136763009033116	-0.0777607526981368\\
-0.136260995524241	-0.0773097302627165\\
-0.135896270394896	-0.0769826158114825\\
-0.135629562551154	-0.0767436917864527\\
-0.135433612638044	-0.0765682980073349\\
-0.135289155042162	-0.0764390694744112\\
-0.135182391085725	-0.0763436003729852\\
-0.135103339569485	-0.0762729331033424\\
-0.135044727494537	-0.0762205488858227\\
-0.135001226280639	-0.0761816761279487\\
-0.134968916076855	-0.0761528070676185\\
-0.134944904627433	-0.0761313547761771\\
-0.13492705309505	-0.0761154068951782\\
-0.134913777157607	-0.0761035472397231\\
-0.134903901786412	-0.0760947256872687\\
-0.134896554703729	-0.0760881627965108\\
-0.134891087932271	-0.0760832796164374\\
-0.134887019871364	-0.0760796458834948\\
-0.134883992441188	-0.0760769417069688\\
-0.134881739326665	-0.0760749291846393\\
-0.134875180997266	-0.0760690712449868\\
-0.134875180971928	-0.0760690712223533\\
};
\addplot [color=black!50!green,solid,forget plot]
  table[row sep=crcr]{%
-3	-1.73205080756888\\
-3.00000000000013	-1.73205080756892\\
-3.00000000000013	-1.73205080756892\\
-3.00000000000013	-1.73205080756892\\
-3.00018574345039	-1.73210999549279\\
-3.00024952467788	-1.73213032886482\\
-3.00033520054779	-1.73215764963554\\
-3.00045028154132	-1.73219436064315\\
-3.00060485006341	-1.73224369235797\\
-3.00081243765992	-1.73230998901812\\
-3.00109119859842	-1.73239909453141\\
-3.00146547765892	-1.73251887357957\\
-3.00186145710671	-1.73264577354041\\
-3.00186331722382	-1.7326463700815\\
-3.00186517733453	-1.73264696662454\\
-3.0019679004465	-1.73267991643415\\
-3.00264215298486	-1.73289649422645\\
-3.0035466645889	-1.73318785704113\\
-3.00475946367187	-1.73358000349137\\
-3.00638453721239	-1.73410810210982\\
-3.00856008284294	-1.73481981859991\\
-3.01146908001185	-1.73577990748758\\
-3.01535258878921	-1.73707657195051\\
-3.02052605019796	-1.73883028872727\\
-3.02739851038195	-1.74120602903249\\
-3.03649397681834	-1.74443003413943\\
-3.04847286372701	-1.74881238821826\\
-3.06414957413294	-1.75477626804255\\
-3.08449983889996	-1.76289340972457\\
-3.09710955106993	-1.76812942496817\\
-3.10942177238854	-1.77338999769741\\
-3.10952063090217	-1.7734328193247\\
-3.10961947067122	-1.77347564204358\\
-3.11064927102963	-1.77392235455516\\
-3.11467281437444	-1.77567725563282\\
-3.11477591089001	-1.77572242111809\\
-3.11487898701366	-1.77576758758909\\
-3.12955787385621	-1.78230000528241\\
-3.14383451446854	-1.78884114691129\\
-3.18533301076605	-1.80886061241473\\
-3.2363683952824	-1.83540203353982\\
-3.29801300845091	-1.87002938698027\\
-3.37111799671745	-1.91434308330117\\
-3.45629541263606	-1.96986177608149\\
-3.55395606203566	-2.03792731170025\\
-3.66438479682309	-2.11965788135532\\
-3.7878256102048	-2.21595334267563\\
-3.92455452562459	-2.32753893105085\\
-4.07493059403701	-2.45502726021708\\
-4.23942558458848	-2.59898182420575\\
-4.41863789267797	-2.75997221768887\\
-4.61329688069298	-2.93861752491266\\
-4.82426255606251	-3.13561821106619\\
-5.05252378567797	-3.35177861306715\\
-5.29919685578166	-3.58802254739917\\
-5.56552526159123	-3.84540431518027\\
-5.85288107189405	-4.12511691924863\\
-6.16276793440798	-4.42849883255713\\
-6.49682566295659	-4.75704025841028\\
-6.856836306824	-5.11238952067469\\
-7.2447316036385	-5.49636000692364\\
-7.66260173629013	-5.91093794148167\\
-8.11270533960169	-6.35829117037366\\
-8.59748072801077	-6.84077908118869\\
-9.11955833902618	-7.36096374643808\\
-167.50374464114	-165.740537665457\\
inf	0\\
};
\addplot [color=white!40!black,dotted,forget plot]
  table[row sep=crcr]{%
0	0\\
-0	14\\
nan	nan\\
0	0\\
-2.24	13.8196382007634\\
nan	nan\\
0	0\\
-4.76	13.1659560989698\\
nan	nan\\
0	0\\
-7	12.1243556529821\\
nan	nan\\
0	0\\
-8.96	10.7572487188872\\
nan	nan\\
0	0\\
-10.64	9.09892301319228\\
nan	nan\\
0	0\\
-12.04	7.14411646041692\\
nan	nan\\
0	0\\
-13.16	4.77644219058495\\
nan	nan\\
0	0\\
-13.79	2.41576074974324\\
nan	nan\\
0	0\\
-14	0\\
nan	nan\\
0	-0\\
-0	-14\\
nan	nan\\
0	-0\\
-2.24	-13.8196382007634\\
nan	nan\\
0	-0\\
-4.76	-13.1659560989698\\
nan	nan\\
0	-0\\
-7	-12.1243556529821\\
nan	nan\\
0	-0\\
-8.96	-10.7572487188872\\
nan	nan\\
0	-0\\
-10.64	-9.09892301319228\\
nan	nan\\
0	-0\\
-12.04	-7.14411646041692\\
nan	nan\\
0	-0\\
-13.16	-4.77644219058495\\
nan	nan\\
0	-0\\
-13.79	-2.41576074974324\\
nan	nan\\
0	-0\\
-14	-0\\
nan	nan\\
};
\addplot [color=white!40!black,dotted,forget plot]
  table[row sep=crcr]{%
-0	0\\
-0	0\\
-0	0\\
-0	0\\
-0	0\\
-0	0\\
-0	0\\
-0	0\\
-0	0\\
-0	0\\
-0	0\\
-0	0\\
-0	0\\
-0	0\\
-0	0\\
-0	0\\
-0	0\\
-0	0\\
-0	0\\
-0	0\\
-0	0\\
-0	0\\
-0	0\\
-0	0\\
-0	0\\
-0	0\\
-0	0\\
-0	0\\
-0	0\\
-0	0\\
-0	0\\
-0	0\\
-0	0\\
-0	0\\
-0	0\\
-0	0\\
-0	0\\
-0	0\\
-0	0\\
-0	0\\
-0	0\\
nan	nan\\
-0	2\\
-0.0785196315181373	1.99845807248145\\
-0.15691819145569	1.99383466746626\\
-0.235074794915675	1.98613691390985\\
-0.312868930080462	1.97537668119028\\
-0.390180644032257	1.96157056080646\\
-0.466890727711811	1.94473984079535\\
-0.542880899730149	1.92491047290729\\
-0.618033988749895	1.90211303259031\\
-0.692234114154986	1.87638267184497\\
-0.76536686473018	1.84775906502257\\
-0.837319475074856	1.81628634765016\\
-0.907980999479094	1.78201304837674\\
-0.97724248299391	1.74499201414559\\
-1.0449971294319	1.70528032870818\\
-1.1111404660392	1.66293922460509\\
-1.17557050458495	1.61803398874989\\
-1.23818789861967	1.57063386176149\\
-1.29889609666037	1.52081193120006\\
-1.35760149106588	1.46864501887137\\
-1.4142135623731	1.41421356237309\\
-1.46864501887137	1.35760149106588\\
-1.52081193120006	1.29889609666037\\
-1.57063386176149	1.23818789861967\\
-1.61803398874989	1.17557050458495\\
-1.66293922460509	1.1111404660392\\
-1.70528032870818	1.0449971294319\\
-1.74499201414559	0.97724248299391\\
-1.78201304837674	0.907980999479093\\
-1.81628634765016	0.837319475074855\\
-1.84775906502257	0.76536686473018\\
-1.87638267184497	0.692234114154986\\
-1.90211303259031	0.618033988749894\\
-1.92491047290729	0.542880899730149\\
-1.94473984079535	0.466890727711811\\
-1.96157056080646	0.390180644032257\\
-1.97537668119028	0.312868930080462\\
-1.98613691390985	0.235074794915674\\
-1.99383466746626	0.156918191455692\\
-1.99845807248145	0.0785196315181388\\
-2	0\\
nan	nan\\
-0	4\\
-0.157039263036275	3.99691614496289\\
-0.31383638291138	3.98766933493251\\
-0.470149589831351	3.97227382781971\\
-0.625737860160924	3.95075336238055\\
-0.780361288064513	3.92314112161292\\
-0.933781455423622	3.88947968159071\\
-1.0857617994603	3.84982094581459\\
-1.23606797749979	3.80422606518061\\
-1.38446822830997	3.75276534368994\\
-1.53073372946036	3.69551813004515\\
-1.67463895014971	3.63257269530033\\
-1.81596199895819	3.56402609675347\\
-1.95448496598782	3.48998402829119\\
-2.0899942588638	3.41056065741637\\
-2.22228093207841	3.32587844921018\\
-2.35114100916989	3.23606797749979\\
-2.47637579723934	3.14126772352298\\
-2.59779219332073	3.04162386240012\\
-2.71520298213177	2.93729003774274\\
-2.82842712474619	2.82842712474619\\
-2.93729003774274	2.71520298213177\\
-3.04162386240012	2.59779219332073\\
-3.14126772352298	2.47637579723934\\
-3.23606797749979	2.35114100916989\\
-3.32587844921018	2.22228093207841\\
-3.41056065741637	2.0899942588638\\
-3.48998402829119	1.95448496598782\\
-3.56402609675347	1.81596199895819\\
-3.63257269530033	1.67463895014971\\
-3.69551813004515	1.53073372946036\\
-3.75276534368994	1.38446822830997\\
-3.80422606518061	1.23606797749979\\
-3.84982094581459	1.0857617994603\\
-3.88947968159071	0.933781455423622\\
-3.92314112161292	0.780361288064513\\
-3.95075336238055	0.625737860160923\\
-3.97227382781971	0.470149589831347\\
-3.98766933493251	0.313836382911385\\
-3.99691614496289	0.157039263036278\\
-4	0\\
nan	nan\\
-0	6\\
-0.235558894554412	5.99537421744434\\
-0.47075457436707	5.98150400239877\\
-0.705224384747026	5.95841074172956\\
-0.938606790241385	5.92613004357083\\
-1.17054193209677	5.88471168241938\\
-1.40067218313543	5.83421952238606\\
-1.62864269919045	5.77473141872188\\
-1.85410196624969	5.70633909777092\\
-2.07670234246496	5.6291480155349\\
-2.29610059419054	5.54327719506772\\
-2.51195842522457	5.44885904295049\\
-2.72394299843728	5.34603914513021\\
-2.93172744898173	5.23497604243678\\
-3.13499138829569	5.11584098612455\\
-3.33342139811761	4.98881767381527\\
-3.52671151375484	4.85410196624968\\
-3.71456369585901	4.71190158528447\\
-3.8966882899811	4.56243579360019\\
-4.07280447319765	4.40593505661411\\
-4.24264068711929	4.24264068711928\\
-4.40593505661411	4.07280447319765\\
-4.56243579360019	3.8966882899811\\
-4.71190158528447	3.714563695859\\
-4.85410196624968	3.52671151375484\\
-4.98881767381527	3.33342139811761\\
-5.11584098612455	3.13499138829569\\
-5.23497604243678	2.93172744898173\\
-5.34603914513021	2.72394299843728\\
-5.44885904295049	2.51195842522457\\
-5.54327719506772	2.29610059419054\\
-5.6291480155349	2.07670234246496\\
-5.70633909777092	1.85410196624968\\
-5.77473141872188	1.62864269919045\\
-5.83421952238606	1.40067218313543\\
-5.88471168241938	1.17054193209677\\
-5.92613004357083	0.938606790241385\\
-5.95841074172956	0.705224384747021\\
-5.98150400239877	0.470754574367077\\
-5.99537421744434	0.235558894554416\\
-6	0\\
nan	nan\\
-0	8\\
-0.314078526072549	7.99383228992578\\
-0.62767276582276	7.97533866986502\\
-0.940299179662701	7.94454765563941\\
-1.25147572032185	7.9015067247611\\
-1.56072257612903	7.84628224322584\\
-1.86756291084724	7.77895936318141\\
-2.17152359892059	7.69964189162918\\
-2.47213595499958	7.60845213036123\\
-2.76893645661994	7.50553068737987\\
-3.06146745892072	7.39103626009029\\
-3.34927790029942	7.26514539060065\\
-3.63192399791637	7.12805219350694\\
-3.90896993197564	6.97996805658238\\
-4.17998851772759	6.82112131483274\\
-4.44456186415682	6.65175689842036\\
-4.70228201833979	6.47213595499958\\
-4.95275159447867	6.28253544704596\\
-5.19558438664147	6.08324772480025\\
-5.43040596426353	5.87458007548549\\
-5.65685424949238	5.65685424949238\\
-5.87458007548549	5.43040596426353\\
-6.08324772480025	5.19558438664147\\
-6.28253544704596	4.95275159447867\\
-6.47213595499958	4.70228201833979\\
-6.65175689842036	4.44456186415682\\
-6.82112131483274	4.17998851772759\\
-6.97996805658238	3.90896993197564\\
-7.12805219350694	3.63192399791637\\
-7.26514539060065	3.34927790029942\\
-7.39103626009029	3.06146745892072\\
-7.50553068737987	2.76893645661994\\
-7.60845213036123	2.47213595499958\\
-7.69964189162918	2.17152359892059\\
-7.77895936318141	1.86756291084724\\
-7.84628224322584	1.56072257612903\\
-7.9015067247611	1.25147572032185\\
-7.94454765563941	0.940299179662694\\
-7.97533866986502	0.62767276582277\\
-7.99383228992578	0.314078526072555\\
-8	0\\
nan	nan\\
-0	10\\
-0.392598157590687	9.99229036240723\\
-0.78459095727845	9.96917333733128\\
-1.17537397457838	9.93068456954926\\
-1.56434465040231	9.87688340595138\\
-1.95090322016128	9.8078528040323\\
-2.33445363855905	9.72369920397677\\
-2.71440449865074	9.62455236453647\\
-3.09016994374947	9.51056516295153\\
-3.46117057077493	9.38191335922484\\
-3.8268343236509	9.23879532511287\\
-4.18659737537428	9.08143173825081\\
-4.53990499739547	8.91006524188368\\
-4.88621241496955	8.72496007072797\\
-5.22498564715949	8.52640164354092\\
-5.55570233019602	8.31469612302545\\
-5.87785252292473	8.09016994374947\\
-6.19093949309834	7.85316930880745\\
-6.49448048330184	7.60405965600031\\
-6.78800745532942	7.34322509435686\\
-7.07106781186548	7.07106781186547\\
-7.34322509435686	6.78800745532942\\
-7.60405965600031	6.49448048330184\\
-7.85316930880745	6.19093949309834\\
-8.09016994374947	5.87785252292473\\
-8.31469612302545	5.55570233019602\\
-8.52640164354092	5.22498564715949\\
-8.72496007072797	4.88621241496955\\
-8.91006524188368	4.53990499739547\\
-9.08143173825082	4.18659737537428\\
-9.23879532511287	3.8268343236509\\
-9.38191335922484	3.46117057077493\\
-9.51056516295154	3.09016994374947\\
-9.62455236453647	2.71440449865074\\
-9.72369920397677	2.33445363855905\\
-9.8078528040323	1.95090322016128\\
-9.87688340595138	1.56434465040231\\
-9.93068456954926	1.17537397457837\\
-9.96917333733128	0.784590957278462\\
-9.99229036240723	0.392598157590694\\
-10	0\\
nan	nan\\
-0	12\\
-0.471117789108824	11.9907484348887\\
-0.94150914873414	11.9630080047975\\
-1.41044876949405	11.9168214834591\\
-1.87721358048277	11.8522600871417\\
-2.34108386419354	11.7694233648388\\
-2.80134436627087	11.6684390447721\\
-3.25728539838089	11.5494628374438\\
-3.70820393249937	11.4126781955418\\
-4.15340468492992	11.2582960310698\\
-4.59220118838108	11.0865543901354\\
-5.02391685044913	10.897718085901\\
-5.44788599687456	10.6920782902604\\
-5.86345489796346	10.4699520848736\\
-6.26998277659139	10.2316819722491\\
-6.66684279623522	9.97763534763054\\
-7.05342302750968	9.70820393249937\\
-7.42912739171801	9.42380317056894\\
-7.7933765799622	9.12487158720037\\
-8.1456089463953	8.81187011322823\\
-8.48528137423857	8.48528137423857\\
-8.81187011322823	8.1456089463953\\
-9.12487158720037	7.7933765799622\\
-9.42380317056894	7.42912739171801\\
-9.70820393249937	7.05342302750968\\
-9.97763534763054	6.66684279623522\\
-10.2316819722491	6.26998277659139\\
-10.4699520848736	5.86345489796346\\
-10.6920782902604	5.44788599687456\\
-10.897718085901	5.02391685044913\\
-11.0865543901354	4.59220118838108\\
-11.2582960310698	4.15340468492991\\
-11.4126781955418	3.70820393249937\\
-11.5494628374438	3.25728539838089\\
-11.6684390447721	2.80134436627087\\
-11.7694233648388	2.34108386419354\\
-11.8522600871417	1.87721358048277\\
-11.9168214834591	1.41044876949404\\
-11.9630080047975	0.941509148734155\\
-11.9907484348887	0.471117789108833\\
-12	0\\
nan	nan\\
-0	14\\
-0.549637420626961	13.9892065073701\\
-1.09842734018983	13.9568426722638\\
-1.64552356440973	13.902958397369\\
-2.19008251056323	13.8276367683319\\
-2.7312645082258	13.7309939256452\\
-3.26823509398268	13.6131788855675\\
-3.80016629811104	13.4743733103511\\
-4.32623792124927	13.3147912281321\\
-4.8456387990849	13.1346787029148\\
-5.35756805311126	12.934313455158\\
-5.86123632552399	12.7140044335511\\
-6.35586699635366	12.4740913386371\\
-6.84069738095737	12.2149440990192\\
-7.31497990602328	11.9369623009573\\
-7.77798326227443	11.6405745722356\\
-8.22899353209462	11.3262379212493\\
-8.66731529033768	10.9944370323304\\
-9.09227267662257	10.6456835184004\\
-9.50321043746118	10.2805151320996\\
-9.89949493661167	9.89949493661166\\
-10.2805151320996	9.50321043746118\\
-10.6456835184004	9.09227267662257\\
-10.9944370323304	8.66731529033768\\
-11.3262379212493	8.22899353209462\\
-11.6405745722356	7.77798326227443\\
-11.9369623009573	7.31497990602328\\
-12.2149440990192	6.84069738095737\\
-12.4740913386371	6.35586699635365\\
-12.7140044335511	5.86123632552399\\
-12.934313455158	5.35756805311126\\
-13.1346787029148	4.8456387990849\\
-13.3147912281322	4.32623792124926\\
-13.4743733103511	3.80016629811104\\
-13.6131788855675	3.26823509398268\\
-13.7309939256452	2.7312645082258\\
-13.8276367683319	2.19008251056323\\
-13.902958397369	1.64552356440972\\
-13.9568426722638	1.09842734018985\\
-13.9892065073701	0.549637420626972\\
-14	0\\
nan	nan\\
-0	-0\\
-0	-0\\
-0	-0\\
-0	-0\\
-0	-0\\
-0	-0\\
-0	-0\\
-0	-0\\
-0	-0\\
-0	-0\\
-0	-0\\
-0	-0\\
-0	-0\\
-0	-0\\
-0	-0\\
-0	-0\\
-0	-0\\
-0	-0\\
-0	-0\\
-0	-0\\
-0	-0\\
-0	-0\\
-0	-0\\
-0	-0\\
-0	-0\\
-0	-0\\
-0	-0\\
-0	-0\\
-0	-0\\
-0	-0\\
-0	-0\\
-0	-0\\
-0	-0\\
-0	-0\\
-0	-0\\
-0	-0\\
-0	-0\\
-0	-0\\
-0	-0\\
-0	-0\\
-0	-0\\
nan	nan\\
-0	-2\\
-0.0785196315181373	-1.99845807248145\\
-0.15691819145569	-1.99383466746626\\
-0.235074794915675	-1.98613691390985\\
-0.312868930080462	-1.97537668119028\\
-0.390180644032257	-1.96157056080646\\
-0.466890727711811	-1.94473984079535\\
-0.542880899730149	-1.92491047290729\\
-0.618033988749895	-1.90211303259031\\
-0.692234114154986	-1.87638267184497\\
-0.76536686473018	-1.84775906502257\\
-0.837319475074856	-1.81628634765016\\
-0.907980999479094	-1.78201304837674\\
-0.97724248299391	-1.74499201414559\\
-1.0449971294319	-1.70528032870818\\
-1.1111404660392	-1.66293922460509\\
-1.17557050458495	-1.61803398874989\\
-1.23818789861967	-1.57063386176149\\
-1.29889609666037	-1.52081193120006\\
-1.35760149106588	-1.46864501887137\\
-1.4142135623731	-1.41421356237309\\
-1.46864501887137	-1.35760149106588\\
-1.52081193120006	-1.29889609666037\\
-1.57063386176149	-1.23818789861967\\
-1.61803398874989	-1.17557050458495\\
-1.66293922460509	-1.1111404660392\\
-1.70528032870818	-1.0449971294319\\
-1.74499201414559	-0.97724248299391\\
-1.78201304837674	-0.907980999479093\\
-1.81628634765016	-0.837319475074855\\
-1.84775906502257	-0.76536686473018\\
-1.87638267184497	-0.692234114154986\\
-1.90211303259031	-0.618033988749894\\
-1.92491047290729	-0.542880899730149\\
-1.94473984079535	-0.466890727711811\\
-1.96157056080646	-0.390180644032257\\
-1.97537668119028	-0.312868930080462\\
-1.98613691390985	-0.235074794915674\\
-1.99383466746626	-0.156918191455692\\
-1.99845807248145	-0.0785196315181388\\
-2	-0\\
nan	nan\\
-0	-4\\
-0.157039263036275	-3.99691614496289\\
-0.31383638291138	-3.98766933493251\\
-0.470149589831351	-3.97227382781971\\
-0.625737860160924	-3.95075336238055\\
-0.780361288064513	-3.92314112161292\\
-0.933781455423622	-3.88947968159071\\
-1.0857617994603	-3.84982094581459\\
-1.23606797749979	-3.80422606518061\\
-1.38446822830997	-3.75276534368994\\
-1.53073372946036	-3.69551813004515\\
-1.67463895014971	-3.63257269530033\\
-1.81596199895819	-3.56402609675347\\
-1.95448496598782	-3.48998402829119\\
-2.0899942588638	-3.41056065741637\\
-2.22228093207841	-3.32587844921018\\
-2.35114100916989	-3.23606797749979\\
-2.47637579723934	-3.14126772352298\\
-2.59779219332073	-3.04162386240012\\
-2.71520298213177	-2.93729003774274\\
-2.82842712474619	-2.82842712474619\\
-2.93729003774274	-2.71520298213177\\
-3.04162386240012	-2.59779219332073\\
-3.14126772352298	-2.47637579723934\\
-3.23606797749979	-2.35114100916989\\
-3.32587844921018	-2.22228093207841\\
-3.41056065741637	-2.0899942588638\\
-3.48998402829119	-1.95448496598782\\
-3.56402609675347	-1.81596199895819\\
-3.63257269530033	-1.67463895014971\\
-3.69551813004515	-1.53073372946036\\
-3.75276534368994	-1.38446822830997\\
-3.80422606518061	-1.23606797749979\\
-3.84982094581459	-1.0857617994603\\
-3.88947968159071	-0.933781455423622\\
-3.92314112161292	-0.780361288064513\\
-3.95075336238055	-0.625737860160923\\
-3.97227382781971	-0.470149589831347\\
-3.98766933493251	-0.313836382911385\\
-3.99691614496289	-0.157039263036278\\
-4	-0\\
nan	nan\\
-0	-6\\
-0.235558894554412	-5.99537421744434\\
-0.47075457436707	-5.98150400239877\\
-0.705224384747026	-5.95841074172956\\
-0.938606790241385	-5.92613004357083\\
-1.17054193209677	-5.88471168241938\\
-1.40067218313543	-5.83421952238606\\
-1.62864269919045	-5.77473141872188\\
-1.85410196624969	-5.70633909777092\\
-2.07670234246496	-5.6291480155349\\
-2.29610059419054	-5.54327719506772\\
-2.51195842522457	-5.44885904295049\\
-2.72394299843728	-5.34603914513021\\
-2.93172744898173	-5.23497604243678\\
-3.13499138829569	-5.11584098612455\\
-3.33342139811761	-4.98881767381527\\
-3.52671151375484	-4.85410196624968\\
-3.71456369585901	-4.71190158528447\\
-3.8966882899811	-4.56243579360019\\
-4.07280447319765	-4.40593505661411\\
-4.24264068711929	-4.24264068711928\\
-4.40593505661411	-4.07280447319765\\
-4.56243579360019	-3.8966882899811\\
-4.71190158528447	-3.714563695859\\
-4.85410196624968	-3.52671151375484\\
-4.98881767381527	-3.33342139811761\\
-5.11584098612455	-3.13499138829569\\
-5.23497604243678	-2.93172744898173\\
-5.34603914513021	-2.72394299843728\\
-5.44885904295049	-2.51195842522457\\
-5.54327719506772	-2.29610059419054\\
-5.6291480155349	-2.07670234246496\\
-5.70633909777092	-1.85410196624968\\
-5.77473141872188	-1.62864269919045\\
-5.83421952238606	-1.40067218313543\\
-5.88471168241938	-1.17054193209677\\
-5.92613004357083	-0.938606790241385\\
-5.95841074172956	-0.705224384747021\\
-5.98150400239877	-0.470754574367077\\
-5.99537421744434	-0.235558894554416\\
-6	-0\\
nan	nan\\
-0	-8\\
-0.314078526072549	-7.99383228992578\\
-0.62767276582276	-7.97533866986502\\
-0.940299179662701	-7.94454765563941\\
-1.25147572032185	-7.9015067247611\\
-1.56072257612903	-7.84628224322584\\
-1.86756291084724	-7.77895936318141\\
-2.17152359892059	-7.69964189162918\\
-2.47213595499958	-7.60845213036123\\
-2.76893645661994	-7.50553068737987\\
-3.06146745892072	-7.39103626009029\\
-3.34927790029942	-7.26514539060065\\
-3.63192399791637	-7.12805219350694\\
-3.90896993197564	-6.97996805658238\\
-4.17998851772759	-6.82112131483274\\
-4.44456186415682	-6.65175689842036\\
-4.70228201833979	-6.47213595499958\\
-4.95275159447867	-6.28253544704596\\
-5.19558438664147	-6.08324772480025\\
-5.43040596426353	-5.87458007548549\\
-5.65685424949238	-5.65685424949238\\
-5.87458007548549	-5.43040596426353\\
-6.08324772480025	-5.19558438664147\\
-6.28253544704596	-4.95275159447867\\
-6.47213595499958	-4.70228201833979\\
-6.65175689842036	-4.44456186415682\\
-6.82112131483274	-4.17998851772759\\
-6.97996805658238	-3.90896993197564\\
-7.12805219350694	-3.63192399791637\\
-7.26514539060065	-3.34927790029942\\
-7.39103626009029	-3.06146745892072\\
-7.50553068737987	-2.76893645661994\\
-7.60845213036123	-2.47213595499958\\
-7.69964189162918	-2.17152359892059\\
-7.77895936318141	-1.86756291084724\\
-7.84628224322584	-1.56072257612903\\
-7.9015067247611	-1.25147572032185\\
-7.94454765563941	-0.940299179662694\\
-7.97533866986502	-0.62767276582277\\
-7.99383228992578	-0.314078526072555\\
-8	-0\\
nan	nan\\
-0	-10\\
-0.392598157590687	-9.99229036240723\\
-0.78459095727845	-9.96917333733128\\
-1.17537397457838	-9.93068456954926\\
-1.56434465040231	-9.87688340595138\\
-1.95090322016128	-9.8078528040323\\
-2.33445363855905	-9.72369920397677\\
-2.71440449865074	-9.62455236453647\\
-3.09016994374947	-9.51056516295153\\
-3.46117057077493	-9.38191335922484\\
-3.8268343236509	-9.23879532511287\\
-4.18659737537428	-9.08143173825081\\
-4.53990499739547	-8.91006524188368\\
-4.88621241496955	-8.72496007072797\\
-5.22498564715949	-8.52640164354092\\
-5.55570233019602	-8.31469612302545\\
-5.87785252292473	-8.09016994374947\\
-6.19093949309834	-7.85316930880745\\
-6.49448048330184	-7.60405965600031\\
-6.78800745532942	-7.34322509435686\\
-7.07106781186548	-7.07106781186547\\
-7.34322509435686	-6.78800745532942\\
-7.60405965600031	-6.49448048330184\\
-7.85316930880745	-6.19093949309834\\
-8.09016994374947	-5.87785252292473\\
-8.31469612302545	-5.55570233019602\\
-8.52640164354092	-5.22498564715949\\
-8.72496007072797	-4.88621241496955\\
-8.91006524188368	-4.53990499739547\\
-9.08143173825082	-4.18659737537428\\
-9.23879532511287	-3.8268343236509\\
-9.38191335922484	-3.46117057077493\\
-9.51056516295154	-3.09016994374947\\
-9.62455236453647	-2.71440449865074\\
-9.72369920397677	-2.33445363855905\\
-9.8078528040323	-1.95090322016128\\
-9.87688340595138	-1.56434465040231\\
-9.93068456954926	-1.17537397457837\\
-9.96917333733128	-0.784590957278462\\
-9.99229036240723	-0.392598157590694\\
-10	-0\\
nan	nan\\
-0	-12\\
-0.471117789108824	-11.9907484348887\\
-0.94150914873414	-11.9630080047975\\
-1.41044876949405	-11.9168214834591\\
-1.87721358048277	-11.8522600871417\\
-2.34108386419354	-11.7694233648388\\
-2.80134436627087	-11.6684390447721\\
-3.25728539838089	-11.5494628374438\\
-3.70820393249937	-11.4126781955418\\
-4.15340468492992	-11.2582960310698\\
-4.59220118838108	-11.0865543901354\\
-5.02391685044913	-10.897718085901\\
-5.44788599687456	-10.6920782902604\\
-5.86345489796346	-10.4699520848736\\
-6.26998277659139	-10.2316819722491\\
-6.66684279623522	-9.97763534763054\\
-7.05342302750968	-9.70820393249937\\
-7.42912739171801	-9.42380317056894\\
-7.7933765799622	-9.12487158720037\\
-8.1456089463953	-8.81187011322823\\
-8.48528137423857	-8.48528137423857\\
-8.81187011322823	-8.1456089463953\\
-9.12487158720037	-7.7933765799622\\
-9.42380317056894	-7.42912739171801\\
-9.70820393249937	-7.05342302750968\\
-9.97763534763054	-6.66684279623522\\
-10.2316819722491	-6.26998277659139\\
-10.4699520848736	-5.86345489796346\\
-10.6920782902604	-5.44788599687456\\
-10.897718085901	-5.02391685044913\\
-11.0865543901354	-4.59220118838108\\
-11.2582960310698	-4.15340468492991\\
-11.4126781955418	-3.70820393249937\\
-11.5494628374438	-3.25728539838089\\
-11.6684390447721	-2.80134436627087\\
-11.7694233648388	-2.34108386419354\\
-11.8522600871417	-1.87721358048277\\
-11.9168214834591	-1.41044876949404\\
-11.9630080047975	-0.941509148734155\\
-11.9907484348887	-0.471117789108833\\
-12	-0\\
nan	nan\\
-0	-14\\
-0.549637420626961	-13.9892065073701\\
-1.09842734018983	-13.9568426722638\\
-1.64552356440973	-13.902958397369\\
-2.19008251056323	-13.8276367683319\\
-2.7312645082258	-13.7309939256452\\
-3.26823509398268	-13.6131788855675\\
-3.80016629811104	-13.4743733103511\\
-4.32623792124927	-13.3147912281321\\
-4.8456387990849	-13.1346787029148\\
-5.35756805311126	-12.934313455158\\
-5.86123632552399	-12.7140044335511\\
-6.35586699635366	-12.4740913386371\\
-6.84069738095737	-12.2149440990192\\
-7.31497990602328	-11.9369623009573\\
-7.77798326227443	-11.6405745722356\\
-8.22899353209462	-11.3262379212493\\
-8.66731529033768	-10.9944370323304\\
-9.09227267662257	-10.6456835184004\\
-9.50321043746118	-10.2805151320996\\
-9.89949493661167	-9.89949493661166\\
-10.2805151320996	-9.50321043746118\\
-10.6456835184004	-9.09227267662257\\
-10.9944370323304	-8.66731529033768\\
-11.3262379212493	-8.22899353209462\\
-11.6405745722356	-7.77798326227443\\
-11.9369623009573	-7.31497990602328\\
-12.2149440990192	-6.84069738095737\\
-12.4740913386371	-6.35586699635365\\
-12.7140044335511	-5.86123632552399\\
-12.934313455158	-5.35756805311126\\
-13.1346787029148	-4.8456387990849\\
-13.3147912281322	-4.32623792124926\\
-13.4743733103511	-3.80016629811104\\
-13.6131788855675	-3.26823509398268\\
-13.7309939256452	-2.7312645082258\\
-13.8276367683319	-2.19008251056323\\
-13.902958397369	-1.64552356440972\\
-13.9568426722638	-1.09842734018985\\
-13.9892065073701	-0.549637420626972\\
-14	-0\\
nan	nan\\
};
\end{axis}
\end{tikzpicture}%}
  \caption{Root locus of the open-loop system with the use of a zero-order hold.}
  \label{fig:Q7.rloc_F_ZOH_G}
\end{figure}

Figures \ref{fig:Q7.1}-\ref{fig:Q7.8} illustrate the step response when between
the continuous controller and the plant a zero-order hold has been inserted, for
sampling time varying between $1$ and $8$ seconds. The values of the $\chi$,
$\zeta$ and $\omega_0$ parameters were chosen to be the ones giving the best
performance among the three sets,
hence $(\chi, \zeta, \omega_0) \equiv (0.5, 0.8, 0.2)$.

Here, up until $h = 5$ sec, as the sampling time increases, so do the rise
time, settling time and overshoot. However, increasing $h$ beyond $7$ sec
make the system critically stable.

\begin{figure}[H]\centering
	\centering
	\scalebox{1}{% This file was created by matlab2tikz.
%
%The latest updates can be retrieved from
%  http://www.mathworks.com/matlabcentral/fileexchange/22022-matlab2tikz-matlab2tikz
%where you can also make suggestions and rate matlab2tikz.
%
\definecolor{mycolor1}{rgb}{0.00000,0.44700,0.74100}%
%
\begin{tikzpicture}

\begin{axis}[%
width=4.133in,
height=3.26in,
at={(0.693in,0.44in)},
scale only axis,
xmin=0,
xmax=200,
xmajorgrids,
ymin=34,
ymax=54,
ymajorgrids,
axis background/.style={fill=white}
]
\addplot [color=mycolor1,solid,forget plot]
  table[row sep=crcr]{%
0	40\\
0.0670825919132015	39.9988385752381\\
0.134165183826403	39.9953626337708\\
0.201247775739604	39.9895846264077\\
0.268330367652806	39.9815169394076\\
0.335412959566007	39.9711718946171\\
0.402495551479209	39.9585617525181\\
0.477501758366411	39.9417949287895\\
0.552507965253614	39.9222285268844\\
0.627514172140816	39.8998794219546\\
0.705588818114703	39.8736783197704\\
0.78366346408859	39.8444992018444\\
0.861738110062477	39.8123608029845\\
0.907825406708318	39.7920081211627\\
0.953912703354159	39.770634588136\\
1	39.7482440111656\\
1.08628811745454	39.7039295742489\\
1.17257623490908	39.6567445984562\\
1.25886435236362	39.6067133696279\\
1.35835823925136	39.5455238123996\\
1.45785212613911	39.4806192776484\\
1.55734601302685	39.4120365944279\\
1.66075710056335	39.3368949100915\\
1.76416818809986	39.2578604524095\\
1.86757927563636	39.1749741449114\\
1.91171951709091	39.1384318452781\\
1.95585975854545	39.1011983588226\\
2	39.0632768467846\\
2.09120259149091	38.9835649390049\\
2.18240518298182	38.9025662449768\\
2.27360777447273	38.8203000247858\\
2.36481036596365	38.7367854528032\\
2.45601295745456	38.6520416144975\\
2.54721554894547	38.5660875103945\\
2.64534117737806	38.4722788735721\\
2.74346680581065	38.3771145989071\\
2.84159243424324	38.2806180184845\\
2.89439495616216	38.2281488819011\\
2.94719747808108	38.1753042869428\\
3	38.1220878281964\\
3.08537191920714	38.0360365409306\\
3.17074383841428	37.9505603167535\\
3.25611575762143	37.8656560307849\\
3.34148767682857	37.7813205668404\\
3.42685959603571	37.6975508176525\\
3.51223151524285	37.6143436847962\\
3.60532541949373	37.5242479986754\\
3.69841932374461	37.4348136431963\\
3.79151322799549	37.3460366374596\\
3.86100881866366	37.280189811428\\
3.93050440933183	37.2147054544706\\
4	37.149581924632\\
4.11108402780871	37.047231504434\\
4.22216805561743	36.9477608925404\\
4.33325208342614	36.8511175405079\\
4.42794224446606	36.7709302983874\\
4.52263240550599	36.6927284489292\\
4.61732256654591	36.6164810331743\\
4.72233976063703	36.5341668196466\\
4.82735695472815	36.4541781690356\\
4.93237414881926	36.3764748288743\\
4.95491609921284	36.3600898964094\\
4.97745804960642	36.3438080451803\\
5	36.3276288890659\\
5.11270975196789	36.248845746864\\
5.22541950393579	36.1737224664249\\
5.33812925590368	36.1021805242655\\
5.446603627054	36.0366373464052\\
5.55507799820431	35.9742732956905\\
5.66355236935463	35.9150227911334\\
5.77570157956975	35.8569703169124\\
5.88785078978488	35.8021078299113\\
6	35.7503673378685\\
6.13144100486509	35.6939069047617\\
6.26288200973019	35.6421181208389\\
6.39432301459528	35.5948797842423\\
6.52375464133709	35.5526954664153\\
6.65318626807889	35.5146992908606\\
6.78261789482069	35.4807842395802\\
6.85507859654712	35.4635402785782\\
6.92753929827356	35.4475247359181\\
7	35.4327199218714\\
7.16457588784749	35.4034882407397\\
7.32915177569498	35.3801891213667\\
7.49372766354247	35.3626300207972\\
7.66248510902831	35.3503900972095\\
7.83124255451416	35.3437946782391\\
8	35.3426550648257\\
8.20742294650158	35.3480897038036\\
8.41484589300315	35.3604469792504\\
8.62226883950473	35.3794449630615\\
8.74817922633648	35.3941006581158\\
8.87408961316824	35.4110441327959\\
9	35.4302177737987\\
9.16091180611379	35.4575919587937\\
9.32182361222757	35.4878424431611\\
9.48273541834136	35.5208767960452\\
9.65515694556091	35.5592615944973\\
9.82757847278045	35.6006320795458\\
10	35.6448838775606\\
10.1795818795046	35.6935836137047\\
10.3591637590093	35.744516517821\\
10.5387456385139	35.7975986030859\\
10.6924970923426	35.8446915262933\\
10.8462485461713	35.8932502767595\\
11	35.9432263360222\\
11.1806205947633	36.0033817551058\\
11.3612411895266	36.0647650507104\\
11.5418617842898	36.1273236756191\\
11.6945745228599	36.1810954686871\\
11.8472872614299	36.2356409503858\\
12	36.290930854059\\
12.1846849401062	36.3584908268905\\
12.3693698802124	36.4265415673965\\
12.5540548203186	36.4950521699232\\
12.7027032135457	36.5505083185195\\
12.8513516067729	36.6062277563371\\
13	36.6621956641521\\
13.1921248483663	36.7346506470072\\
13.3842496967327	36.8070223307294\\
13.576374545099	36.8792960761916\\
13.717583030066	36.9323453776143\\
13.858791515033	36.9853287135197\\
14	37.0382408034469\\
14.201821249267	37.1135293619717\\
14.4036424985341	37.1882528434436\\
14.6054637478011	37.2624106639627\\
14.7369758318674	37.3104292710988\\
14.8684879159337	37.3582074356561\\
15	37.4057451114208\\
15.2129644047453	37.4820339237975\\
15.4259288094907	37.5573378463602\\
15.638893214236	37.6316695251297\\
15.759262142824	37.6732569640421\\
15.879631071412	37.71454002644\\
16	37.7555209229447\\
16.2250935082959	37.8311895915911\\
16.4501870165918	37.9055042235078\\
16.6752805248877	37.9784903958166\\
16.7835203499251	38.0131213647583\\
16.8917601749626	38.0474536907559\\
17	38.0814900876962\\
17.2380615901245	38.1551745175648\\
17.4761231802489	38.2271821425647\\
17.7141847703734	38.2975513399227\\
17.8094565135823	38.3252623192407\\
17.9047282567911	38.3527192601107\\
18	38.3799245002212\\
18.252026091304	38.4505647504476\\
18.5040521826079	38.5192478329826\\
18.7560782739119	38.5860249089952\\
18.8373855159412	38.6071697799916\\
18.9186927579706	38.6281231130622\\
19	38.6488865380231\\
19.2674437758336	38.7157501231683\\
19.5348875516672	38.780411493582\\
19.8023313275008	38.8429348479447\\
19.8682208850005	38.8580175516427\\
19.9341104425003	38.8729751955376\\
20	38.8878086896527\\
20.2850850877728	38.9504832474696\\
20.5701701755455	39.0107342014931\\
20.8552552633183	39.0686396222405\\
20.9035035088789	39.0782130059521\\
20.9517517544394	39.0877217414587\\
21	39.0971661873196\\
21.2412412278028	39.1433868788165\\
21.4824824556057	39.1879562937511\\
21.7237236834085	39.2309208496473\\
21.815815788939	39.2469085097937\\
21.9079078944695	39.2626713699055\\
22	39.2782118758794\\
22.2664844577584	39.3219105138801\\
22.5329689155167	39.3637471466299\\
22.7994533732751	39.4037809220672\\
22.8663022488501	39.4135475306999\\
22.933151124425	39.4232052032538\\
23	39.4327548238569\\
23.3333333333333	39.4787541812901\\
23.6666666666667	39.5221249817807\\
24	39.5629735579325\\
24.292141752894	39.5967694234001\\
24.584283505788	39.6287504318333\\
24.876425258682	39.6589822255406\\
24.9176168391213	39.6631078355195\\
24.9588084195607	39.6672001120186\\
25	39.6712592291757\\
25.2059579021967	39.6910636486322\\
25.4119158043933	39.7100651125171\\
25.61787370659	39.7282845088497\\
25.7452491377266	39.7391699008002\\
25.8726245688633	39.7497687242706\\
26	39.7600856549505\\
26.3333333333333	39.7857957212109\\
26.6666666666667	39.8097103879323\\
27	39.8319060941284\\
27.3333333333333	39.8524763762326\\
27.6666666666667	39.8715106544135\\
28	39.8890753241677\\
28.3333333333333	39.905259029482\\
28.6666666666667	39.9201463089155\\
29	39.9337940420656\\
29.3333333333333	39.9462841139504\\
29.6666666666667	39.9576944630473\\
30	39.9680731733187\\
30.3333333333333	39.9774942814516\\
30.6666666666667	39.9860281375646\\
31	39.9937148639681\\
31.3333333333333	40.0006201804822\\
31.6666666666667	40.006806442233\\
32	40.0123066939301\\
32.3333333333333	40.0171783392735\\
32.6666666666667	40.0214757699242\\
33	40.0252258384477\\
33.3333333333333	40.0284779660328\\
33.6666666666667	40.0312789217942\\
34	40.0336502180421\\
34.3333333333333	40.0356338446133\\
34.6666666666667	40.0372694898393\\
35	40.0385741208683\\
35.3333333333333	40.0395829728116\\
35.6666666666667	40.0403293093783\\
36	40.0408262771402\\
36.3333333333333	40.0411030889515\\
36.6666666666667	40.0411872892772\\
37	40.0410888517796\\
37.3333333333333	40.0408317056022\\
37.6666666666667	40.040438384133\\
38	40.0399162577539\\
38.3333333333333	40.0392846831515\\
38.6666666666667	40.0385618625955\\
39	40.0377530572705\\
39.3333333333333	40.036873714753\\
39.6666666666667	40.0359383389296\\
40	40.0349505050949\\
40.3333333333333	40.0339223559536\\
40.6666666666667	40.0328652723279\\
41	40.0317815022028\\
41.3333333333333	40.0306804239406\\
41.6666666666667	40.0295708091743\\
42	40.0284538790552\\
42.3333333333333	40.0273367242065\\
42.6666666666667	40.0262259595633\\
43	40.025122028527\\
43.3333333333333	40.0240301491854\\
43.6666666666667	40.0229551734827\\
44	40.0218969712612\\
44.3333333333333	40.0208592459845\\
44.6666666666667	40.019845425584\\
45	40.0188549717435\\
45.3333333333333	40.0178903788275\\
45.6666666666667	40.0169539397829\\
46	40.0160448403542\\
46.3333333333333	40.015164624153\\
46.6666666666667	40.0143146932531\\
47	40.0134940615711\\
47.3333333333333	40.0127035383269\\
47.6666666666667	40.0119438389184\\
48	40.011213885969\\
48.3333333333333	40.0105139336854\\
48.6666666666667	40.0098441798736\\
49	40.0092035168248\\
49.3333333333333	40.0085917908372\\
49.6666666666667	40.0080088206044\\
50	40.0074535129747\\
50.3333333333333	40.0069254255304\\
50.6666666666667	40.0064241098964\\
51	40.0059485193302\\
51.3333333333333	40.0054980179707\\
51.6666666666667	40.005071979789\\
52	40.0046694258729\\
52.3333333333333	40.004289601884\\
52.6666666666667	40.003931774369\\
53	40.0035950454491\\
53.3333333333333	40.00327860021\\
53.6666666666667	40.0029816518995\\
54	40.0027033905208\\
54.3333333333333	40.0024429842983\\
54.6666666666667	40.0021996339342\\
55	40.0019726193591\\
55.3333333333333	40.0017611239929\\
55.6666666666667	40.0015643657744\\
56	40.0013817130773\\
56.3333333333333	40.0012123871192\\
56.6666666666667	40.0010556439708\\
57	40.0009109364704\\
57.3333333333333	40.000777538677\\
57.6666666666667	40.0006547585959\\
58	40.0005421278854\\
58.3333333333333	40.0004389825456\\
58.6666666666667	40.0003446907522\\
59	40.0002588563247\\
59.3333333333333	40.0001808817207\\
59.6666666666667	40.000110199239\\
60	40.0000464776897\\
60.3333333333333	39.9999891870527\\
60.6666666666667	39.9999378244855\\
61	39.9998921164861\\
61.3333333333333	39.9998515990998\\
61.6666666666667	39.9998158327365\\
62	39.9997845944251\\
62.3333333333333	39.9997574830505\\
62.6666666666667	39.9997341190349\\
63	39.9997143231213\\
63.3333333333333	39.9996977526417\\
63.6666666666667	39.9996840837201\\
64	39.9996731744669\\
64.3333333333333	39.9996647355727\\
64.6666666666667	39.9996584939251\\
65	39.9996543392025\\
65.3333333333333	39.9996520300326\\
65.6666666666667	39.9996513388387\\
66	39.9996521816529\\
66.3333333333333	39.999654359558\\
66.6666666666667	39.9996576852491\\
67	39.9996620964911\\
67.3333333333333	39.9996674314763\\
67.6666666666667	39.9996735380561\\
68	39.9996803716856\\
68.3333333333333	39.9996878025971\\
68.6666666666667	39.9996957089566\\
69	39.9997040604184\\
69.3333333333333	39.9997127545489\\
69.6666666666667	39.9997216953461\\
70	39.999730863682\\
70.3333333333333	39.9997401801704\\
70.6666666666667	39.999749570563\\
71	39.999759024435\\
71.3333333333333	39.999768481602\\
71.6666666666667	39.9997778859181\\
72	39.9997872335646\\
72.3333333333333	39.9997964801593\\
72.6666666666667	39.9998055844337\\
73	39.9998145474486\\
73.3333333333333	39.9998233376564\\
73.6666666666667	39.999831925855\\
74	39.9998403165814\\
74.3333333333333	39.9998484885638\\
74.6666666666667	39.9998564222457\\
75	39.9998641245162\\
75.3333333333333	39.999871582199\\
75.6666666666667	39.9998787833233\\
76	39.9998857362436\\
76.3333333333333	39.9998924340435\\
76.6666666666667	39.9998988706056\\
77	39.9999050550634\\
77.3333333333333	39.9999109852331\\
77.6666666666667	39.9999166594108\\
78	39.9999220869894\\
78.3333333333333	39.9999272692629\\
78.6666666666667	39.9999322077594\\
79	39.9999369117489\\
79.3333333333333	39.999941384985\\
79.6666666666667	39.9999456312714\\
80	39.9999496594838\\
80.3333333333333	39.9999534750221\\
80.6666666666667	39.9999570832019\\
81	39.9999604923228\\
81.3333333333333	39.9999637087911\\
81.6666666666667	39.9999667388353\\
82	39.9999695900665\\
82.3333333333333	39.9999722694043\\
82.6666666666667	39.9999747835284\\
83	39.9999771393039\\
83.3333333333333	39.9999793437912\\
83.6666666666667	39.9999814037749\\
84	39.9999833253565\\
84.3333333333333	39.9999851154654\\
84.6666666666667	39.9999867807375\\
85	39.999988326524\\
85.3333333333333	39.9999897594314\\
85.6666666666667	39.99999108577\\
86	39.9999923101745\\
86.3333333333333	39.9999934388011\\
86.6666666666667	39.999994477518\\
87	39.9999954302914\\
87.3333333333333	39.9999963027507\\
87.6666666666667	39.9999971002521\\
88	39.9999978261503\\
88.3333333333333	39.9999984855098\\
88.6666666666667	39.999999083142\\
89	39.9999996218508\\
89.3333333333333	40.0000001061274\\
89.6666666666667	40.0000005402323\\
90	40.0000009264808\\
90.3333333333333	40.0000012688033\\
90.6666666666667	40.0000015709238\\
91	40.0000018347292\\
91.3333333333333	40.000002063617\\
91.6666666666667	40.0000022608021\\
92	40.0000024278014\\
92.3333333333333	40.0000025675166\\
92.6666666666667	40.0000026826905\\
93	40.0000027745233\\
93.3333333333333	40.0000028454647\\
93.6666666666667	40.0000028978271\\
94	40.0000029325434\\
94.3333333333333	40.0000029516572\\
94.6666666666667	40.0000029570948\\
95	40.0000029495662\\
95.3333333333333	40.0000029307553\\
95.6666666666667	40.0000029022472\\
96	40.000002864568\\
96.3333333333333	40.0000028190871\\
96.6666666666667	40.000002767092\\
97	40.0000027089591\\
97.3333333333333	40.0000026457864\\
97.6666666666667	40.0000025786047\\
98	40.0000025076703\\
98.3333333333333	40.0000024338496\\
98.6666666666667	40.0000023579548\\
99	40.0000022801472\\
99.3333333333333	40.0000022010982\\
99.6666666666667	40.0000021214358\\
99.9999999999991	40.0000020412478\\
100	40.0000020412478\\
100.000000000001	40.0000020412478\\
100.051963134241	40.0012071941124\\
100.103926268481	40.0047979054622\\
100.155889402721	40.0107376371509\\
100.207582010479	40.0189417720223\\
100.259274618237	40.0294007855491\\
100.310967225995	40.0420810183344\\
100.364545577105	40.0575328864236\\
100.418123928215	40.0752997115387\\
100.471702279325	40.0953463648469\\
100.527190248097	40.1184740094981\\
100.582678216869	40.1439726452724\\
100.638166185641	40.171805691924\\
100.697156973014	40.2039158030199\\
100.756147760387	40.2385818085934\\
100.81513854776	40.2757625334927\\
100.87675903184	40.3172424335724\\
100.93837951592	40.3613774074291\\
101	40.4081235394367\\
101.067110212809	40.4619544254583\\
101.134220425618	40.5187772874578\\
101.201330638428	40.578539147783\\
101.272378103611	40.6449523583619\\
101.343425568794	40.7145415732072\\
101.414473033977	40.7872482389985\\
101.4896356513	40.8674962765185\\
101.564798268624	40.9511033170341\\
101.639960885947	41.0380048889285\\
101.719366714915	41.1333215416946\\
101.798772543883	41.2321726397746\\
101.878178372851	41.3344876104923\\
101.918785581901	41.3881265776738\\
101.95939279095	41.4426443498277\\
102	41.4980320056366\\
102.089473927374	41.6230997458632\\
102.178947854748	41.7522557469137\\
102.268421782122	41.8854100702741\\
102.359159714698	42.0244391016579\\
102.449897647273	42.1674012978546\\
102.540635579849	42.3142100916463\\
102.636922367988	42.4741092200228\\
102.733209156126	42.6381468478399\\
102.829495944265	42.8062275777354\\
102.88633062951	42.9072995546425\\
102.943165314755	43.0097291317724\\
103	43.1134979213554\\
103.109896663079	43.316297966749\\
103.219793326159	43.5208025541632\\
103.329689989238	43.7269664246864\\
103.430663309574	43.9178150211604\\
103.53163662991	44.1099935392874\\
103.632609950246	44.3034691719732\\
103.741771153587	44.5140557316533\\
103.850932356927	44.7260811048192\\
103.960093560268	44.9395065106615\\
103.973395706845	44.9656079217598\\
103.986697853423	44.991729490874\\
104	45.0178711504889\\
104.066510732887	45.1484183775866\\
104.133021465774	45.2785446059912\\
104.199532198661	45.4082518787554\\
104.311106968414	45.6249063631952\\
104.422681738167	45.8403969770318\\
104.53425650792	46.0547330283892\\
104.649748073758	46.2753869801891\\
104.765239639595	46.4948237352648\\
104.880731205433	46.7130531686734\\
104.920487470288	46.7878978555899\\
104.960243735144	46.8626010245585\\
105	46.9371630665723\\
105.138086328167	47.1937522729162\\
105.276172656335	47.4460943217876\\
105.414258984502	47.6942501941458\\
105.533996305177	47.9060885883234\\
105.653733625852	48.1148629324407\\
105.773470946527	48.320611108592\\
105.848980631018	48.4488225494591\\
105.924490315509	48.5758546156523\\
106	48.7017164738036\\
106.145869071484	48.9406175887658\\
106.291738142968	49.1733536351734\\
106.437607214451	49.4000163819098\\
106.567893677114	49.5974045496235\\
106.698180139776	49.790083140744\\
106.828466602438	49.9781146755886\\
106.885644401626	50.0591832399142\\
106.942822200813	50.1393738291211\\
107	50.2186915735267\\
107.161315303939	50.4370604927872\\
107.322630607877	50.6471520966493\\
107.483945911816	50.8490984185137\\
107.625320439701	51.0194867449029\\
107.766694967587	51.183805207383\\
107.908069495472	51.3421389515227\\
107.938712996982	51.37567744828\\
107.969356498491	51.4089395710241\\
108	51.4419261666324\\
108.153217507546	51.6023737537493\\
108.306435015092	51.7553009264248\\
108.459652522638	51.9008199211689\\
108.61529485509	52.0411709652706\\
108.770937187542	52.1741064516232\\
108.926579519995	52.2997395780725\\
108.95105301333	52.3188373063359\\
108.975526506665	52.3377576817126\\
109	52.3565011348233\\
109.122367466676	52.4474890465803\\
109.244734933351	52.5339380274203\\
109.367102400027	52.615902596561\\
109.544788176568	52.7270476737764\\
109.722473953108	52.8290146768853\\
109.900159729648	52.9219650017972\\
109.933439819766	52.9383847553071\\
109.966719909883	52.9544948344093\\
110	52.9702962755152\\
110.166400450586	53.0447765910527\\
110.332800901172	53.1118270708501\\
110.499201351758	53.1715729486856\\
110.666134234505	53.2242948297214\\
110.833067117253	53.2699134440728\\
111	53.3085509824856\\
111.234867785976	53.3518228519423\\
111.469735571953	53.3828321061869\\
111.704603357929	53.4018874036236\\
111.803068905286	53.4063928969054\\
111.901534452643	53.4088729712652\\
112	53.4093496328592\\
112.179485671192	53.40557855597\\
112.358971342385	53.396230761328\\
112.538457013577	53.3814220854685\\
112.692304675718	53.3644689584944\\
112.846152337859	53.3436589530597\\
113	53.3190625816225\\
113.17196082367	53.2876580219303\\
113.34392164734	53.2526618709667\\
113.51588247101	53.2141530347773\\
113.677254980674	53.174889859353\\
113.838627490337	53.1326654757092\\
114	53.0875427259343\\
114.198208114676	53.0289092661974\\
114.396416229352	52.9674191069537\\
114.594624344028	52.9031551327104\\
114.729749562685	52.8577959471382\\
114.864874781343	52.8112109902538\\
115	52.7634254472215\\
115.196808560005	52.6923425160877\\
115.39361712001	52.6200471366407\\
115.590425680016	52.5465846724208\\
115.726950453344	52.494962160307\\
115.863475226672	52.4428140823709\\
116	52.3901548585875\\
116.195603361127	52.314366161237\\
116.391206722254	52.2386211706076\\
116.586810083381	52.1629336199074\\
116.724540055588	52.1096815426631\\
116.862270027794	52.0564691505467\\
117	52.0033009253253\\
117.198623631288	51.9271392141941\\
117.397247262576	51.8519291106972\\
117.595870893864	51.7776597070607\\
117.730580595909	51.7278188724189\\
117.865290297955	51.6784024213198\\
118	51.6294070310734\\
118.206129159229	51.555582161134\\
118.412258318458	51.4833880971594\\
118.618387477688	51.4127931756319\\
118.745591651792	51.3700127630368\\
118.872795825896	51.3278222783187\\
119	51.2862145938467\\
119.218331905276	51.2163963527129\\
119.436663810552	51.1487448602131\\
119.654995715829	51.0832091113537\\
119.769997143886	51.049523377478\\
119.884998571943	51.0164035818065\\
120	50.9838426204225\\
120.23607336248	50.9189086617034\\
120.47214672496	50.8566022417263\\
120.70822008744	50.7968519963945\\
120.805480058293	50.7729626091593\\
120.902740029147	50.7494904589936\\
121	50.7264308016748\\
121.261351857623	50.6665918128682\\
121.522703715245	50.609838728905\\
121.784055572868	50.5560752077286\\
121.856037048579	50.5417801965931\\
121.928018524289	50.5277029098029\\
122	50.5138414366912\\
122.298877498108	50.4586077000813\\
122.597754996216	50.4070340485279\\
122.896632494324	50.3589866422946\\
122.931088329549	50.353668310549\\
122.965544164775	50.3483949209929\\
123	50.3431662791411\\
123.172279176127	50.3176833340002\\
123.344558352254	50.2932820737232\\
123.516837528382	50.2699391873494\\
123.677891685588	50.2490541534639\\
123.838945842794	50.2290556907784\\
124	50.2099257400467\\
124.333333333333	50.1729493692338\\
124.666666666667	50.1393368753542\\
125	50.1089467048245\\
125.333333333333	50.0815491540078\\
125.666666666667	50.0569255535634\\
126	50.0349576966831\\
126.300563509414	50.0172469076872\\
126.601127018828	50.0013571103649\\
126.901690528242	49.9872174086448\\
126.934460352161	49.9857786532544\\
126.967230176081	49.9843597959495\\
127	49.9829607496714\\
127.163849119596	49.9762340981721\\
127.327698239193	49.9699395431447\\
127.491547358789	49.9640677597506\\
127.661031572526	49.9584291157092\\
127.830515786263	49.9532230608176\\
128	49.9484398142544\\
128.333333333333	49.9401297642134\\
128.666666666667	49.9331550927635\\
129	49.9274565683217\\
129.285127234098	49.9234856802966\\
129.570254468197	49.9202439106073\\
129.855381702296	49.9177030575764\\
129.90358780153	49.9173407811452\\
129.951793900765	49.9169976265762\\
130	49.9166734647856\\
130.241030496174	49.9152915922272\\
130.482060992348	49.9142845241949\\
130.723091488522	49.9136397037854\\
130.815394325682	49.9134861093265\\
130.907697162841	49.9133831673092\\
131	49.9133302099546\\
131.333333333333	49.9134773259392\\
131.666666666667	49.9141064835112\\
132	49.9151946505352\\
132.333333333333	49.9166630984322\\
132.666666666667	49.9184375146639\\
133	49.9205025260823\\
133.333333333333	49.9227970072683\\
133.666666666667	49.9252633274626\\
134	49.927892036307\\
134.333333333333	49.9306368907076\\
134.666666666667	49.9334543508134\\
135	49.9363394303689\\
135.333333333333	49.9392582049348\\
135.666666666667	49.9421787878334\\
136	49.9450994557379\\
136.333333333333	49.9479962730723\\
136.666666666667	49.9508467934788\\
137	49.9536515867508\\
137.333333333333	49.9563946471231\\
137.666666666667	49.9590610147793\\
138	49.9616527855397\\
138.333333333333	49.9641601058557\\
138.666666666667	49.9665738173116\\
139	49.9688969485051\\
139.333333333333	49.9711242957247\\
139.666666666667	49.9732510790313\\
140	49.9752808127291\\
140.333333333333	49.9772116987911\\
140.666666666667	49.9790421585241\\
141	49.9807758654416\\
141.333333333333	49.9824134178611\\
141.666666666667	49.9839554842547\\
142	49.9854056683604\\
142.333333333333	49.9867661635284\\
142.666666666667	49.9880391287962\\
143	49.9892279463589\\
143.333333333333	49.990335783358\\
143.666666666667	49.991365703555\\
144	49.9923207755708\\
144.333333333333	49.9932046716009\\
144.666666666667	49.9940209187602\\
145	49.9947722250422\\
145.333333333333	49.995462425388\\
145.666666666667	49.9960951885377\\
146	49.9966728475794\\
146.333333333333	49.9971991604521\\
146.666666666667	49.9976777127413\\
147	49.9981104712453\\
147.333333333333	49.9985009584733\\
147.666666666667	49.9988525288556\\
148	49.9991668065869\\
148.333333333333	49.9994469832084\\
148.666666666667	49.9996960932947\\
149	49.9999154511516\\
149.333333333333	50.0001078690876\\
149.666666666667	50.0002760176276\\
150	50.0004209383968\\
150.333333333333	50.0005450505494\\
150.666666666667	50.0006506486372\\
151	50.0007385400182\\
151.333333333333	50.0008107606514\\
151.666666666667	50.0008692395745\\
152	50.0009145872155\\
152.333333333333	50.0009484819142\\
152.666666666667	50.0009725122353\\
153	50.0009871264503\\
153.333333333333	50.0009936801149\\
153.666666666667	50.0009934549474\\
154	50.0009867683996\\
154.333333333333	50.0009746925836\\
154.666666666667	50.0009582401141\\
155	50.0009376251019\\
155.333333333333	50.0009136766378\\
155.666666666667	50.0008871768717\\
156	50.0008582601052\\
156.333333333333	50.0008275515515\\
156.666666666667	50.0007956402215\\
157	50.0007626003256\\
157.333333333333	50.0007288895429\\
157.666666666667	50.0006949383372\\
158	50.0006607770128\\
158.333333333333	50.0006267283746\\
158.666666666667	50.0005930953799\\
159	50.0005598774416\\
159.333333333333	50.0005272910587\\
159.666666666667	50.0004955387997\\
160	50.0004645994314\\
160.333333333333	50.0004346075536\\
160.666666666667	50.0004056884876\\
161	50.0003778082365\\
161.333333333333	50.0003510399065\\
161.666666666667	50.0003254508998\\
162	50.0003010003645\\
162.333333333333	50.0002777166259\\
162.666666666667	50.0002556249802\\
163	50.0002346820153\\
163.333333333333	50.0002148846844\\
163.666666666667	50.000196228852\\
164	50.0001786715467\\
164.333333333333	50.000162188881\\
164.666666666667	50.0001467572306\\
165	50.0001323360513\\
165.333333333333	50.0001188886962\\
165.666666666667	50.0001063796739\\
166	50.0000947720754\\
166.333333333333	50.000084022536\\
166.666666666667	50.0000740893857\\
167	50.0000649399787\\
167.333333333333	50.0000565286081\\
167.666666666667	50.0000488115379\\
168	50.0000417605946\\
168.333333333333	50.0000353307722\\
168.666666666667	50.0000294791234\\
169	50.0000241818643\\
169.333333333333	50.0000193966917\\
169.666666666667	50.0000150833168\\
170	50.0000112220729\\
170.333333333333	50.0000077745638\\
170.666666666667	50.0000047042791\\
171	50.0000019952818\\
171.333333333333	49.9999996136953\\
171.666666666667	49.9999975273484\\
172	49.9999957235901\\
172.333333333333	49.9999941732484\\
172.666666666667	49.9999928486513\\
173	49.999991739974\\
173.333333333333	49.9999908226389\\
173.666666666667	49.9999900733575\\
174	49.9999894846827\\
174.333333333333	49.9999890363305\\
174.666666666667	49.9999887091\\
175	49.9999884975044\\
175.333333333333	49.9999883851384\\
175.666666666667	49.9999883564884\\
176	49.9999884076512\\
176.333333333333	49.9999885256302\\
176.666666666667	49.9999886981482\\
177	49.9999889225553\\
177.333333333333	49.9999891887797\\
177.666666666667	49.9999894873186\\
178	49.9999898164918\\
178.333333333333	49.9999901686842\\
178.666666666667	49.9999905367203\\
179	49.9999909196519\\
179.333333333333	49.9999913118851\\
179.666666666667	49.9999917081574\\
180	49.9999921080574\\
180.333333333333	49.9999925076207\\
180.666666666667	49.9999929031257\\
181	49.9999932945405\\
181.333333333333	49.9999936791876\\
181.666666666667	49.9999940545606\\
182	49.9999944208832\\
182.333333333333	49.9999947764716\\
182.666666666667	49.9999951197568\\
183	49.9999954511223\\
183.333333333333	49.9999957696325\\
183.666666666667	49.9999960744234\\
184	49.9999963659663\\
184.333333333333	49.9999966438728\\
184.666666666667	49.9999969077932\\
185	49.9999971582342\\
185.333333333333	49.9999973951925\\
185.666666666667	49.99999761868\\
186	49.999997829202\\
186.333333333333	49.9999980270131\\
186.666666666667	49.9999982123663\\
187	49.9999983857409\\
187.333333333333	49.9999985475513\\
187.666666666667	49.9999986981994\\
188	49.9999988381233\\
188.333333333333	49.9999989678238\\
188.666666666667	49.9999990877824\\
189	49.999999198388\\
189.333333333333	49.9999993001743\\
189.666666666667	49.9999993936521\\
190	49.9999994791581\\
190.333333333333	49.9999995572215\\
190.666666666667	49.9999996283476\\
191	49.9999996928212\\
191.333333333333	49.9999997511424\\
191.666666666667	49.9999998037877\\
192	49.9999998509931\\
192.333333333333	49.9999998932145\\
192.666666666667	49.9999999308852\\
193	49.9999999641971\\
193.333333333333	49.9999999935538\\
193.666666666667	50.0000000193383\\
194	50.0000000417035\\
194.333333333333	50.0000000609978\\
194.666666666667	50.0000000775519\\
195	50.0000000914846\\
195.333333333333	50.0000001030905\\
195.666666666667	50.0000001126484\\
196	50.000000120249\\
196.333333333333	50.0000001261358\\
196.666666666667	50.0000001305394\\
197	50.0000001335269\\
197.333333333333	50.0000001352958\\
197.666666666667	50.0000001360329\\
198	50.0000001357862\\
198.333333333333	50.0000001347125\\
198.666666666667	50.0000001329603\\
199	50.0000001305622\\
199.333333333333	50.0000001276403\\
199.666666666667	50.0000001243096\\
200	50.0000001205913\\
};
\end{axis}
\end{tikzpicture}%}
  \caption{Step response using a zero-order hold of sample time $1$ sec.}
  \label{fig:Q7.1}
\end{figure}

\begin{figure}[H]\centering
	\centering
	\scalebox{1}{% This file was created by matlab2tikz.
%
%The latest updates can be retrieved from
%  http://www.mathworks.com/matlabcentral/fileexchange/22022-matlab2tikz-matlab2tikz
%where you can also make suggestions and rate matlab2tikz.
%
\definecolor{mycolor1}{rgb}{0.00000,0.44700,0.74100}%
%
\begin{tikzpicture}

\begin{axis}[%
width=4.133in,
height=3.26in,
at={(0.693in,0.44in)},
scale only axis,
xmin=0,
xmax=200,
xmajorgrids,
ymin=30,
ymax=55,
ymajorgrids,
axis background/.style={fill=white}
]
\addplot [color=mycolor1,solid,forget plot]
  table[row sep=crcr]{%
0	40\\
0.0666666666666667	39.9988529199354\\
0.133333333333333	39.9954198587329\\
0.2	39.9897130376372\\
0.266666666666667	39.9817446149091\\
0.333333333333333	39.9715266859614\\
0.4	39.959071286313\\
0.47499490925035	39.942400621861\\
0.5499898185007	39.9229306571585\\
0.62498472775105	39.9006782627291\\
0.703019200761635	39.8745881554063\\
0.78105367377222	39.8455224785902\\
0.859088146782806	39.8134999414055\\
0.940721928520819	39.7768560227102\\
1.02235571025883	39.7370177534163\\
1.10398949199685	39.6940062251002\\
1.18937445628243	39.6456457248368\\
1.27475942056802	39.5938604847577\\
1.36014438485361	39.5386742795039\\
1.44941284679236	39.4773672565188\\
1.53868130873111	39.4123956293716\\
1.62794977066986	39.3437861740105\\
1.72115687139107	39.2682963135632\\
1.81436397211228	39.1888997791019\\
1.90757107283349	39.1056266274356\\
1.93838071522233	39.0772532766076\\
1.96919035761116	39.0484607009771\\
2	39.0192499766143\\
2.10280158530272	38.9204150008745\\
2.20560317060544	38.8202527136271\\
2.30840475590817	38.7187881496068\\
2.39217645422804	38.6351602646607\\
2.47594815254792	38.5506975512143\\
2.55971985086779	38.4654133506129\\
2.65035248755473	38.3722345382189\\
2.74098512424168	38.2781264478673\\
2.83161776092862	38.1831057321341\\
2.92774506709594	38.0813454925111\\
3.02387237326326	37.9785969186242\\
3.11999967943058	37.8748795694164\\
3.22554318749687	37.7599098975372\\
3.33108669556316	37.6438214083989\\
3.43663020362945	37.5266395469625\\
3.55442732437873	37.3945927247898\\
3.67222444512802	37.2612503430385\\
3.7900215658773	37.1266470934711\\
3.86001437725153	37.0460853530923\\
3.93000718862577	36.9650978161219\\
4	36.8836916387702\\
4.084026152496	36.7870247096103\\
4.168052304992	36.6929479594034\\
4.25207845748801	36.6014191408825\\
4.33610460998401	36.5123967818476\\
4.42013076248001	36.4258401647491\\
4.50415691497601	36.341709300342\\
4.59533018726636	36.253120763005\\
4.68650345955671	36.1672927168842\\
4.77767673184706	36.0841768491583\\
4.87628569769649	35.9972806184676\\
4.97489466354593	35.9134427675073\\
5.07350362939537	35.8326056757701\\
5.18063700929633	35.7481157495863\\
5.28777038919728	35.6670302663071\\
5.39490376909824	35.5892798109623\\
5.51185402760203	35.5081324277794\\
5.62880428610583	35.4307918945455\\
5.74575454460962	35.3571737298582\\
5.83050302973975	35.3061051583374\\
5.91525151486987	35.2569170868271\\
6	35.2095792060385\\
6.11052917961399	35.1515428985431\\
6.22105835922798	35.0984402854427\\
6.33158753884198	35.0501493727963\\
6.44211671845597	35.0065515787593\\
6.55264589806996	34.967531594126\\
6.66317507768395	34.932977223695\\
6.78544037940584	34.8998242881971\\
6.90770568112773	34.8718585229317\\
7.02997098284962	34.8489404668414\\
7.16439929713635	34.8294070437655\\
7.29882761142308	34.8156373499137\\
7.43325592570981	34.8074621554466\\
7.58318995160709	34.8047429072671\\
7.73312397750436	34.8085590544253\\
7.88305800340164	34.8186976907307\\
7.92203866893443	34.8223435936097\\
7.96101933446721	34.8263993631492\\
8	34.8308614896486\\
8.1491425417522	34.8510070946668\\
8.29828508350441	34.8756573539332\\
8.44742762525661	34.9046715098583\\
8.59657016700882	34.9379129590438\\
8.74571270876102	34.9752491247576\\
8.89485525051322	35.0165512690264\\
9.0594247179955	35.0665782033574\\
9.22399418547779	35.1211191184216\\
9.38856365296007	35.1800163541979\\
9.56617752048635	35.2482942530195\\
9.74379138801264	35.3212831019091\\
9.92140525553892	35.3988035353563\\
9.94760350369262	35.4106108159973\\
9.97380175184631	35.422512362722\\
10	35.4345076330473\\
10.1309912407685	35.495057241796\\
10.2619824815369	35.5562711808732\\
10.3929737223054	35.6181267798147\\
10.5231695337911	35.680220674964\\
10.6533653452769	35.7429052874002\\
10.7835611567627	35.8061598095524\\
10.9260985340574	35.8760398257142\\
11.0686359113522	35.9465524683477\\
11.211173288647	36.0176722931975\\
11.3665561224047	36.095864040564\\
11.5219389561624	36.1747165104869\\
11.6773217899201	36.2541991888802\\
11.7848811932801	36.3095723069258\\
11.89244059664	36.3652235238133\\
12	36.4211433869249\\
12.1370411138001	36.4921631776317\\
12.2740822276002	36.562399870204\\
12.4111233414003	36.6318628823105\\
12.5481644552004	36.7005614996648\\
12.6852055690005	36.7685048776944\\
12.8222466828006	36.8357020451163\\
12.975201860551	36.9098322575142\\
13.1281570383013	36.9830561733408\\
13.2811122160516	37.055385804101\\
13.4487700363402	37.1336546967418\\
13.6164278566288	37.2108786513634\\
13.7840856769174	37.2870727416355\\
13.8560571179449	37.3194687716989\\
13.9280285589725	37.351678914167\\
14	37.3837043195419\\
14.1590988325837	37.4535029540272\\
14.3181976651673	37.5217354882392\\
14.477296497751	37.5884308806629\\
14.6363953303347	37.6536175746264\\
14.7954941629183	37.717323503011\\
14.954592995502	37.7795761067536\\
15.134927877162	37.8484147807504\\
15.3152627588219	37.9154593919385\\
15.4955976404819	37.9807476849207\\
15.6637317603213	38.0400693079075\\
15.8318658801606	38.0979257867839\\
16	38.1543459790778\\
16.2282853176513	38.2285910742229\\
16.4565706353026	38.3000919572608\\
16.6848559529539	38.3689238324263\\
16.8993876723249	38.4312416309378\\
17.1139193916959	38.4913264665264\\
17.3284511110669	38.5492363183925\\
17.5523007407112	38.6074038978111\\
17.7761503703556	38.6633278574768\\
18	38.7170694443995\\
18.2928624601938	38.7843752913728\\
18.5857249203877	38.8485498757785\\
18.8785873805815	38.9097024002744\\
19.1217935840583	38.9582652681366\\
19.3649997875352	39.0048757878162\\
19.608205991012	39.0495912194539\\
19.738803994008	39.0728401956449\\
19.869401997004	39.0955672763524\\
20	39.1177807984113\\
20.2260668109429	39.1551598066438\\
20.4521336218859	39.1912947311899\\
20.6782004328288	39.2262183152966\\
20.9042672437718	39.2599625158377\\
21.1303340547147	39.2925585081959\\
21.3564008656576	39.3240367243378\\
21.5709339104384	39.3529022632402\\
21.7854669552192	39.380812702355\\
22	39.407792257176\\
22.3202318024385	39.4464765450369\\
22.640463604877	39.4833952183368\\
22.9606954073156	39.518613118539\\
23.226643482588	39.5466133527644\\
23.4925915578605	39.5735185777666\\
23.7585396331329	39.5993625985648\\
23.8390264220886	39.6069798741802\\
23.9195132110443	39.6145038618878\\
24	39.6219354478601\\
24.4024339447784	39.6577131310881\\
24.8048678895569	39.6912619658454\\
25.2073018343353	39.7226874759396\\
25.4715345562236	39.7422151344367\\
25.7357672781118	39.7608986354856\\
26	39.7787647836766\\
26.2991886251125	39.7980015011387\\
26.5983772502251	39.81618460871\\
26.8975658753376	39.8333531818913\\
27.1967545004502	39.849545041174\\
27.4959431255627	39.8647967578398\\
27.7951317506753	39.8791437353967\\
27.8634211671169	39.882295184037\\
27.9317105835584	39.8854016861226\\
28	39.8884636376839\\
28.3414470822079	39.9030955023857\\
28.6828941644157	39.9166263733936\\
29.0243412466236	39.9291049240536\\
29.3495608310824	39.9400548373682\\
29.6747804155412	39.9501315788277\\
30	39.9593727486098\\
30.4607574449338	39.9711370224952\\
30.9215148898676	39.9814557978763\\
31.3822723348014	39.9904182791666\\
31.5881815565343	39.9940075482554\\
31.7940907782671	39.9973501789157\\
32	40.0004532715921\\
32.3135635903446	40.0047702967245\\
32.6271271806893	40.0086414476255\\
32.9406907710339	40.0120861829015\\
33.2542543613786	40.0151232762012\\
33.5678179517232	40.017770818754\\
33.8813815420679	40.0200462671524\\
33.9209210280452	40.0203076304559\\
33.9604605140226	40.0205633787407\\
34	40.0208135449063\\
34.1976974298869	40.0219975040755\\
34.3953948597738	40.0230775389986\\
34.5930922896607	40.0240566256708\\
34.9257236660814	40.0254841818704\\
35.2583550425021	40.0266475950156\\
35.5909864189228	40.0275597834539\\
35.7273242792819	40.0278639804722\\
35.8636621396409	40.0281288961617\\
36	40.0283553489978\\
36.3233466415018	40.0287762392971\\
36.6466932830036	40.0290593654254\\
36.9700399245054	40.0292115112489\\
37.3133599496703	40.0292369780929\\
37.6566799748351	40.0291296424187\\
38	40.0288967073387\\
38.4179047967019	40.0284885862398\\
38.8358095934037	40.0279832655716\\
39.2537143901056	40.027387609267\\
39.5024762600704	40.0269928398617\\
39.7512381300352	40.0265696946308\\
40	40.026119451176\\
40.3922815228351	40.0253744632188\\
40.7845630456702	40.0246025813638\\
41.1768445685053	40.0238061929149\\
41.4512297123369	40.0232358023216\\
41.7256148561684	40.0226552731911\\
42	40.0220653192403\\
42.5404692055779	40.0208957548615\\
43.0809384111559	40.0197301005514\\
43.6214076167338	40.0185701393012\\
43.7476050778226	40.0183002902946\\
43.8738025389113	40.0180308603351\\
44	40.0177618684613\\
44.4703868427218	40.0167715340475\\
44.9407736854437	40.0158039144348\\
45.4111605281655	40.014858688047\\
45.6074403521103	40.0144708138758\\
45.8037201760552	40.0140867572762\\
46	40.0137064938111\\
46.5020044119357	40.0127580544558\\
47.0040088238714	40.0118471765723\\
47.506013235807	40.0109723727806\\
47.670675490538	40.0106930425443\\
47.835337745269	40.0104173903291\\
48	40.0101453677169\\
48.529852964495	40.0093001168072\\
49.05970592899	40.0085017387463\\
49.589558893485	40.0077477937477\\
49.7263725956567	40.0075600485917\\
49.8631862978283	40.007375072169\\
50	40.0071928263741\\
50.6666666666667	40.0063490716525\\
51.3333333333333	40.0055770883808\\
52	40.0048716814963\\
52.5691423420476	40.0043209209872\\
53.1382846840952	40.0038169021933\\
53.7074270261428	40.0033565446988\\
53.8049513507619	40.003281829344\\
53.9024756753809	40.0032082964054\\
54	40.0031359319704\\
54.4876216230953	40.002791902113\\
54.9752432461907	40.0024764888706\\
55.462864869286	40.0021879999288\\
55.6419099128574	40.0020885098789\\
55.8209549564287	40.0019923561518\\
56	40.0018994636999\\
56.6666666666667	40.0015807819101\\
57.3333333333333	40.0013022099768\\
58	40.0010604277779\\
58.6666666666667	40.0008513053059\\
59.3333333333333	40.0006710913097\\
60	40.0005173368297\\
60.6666666666667	40.0003866131534\\
61.3333333333333	40.000275826128\\
62	40.0001832500542\\
62.6666666666667	40.0001062732968\\
63.3333333333333	40.000042548459\\
64	39.9999909014454\\
64.6666666666667	39.9999494485743\\
65.3333333333333	39.999916501732\\
66	39.9998912856168\\
66.6666666666667	39.9998724811295\\
67.3333333333333	39.9998589095072\\
68	39.9998500755359\\
68.6666666666667	39.9998450681474\\
69.3333333333333	39.9998430760385\\
70	39.9998437979724\\
70.6666666666667	39.9998466121402\\
71.3333333333333	39.9998509673229\\
72	39.9998566949224\\
72.6666666666667	39.9998633809624\\
73.3333333333333	39.9998706605847\\
74	39.9998784535777\\
74.6666666666667	39.9998864977804\\
75.3333333333333	39.9998945638837\\
76	39.9999026276748\\
76.6666666666667	39.9999105371136\\
77.3333333333333	39.9999181606837\\
78	39.9999255066795\\
78.6666666666667	39.9999324997395\\
79.3333333333333	39.9999390760242\\
80	39.9999452599963\\
80.6666666666667	39.9999510258853\\
81.3333333333333	39.9999563533067\\
82	39.9999612723334\\
82.6666666666667	39.9999657859688\\
83.3333333333333	39.9999698987943\\
84	39.9999736403634\\
84.6666666666667	39.9999770278249\\
85.3333333333333	39.9999800778144\\
86	39.9999828163406\\
86.6666666666667	39.9999852654974\\
87.3333333333333	39.9999874459078\\
88	39.9999893789612\\
88.6666666666667	39.9999910866241\\
89.3333333333333	39.9999925890996\\
90	39.9999939031606\\
90.6666666666667	39.9999950482957\\
91.3333333333333	39.9999960422791\\
92	39.9999968977741\\
92.6666666666667	39.9999976309962\\
93.3333333333333	39.9999982566466\\
94	39.9999987839604\\
94.6666666666667	39.9999992258562\\
95.3333333333333	39.9999995939882\\
96	39.9999998948551\\
96.6666666666667	40.0000001383848\\
97.3333333333333	40.000000333493\\
98	40.0000004845728\\
98.6666666666667	40.000000598987\\
99.3333333333333	40.0000006833196\\
99.9999999999991	40.0000007404001\\
100	40.0000007404001\\
100.000000000001	40.0000007404001\\
100.033333333334	40.0004975857709\\
100.066666666668	40.0019815376542\\
100.100000000001	40.0044428335458\\
100.133333333334	40.0078718527244\\
100.166666666668	40.0122591125174\\
100.200000000001	40.0175952645405\\
100.254274164709	40.0282902398129\\
100.308548329416	40.0414373957124\\
100.362822494124	40.0569985428575\\
100.41803748384	40.0752680217827\\
100.473252473556	40.095958804653\\
100.528467463273	40.1190332694758\\
100.585766063412	40.145459387482\\
100.643064663551	40.1743726118562\\
100.700363263691	40.2057335748845\\
100.76081717281	40.2414326459742\\
100.82127108193	40.2797693510467\\
100.881724991049	40.3207004212347\\
100.946273824899	40.367220177493\\
101.010822658749	40.4165990814328\\
101.075371492599	40.4687879214533\\
101.143924281103	40.5272370869229\\
101.212477069608	40.5887450823957\\
101.281029858113	40.6532569351475\\
101.353634408266	40.72479744793\\
101.426238958419	40.7995848977796\\
101.498843508573	40.8775584885077\\
101.575602947023	40.9633933247244\\
101.652362385473	41.0526546439006\\
101.729121823924	41.1452756799471\\
101.810187716988	41.2466687789199\\
101.891253610052	41.3516616864428\\
101.972319503116	41.4601814390083\\
101.98154633541	41.472753400652\\
101.990773167705	41.4853700247536\\
102	41.4980312081635\\
102.046134161474	41.5620019248484\\
102.092268322948	41.6270713181892\\
102.138402484421	41.6932268164349\\
102.230246616295	41.8281139963288\\
102.322090748169	41.9671609691358\\
102.413934880042	42.1102745009525\\
102.507484608774	42.2601324325319\\
102.601034337507	42.4140219321834\\
102.694584066239	42.5718521696631\\
102.793972553828	42.7437516405134\\
102.893361041418	42.9198961925135\\
102.992749529007	43.1001855045723\\
103.098276007634	43.2960367249455\\
103.203802486261	43.4963369884492\\
103.309328964888	43.7009760001321\\
103.421697878597	43.9235335547472\\
103.534066792305	44.1507636408656\\
103.646435706014	44.3825443042402\\
103.764290470676	44.6304005426399\\
103.882145235338	44.8829998586948\\
104	45.1402137331725\\
104.101469251519	45.3628130736932\\
104.202938503038	45.583814619065\\
104.304407754556	45.8032340464065\\
104.405877006075	46.0210867999223\\
104.507346257594	46.2373880961723\\
104.608815509113	46.4521529303088\\
104.718528428149	46.6826542052029\\
104.828241347185	46.911394894137\\
104.937954266221	47.1383931128812\\
105.055499926632	47.3796841769273\\
105.173045587043	47.6190171504455\\
105.290591247454	47.8564132760171\\
105.417301859542	48.1101737612527\\
105.544012471631	48.3617336228118\\
105.670723083719	48.6111181707124\\
105.780482055813	48.8254004468978\\
105.890241027906	49.0380851067573\\
106	49.2491878391875\\
106.114259224274	49.4654990008019\\
106.228518448549	49.6765908688701\\
106.342777672823	49.8825210517944\\
106.457036897098	50.0833464330912\\
106.571296121372	50.2791231878583\\
106.685555345647	50.4699068106986\\
106.810316963616	50.6725886011237\\
106.935078581585	50.8694530088901\\
107.059840199554	51.0605696313335\\
107.194301523506	51.2601882554572\\
107.328762847457	51.453295352062\\
107.463224171409	51.6399747394876\\
107.609037538345	51.8352462377171\\
107.754850905281	52.0231602682223\\
107.900664272217	52.2038195447009\\
107.933776181478	52.243844144323\\
107.966888090739	52.2835010449697\\
108	52.322791420263\\
108.165559546304	52.5122894554199\\
108.331119092609	52.689824327722\\
108.496678638913	52.8555679757899\\
108.630842727954	52.9813486427892\\
108.765006816994	53.0995866533566\\
108.899170906035	53.2103702526823\\
109.049157837651	53.3255017460703\\
109.199144769267	53.431546718846\\
109.349131700883	53.5286246010183\\
109.510557860468	53.6232227672134\\
109.671984020053	53.7077164413944\\
109.833410179638	53.78224977285\\
109.888940119759	53.8056122623193\\
109.944470059879	53.8278187674797\\
110	53.8488750331026\\
110.192543987691	53.913399334135\\
110.385087975381	53.9650797862923\\
110.577631963072	54.0041440614322\\
110.822248729758	54.0359312729087\\
111.066865496443	54.0481744443582\\
111.311482263128	54.0413233326328\\
111.540988175419	54.017929493365\\
111.770494087709	53.9784805300522\\
112	53.9233345488446\\
112.142803465281	53.8823710225739\\
112.285606930562	53.8377674695639\\
112.428410395842	53.7895855437377\\
112.571213861123	53.7378863814822\\
112.714017326404	53.6827305925928\\
112.856820791685	53.624178282685\\
113.018722337529	53.55376240709\\
113.180623883374	53.4791434488655\\
113.342525429219	53.400406668928\\
113.522231350578	53.308291701189\\
113.701937271938	53.2113218004926\\
113.881643193297	53.1096101064313\\
113.921095462198	53.0866568707625\\
113.960547731099	53.0634816726758\\
114	53.040085686395\\
114.133349880282	52.9605999604485\\
114.266699760565	52.8810504649274\\
114.400049640847	52.8014457247983\\
114.53339952113	52.7217941337212\\
114.666749401412	52.6421039531797\\
114.800099281695	52.5623833160781\\
114.948836958371	52.4734372554374\\
115.097574635048	52.3844741795348\\
115.246312311724	52.2955048147834\\
115.408860634146	52.1982796567856\\
115.571408956569	52.1010730521126\\
115.733957278991	52.0038982200703\\
115.822638185994	51.9509009350274\\
115.911319092997	51.897919037229\\
116	51.8449545738399\\
116.139563397847	51.7625909226542\\
116.279126795694	51.6821569294565\\
116.418690193542	51.6036209936437\\
116.558253591389	51.5269519703344\\
116.697816989236	51.4521191685854\\
116.837380387083	51.3790923391437\\
116.993818258117	51.299345769519\\
117.150256129151	51.221789510435\\
117.306694000185	51.1463828953579\\
117.478405328439	51.0660412937804\\
117.650116656693	50.9881894004937\\
117.821827984947	50.9127761481275\\
117.881218656631	50.8872510268095\\
117.940609328316	50.862009581853\\
118	50.8370497762349\\
118.164690007043	50.7697923383388\\
118.329380014085	50.705623137672\\
118.494070021128	50.6444734945263\\
118.65876002817	50.5862760710435\\
118.823450035213	50.5309648577043\\
118.988140042255	50.4784751278107\\
119.176020110948	50.4219586191469\\
119.363900179642	50.3689393423409\\
119.551780248335	50.3193271095808\\
119.701186832223	50.2822475551558\\
119.850593416112	50.2472233062244\\
120	50.2142116885825\\
120.244413009701	50.1642877418166\\
120.488826019402	50.1191657383097\\
120.733239029102	50.0786834157326\\
121.085644372993	50.0281644696925\\
121.438049716883	49.9865084235577\\
121.790455060774	49.9532780963349\\
121.860303373849	49.9476554269789\\
121.930151686925	49.9423440639636\\
122	49.9373408532308\\
122.188408908929	49.9249970183925\\
122.376817817858	49.9141053866112\\
122.565226726787	49.9046277417478\\
122.753635635716	49.8965266805363\\
122.942044544644	49.8897656067465\\
123.130453453573	49.8843087005069\\
123.353496748022	49.8794864445646\\
123.576540042472	49.8763860313422\\
123.799583336921	49.8749521090853\\
123.866388891281	49.8748388633172\\
123.93319444564	49.8748688469558\\
124	49.8750406458624\\
124.190107212195	49.8759522599024\\
124.380214424391	49.8773001971165\\
124.570321636586	49.8790722864698\\
124.760428848782	49.8812566126782\\
124.950536060977	49.8838415149332\\
125.140643273173	49.8868155771081\\
125.3637442663	49.8907870663061\\
125.586845259428	49.8952614977556\\
125.809946252555	49.9002216683039\\
125.873297501703	49.9017163562577\\
125.936648750852	49.9032484776\\
126	49.9048176572858\\
126.20756003604	49.9099676615454\\
126.41512007208	49.9150160236589\\
126.622680108121	49.9199646277236\\
126.830240144161	49.9248153242801\\
127.037800180201	49.9295699304326\\
127.245360216241	49.9342302310724\\
127.496906810828	49.9397542644465\\
127.748453405414	49.9451454173458\\
128	49.9504066823066\\
128.28078119446	49.9559592372164\\
128.56156238892	49.9610259060065\\
128.84234358338	49.9656238203714\\
129.090904130448	49.9693163236762\\
129.339464677516	49.9726657652231\\
129.588025224583	49.9756830950609\\
129.725350149722	49.9772115884651\\
129.862675074861	49.9786437354566\\
130	49.9799812747432\\
130.372357210488	49.9831491484512\\
130.744714420976	49.9856743893093\\
131.117071631465	49.9875886269729\\
131.411381087643	49.9886893722913\\
131.705690543822	49.9894419255151\\
132	49.9898603387024\\
132.300347805088	49.990034354802\\
132.600695610176	49.9900389419751\\
132.901043415265	49.9898812498705\\
133.201391220353	49.9895682019766\\
133.501739025441	49.9891064964652\\
133.802086830529	49.9885026199359\\
133.868057887019	49.9883515619892\\
133.93402894351	49.9881940137605\\
134	49.9880300399789\\
134.329855282451	49.9872051840036\\
134.659710564903	49.9864018883091\\
134.989565847354	49.9856195954302\\
135.326377231569	49.9848419127234\\
135.663188615785	49.9840849985226\\
136	49.983348304252\\
136.387145869272	49.9825922849973\\
136.774291738544	49.9819895799365\\
137.161437607815	49.981532331031\\
137.44095840521	49.981288351061\\
137.720479202605	49.9811135199806\\
138	49.9810051795132\\
138.452950298499	49.9809813806334\\
138.905900596998	49.9811477588245\\
139.358850895497	49.9814921342498\\
139.572567263664	49.981713087647\\
139.786283631832	49.9819699052696\\
140	49.9822614406545\\
140.439229850775	49.9829417458365\\
140.87845970155	49.9837102794753\\
141.317689552326	49.9845608925962\\
141.545126368217	49.9850316595129\\
141.772563184109	49.9855220786947\\
142	49.9860313882083\\
142.394571764462	49.9869309978142\\
142.789143528923	49.9878302778749\\
143.183715293385	49.9887285426891\\
143.45581019559	49.9893470509826\\
143.727905097795	49.9899645662606\\
144	49.9905808948056\\
144.419048513668	49.9915043198229\\
144.838097027336	49.9923796236225\\
145.257145541004	49.9932088497029\\
145.504763694003	49.9936779867298\\
145.752381847001	49.994132114873\\
146	49.9945716177962\\
146.472114281842	49.9953561229957\\
146.944228563683	49.9960640358864\\
147.416342845525	49.9966995957194\\
147.61089523035	49.996941403169\\
147.805447615175	49.9971718857061\\
148	49.9973913112911\\
148.666666666667	49.9980586937726\\
149.333333333333	49.998601466483\\
150	49.9990298349003\\
150.666666666667	49.9993619660413\\
151.333333333333	49.9996142785255\\
152	49.9997935188618\\
152.666666666667	49.9999170519692\\
153.333333333333	50.0000004130571\\
154	50.0000470934354\\
154.571808103044	50.0000669626872\\
155.143616206088	50.0000766100828\\
155.715424309132	50.0000768271711\\
155.810282872755	50.0000760039003\\
155.905141436377	50.0000749449635\\
156	50.0000736536031\\
156.474292818113	50.0000665835793\\
156.948585636227	50.0000594471508\\
157.42287845434	50.0000522567238\\
157.615252302894	50.0000493275729\\
157.807626151447	50.0000463922056\\
158	50.0000434513457\\
158.666666666667	50.0000351044108\\
159.333333333333	50.0000302478209\\
160	50.0000285729773\\
160.666666666667	50.0000295279998\\
161.333333333333	50.0000326148809\\
162	50.0000376318614\\
162.666666666667	50.0000436110357\\
163.333333333333	50.0000496957797\\
164	50.0000558630967\\
164.666666666667	50.000061445667\\
165.333333333333	50.0000658586372\\
166	50.0000691993093\\
166.666666666667	50.0000712692611\\
167.333333333333	50.0000718996888\\
168	50.000071220876\\
168.666666666667	50.000069376318\\
169.333333333333	50.0000664978436\\
170	50.0000626863307\\
170.666666666667	50.0000582211172\\
171.333333333333	50.0000533513363\\
172	50.0000481246686\\
172.666666666667	50.0000427958376\\
173.333333333333	50.0000375897888\\
174	50.000032506706\\
174.666666666667	50.0000277010469\\
175.333333333333	50.0000233077171\\
176	50.0000192986352\\
176.666666666667	50.0000157233011\\
177.333333333333	50.0000126234543\\
178	50.0000099624006\\
178.666666666667	50.0000077177453\\
179.333333333333	50.0000058679637\\
180	50.0000043809941\\
180.666666666667	50.000003203271\\
181.333333333333	50.0000022862779\\
182	50.0000016081774\\
182.666666666667	50.000001114981\\
183.333333333333	50.0000007583413\\
184	50.00000052655\\
184.666666666667	50.000000380699\\
185.333333333333	50.0000002861547\\
186	50.0000002384316\\
186.666666666667	50.0000002165815\\
187.333333333333	50.0000002020494\\
188	50.000000194198\\
188.666666666667	50.000000185845\\
189.333333333333	50.0000001706727\\
190	50.000000149338\\
190.666666666667	50.0000001222208\\
191.333333333333	50.0000000896838\\
192	50.000000052294\\
192.666666666667	50.0000000128743\\
193.333333333333	49.9999999739207\\
194	49.9999999354758\\
194.666666666667	49.9999998998716\\
195.333333333333	49.9999998691488\\
196	49.9999998429283\\
196.666666666667	49.999999822078\\
197.333333333333	49.999999807338\\
198	49.9999997981817\\
198.666666666667	49.9999997942318\\
199.333333333333	49.9999997951322\\
200	49.9999998004375\\
};
\end{axis}
\end{tikzpicture}%}
  \caption{Step response using a zero-order hold of sample time $2$ sec.}
  \label{fig:Q7.2}
\end{figure}

\begin{figure}[H]\centering
	\centering
	\scalebox{1}{\input{./images/7/zoh_3.tex}}
  \caption{Step response using a zero-order hold of sample time $3$ sec.}
  \label{fig:Q7.3}
\end{figure}

\begin{figure}[H]\centering
	\centering
	\scalebox{1}{% This file was created by matlab2tikz.
%
%The latest updates can be retrieved from
%  http://www.mathworks.com/matlabcentral/fileexchange/22022-matlab2tikz-matlab2tikz
%where you can also make suggestions and rate matlab2tikz.
%
\definecolor{mycolor1}{rgb}{0.00000,0.44700,0.74100}%
%
\begin{tikzpicture}

\begin{axis}[%
width=4.133in,
height=3.26in,
at={(0.693in,0.44in)},
scale only axis,
xmin=0,
xmax=200,
xmajorgrids,
ymin=30,
ymax=60,
ymajorgrids,
axis background/.style={fill=white}
]
\addplot [color=mycolor1,solid,forget plot]
  table[row sep=crcr]{%
0	40\\
0.0666666666666667	39.9988529199354\\
0.133333333333333	39.9954198587329\\
0.2	39.9897130376372\\
0.266666666666667	39.9817446149091\\
0.333333333333333	39.9715266859614\\
0.4	39.959071286313\\
0.47499490925035	39.942400621861\\
0.5499898185007	39.9229306571585\\
0.62498472775105	39.9006782627291\\
0.703019200761635	39.8745881554063\\
0.78105367377222	39.8455224785902\\
0.859088146782806	39.8134999414055\\
0.940721928520819	39.7768560227102\\
1.02235571025883	39.7370177534163\\
1.10398949199685	39.6940062251002\\
1.18937445628243	39.6456457248368\\
1.27475942056802	39.5938604847577\\
1.36014438485361	39.5386742795039\\
1.44941284679236	39.4773672565188\\
1.53868130873111	39.4123956293716\\
1.62794977066986	39.3437861740105\\
1.72115687139107	39.2682963135632\\
1.81436397211228	39.1888997791019\\
1.90757107283349	39.1056266274356\\
2.00466164289194	39.0147939058355\\
2.10175221295039	38.9198209377759\\
2.19884278300884	38.820741250687\\
2.29961827190097	38.7135934398924\\
2.4003937607931	38.6020944165113\\
2.50116924968523	38.4862812151995\\
2.60526965685247	38.3621560855701\\
2.70937006401971	38.2335072833433\\
2.81347047118694	38.1003751782987\\
2.92038536533207	37.9590191789915\\
3.02730025947719	37.8130200499409\\
3.13421515362232	37.6624210955585\\
3.24774120796246	37.4975227201725\\
3.36126726230261	37.3275382844275\\
3.47479331664276	37.1525191924883\\
3.59549874757794	36.9609670376133\\
3.71620417851312	36.7638427524235\\
3.8369096094483	36.5612076866583\\
3.89127307296553	36.4681611053372\\
3.94563653648277	36.3740147074874\\
4	36.2787740753648\\
4.07306942194238	36.152095843641\\
4.14613884388476	36.0294931183831\\
4.21920826582714	35.9108811838989\\
4.29227768776952	35.796178044701\\
4.3653471097119	35.6853042573769\\
4.43841653165428	35.5781827711582\\
4.5171075185368	35.46693067802\\
4.59579850541931	35.3598532595318\\
4.67448949230183	35.2568627152937\\
4.7592944849153	35.1503479371381\\
4.84409947752878	35.0483775603095\\
4.92890447014226	34.9508513932065\\
5.02063315931593	34.8502539384762\\
5.1123618484896	34.7546232236021\\
5.20409053766328	34.6638437873081\\
5.30372507114198	34.570606453934\\
5.40335960462069	34.4828218084415\\
5.50299413809939	34.4003553458356\\
5.61179781360876	34.3162212594037\\
5.72060148911813	34.238109789708\\
5.8294051646275	34.1658621089286\\
5.94908455034606	34.0929817836071\\
6.06876393606462	34.0268125816024\\
6.18844332178317	33.9671635161121\\
6.3214736675571	33.9082875860319\\
6.45450401333103	33.8569952407483\\
6.58753435910497	33.8130506489221\\
6.7379330059141	33.7719325624768\\
6.88833165272323	33.7395992897998\\
7.03873029953236	33.71574659807\\
7.21451342374266	33.6982254123178\\
7.39029654795296	33.691441932337\\
7.56607967216326	33.6949672827354\\
7.71071978144218	33.7053088774931\\
7.85535989072109	33.7221319822755\\
8	33.74522419756\\
8.15252700317569	33.7767894410732\\
8.30505400635138	33.8161272006975\\
8.45758100952708	33.8629608681523\\
8.61010801270277	33.9170242882754\\
8.76263501587846	33.9780612407922\\
8.91516201905415	34.0458248474517\\
9.09931712462248	34.1362652848691\\
9.28347223019081	34.2357631553153\\
9.46762733575914	34.3439336416887\\
9.69963552431586	34.4919899943087\\
9.93164371287258	34.6525267448353\\
10.1636519014293	34.824878560811\\
10.4391405218495	35.0440063327187\\
10.7146291422698	35.2779081358734\\
10.99011776269	35.52564605868\\
11.2142019674259	35.7367503696152\\
11.4382861721618	35.9559754962579\\
11.6623703768977	36.1828934821622\\
11.7749135845985	36.2996329541861\\
11.8874567922992	36.4181599860009\\
12	36.538425967994\\
12.1046532338433	36.6499871442254\\
12.2093064676866	36.7594320430574\\
12.3139597015299	36.866787413311\\
12.4186129353732	36.9720796627864\\
12.5232661692165	37.0753348627797\\
12.6279194030599	37.1765787581565\\
12.7443083172095	37.2868447107008\\
12.8606972313592	37.3946889206864\\
12.9770861455089	37.5001454248574\\
13.101868051356	37.6105925589471\\
13.2266499572031	37.7183745757524\\
13.3514318630503	37.8235316747092\\
13.4864175167662	37.9343779025481\\
13.6214031704822	38.0422481322269\\
13.7563888241981	38.1471910175533\\
13.903174783484	38.2580414566287\\
14.04996074277	38.3655476354312\\
14.1967467020559	38.4697691913331\\
14.3573830489624	38.5801300352673\\
14.5180193958688	38.686702967507\\
14.6786557427753	38.7895622395453\\
14.8556869209316	38.8987053126733\\
15.0327180990878	39.0035224114959\\
15.2097492772441	39.1041075609045\\
15.406333661228	39.2109555185282\\
15.6029180452118	39.3128225199338\\
15.7995024291957	39.4098297673651\\
15.8663349527971	39.4417234237903\\
15.9331674763986	39.4730737322289\\
16	39.5038852582858\\
16.1327262155386	39.5626547181104\\
16.2654524310771	39.6176875327205\\
16.3981786466157	39.6690445179591\\
16.5309048621542	39.7167857686254\\
16.6636310776928	39.760970660097\\
16.7963572932313	39.8016578775067\\
16.9460517610879	39.8434222604604\\
17.0957462289444	39.8808934293317\\
17.245440696801	39.9141525042314\\
17.4078633004524	39.9455681024085\\
17.5702859041037	39.9722200303759\\
17.7327085077551	39.9942078126419\\
17.9104071135667	40.0130367284705\\
18.0881057193784	40.026527054908\\
18.26580432519	40.034803605116\\
18.4603184206402	40.0380315578811\\
18.6548325160904	40.035318368683\\
18.8493466115407	40.0268202732182\\
19.0616820126806	40.0111202096076\\
19.2740174138205	39.9889064142977\\
19.4863528149604	39.9603720992146\\
19.6575685433069	39.9328900069572\\
19.8287842716535	39.9015198267338\\
20	39.8663581567543\\
20.1208933589023	39.8404873279588\\
20.2417867178047	39.8151559384308\\
20.362680076707	39.7903558348309\\
20.4835734356094	39.7660789713245\\
20.6044667945117	39.7423174093784\\
20.7253601534141	39.7190633150461\\
20.8612333973674	39.6935238882434\\
20.9971066413207	39.6686049006851\\
21.132979885274	39.6442956945198\\
21.2801979593893	39.6186328762818\\
21.4274160335047	39.5936604523864\\
21.57463410762	39.5693654912071\\
21.7359761404339	39.543502695946\\
21.8973181732478	39.5184218568609\\
22.0586602060616	39.4941068095691\\
22.2367662909783	39.4681355495683\\
22.414872375895	39.4430571979824\\
22.5929784608116	39.4188512186743\\
22.7914222337681	39.3928840773718\\
22.9898660067246	39.3679476850679\\
23.1883097796811	39.3440153925612\\
23.411858463321	39.3182256223527\\
23.6354071469609	39.2936406682986\\
23.8589558306008	39.2702250791665\\
23.9059705537339	39.2654460946238\\
23.9529852768669	39.260716983882\\
24	39.2560374321674\\
24.1385694113088	39.2432962426967\\
24.2771388226177	39.2324812136223\\
24.4157082339265	39.2235470180477\\
24.5542776452353	39.2164492395379\\
24.6928470565442	39.2111443608442\\
24.831416467853	39.2075897336139\\
24.9880803400675	39.2056264474203\\
25.144744212282	39.2057878676585\\
25.3014080844964	39.2080162342612\\
25.4736056180221	39.2127828340913\\
25.6458031515477	39.2199052557249\\
25.8180006850733	39.2293121500175\\
26.0102280851084	39.2424264604957\\
26.2024554851435	39.2582067533857\\
26.3946828851787	39.2765614264422\\
26.6122894085667	39.3003339577143\\
26.8298959319547	39.3271661564961\\
27.0475024553426	39.3569366215433\\
27.2994121284508	39.3949152379287\\
27.5513218015589	39.4364979428166\\
27.803231474667	39.4815149615515\\
27.8688209831113	39.4937791475988\\
27.9344104915557	39.5062622145298\\
28	39.5189613701785\\
28.1392395133098	39.5459325702102\\
28.2784790266195	39.572477640315\\
28.4177185399293	39.5986027835891\\
28.556958053239	39.6243141151023\\
28.6961975665488	39.6496176623588\\
28.8354370798585	39.6745193677976\\
28.9962072431167	39.7027794390204\\
29.1569774063748	39.7305204722606\\
29.317747569633	39.7577512021688\\
29.4938952930581	39.787010855333\\
29.6700430164833	39.8156793337178\\
29.8461907399085	39.8437675460977\\
30.0424314412404	39.8743895139419\\
30.2386721425724	39.9043191108213\\
30.4349128439043	39.9335705828379\\
30.6556582436586	39.9656816481366\\
30.8764036434129	39.9969717720502\\
31.0971490431671	40.0274599763527\\
31.348749521522	40.0612555543731\\
31.6003499998768	40.0940603893223\\
31.8519504782316	40.1259006815833\\
31.9013003188211	40.1320350091484\\
31.9506501594105	40.1381334026344\\
32	40.1441960508241\\
32.1538964440721	40.1623176487676\\
32.3077928881442	40.1790018873996\\
32.4616893322163	40.194279663293\\
32.6155857762884	40.208181372883\\
32.7694822203606	40.2207369124273\\
32.9233786644327	40.2319756954817\\
33.0985928776949	40.2432051237268\\
33.2738070909571	40.2528071774334\\
33.4490213042194	40.2608230330602\\
33.6427017183607	40.2678865853065\\
33.836382132502	40.2731148379764\\
34.0300625466434	40.2765601963564\\
34.2472583056547	40.2783666155159\\
34.464454064666	40.278066212083\\
34.6816498236773	40.2757281241652\\
34.9274613097975	40.2707089006609\\
35.1732727959177	40.2632615547545\\
35.4190842820379	40.2534789946514\\
35.6127228546919	40.2441876775045\\
35.806361427346	40.2335465390153\\
36	40.2215978087901\\
36.1641111483948	40.2109068518111\\
36.3282222967896	40.2001740895774\\
36.4923334451844	40.189402454527\\
36.6564445935791	40.1785948111575\\
36.8205557419739	40.1677539561493\\
36.9846668903687	40.156882621057\\
37.176337987329	40.144150586511\\
37.3680090842892	40.1313847493276\\
37.5596801812495	40.1185891338382\\
37.7730131918708	40.1043171611969\\
37.9863462024922	40.0900183729214\\
38.1996792131136	40.0756978357886\\
38.4420904314972	40.0594051687731\\
38.6845016498809	40.0430976965517\\
38.9269128682646	40.0267821135723\\
39.206029494205	40.0079943896947\\
39.4851461201455	39.9892138891339\\
39.7642627460859	39.9704496437213\\
39.8428418307239	39.9651711140647\\
39.921420915362	39.9598947463623\\
40	39.9546207252474\\
40.1617428209242	39.9441764447705\\
40.3234856418483	39.9345367957106\\
40.4852284627725	39.9256822216263\\
40.6469712836966	39.917593551581\\
40.8087141046208	39.9102519980419\\
40.9704569255449	39.9036391430925\\
41.1620194431999	39.8967249474144\\
41.3535819608549	39.8907782710226\\
41.5451444785098	39.8857704654969\\
41.7580682553436	39.8812719609\\
41.9709920321773	39.8778614025494\\
42.183915809011	39.8755023244266\\
42.4262993247523	39.8740514466391\\
42.6686828404936	39.8738662810997\\
42.911066356235	39.8748974692764\\
43.1911032531322	39.8775406943499\\
43.4711401500294	39.8816723121499\\
43.7511770469267	39.8872234155529\\
43.8341180312845	39.889130174913\\
43.9170590156422	39.8911539198827\\
44	39.8932929812234\\
44.1942676071621	39.8985117271784\\
44.3885352143242	39.9038761287946\\
44.5828028214863	39.9093803893127\\
44.7770704286484	39.9150188584413\\
44.9713380358105	39.9207860316221\\
45.1656056429726	39.9266765433991\\
45.4000078624982	39.9339407645848\\
45.6344100820239	39.9413680250569\\
45.8688123015495	39.9489496781365\\
46.1343665285908	39.9577149057466\\
46.3999207556321	39.9666557771843\\
46.6654749826733	39.975760938339\\
46.9754659175433	39.9865829429793\\
47.2854568524132	39.9975971907879\\
47.5954477872831	40.0087876520462\\
47.7302985248554	40.013706759434\\
47.8651492624277	40.0186550763682\\
48	40.0236314018587\\
48.1791023649896	40.030010312875\\
48.3582047299792	40.0359007160372\\
48.5373070949688	40.041315114199\\
48.7164094599584	40.0462657581187\\
48.895511824948	40.0507646470745\\
49.0746141899376	40.0548235388299\\
49.2922357639058	40.0591798471408\\
49.509857337874	40.0629237341656\\
49.7274789118421	40.0660748969728\\
49.9715415704278	40.0689275640875\\
50.2156042290134	40.0710852438324\\
50.459666887599	40.0725735770596\\
50.7416478314561	40.0734923367446\\
51.0236287753133	40.0735881109836\\
51.3056097191704	40.0728969565762\\
51.5370731461136	40.0717660854348\\
51.7685365730568	40.070147216931\\
52	40.0680585546281\\
52.2416113218349	40.065521268362\\
52.4832226436699	40.0627544824453\\
52.7248339655048	40.0597679205184\\
52.9664452873398	40.0565710217197\\
53.2080566091747	40.0531729415119\\
53.4496679310097	40.0495825674365\\
53.746869756334	40.0449150780345\\
54.0440715816584	40.039985170366\\
54.3412734069828	40.0348077895345\\
54.6880687188549	40.0284727901701\\
55.034864030727	40.0218423444718\\
55.3816593425991	40.0149374413684\\
55.5877728950661	40.0107118238305\\
55.793886447533	40.006400424998\\
56	40.0020071995535\\
56.1976456098025	39.997907289364\\
56.3952912196051	39.9941033445803\\
56.5929368294076	39.9905870021667\\
56.7905824392101	39.9873500901113\\
56.9882280490127	39.9843846266481\\
57.1858736588152	39.9816828119665\\
57.4321542136346	39.9786735806126\\
57.678434768454	39.9760475385348\\
57.9247153232733	39.9737907384756\\
58.2039701754422	39.971661381916\\
58.483225027611	39.9699702290155\\
58.7624798797798	39.9686987732249\\
59.0915172085873	39.967714951067\\
59.4205545373948	39.967260774192\\
59.7495918662023	39.9673091242332\\
59.8330612441348	39.9673981624197\\
59.9165306220674	39.9675174430507\\
60	39.9676665557172\\
60.3241863499502	39.968467779083\\
60.6483726999005	39.9695867415144\\
60.9725590498507	39.9710065061803\\
61.296745399801	39.9727107975248\\
61.6209317497512	39.9746839979357\\
61.9451180997015	39.9769110985624\\
62.3591935902716	39.9801021405801\\
62.7732690808418	39.9836552946801\\
63.187344571412	39.9875433597474\\
63.4582297142746	39.9902557058266\\
63.7291148571373	39.9930934076444\\
64	39.9960497933183\\
64.2185746848966	39.998396866915\\
64.4371493697932	40.0005782858387\\
64.6557240546898	40.0025990962676\\
64.8742987395863	40.0044642197744\\
65.0928734244829	40.0061784537705\\
65.3114481093795	40.0077464774894\\
65.5937631881889	40.0095627783448\\
65.8760782669982	40.0111523083222\\
66.1583933458076	40.0125243092919\\
66.4819663213369	40.0138408023033\\
66.8055392968661	40.0148962681404\\
67.1291122723954	40.0157032173664\\
67.419408181597	40.0162255927292\\
67.7097040907985	40.0165660194411\\
68	40.0167325977132\\
68.4925710113153	40.0166245789218\\
68.9851420226306	40.0160467168743\\
69.4777130339459	40.0150355988888\\
69.9245294540649	40.0137725903838\\
70.3713458741839	40.0122057245573\\
70.8181622943029	40.0103579861579\\
71.212108196202	40.008513262866\\
71.606054098101	40.00648125564\\
72	40.0042754210235\\
72.2402078460706	40.002925944353\\
72.4804156921412	40.0016662381764\\
72.7206235382119	40.0004933587746\\
72.9608313842825	39.999404441931\\
73.2010392303531	39.9983967026153\\
73.4412470764237	39.9974674308049\\
73.7665271812036	39.9963294336038\\
74.0918072859835	39.9953241683563\\
74.4170873907634	39.994445533977\\
74.7937240519241	39.9935786598981\\
75.1703607130848	39.9928649432311\\
75.5469973742455	39.992296009304\\
75.697998249497	39.9921067588306\\
75.8489991247485	39.9919389885465\\
76	39.9917922071436\\
76.3621152547958	39.9915486878758\\
76.7242305095916	39.9914663319195\\
77.0863457643875	39.9915360785927\\
77.4484610191833	39.9917492523827\\
77.8105762739791	39.992097561291\\
78.1726915287749	39.9925730649719\\
78.6678116539695	39.9934153230999\\
79.162931779164	39.994462815525\\
79.6580519043585	39.9956982207785\\
79.7720346029057	39.9960074998632\\
79.8860173014528	39.9963256800456\\
80	39.996652573054\\
80.2932950780083	39.9974764289363\\
80.5865901560166	39.9982450560415\\
80.8798852340248	39.9989605805011\\
81.1731803120331	39.9996250597638\\
81.4664753900414	40.000240482874\\
81.7597704680497	40.0008087748744\\
82.1425089145372	40.0014825692436\\
82.5252473610248	40.0020832479359\\
82.9079858075124	40.0026146226689\\
83.2719905383416	40.0030590219016\\
83.6359952691708	40.003447040549\\
84	40.00378156052\\
84.3714348664166	40.0040449898533\\
84.7428697328331	40.0042091335041\\
85.1143045992497	40.0042796027259\\
85.4857394656663	40.0042617667842\\
85.8571743320829	40.0041607538633\\
86.2286091984994	40.0039814715641\\
86.7339256393048	40.0036203056519\\
87.2392420801101	40.003134122031\\
87.7445585209154	40.0025334456604\\
87.8297056806103	40.0024216718682\\
87.9148528403051	40.0023069748745\\
88	40.0021893997484\\
88.4257357984743	40.0016165787972\\
88.8514715969487	40.0010865696839\\
89.277207395423	40.0005971392053\\
89.6274766144983	40.0002233852011\\
89.9777458335736	39.9998745175628\\
90.3280150526488	39.9995494421308\\
90.8049953568355	39.999143137772\\
91.2819756610221	39.9987764688765\\
91.7589559652087	39.9984470273529\\
91.8393039768058	39.9983950318269\\
91.9196519884029	39.9983440170392\\
92	39.9982939725144\\
92.4017400579855	39.9980799763812\\
92.803480115971	39.9979309167238\\
93.2052201739564	39.997842879746\\
93.6185534012421	39.9978120603774\\
94.0318866285278	39.9978380219193\\
94.4452198558134	39.99791708354\\
94.9634799038756	39.99808589386\\
95.4817399519378	39.9983261802131\\
96	39.9986318237591\\
96.3852185921925	39.9988763774502\\
96.770437184385	39.9991102766698\\
97.1556557765776	39.9993339438858\\
97.5408743687701	39.9995477858224\\
97.9260929609626	39.9997521935216\\
98.3113115531552	39.9999475435788\\
98.8742077021034	40.0002174579237\\
99.4371038510517	40.000469881138\\
99.9999999999991	40.0007058329336\\
100	40.0007058329336\\
100.000000000001	40.0007058329336\\
100.056289614896	40.0021420234399\\
100.112579229791	40.0063735495015\\
100.168868844685	40.0133541055031\\
100.223663179069	40.0227466664\\
100.278457513453	40.034660616682\\
100.333251847836	40.0490561688067\\
100.390233326462	40.0666166964019\\
100.447214805087	40.0867760855073\\
100.504196283712	40.1094925486376\\
100.563329561955	40.1357268731215\\
100.622462840197	40.164626433793\\
100.681596118439	40.1961476085594\\
100.743052471355	40.2316394581531\\
100.804508824272	40.269870104653\\
100.865965177188	40.3107938083631\\
100.930092035253	40.3563185847931\\
100.994218893318	40.4046772244906\\
101.058345751384	40.4558212008061\\
101.126558721887	40.5132280951283\\
101.194771692389	40.5736771660847\\
101.262984662892	40.6371139360197\\
101.335237819715	40.7075072211167\\
101.407490976537	40.7811311660149\\
101.47974413336	40.8579254916164\\
101.556141462588	40.9425076051432\\
101.632538791816	41.0305005522234\\
101.708936121044	41.1218381001056\\
101.789625003677	41.2218666957692\\
101.87031388631	41.3254798828487\\
101.951002768943	41.4326052548281\\
102.036172059064	41.5494111192955\\
102.121341349185	41.6699704635271\\
102.206510639307	41.7942044171415\\
102.296383365436	41.929198893435\\
102.386256091566	42.0681115074527\\
102.476128817695	42.2108565817301\\
102.571144709898	42.365845842281\\
102.666160602101	42.5249295346105\\
102.761176494304	42.6880142483091\\
102.86188394078	42.8651344350987\\
102.962591387256	43.0465419581533\\
103.063298833732	43.2321344444255\\
103.170315104109	43.4338284897854\\
103.277331374487	43.6400198112844\\
103.384347644865	43.8505956467482\\
103.498368889654	44.0796564711345\\
103.612390134443	44.3134416706109\\
103.726411379232	44.5518264369992\\
103.817607586155	44.7457199167985\\
103.908803793077	44.9424179187508\\
104	45.1418615226792\\
104.101372757585	45.364261975954\\
104.20274551517	45.5850659187867\\
104.304118272755	45.8042890071934\\
104.40549103034	46.0219466648174\\
104.506863787925	46.2380540881801\\
104.60823654551	46.4526262528993\\
104.717928020219	46.6830940244043\\
104.827619494929	46.9117999232302\\
104.937310969638	47.1387620830928\\
105.054823333207	47.3799920244127\\
105.172335696776	47.6192628029653\\
105.289848060346	47.8565956770188\\
105.416521191209	48.1102836466395\\
105.543194322072	48.3617698570411\\
105.669867452935	48.6110796363428\\
105.807170250762	48.8788808851525\\
105.944473048589	49.1441855104142\\
106.081775846417	49.407024107773\\
106.231598949303	49.6910520352359\\
106.381422052189	49.9722177717028\\
106.531245155075	50.2505589045418\\
106.696061303514	50.5535353457014\\
106.860877451952	50.8531862940785\\
107.025693600391	51.1495588862648\\
107.208876051663	51.4751672532267\\
107.392058502935	51.7968449778652\\
107.575240954207	52.1146527490234\\
107.716827302805	52.3576804750599\\
107.858413651402	52.5984586032473\\
108	52.8370135665447\\
108.103944338247	53.0081945502034\\
108.207888676494	53.1731245233163\\
108.311833014741	53.3318535076323\\
108.415777352989	53.4844309811913\\
108.519721691236	53.6309058907536\\
108.623666029483	53.7713266757858\\
108.739071094181	53.920196029503\\
108.85447615888	54.0617266232072\\
108.969881223578	54.1959826670418\\
109.092978799717	54.3312412690399\\
109.216076375856	54.4583717239816\\
109.339173951995	54.5774494618966\\
109.471266741673	54.6963568008403\\
109.603359531351	54.8061686687541\\
109.735452321029	54.906975296179\\
109.877261216949	55.0052753842589\\
110.01907011287	55.0934088937176\\
110.160879008791	55.1714839322835\\
110.312924241998	55.2441426573011\\
110.464969475205	55.3054918015733\\
110.617014708413	55.3556604663421\\
110.779274521898	55.3970112271523\\
110.941534335383	55.4259280329553\\
111.103794148868	55.4425630523621\\
111.275412522636	55.4469598616497\\
111.447030896404	55.437962369565\\
111.618649270171	55.4157454471365\\
111.745766180114	55.3908695209754\\
111.872883090057	55.3589059776666\\
112	55.3199242437082\\
112.187430766437	55.2513168465795\\
112.374861532875	55.1708660851557\\
112.562292299312	55.0787748920327\\
112.721696736736	54.991445000871\\
112.881101174159	54.8959644580365\\
113.040505611583	54.792455712049\\
113.233491997705	54.656571703545\\
113.426478383827	54.5093138613237\\
113.61946476995	54.350895755505\\
113.864481102243	54.1340217050855\\
114.109497434535	53.8999324665019\\
114.354513766828	53.6490571185635\\
114.619217866961	53.3596519100087\\
114.883921967093	53.0516867951783\\
115.148626067225	52.7256933307935\\
115.367489858817	52.442914949849\\
115.586353650409	52.1484707651387\\
115.805217442001	51.8426572214179\\
115.870144961334	51.7497922863971\\
115.935072480667	51.6559604880012\\
116	51.561169514966\\
116.098585572433	51.4184856159539\\
116.197171144866	51.2796985420527\\
116.295756717299	51.1447430421525\\
116.394342289732	51.0135551366924\\
116.492927862166	50.8860720799188\\
116.591513434599	50.7622323137589\\
116.700616904414	50.6293545285979\\
116.80972037423	50.500784705563\\
116.918823844045	50.3764442300188\\
117.035583568023	50.2479759233426\\
117.152343292002	50.1241707588591\\
117.26910301598	50.0049382597204\\
117.395190415579	49.881213680235\\
117.521277815178	49.7626091954096\\
117.647365214777	49.649018123999\\
117.784386520113	49.5311395393886\\
117.921407825448	49.4189276216757\\
118.058429130784	49.3122546486754\\
118.208596659293	49.2015609897603\\
118.358764187801	49.0972102114757\\
118.50893171631	48.9990461784627\\
118.675375310337	48.8972873007301\\
118.841818904364	48.8027402568067\\
119.00826249839	48.7152087258118\\
119.195854814092	48.6247260753492\\
119.383447129793	48.5426470360439\\
119.571039445494	48.4687140095769\\
119.714026296996	48.4176767755943\\
119.857013148498	48.3711191890112\\
120	48.3289359552343\\
120.14499286046	48.2913949136359\\
120.289985720921	48.2598505365815\\
120.434978581381	48.2341475769488\\
120.579971441842	48.2141348902271\\
120.724964302302	48.1996653033494\\
120.869957162762	48.1905954415211\\
121.037927468915	48.1866556916935\\
121.205897775068	48.1895642863599\\
121.37386808122	48.199116293481\\
121.562885150616	48.2175617996534\\
121.751902220012	48.2438935034054\\
121.940919289408	48.2778456150564\\
122.162912323423	48.3271045732955\\
122.384905357438	48.3861183805533\\
122.606898391453	48.454499900788\\
122.897707441356	48.5576323670576\\
123.188516491258	48.6754003564846\\
123.47932554116	48.8070397797807\\
123.652883694107	48.8919118429555\\
123.826441847053	48.9813161449582\\
124	49.075108133676\\
124.119169276447	49.1404102982759\\
124.238338552895	49.2045787395029\\
124.357507829342	49.2676277922734\\
124.476677105789	49.3295716230919\\
124.595846382236	49.3904242310708\\
124.715015658684	49.4501994520246\\
124.849984211697	49.5166157020167\\
124.98495276471	49.5816871030345\\
125.119921317723	49.6454330138568\\
125.265949083627	49.712931455945\\
125.411976849532	49.7789245090303\\
125.558004615436	49.8434356968754\\
125.717944851942	49.9124201273604\\
125.877885088447	49.9796847058604\\
126.037825324953	50.0452589838576\\
126.214226658327	50.1156568683257\\
126.390627991701	50.1840723918255\\
126.567029325075	50.2505432999515\\
126.763346212026	50.3222776337343\\
126.959663098977	50.3916996352625\\
127.155979985928	50.4588585837064\\
127.376764490746	50.5317441985622\\
127.597548995565	50.6018960252023\\
127.818333500384	50.6693800415552\\
127.878889000256	50.6874315570406\\
127.939444500128	50.7052885635689\\
128	50.722952367912\\
128.136833190379	50.7611513252548\\
128.273666380758	50.796386799483\\
128.410499571137	50.828705531918\\
128.547332761516	50.858153749981\\
128.684165951895	50.8847771663346\\
128.820999142274	50.9086209958047\\
128.977565441666	50.9325511192653\\
129.134131741057	50.9529668783336\\
129.290698040448	50.9699336549158\\
129.461430743874	50.9845802098282\\
129.632163447299	50.995284808828\\
129.802896150724	51.0021289774934\\
129.991403489444	51.0052981121571\\
130.179910828164	51.0039658980726\\
130.368418166885	50.9982374644121\\
130.577390697177	50.9868744771434\\
130.786363227468	50.970375022017\\
130.99533575776	50.9488756993753\\
131.227543695069	50.9192846040348\\
131.459751632377	50.8838669175088\\
131.691959569686	50.8428004532423\\
131.794639713124	50.8228855082267\\
131.897319856562	50.8019151074305\\
132	50.7799040196196\\
132.144987737073	50.7479787440406\\
132.289975474146	50.7158247929175\\
132.434963211219	50.6834502436576\\
132.579950948293	50.6508630426241\\
132.724938685366	50.6180710048799\\
132.869926422439	50.5850818181198\\
133.036404447837	50.546969560501\\
133.202882473235	50.5086185850486\\
133.369360498633	50.470039919975\\
133.552500343897	50.4273500344868\\
133.73564018916	50.3844117960375\\
133.918780034424	50.3412390133612\\
134.123561261042	50.2927034315644\\
134.328342487659	50.243909945282\\
134.533123714277	50.194876568623\\
134.764518733143	50.1392040829785\\
134.99591375201	50.0832724194004\\
135.227308770876	50.0271055834334\\
135.484872513918	49.9643386607768\\
135.742436256959	49.9013404612775\\
136	49.8381412054251\\
136.139434821275	49.8048034429062\\
136.27886964255	49.7732913190431\\
136.418304463825	49.7435690063821\\
136.5577392851	49.7156012809997\\
136.697174106375	49.6893535173048\\
136.83660892765	49.664791670648\\
136.997053755871	49.6385714887434\\
137.157498584092	49.6144892092718\\
137.317943412313	49.5924957362575\\
137.494016660313	49.5707066835169\\
137.670089908314	49.5513129064653\\
137.846163156315	49.534253407806\\
138.042744661984	49.5178916103191\\
138.239326167653	49.5042839681876\\
138.435907673323	49.4933512189405\\
138.658019376387	49.4841201079623\\
138.880131079452	49.4780963313568\\
139.102242782517	49.475174064887\\
139.357795606742	49.4755152803325\\
139.613348430967	49.4796738067481\\
139.868901255192	49.4875021142089\\
139.912600836795	49.4891977526081\\
139.956300418397	49.4909958082278\\
140	49.4928955834719\\
140.218497908013	49.5034663741792\\
140.436995816025	49.515637699441\\
140.655493724038	49.5293551024147\\
140.856785777021	49.5433148292113\\
141.058077830003	49.5585012689867\\
141.259369882986	49.5748749621875\\
141.494156127914	49.5954218895921\\
141.728942372841	49.6174721994005\\
141.963728617769	49.6409680560913\\
142.230472823262	49.6693449028932\\
142.497217028755	49.6994351964952\\
142.763961234248	49.7311615768346\\
143.072856046146	49.7698472754185\\
143.381750858043	49.810514507583\\
143.690645669941	49.8530552679238\\
143.793763779961	49.8676556107174\\
143.89688188998	49.8824493602306\\
144	49.8974327965723\\
144.151796356991	49.9191973166272\\
144.303592713982	49.9401125659099\\
144.455389070973	49.9601940172169\\
144.607185427964	49.9794569142501\\
144.758981784955	49.9979162721328\\
144.910778141946	50.015586884266\\
145.092033104518	50.0356738812103\\
145.27328806709	50.0546814599989\\
145.454543029663	50.0726335929768\\
145.654418096119	50.0912344967879\\
145.854293162575	50.1086115684767\\
146.05416822903	50.124795180994\\
146.279726867829	50.1416622803327\\
146.505285506627	50.1570892898294\\
146.730844145425	50.1711171708073\\
146.988136969041	50.185461565623\\
147.245429792656	50.1980955363494\\
147.502722616272	50.2090756230638\\
147.668481744181	50.2152992763871\\
147.834240872091	50.22087379957\\
148	50.2258133721318\\
148.217856233162	50.231140940793\\
148.435712466324	50.2349911035592\\
148.653568699486	50.2374071366105\\
148.871424932648	50.2384313850157\\
149.08928116581	50.2381052626444\\
149.307137398971	50.2364692929603\\
149.569226283693	50.2328214289301\\
149.831315168414	50.2274025055355\\
150.093404053136	50.2202781143348\\
150.393910415926	50.2100930026224\\
150.694416778717	50.1978431966913\\
150.994923141508	50.1836193751048\\
151.329948761005	50.1655420160385\\
151.664974380503	50.1452380962416\\
152	50.1228214223994\\
152.175472627205	50.1106986453673\\
152.350945254409	50.0988762240071\\
152.526417881614	50.0873485182431\\
152.701890508818	50.0761099817304\\
152.877363136023	50.0651551616142\\
153.052835763228	50.0544786954697\\
153.263224213894	50.0420373483378\\
153.473612664561	50.0299796556019\\
153.684001115227	50.0182968753822\\
153.919610692202	50.0056480307031\\
154.155220269177	49.9934468635985\\
154.390829846151	49.9816818655926\\
154.661871097701	49.9686725022973\\
154.93291234925	49.9562087753569\\
155.2039536008	49.9442744122648\\
155.469302400533	49.9330882524143\\
155.734651200267	49.9223797824887\\
156	49.9121348900671\\
156.200535306064	49.9049813921875\\
156.401070612129	49.898647339247\\
156.601605918193	49.8931103201141\\
156.802141224257	49.8883484106242\\
157.002676530322	49.8843401709817\\
157.203211836386	49.8810646263581\\
157.444292651083	49.8780677370386\\
157.68537346578	49.876065374069\\
157.926454280478	49.8750236726226\\
158.201155512446	49.8749656393331\\
158.475856744414	49.8760647377202\\
158.750557976382	49.8782750126861\\
159.073679079109	49.882236794974\\
159.396800181836	49.8876043089297\\
159.719921284564	49.8943099497139\\
159.813280856376	49.8964872606987\\
159.906640428188	49.8987692899034\\
160	49.9011545189018\\
160.212244781889	49.9067242535161\\
160.424489563778	49.9123538044012\\
160.636734345667	49.9180400769628\\
160.848979127556	49.9237800570945\\
161.061223909445	49.9295708108131\\
161.273468691334	49.9354094806076\\
161.534665509191	49.9426564397654\\
161.795862327047	49.9499667277904\\
162.057059144903	49.9573354779653\\
162.356112023017	49.9658379378066\\
162.655164901131	49.9744040599459\\
162.954217779244	49.9830272918345\\
163.30281185283	49.9931427549124\\
163.651405926415	50.0033177081999\\
164	50.0135430238471\\
164.209454497015	50.0194645554228\\
164.41890899403	50.0249229646432\\
164.628363491045	50.0299306656242\\
164.83781798806	50.0344998090999\\
165.047272485075	50.0386422831573\\
165.25672698209	50.0423697241207\\
165.512181750454	50.0463705408107\\
165.767636518818	50.0497910939611\\
166.023091287182	50.0526510222405\\
166.31520741254	50.0552588439669\\
166.607323537899	50.0571863495976\\
166.899439663258	50.0584604899379\\
167.266293108838	50.0591752339946\\
167.633146554419	50.0589506614714\\
168	50.0578348664867\\
168.306630626399	50.0563365007402\\
168.613261252798	50.0544378902782\\
168.919891879198	50.0521571222429\\
169.226522505597	50.0495116945681\\
169.533153131996	50.0465185177726\\
169.839783758395	50.0431939518395\\
170.228887365157	50.038522883298\\
170.617990971919	50.0333747052933\\
171.00709457868	50.0277792061269\\
171.338063052454	50.0226890237434\\
171.669031526227	50.0173127023491\\
172	50.0116664754093\\
172.222327405684	50.0078957678188\\
172.444654811368	50.004339584226\\
172.666982217051	50.000992050082\\
172.889309622735	49.9978474232132\\
173.111637028419	49.9949000933253\\
173.333964434103	49.992144576267\\
173.618310255854	49.988891416411\\
173.902656077605	49.9859323327634\\
174.187001899356	49.9832567220861\\
174.513926910676	49.980517434178\\
174.840851921996	49.9781241283637\\
175.167776933316	49.9760621630424\\
175.445184622211	49.9745618233693\\
175.722592311105	49.9732814916936\\
176	49.972213040173\\
176.31659713458	49.9713046034212\\
176.633194269159	49.9707710524441\\
176.949791403739	49.9705959086619\\
177.266388538319	49.9707632459907\\
177.582985672899	49.9712576881689\\
177.899582807478	49.97206437342\\
178.311417473493	49.9735574019187\\
178.723252139508	49.9755241953193\\
179.135086805523	49.9779357431701\\
179.423391203682	49.9798737030556\\
179.711695601841	49.982006878395\\
180	49.9843264193407\\
180.253968662071	49.9864132462242\\
180.507937324143	49.9884293153049\\
180.761905986214	49.9903765039056\\
181.015874648285	49.9922566447357\\
181.269843310356	49.9940715260611\\
181.523811972428	49.9958228938284\\
181.857959315754	49.9980332821757\\
182.192106659081	50.0001404424018\\
182.526254002408	50.0021480295681\\
182.917314027394	50.0043758419304\\
183.30837405238	50.0064775947733\\
183.699434077366	50.0084585651529\\
183.799622718244	50.0089472684404\\
183.899811359122	50.0094284617914\\
184	50.0099022278155\\
184.297575091091	50.0111889460766\\
184.595150182181	50.0122624405542\\
184.892725273272	50.0131311452604\\
185.190300364362	50.0138032349745\\
185.487875455453	50.0142866262126\\
185.785450546543	50.0145889926728\\
186.168835071391	50.0147236865247\\
186.552219596239	50.0145855454012\\
186.935604121088	50.0141892827359\\
187.290402747392	50.0136048739091\\
187.645201373696	50.0128222420861\\
188	50.0118517293133\\
188.31722660585	50.0108919821634\\
188.6344532117	50.009911197877\\
188.95167981755	50.008910813385\\
189.2689064234	50.0078922118986\\
189.58613302925	50.0068567231282\\
189.903359635101	50.0058056269251\\
190.324943908743	50.0043867070685\\
190.746528182385	50.0029451484932\\
191.168112456028	50.0014835690827\\
191.445408304019	50.0005125088914\\
191.722704152009	49.9995345384083\\
192	49.9985503091617\\
192.300862356292	49.9975370942553\\
192.601724712583	49.9966365563887\\
192.902587068875	49.9958443316361\\
193.203449425166	49.9951561907486\\
193.504311781458	49.9945680385896\\
193.805174137749	49.9940759060792\\
194.210385236881	49.9935580727159\\
194.615596336013	49.9931984103255\\
195.020807435145	49.992988251941\\
195.347204956763	49.9929220265847\\
195.673602478382	49.992943244549\\
196	49.9930478806904\\
196.456850935189	49.9933073664397\\
196.913701870379	49.9936747981873\\
197.370552805568	49.9941429220798\\
197.827403740758	49.9947048375623\\
198.284254675947	49.9953539960359\\
198.741105611137	49.9960841673222\\
199.160737074091	49.9968211528602\\
199.580368537046	49.9976171307471\\
200	49.9984679404915\\
};
\end{axis}
\end{tikzpicture}%}
  \caption{Step response using a zero-order hold of sample time $4$ sec.}
  \label{fig:Q7.4}
\end{figure}

\begin{figure}[H]\centering
	\centering
	\scalebox{1}{\input{./images/7/zoh_5.tex}}
  \caption{Step response using a zero-order hold of sample time $5$ sec.}
  \label{fig:Q7.5}
\end{figure}

\begin{figure}[H]\centering
	\centering
	\scalebox{1}{\input{./images/7/zoh_6.tex}}
  \caption{Step response using a zero-order hold of sample time $6$ sec.}
  \label{fig:Q7.6}
\end{figure}

\begin{figure}[H]\centering
	\centering
	\scalebox{1}{\input{./images/7/zoh_7.tex}}
  \caption{Step response using a zero-order hold of sample time $7$ sec.}
  \label{fig:Q7.7}
\end{figure}

\begin{figure}[H]\centering
	\centering
	\scalebox{1}{\input{./images/7/zoh_8.tex}}
  \caption{Step response using a zero-order hold of sample time $8$ sec.}
  \label{fig:Q7.8}
\end{figure}
