\section{Question 7}

Figure \ref{fig:Q7.rloc_F_G} shows the root locus of the open-loop transfer
function without the use of a zero-order hold, while figure
\ref{fig:Q7.rloc_F_ZOH_G} shows exacly the same, but with the addition of a
zero-order hold between the controller and the process, with sampling time of
$h=1$ sec. It is apparent that without the zero-order hold, the system is stable,
since all poles have negative real values. However, the time delay and the pole
at zero that the zero-order hold introduces deliver a reduction in the degree of
stability.

\begin{figure}[H]\centering
	\centering
	\scalebox{1}{% This file was created by matlab2tikz.
%
%The latest updates can be retrieved from
%  http://www.mathworks.com/matlabcentral/fileexchange/22022-matlab2tikz-matlab2tikz
%where you can also make suggestions and rate matlab2tikz.
%
\definecolor{mycolor1}{rgb}{0.00000,0.44700,0.74100}%
\definecolor{mycolor2}{rgb}{0.00000,0.75000,0.75000}%
%
\begin{tikzpicture}

\begin{axis}[%
width=4.008in,
height=3.052in,
at={(0.818in,0.44in)},
scale only axis,
unbounded coords=jump,
separate axis lines,
every outer x axis line/.append style={white!40!black},
every x tick label/.append style={font=\color{white!40!black}},
xmin=-1.2,
xmax=0.2,
every outer y axis line/.append style={white!40!black},
every y tick label/.append style={font=\color{white!40!black}},
ymin=-1,
ymax=1,
axis background/.style={fill=white}
]
\addplot [color=white,solid,forget plot]
  table[row sep=crcr]{%
0	0\\
};
\addplot [color=mycolor1,only marks,mark=o,mark options={solid},forget plot]
  table[row sep=crcr]{%
-0.134875180971928	0.0760690712223554\\
-0.134875180971928	-0.0760690712223554\\
};
\addplot [color=mycolor1,only marks,mark=x,mark options={solid},forget plot]
  table[row sep=crcr]{%
0	0\\
-1.15912180370184	0\\
-0.0804391007994334	0\\
-0.0804390954987234	0\\
};
\addplot [color=blue,solid,forget plot]
  table[row sep=crcr]{%
0	0\\
-0.00172258141172093	0\\
-0.00212735127812858	0\\
-0.00263131770774314	0\\
-0.00326121735506603	0\\
-0.00405257004497591	0\\
-0.00505372231013738	0\\
-0.00633275328735025	0\\
-0.00799037693414231	0\\
-0.0101872515208663	0\\
-0.0132136773986536	0\\
-0.0177357655010795	0\\
-0.026965149880164	0\\
-0.031861170880117	0\\
-0.0329210964874718	0\\
-0.0329191812447814	0.00106171833401894\\
-0.0326002115036228	0.0144153917668627\\
-0.0322532575366739	0.0222133400181056\\
-0.0319619413767979	0.0287786677028206\\
-0.0317503952262301	0.0349437933283371\\
-0.0316490597307499	0.0410187391041332\\
-0.0316962291519887	0.0471699250742801\\
-0.031940010304282	0.0535065627621177\\
-0.0324408279090809	0.0601100135750878\\
-0.033274662303877	0.0670456186005314\\
-0.0345372857097423	0.0743673458050051\\
-0.0363498872538261	0.0821184817209548\\
-0.0388666735863987	0.0903291485867909\\
-0.0422853521405217	0.099010100663705\\
-0.0468619379890806	0.108140874273028\\
-0.0529322272301958	0.117648003902467\\
-0.0609437701942948	0.127363885294016\\
-0.0715042606053031	0.13694405539213\\
-0.0783478676713363	0.141710946309434\\
-0.0854522168936597	0.145684115756799\\
-0.0945139963054132	0.14948467190357\\
-0.103927241473383	0.152055633108492\\
-0.115887690813282	0.153366226022844\\
-0.121981270592794	0.153161829061482\\
-0.128092091010695	0.152307438983691\\
-0.135502275953594	0.150266973358649\\
-0.142593352928631	0.1470124968458\\
-0.149000509461001	0.142473795143078\\
-0.154225256949533	0.136771887844467\\
-0.158588224713188	0.128207122017414\\
-0.159995472464146	0.120133925111416\\
-0.16	0.12\\
-0.160003931238963	0.11986661611598\\
-0.158775086647782	0.11003569436446\\
-0.155620608994788	0.101890815765771\\
-0.152848362285918	0.0969891014951801\\
-0.148240278725404	0.0904996916159923\\
-0.145010907650277	0.0866083592419506\\
-0.14269208086119	0.0840206196541003\\
-0.140980065689828	0.0821926208248014\\
-0.139687861557493	0.0808507795728236\\
-0.138696039618838	0.0798399007291188\\
-0.137925014775366	0.079064249399346\\
-0.137319748803138	0.0784610606268747\\
-0.136840995561178	0.0779872634890172\\
-0.13646006354647	0.0776122501906733\\
-0.13615554897753	0.0773136698302359\\
-0.135911220601803	0.0770748475944081\\
-0.135714604605612	0.0768831298834395\\
-0.135556010412003	0.0767287831820409\\
-0.135427843048246	0.0766042380585054\\
-0.134876554458581	0.0760703975591964\\
-0.134875180971928	0.0760690712223554\\
};
\addplot [color=mycolor2,solid,forget plot]
  table[row sep=crcr]{%
-0.0804390954987234	0\\
-0.0723951907194905	0\\
-0.0714129432946076	0\\
-0.0702910149341413	0\\
-0.0690026907496458	0\\
-0.0675133206154549	0\\
-0.0657765240062594	0\\
-0.0637275885730999	0\\
-0.061270940685456	0\\
-0.0582532920812395	0\\
-0.0543945004335206	0\\
-0.0490420849527246	0\\
-0.0390023534640328	0\\
-0.0339848604910167	0\\
-0.0329210979667503	0\\
-0.0329191812447814	-0.00106171833401894\\
-0.0326002115036228	-0.0144153917668627\\
-0.0322532575366739	-0.0222133400181056\\
-0.0319619413767979	-0.0287786677028206\\
-0.0317503952262301	-0.0349437933283371\\
-0.0316490597307499	-0.0410187391041332\\
-0.0316962291519887	-0.0471699250742801\\
-0.031940010304282	-0.0535065627621177\\
-0.0324408279090809	-0.0601100135750878\\
-0.033274662303877	-0.0670456186005314\\
-0.0345372857097423	-0.0743673458050051\\
-0.0363498872538261	-0.0821184817209548\\
-0.0388666735863987	-0.0903291485867909\\
-0.0422853521405217	-0.099010100663705\\
-0.0468619379890806	-0.108140874273028\\
-0.0529322272301958	-0.117648003902467\\
-0.0609437701942948	-0.127363885294016\\
-0.0715042606053031	-0.13694405539213\\
-0.0783478676713363	-0.141710946309434\\
-0.0854522168936597	-0.145684115756799\\
-0.0945139963054132	-0.14948467190357\\
-0.103927241473383	-0.152055633108492\\
-0.115887690813282	-0.153366226022844\\
-0.121981270592794	-0.153161829061482\\
-0.128092091010695	-0.152307438983691\\
-0.135502275953594	-0.150266973358649\\
-0.142593352928631	-0.1470124968458\\
-0.149000509461001	-0.142473795143078\\
-0.154225256949533	-0.136771887844467\\
-0.158588224713188	-0.128207122017414\\
-0.159995472464146	-0.120133925111416\\
-0.16	-0.12\\
-0.160003931238963	-0.11986661611598\\
-0.158775086647782	-0.11003569436446\\
-0.155620608994788	-0.101890815765771\\
-0.152848362285918	-0.0969891014951801\\
-0.148240278725404	-0.0904996916159923\\
-0.145010907650277	-0.0866083592419506\\
-0.14269208086119	-0.0840206196541003\\
-0.140980065689828	-0.0821926208248014\\
-0.139687861557493	-0.0808507795728236\\
-0.138696039618838	-0.0798399007291188\\
-0.137925014775366	-0.079064249399346\\
-0.137319748803138	-0.0784610606268747\\
-0.136840995561178	-0.0779872634890172\\
-0.13646006354647	-0.0776122501906733\\
-0.13615554897753	-0.0773136698302359\\
-0.135911220601803	-0.0770748475944081\\
-0.135714604605612	-0.0768831298834395\\
-0.135556010412003	-0.0767287831820409\\
-0.135427843048246	-0.0766042380585054\\
-0.134876554458581	-0.0760703975591964\\
-0.134875180971928	-0.0760690712223554\\
};
\addplot [color=red,solid,forget plot]
  table[row sep=crcr]{%
-0.0804391007994334	0\\
-0.0871714308139377	0\\
-0.0878424493963486	0\\
-0.0885752515881706	0\\
-0.0893746706407535	0\\
-0.0902458011796924	0\\
-0.0911940056346637	0\\
-0.0922249237841842	0\\
-0.0933444873355378	0\\
-0.0945589420922342	0\\
-0.0958748810682332	0\\
-0.0972992929541448	0\\
-0.098839631723794	0\\
-0.0990873538657956	0\\
-0.0990952628791606	0\\
-0.0991031669661326	0\\
-0.100503915040895	0\\
-0.102300861719523	0\\
-0.104240082177677	0\\
-0.106332341190221	0\\
-0.1085899202569	0\\
-0.111027119154807	0\\
-0.113660955484353	0\\
-0.116512152078757	0\\
-0.119606553851946	0\\
-0.122977204738043	0\\
-0.126667474976819	0\\
-0.130735928223321	0\\
-0.135264209614481	0\\
-0.140370482406512	0\\
-0.146233777005586	0\\
-0.153141716997292	0\\
-0.161594175548456	0\\
-0.166949896198557	0\\
-0.172562276926219	0\\
-0.179986616189226	0\\
-0.188281700676067	0\\
-0.200298592453817	0\\
-0.207422830374181	0\\
-0.215606650038428	0\\
-0.227641386353211	0\\
-0.242830362304848	0\\
-0.262768087997266	0\\
-0.28996366787719	0\\
-0.345167432953542	0\\
-0.478882659159308	0\\
-0.5	1.83497017401247e-08\\
-0.499996068761037	0.0211275473195168\\
-0.501224913352219	0.203031956385616\\
-0.504379391005213	0.311290201465268\\
-0.507151637714082	0.389356648767733\\
-0.511759721274597	0.532189945695628\\
-0.514989092349724	0.665058793572132\\
-0.51730791913881	0.797486665596829\\
-0.519019934310172	0.934163041764775\\
-0.520312138442508	1.07815796746604\\
-0.521303960381161	1.23189193181622\\
-0.522074985224634	1.39752275865309\\
-0.522680251196862	1.57713196961159\\
-0.523159004438822	1.77282939436154\\
-0.523539936453529	1.98682063109345\\
-0.523844451022471	2.22145655067634\\
-0.524088779398197	2.47927403302538\\
-0.524285395394388	2.76303275379165\\
-0.524443989587997	3.07575077282271\\
-0.524572156951753	3.42074063839496\\
-0.52512344554142	69.2117540067618\\
inf	0\\
};
\addplot [color=black!50!green,solid,forget plot]
  table[row sep=crcr]{%
-1.15912180370184	0\\
-1.15871079705485	0\\
-1.15861725603092	0\\
-1.15850241576994	0\\
-1.15836142125453	0\\
-1.15818830815988	0\\
-1.15797574804894	0\\
-1.15771473435537	0\\
-1.15739419504487	0\\
-1.15700051430566	0\\
-1.15651694109959	0\\
-1.15592285659205	0\\
-1.15519286493201	0\\
-1.15506661476307	0\\
-1.15506254266662	0\\
-1.1550584705443	0\\
-1.15429566195186	0\\
-1.15319262320713	0\\
-1.15183603506873	0\\
-1.15016686835732	0\\
-1.1481119602816	0\\
-1.14558042254122	0\\
-1.14245902390708	0\\
-1.13860619210308	0\\
-1.1338441215403	0\\
-1.12794822384247	0\\
-1.12063275051553	0\\
-1.11153072460388	0\\
-1.10016508610448	0\\
-1.08590564161533	0\\
-1.06790176853402	0\\
-1.04497074261412	0\\
-1.01539730324094	0\\
-0.996354368458771	0\\
-0.976533289286463	0\\
-0.950985391199949	0\\
-0.923863816377166	0\\
-0.887926025919618	0\\
-0.868614628440231	0\\
-0.848209167940182	0\\
-0.8213540617396	0\\
-0.791982931837891	0\\
-0.759230893080733	0\\
-0.721585818223746	0\\
-0.657656117620083	0\\
-0.521126395912401	0\\
-0.5	-1.83497017401247e-08\\
-0.499996068761037	-0.0211275473195168\\
-0.501224913352219	-0.203031956385616\\
-0.504379391005213	-0.311290201465268\\
-0.507151637714082	-0.389356648767733\\
-0.511759721274597	-0.532189945695628\\
-0.514989092349724	-0.665058793572132\\
-0.51730791913881	-0.797486665596829\\
-0.519019934310172	-0.934163041764775\\
-0.520312138442508	-1.07815796746604\\
-0.521303960381161	-1.23189193181622\\
-0.522074985224634	-1.39752275865309\\
-0.522680251196862	-1.57713196961159\\
-0.523159004438822	-1.77282939436154\\
-0.523539936453529	-1.98682063109345\\
-0.523844451022471	-2.22145655067634\\
-0.524088779398197	-2.47927403302538\\
-0.524285395394388	-2.76303275379165\\
-0.524443989587997	-3.07575077282271\\
-0.524572156951753	-3.42074063839496\\
-0.52512344554142	-69.2117540067618\\
inf	0\\
};
\addplot [color=white!40!black,dotted,forget plot]
  table[row sep=crcr]{%
0	0\\
-0	1.6\\
nan	nan\\
0	0\\
-0.192	1.58843822668683\\
nan	nan\\
0	0\\
-0.384	1.55323662073748\\
nan	nan\\
0	0\\
-0.608	1.47997837822044\\
nan	nan\\
0	0\\
-0.8	1.3856406460551\\
nan	nan\\
0	0\\
-1.024	1.22939985358711\\
nan	nan\\
0	0\\
-1.216	1.0398769157934\\
nan	nan\\
0	0\\
-1.408	0.759957893570427\\
nan	nan\\
0	0\\
-1.552	0.388967864996583\\
nan	nan\\
0	0\\
-1.6	0\\
nan	nan\\
0	-0\\
-0	-1.6\\
nan	nan\\
0	-0\\
-0.192	-1.58843822668683\\
nan	nan\\
0	-0\\
-0.384	-1.55323662073748\\
nan	nan\\
0	-0\\
-0.608	-1.47997837822044\\
nan	nan\\
0	-0\\
-0.8	-1.3856406460551\\
nan	nan\\
0	-0\\
-1.024	-1.22939985358711\\
nan	nan\\
0	-0\\
-1.216	-1.0398769157934\\
nan	nan\\
0	-0\\
-1.408	-0.759957893570427\\
nan	nan\\
0	-0\\
-1.552	-0.388967864996583\\
nan	nan\\
0	-0\\
-1.6	-0\\
nan	nan\\
};
\addplot [color=white!40!black,dotted,forget plot]
  table[row sep=crcr]{%
-0	0\\
-0	0\\
-0	0\\
-0	0\\
-0	0\\
-0	0\\
-0	0\\
-0	0\\
-0	0\\
-0	0\\
-0	0\\
-0	0\\
-0	0\\
-0	0\\
-0	0\\
-0	0\\
-0	0\\
-0	0\\
-0	0\\
-0	0\\
-0	0\\
-0	0\\
-0	0\\
-0	0\\
-0	0\\
-0	0\\
-0	0\\
-0	0\\
-0	0\\
-0	0\\
-0	0\\
-0	0\\
-0	0\\
-0	0\\
-0	0\\
-0	0\\
-0	0\\
-0	0\\
-0	0\\
-0	0\\
-0	0\\
nan	nan\\
-0	0.2\\
-0.0054985357753655	0.199924400972785\\
-0.0110015565231823	0.199697185143074\\
-0.0165135128595992	0.199317093830499\\
-0.0220387854178422	0.198782021162142\\
-0.0275816466568833	0.198089002137158\\
-0.0331462187635179	0.19723419627864\\
-0.0387364262364926	0.196212867270281\\
-0.0443559416523546	0.19501935914194\\
-0.0500081230081741	0.193647069776951\\
-0.0556959409222359	0.192088422776561\\
-0.0614218938590815	0.190334839046255\\
-0.0671879094426594	0.188376709878703\\
-0.0729952298487411	0.186203373812961\\
-0.0788442792497768	0.183803100163145\\
-0.0847345113550247	0.181163082843678\\
-0.0906642352893107	0.17826944897936\\
-0.0966304184403005	0.175107287775388\\
-0.10262846554481	0.171660706219909\\
-0.108651974260549	0.167912919363833\\
-0.114692468872666	0.163846384103808\\
-0.120739115711546	0.159442986478533\\
-0.126778426401034	0.154684293317963\\
-0.132793958282688	0.149551879438594\\
-0.138766025290247	0.144027741165188\\
-0.144671437110985	0.138094805420922\\
-0.150483289467329	0.131737540553527\\
-0.156170833400534	0.12494267003303\\
-0.161699455927761	0.117699982806567\\
-0.167030807509977	0.110003224237133\\
-0.17212311231062	0.101851039309895\\
-0.176931693997889	0.0932479257626533\\
-0.181409741640536	0.0842051402107613\\
-0.185509326192266	0.0747414871118537\\
-0.189182657981152	0.0648839110965614\\
-0.192383550434224	0.054667810659679\\
-0.19506902725479	0.0441369981520023\\
-0.197200983091473	0.0333432492081472\\
-0.198747786053174	0.022345414271427\\
-0.199685699074389	0.0112081035493346\\
-0.2	0\\
nan	nan\\
-0	0.4\\
-0.010997071550731	0.399848801945571\\
-0.0220031130463645	0.399394370286148\\
-0.0330270257191984	0.398634187660998\\
-0.0440775708356844	0.397564042324284\\
-0.0551632933137667	0.396178004274315\\
-0.0662924375270359	0.39446839255728\\
-0.0774728524729852	0.392425734540561\\
-0.0887118833047092	0.39003871828388\\
-0.100016246016348	0.387294139553902\\
-0.111391881844472	0.384176845553122\\
-0.122843787718163	0.38066967809251\\
-0.134375818885319	0.376753419757406\\
-0.145990459697482	0.372406747625923\\
-0.157688558499554	0.367606200326291\\
-0.169469022710049	0.362326165687355\\
-0.181328470578621	0.356538897958719\\
-0.193260836880601	0.350214575550775\\
-0.205256931089619	0.343321412439817\\
-0.217303948521098	0.335825838727666\\
-0.229384937745331	0.327692768207616\\
-0.241478231423093	0.318885972957067\\
-0.253556852802067	0.309368586635927\\
-0.265587916565376	0.299103758877188\\
-0.277532050580494	0.288055482330377\\
-0.28934287422197	0.276189610841844\\
-0.300966578934658	0.263475081107053\\
-0.312341666801068	0.249885340066061\\
-0.323398911855522	0.235399965613134\\
-0.334061615019954	0.220006448474266\\
-0.34424622462124	0.203702078619789\\
-0.353863387995778	0.186495851525307\\
-0.362819483281072	0.168410280421523\\
-0.371018652384532	0.149482974223707\\
-0.378365315962304	0.129767822193123\\
-0.384767100868447	0.109335621319358\\
-0.390138054509581	0.0882739963040046\\
-0.394401966182946	0.0666864984162943\\
-0.397495572106348	0.044690828542854\\
-0.399371398148778	0.0224162070986691\\
-0.4	0\\
nan	nan\\
-0	0.6\\
-0.0164956073260965	0.599773202918356\\
-0.0330046695695468	0.599091555429222\\
-0.0495405385787977	0.597951281491496\\
-0.0661163562535266	0.596346063486426\\
-0.08274493997065	0.594267006411473\\
-0.0994386562905538	0.59170258883592\\
-0.116209278709478	0.588638601810842\\
-0.133067824957064	0.58505807742582\\
-0.150024369024522	0.580941209330853\\
-0.167087822766708	0.576265268329683\\
-0.184265681577245	0.571004517138765\\
-0.201563728327978	0.565130129636109\\
-0.218985689546223	0.558610121438884\\
-0.236532837749331	0.551409300489436\\
-0.254203534065074	0.543489248531033\\
-0.271992705867932	0.534808346938079\\
-0.289891255320901	0.525321863326163\\
-0.307885396634429	0.514982118659726\\
-0.325955922781647	0.503738758091499\\
-0.344077406617997	0.491539152311424\\
-0.362217347134639	0.478328959435601\\
-0.380335279203101	0.46405287995389\\
-0.398381874848064	0.448655638315782\\
-0.416298075870741	0.432083223495565\\
-0.434014311332955	0.414284416262766\\
-0.451449868401987	0.39521262166058\\
-0.468512500201602	0.374828010099091\\
-0.485098367783282	0.353099948419701\\
-0.501092422529931	0.3300096727114\\
-0.51636933693186	0.305553117929684\\
-0.530795081993668	0.27974377728796\\
-0.544229224921608	0.252615420632284\\
-0.556527978576798	0.224224461335561\\
-0.567547973943456	0.194651733289684\\
-0.577150651302671	0.164003431979037\\
-0.585207081764371	0.132410994456007\\
-0.591602949274419	0.100029747624441\\
-0.596243358159523	0.0670362428142811\\
-0.599057097223167	0.0336243106480037\\
-0.6	0\\
nan	nan\\
-0	0.8\\
-0.021994143101462	0.799697603891141\\
-0.044006226092729	0.798788740572297\\
-0.0660540514383969	0.797268375321995\\
-0.0881551416713688	0.795128084648568\\
-0.110326586627533	0.792356008548631\\
-0.132584875054072	0.788936785114559\\
-0.15494570494597	0.784851469081123\\
-0.177423766609418	0.780077436567759\\
-0.200032492032696	0.774588279107804\\
-0.222783763688944	0.768353691106244\\
-0.245687575436326	0.761339356185019\\
-0.268751637770638	0.753506839514812\\
-0.291980919394964	0.744813495251846\\
-0.315377116999107	0.735212400652581\\
-0.338938045420099	0.72465233137471\\
-0.362656941157243	0.713077795917439\\
-0.386521673761202	0.700429151101551\\
-0.410513862179239	0.686642824879635\\
-0.434607897042195	0.671651677455331\\
-0.458769875490663	0.655385536415232\\
-0.482956462846185	0.637771945914134\\
-0.507113705604134	0.618737173271854\\
-0.531175833130752	0.598207517754376\\
-0.555064101160988	0.576110964660754\\
-0.57868574844394	0.552379221683688\\
-0.601933157869316	0.526950162214107\\
-0.624683333602136	0.499770680132121\\
-0.646797823711043	0.470799931226268\\
-0.668123230039908	0.440012896948533\\
-0.68849244924248	0.407404157239578\\
-0.707726775991557	0.372991703050613\\
-0.725638966562144	0.336820560843045\\
-0.742037304769065	0.298965948447415\\
-0.756730631924608	0.259535644386246\\
-0.769534201736895	0.218671242638716\\
-0.780276109019161	0.176547992608009\\
-0.788803932365892	0.133372996832589\\
-0.794991144212697	0.0893816570857081\\
-0.798742796297556	0.0448324141973383\\
-0.8	0\\
nan	nan\\
-0	1\\
-0.0274926788768275	0.999622004863926\\
-0.0550077826159112	0.998485925715371\\
-0.0825675642979961	0.996585469152494\\
-0.110193927089211	0.99391010581071\\
-0.137908233284417	0.990445010685788\\
-0.16573109381759	0.986170981393199\\
-0.193682131182463	0.981064336351403\\
-0.221779708261773	0.975096795709699\\
-0.25004061504087	0.968235348884755\\
-0.27847970461118	0.960442113882805\\
-0.307109469295408	0.951674195231274\\
-0.335939547213297	0.941883549393514\\
-0.364976149243705	0.931016869064807\\
-0.394221396248884	0.919015500815726\\
-0.423672556775123	0.905815414218388\\
-0.453321176446553	0.891347244896798\\
-0.483152092201502	0.875536438876938\\
-0.513142327724049	0.858303531099544\\
-0.543259871302744	0.839564596819164\\
-0.573462344363328	0.81923192051904\\
-0.603695578557732	0.797214932392667\\
-0.633892132005168	0.773421466589817\\
-0.66396979141344	0.74775939719297\\
-0.693830126451235	0.720138705825942\\
-0.723357185554925	0.69047402710461\\
-0.752416447336645	0.658687702767634\\
-0.78085416700267	0.624713350165151\\
-0.808497279638804	0.588499914032835\\
-0.835154037549885	0.550016121185666\\
-0.8606155615531	0.509255196549473\\
-0.884658469989446	0.466239628813266\\
-0.90704870820268	0.421025701053807\\
-0.927546630961331	0.373707435559269\\
-0.94591328990576	0.324419555482807\\
-0.961917752171118	0.273339053298395\\
-0.975345136273952	0.220684990760011\\
-0.986004915457365	0.166716246040736\\
-0.993738930265871	0.111727071357135\\
-0.998428495371944	0.0560405177466728\\
-1	0\\
nan	nan\\
-0	1.2\\
-0.032991214652193	1.19954640583671\\
-0.0660093391390935	1.19818311085844\\
-0.0990810771575953	1.19590256298299\\
-0.132232712507053	1.19269212697285\\
-0.1654898799413	1.18853401282295\\
-0.198877312581108	1.18340517767184\\
-0.232418557418956	1.17727720362168\\
-0.266135649914128	1.17011615485164\\
-0.300048738049045	1.16188241866171\\
-0.334175645533416	1.15253053665937\\
-0.368531363154489	1.14200903427753\\
-0.403127456655956	1.13026025927222\\
-0.437971379092446	1.11722024287777\\
-0.473065675498661	1.10281860097887\\
-0.508407068130148	1.08697849706207\\
-0.543985411735864	1.06961669387616\\
-0.579782510641803	1.05064372665233\\
-0.615770793268858	1.02996423731945\\
-0.651911845563293	1.007477516183\\
-0.688154813235994	0.983078304622849\\
-0.724434694269278	0.956657918871201\\
-0.760670558406201	0.92810575990778\\
-0.796763749696128	0.897311276631564\\
-0.832596151741482	0.864166446991131\\
-0.86802862266591	0.828568832525532\\
-0.902899736803974	0.79042524332116\\
-0.937025000403204	0.749656020198182\\
-0.970196735566565	0.706199896839402\\
-1.00218484505986	0.660019345422799\\
-1.03273867386372	0.611106235859368\\
-1.06159016398734	0.55948755457592\\
-1.08845844984322	0.505230841264568\\
-1.1130559571536	0.448448922671122\\
-1.13509594788691	0.389303466579369\\
-1.15430130260534	0.328006863958074\\
-1.17041416352874	0.264821988912014\\
-1.18320589854884	0.200059495248883\\
-1.19248671631905	0.134072485628562\\
-1.19811419444633	0.0672486212960074\\
-1.2	0\\
nan	nan\\
-0	1.4\\
-0.0384897504275585	1.3994708068095\\
-0.0770108956622757	1.39788029600152\\
-0.115594590017195	1.39521965681349\\
-0.154271497924896	1.39147414813499\\
-0.193071526598183	1.3866230149601\\
-0.232023531344626	1.38063937395048\\
-0.271154983655448	1.37349007089196\\
-0.310491591566482	1.36513551399358\\
-0.350056861057219	1.35552948843866\\
-0.389871586455652	1.34461895943593\\
-0.429953257013571	1.33234387332378\\
-0.470315366098616	1.31863696915092\\
-0.510966608941187	1.30342361669073\\
-0.551909954748438	1.28662170114202\\
-0.593141579485173	1.26814157990574\\
-0.634649647025175	1.24788614285552\\
-0.676412929082103	1.22575101442771\\
-0.718399258813668	1.20162494353936\\
-0.760563819823842	1.17539043554683\\
-0.80284728210866	1.14692468872666\\
-0.845173809980825	1.11610090534973\\
-0.887448984807235	1.08279005322574\\
-0.929557707978816	1.04686315607016\\
-0.97136217703173	1.00819418815632\\
-1.0127000597769	0.966663637946454\\
-1.0533830262713	0.922162783874687\\
-1.09319583380374	0.874598690231212\\
-1.13189619149433	0.823899879645969\\
-1.16921565256984	0.770022569659933\\
-1.20486178617434	0.712957275169262\\
-1.23852185798522	0.652735480338573\\
-1.26986819148375	0.589435981475329\\
-1.29856528334586	0.523190409782976\\
-1.32427860586806	0.45418737767593\\
-1.34668485303957	0.382674674617753\\
-1.36548319078353	0.308958987064016\\
-1.38040688164031	0.23340274445703\\
-1.39123450237222	0.156417899899989\\
-1.39779989352072	0.078456724845342\\
-1.4	0\\
nan	nan\\
-0	1.6\\
-0.043988286202924	1.59939520778228\\
-0.088012452185458	1.59757748114459\\
-0.132108102876794	1.59453675064399\\
-0.176310283342738	1.59025616929714\\
-0.220653173255067	1.58471201709726\\
-0.265169750108143	1.57787357022912\\
-0.309891409891941	1.56970293816225\\
-0.354847533218837	1.56015487313552\\
-0.400064984065393	1.54917655821561\\
-0.445567527377887	1.53670738221249\\
-0.491375150872652	1.52267871237004\\
-0.537503275541275	1.50701367902962\\
-0.583961838789928	1.48962699050369\\
-0.630754233998215	1.47042480130516\\
-0.677876090840198	1.44930466274942\\
-0.725313882314485	1.42615559183488\\
-0.773043347522404	1.4008583022031\\
-0.821027724358478	1.37328564975927\\
-0.869215794084391	1.34330335491066\\
-0.917539750981326	1.31077107283046\\
-0.965912925692371	1.27554389182827\\
-1.01422741120827	1.23747434654371\\
-1.0623516662615	1.19641503550875\\
-1.11012820232198	1.15222192932151\\
-1.15737149688788	1.10475844336738\\
-1.20386631573863	1.05390032442821\\
-1.24936666720427	0.999541360264242\\
-1.29359564742209	0.941599862452536\\
-1.33624646007982	0.880025793897066\\
-1.37698489848496	0.814808314479157\\
-1.41545355198311	0.745983406101226\\
-1.45127793312429	0.673641121686091\\
-1.48407460953813	0.59793189689483\\
-1.51346126384922	0.519071288772491\\
-1.53906840347379	0.437342485277432\\
-1.56055221803832	0.353095985216018\\
-1.57760786473178	0.266745993665177\\
-1.58998228842539	0.178763314171416\\
-1.59748559259511	0.0896648283946765\\
-1.6	0\\
nan	nan\\
-0	-0\\
-0	-0\\
-0	-0\\
-0	-0\\
-0	-0\\
-0	-0\\
-0	-0\\
-0	-0\\
-0	-0\\
-0	-0\\
-0	-0\\
-0	-0\\
-0	-0\\
-0	-0\\
-0	-0\\
-0	-0\\
-0	-0\\
-0	-0\\
-0	-0\\
-0	-0\\
-0	-0\\
-0	-0\\
-0	-0\\
-0	-0\\
-0	-0\\
-0	-0\\
-0	-0\\
-0	-0\\
-0	-0\\
-0	-0\\
-0	-0\\
-0	-0\\
-0	-0\\
-0	-0\\
-0	-0\\
-0	-0\\
-0	-0\\
-0	-0\\
-0	-0\\
-0	-0\\
-0	-0\\
nan	nan\\
-0	-0.2\\
-0.0054985357753655	-0.199924400972785\\
-0.0110015565231823	-0.199697185143074\\
-0.0165135128595992	-0.199317093830499\\
-0.0220387854178422	-0.198782021162142\\
-0.0275816466568833	-0.198089002137158\\
-0.0331462187635179	-0.19723419627864\\
-0.0387364262364926	-0.196212867270281\\
-0.0443559416523546	-0.19501935914194\\
-0.0500081230081741	-0.193647069776951\\
-0.0556959409222359	-0.192088422776561\\
-0.0614218938590815	-0.190334839046255\\
-0.0671879094426594	-0.188376709878703\\
-0.0729952298487411	-0.186203373812961\\
-0.0788442792497768	-0.183803100163145\\
-0.0847345113550247	-0.181163082843678\\
-0.0906642352893107	-0.17826944897936\\
-0.0966304184403005	-0.175107287775388\\
-0.10262846554481	-0.171660706219909\\
-0.108651974260549	-0.167912919363833\\
-0.114692468872666	-0.163846384103808\\
-0.120739115711546	-0.159442986478533\\
-0.126778426401034	-0.154684293317963\\
-0.132793958282688	-0.149551879438594\\
-0.138766025290247	-0.144027741165188\\
-0.144671437110985	-0.138094805420922\\
-0.150483289467329	-0.131737540553527\\
-0.156170833400534	-0.12494267003303\\
-0.161699455927761	-0.117699982806567\\
-0.167030807509977	-0.110003224237133\\
-0.17212311231062	-0.101851039309895\\
-0.176931693997889	-0.0932479257626533\\
-0.181409741640536	-0.0842051402107613\\
-0.185509326192266	-0.0747414871118537\\
-0.189182657981152	-0.0648839110965614\\
-0.192383550434224	-0.054667810659679\\
-0.19506902725479	-0.0441369981520023\\
-0.197200983091473	-0.0333432492081472\\
-0.198747786053174	-0.022345414271427\\
-0.199685699074389	-0.0112081035493346\\
-0.2	-0\\
nan	nan\\
-0	-0.4\\
-0.010997071550731	-0.399848801945571\\
-0.0220031130463645	-0.399394370286148\\
-0.0330270257191984	-0.398634187660998\\
-0.0440775708356844	-0.397564042324284\\
-0.0551632933137667	-0.396178004274315\\
-0.0662924375270359	-0.39446839255728\\
-0.0774728524729852	-0.392425734540561\\
-0.0887118833047092	-0.39003871828388\\
-0.100016246016348	-0.387294139553902\\
-0.111391881844472	-0.384176845553122\\
-0.122843787718163	-0.38066967809251\\
-0.134375818885319	-0.376753419757406\\
-0.145990459697482	-0.372406747625923\\
-0.157688558499554	-0.367606200326291\\
-0.169469022710049	-0.362326165687355\\
-0.181328470578621	-0.356538897958719\\
-0.193260836880601	-0.350214575550775\\
-0.205256931089619	-0.343321412439817\\
-0.217303948521098	-0.335825838727666\\
-0.229384937745331	-0.327692768207616\\
-0.241478231423093	-0.318885972957067\\
-0.253556852802067	-0.309368586635927\\
-0.265587916565376	-0.299103758877188\\
-0.277532050580494	-0.288055482330377\\
-0.28934287422197	-0.276189610841844\\
-0.300966578934658	-0.263475081107053\\
-0.312341666801068	-0.249885340066061\\
-0.323398911855522	-0.235399965613134\\
-0.334061615019954	-0.220006448474266\\
-0.34424622462124	-0.203702078619789\\
-0.353863387995778	-0.186495851525307\\
-0.362819483281072	-0.168410280421523\\
-0.371018652384532	-0.149482974223707\\
-0.378365315962304	-0.129767822193123\\
-0.384767100868447	-0.109335621319358\\
-0.390138054509581	-0.0882739963040046\\
-0.394401966182946	-0.0666864984162943\\
-0.397495572106348	-0.044690828542854\\
-0.399371398148778	-0.0224162070986691\\
-0.4	-0\\
nan	nan\\
-0	-0.6\\
-0.0164956073260965	-0.599773202918356\\
-0.0330046695695468	-0.599091555429222\\
-0.0495405385787977	-0.597951281491496\\
-0.0661163562535266	-0.596346063486426\\
-0.08274493997065	-0.594267006411473\\
-0.0994386562905538	-0.59170258883592\\
-0.116209278709478	-0.588638601810842\\
-0.133067824957064	-0.58505807742582\\
-0.150024369024522	-0.580941209330853\\
-0.167087822766708	-0.576265268329683\\
-0.184265681577245	-0.571004517138765\\
-0.201563728327978	-0.565130129636109\\
-0.218985689546223	-0.558610121438884\\
-0.236532837749331	-0.551409300489436\\
-0.254203534065074	-0.543489248531033\\
-0.271992705867932	-0.534808346938079\\
-0.289891255320901	-0.525321863326163\\
-0.307885396634429	-0.514982118659726\\
-0.325955922781647	-0.503738758091499\\
-0.344077406617997	-0.491539152311424\\
-0.362217347134639	-0.478328959435601\\
-0.380335279203101	-0.46405287995389\\
-0.398381874848064	-0.448655638315782\\
-0.416298075870741	-0.432083223495565\\
-0.434014311332955	-0.414284416262766\\
-0.451449868401987	-0.39521262166058\\
-0.468512500201602	-0.374828010099091\\
-0.485098367783282	-0.353099948419701\\
-0.501092422529931	-0.3300096727114\\
-0.51636933693186	-0.305553117929684\\
-0.530795081993668	-0.27974377728796\\
-0.544229224921608	-0.252615420632284\\
-0.556527978576798	-0.224224461335561\\
-0.567547973943456	-0.194651733289684\\
-0.577150651302671	-0.164003431979037\\
-0.585207081764371	-0.132410994456007\\
-0.591602949274419	-0.100029747624441\\
-0.596243358159523	-0.0670362428142811\\
-0.599057097223167	-0.0336243106480037\\
-0.6	-0\\
nan	nan\\
-0	-0.8\\
-0.021994143101462	-0.799697603891141\\
-0.044006226092729	-0.798788740572297\\
-0.0660540514383969	-0.797268375321995\\
-0.0881551416713688	-0.795128084648568\\
-0.110326586627533	-0.792356008548631\\
-0.132584875054072	-0.788936785114559\\
-0.15494570494597	-0.784851469081123\\
-0.177423766609418	-0.780077436567759\\
-0.200032492032696	-0.774588279107804\\
-0.222783763688944	-0.768353691106244\\
-0.245687575436326	-0.761339356185019\\
-0.268751637770638	-0.753506839514812\\
-0.291980919394964	-0.744813495251846\\
-0.315377116999107	-0.735212400652581\\
-0.338938045420099	-0.72465233137471\\
-0.362656941157243	-0.713077795917439\\
-0.386521673761202	-0.700429151101551\\
-0.410513862179239	-0.686642824879635\\
-0.434607897042195	-0.671651677455331\\
-0.458769875490663	-0.655385536415232\\
-0.482956462846185	-0.637771945914134\\
-0.507113705604134	-0.618737173271854\\
-0.531175833130752	-0.598207517754376\\
-0.555064101160988	-0.576110964660754\\
-0.57868574844394	-0.552379221683688\\
-0.601933157869316	-0.526950162214107\\
-0.624683333602136	-0.499770680132121\\
-0.646797823711043	-0.470799931226268\\
-0.668123230039908	-0.440012896948533\\
-0.68849244924248	-0.407404157239578\\
-0.707726775991557	-0.372991703050613\\
-0.725638966562144	-0.336820560843045\\
-0.742037304769065	-0.298965948447415\\
-0.756730631924608	-0.259535644386246\\
-0.769534201736895	-0.218671242638716\\
-0.780276109019161	-0.176547992608009\\
-0.788803932365892	-0.133372996832589\\
-0.794991144212697	-0.0893816570857081\\
-0.798742796297556	-0.0448324141973383\\
-0.8	-0\\
nan	nan\\
-0	-1\\
-0.0274926788768275	-0.999622004863926\\
-0.0550077826159112	-0.998485925715371\\
-0.0825675642979961	-0.996585469152494\\
-0.110193927089211	-0.99391010581071\\
-0.137908233284417	-0.990445010685788\\
-0.16573109381759	-0.986170981393199\\
-0.193682131182463	-0.981064336351403\\
-0.221779708261773	-0.975096795709699\\
-0.25004061504087	-0.968235348884755\\
-0.27847970461118	-0.960442113882805\\
-0.307109469295408	-0.951674195231274\\
-0.335939547213297	-0.941883549393514\\
-0.364976149243705	-0.931016869064807\\
-0.394221396248884	-0.919015500815726\\
-0.423672556775123	-0.905815414218388\\
-0.453321176446553	-0.891347244896798\\
-0.483152092201502	-0.875536438876938\\
-0.513142327724049	-0.858303531099544\\
-0.543259871302744	-0.839564596819164\\
-0.573462344363328	-0.81923192051904\\
-0.603695578557732	-0.797214932392667\\
-0.633892132005168	-0.773421466589817\\
-0.66396979141344	-0.74775939719297\\
-0.693830126451235	-0.720138705825942\\
-0.723357185554925	-0.69047402710461\\
-0.752416447336645	-0.658687702767634\\
-0.78085416700267	-0.624713350165151\\
-0.808497279638804	-0.588499914032835\\
-0.835154037549885	-0.550016121185666\\
-0.8606155615531	-0.509255196549473\\
-0.884658469989446	-0.466239628813266\\
-0.90704870820268	-0.421025701053807\\
-0.927546630961331	-0.373707435559269\\
-0.94591328990576	-0.324419555482807\\
-0.961917752171118	-0.273339053298395\\
-0.975345136273952	-0.220684990760011\\
-0.986004915457365	-0.166716246040736\\
-0.993738930265871	-0.111727071357135\\
-0.998428495371944	-0.0560405177466728\\
-1	-0\\
nan	nan\\
-0	-1.2\\
-0.032991214652193	-1.19954640583671\\
-0.0660093391390935	-1.19818311085844\\
-0.0990810771575953	-1.19590256298299\\
-0.132232712507053	-1.19269212697285\\
-0.1654898799413	-1.18853401282295\\
-0.198877312581108	-1.18340517767184\\
-0.232418557418956	-1.17727720362168\\
-0.266135649914128	-1.17011615485164\\
-0.300048738049045	-1.16188241866171\\
-0.334175645533416	-1.15253053665937\\
-0.368531363154489	-1.14200903427753\\
-0.403127456655956	-1.13026025927222\\
-0.437971379092446	-1.11722024287777\\
-0.473065675498661	-1.10281860097887\\
-0.508407068130148	-1.08697849706207\\
-0.543985411735864	-1.06961669387616\\
-0.579782510641803	-1.05064372665233\\
-0.615770793268858	-1.02996423731945\\
-0.651911845563293	-1.007477516183\\
-0.688154813235994	-0.983078304622849\\
-0.724434694269278	-0.956657918871201\\
-0.760670558406201	-0.92810575990778\\
-0.796763749696128	-0.897311276631564\\
-0.832596151741482	-0.864166446991131\\
-0.86802862266591	-0.828568832525532\\
-0.902899736803974	-0.79042524332116\\
-0.937025000403204	-0.749656020198182\\
-0.970196735566565	-0.706199896839402\\
-1.00218484505986	-0.660019345422799\\
-1.03273867386372	-0.611106235859368\\
-1.06159016398734	-0.55948755457592\\
-1.08845844984322	-0.505230841264568\\
-1.1130559571536	-0.448448922671122\\
-1.13509594788691	-0.389303466579369\\
-1.15430130260534	-0.328006863958074\\
-1.17041416352874	-0.264821988912014\\
-1.18320589854884	-0.200059495248883\\
-1.19248671631905	-0.134072485628562\\
-1.19811419444633	-0.0672486212960074\\
-1.2	-0\\
nan	nan\\
-0	-1.4\\
-0.0384897504275585	-1.3994708068095\\
-0.0770108956622757	-1.39788029600152\\
-0.115594590017195	-1.39521965681349\\
-0.154271497924896	-1.39147414813499\\
-0.193071526598183	-1.3866230149601\\
-0.232023531344626	-1.38063937395048\\
-0.271154983655448	-1.37349007089196\\
-0.310491591566482	-1.36513551399358\\
-0.350056861057219	-1.35552948843866\\
-0.389871586455652	-1.34461895943593\\
-0.429953257013571	-1.33234387332378\\
-0.470315366098616	-1.31863696915092\\
-0.510966608941187	-1.30342361669073\\
-0.551909954748438	-1.28662170114202\\
-0.593141579485173	-1.26814157990574\\
-0.634649647025175	-1.24788614285552\\
-0.676412929082103	-1.22575101442771\\
-0.718399258813668	-1.20162494353936\\
-0.760563819823842	-1.17539043554683\\
-0.80284728210866	-1.14692468872666\\
-0.845173809980825	-1.11610090534973\\
-0.887448984807235	-1.08279005322574\\
-0.929557707978816	-1.04686315607016\\
-0.97136217703173	-1.00819418815632\\
-1.0127000597769	-0.966663637946454\\
-1.0533830262713	-0.922162783874687\\
-1.09319583380374	-0.874598690231212\\
-1.13189619149433	-0.823899879645969\\
-1.16921565256984	-0.770022569659933\\
-1.20486178617434	-0.712957275169262\\
-1.23852185798522	-0.652735480338573\\
-1.26986819148375	-0.589435981475329\\
-1.29856528334586	-0.523190409782976\\
-1.32427860586806	-0.45418737767593\\
-1.34668485303957	-0.382674674617753\\
-1.36548319078353	-0.308958987064016\\
-1.38040688164031	-0.23340274445703\\
-1.39123450237222	-0.156417899899989\\
-1.39779989352072	-0.078456724845342\\
-1.4	-0\\
nan	nan\\
-0	-1.6\\
-0.043988286202924	-1.59939520778228\\
-0.088012452185458	-1.59757748114459\\
-0.132108102876794	-1.59453675064399\\
-0.176310283342738	-1.59025616929714\\
-0.220653173255067	-1.58471201709726\\
-0.265169750108143	-1.57787357022912\\
-0.309891409891941	-1.56970293816225\\
-0.354847533218837	-1.56015487313552\\
-0.400064984065393	-1.54917655821561\\
-0.445567527377887	-1.53670738221249\\
-0.491375150872652	-1.52267871237004\\
-0.537503275541275	-1.50701367902962\\
-0.583961838789928	-1.48962699050369\\
-0.630754233998215	-1.47042480130516\\
-0.677876090840198	-1.44930466274942\\
-0.725313882314485	-1.42615559183488\\
-0.773043347522404	-1.4008583022031\\
-0.821027724358478	-1.37328564975927\\
-0.869215794084391	-1.34330335491066\\
-0.917539750981326	-1.31077107283046\\
-0.965912925692371	-1.27554389182827\\
-1.01422741120827	-1.23747434654371\\
-1.0623516662615	-1.19641503550875\\
-1.11012820232198	-1.15222192932151\\
-1.15737149688788	-1.10475844336738\\
-1.20386631573863	-1.05390032442821\\
-1.24936666720427	-0.999541360264242\\
-1.29359564742209	-0.941599862452536\\
-1.33624646007982	-0.880025793897066\\
-1.37698489848496	-0.814808314479157\\
-1.41545355198311	-0.745983406101226\\
-1.45127793312429	-0.673641121686091\\
-1.48407460953813	-0.59793189689483\\
-1.51346126384922	-0.519071288772491\\
-1.53906840347379	-0.437342485277432\\
-1.56055221803832	-0.353095985216018\\
-1.57760786473178	-0.266745993665177\\
-1.58998228842539	-0.178763314171416\\
-1.59748559259511	-0.0896648283946765\\
-1.6	-0\\
nan	nan\\
};
\end{axis}
\end{tikzpicture}%}
  \caption{Root locus of the open-loop system without the use of a zero-order hold.}
  \label{fig:Q7.rloc_F_G}
\end{figure}

\begin{figure}[H]\centering
	\centering
	\scalebox{1}{% This file was created by matlab2tikz.
%
%The latest updates can be retrieved from
%  http://www.mathworks.com/matlabcentral/fileexchange/22022-matlab2tikz-matlab2tikz
%where you can also make suggestions and rate matlab2tikz.
%
\definecolor{mycolor1}{rgb}{0.00000,0.44700,0.74100}%
\definecolor{mycolor2}{rgb}{0.75000,0.75000,0.00000}%
\definecolor{mycolor3}{rgb}{0.75000,0.00000,0.75000}%
\definecolor{mycolor4}{rgb}{0.00000,0.75000,0.75000}%
%
\begin{tikzpicture}

\begin{axis}[%
width=4.008in,
height=3.052in,
at={(0.818in,0.44in)},
scale only axis,
unbounded coords=jump,
separate axis lines,
every outer x axis line/.append style={white!40!black},
every x tick label/.append style={font=\color{white!40!black}},
xmin=-10,
xmax=6,
every outer y axis line/.append style={white!40!black},
every y tick label/.append style={font=\color{white!40!black}},
ymin=-8,
ymax=8,
axis background/.style={fill=white}
]
\addplot [color=white,solid,forget plot]
  table[row sep=crcr]{%
0	0\\
};
\addplot [color=mycolor1,only marks,mark=o,mark options={solid},forget plot]
  table[row sep=crcr]{%
-0.134875180971928	0.0760690712223533\\
-0.134875180971928	-0.0760690712223533\\
-2.19779182178189e-16	0\\
};
\addplot [color=mycolor1,only marks,mark=x,mark options={solid},forget plot]
  table[row sep=crcr]{%
-3	1.73205080756888\\
-3	-1.73205080756888\\
-0.0804390944323632	0\\
-0.0804391018657927	0\\
-1.15912180370184	0\\
4.8806657811541e-17	0\\
-1.65289318632125e-18	0\\
};
\addplot [color=blue,solid,forget plot]
  table[row sep=crcr]{%
-3	1.73205080756888\\
-3.00000000000013	1.73205080756892\\
-3.00000000000013	1.73205080756892\\
-3.00000000000013	1.73205080756892\\
-3.00018574345039	1.73210999549279\\
-3.00024952467788	1.73213032886482\\
-3.00033520054779	1.73215764963554\\
-3.00045028154132	1.73219436064315\\
-3.00060485006341	1.73224369235797\\
-3.00081243765992	1.73230998901812\\
-3.00109119859842	1.73239909453141\\
-3.00146547765892	1.73251887357957\\
-3.00186145710671	1.73264577354041\\
-3.00186331722382	1.7326463700815\\
-3.00186517733453	1.73264696662454\\
-3.0019679004465	1.73267991643415\\
-3.00264215298486	1.73289649422645\\
-3.0035466645889	1.73318785704113\\
-3.00475946367187	1.73358000349137\\
-3.00638453721239	1.73410810210982\\
-3.00856008284294	1.73481981859991\\
-3.01146908001185	1.73577990748758\\
-3.01535258878921	1.73707657195051\\
-3.02052605019796	1.73883028872727\\
-3.02739851038195	1.74120602903249\\
-3.03649397681834	1.74443003413943\\
-3.04847286372701	1.74881238821826\\
-3.06414957413294	1.75477626804255\\
-3.08449983889996	1.76289340972457\\
-3.09710955106993	1.76812942496817\\
-3.10942177238854	1.77338999769741\\
-3.10952063090217	1.7734328193247\\
-3.10961947067122	1.77347564204358\\
-3.11064927102963	1.77392235455516\\
-3.11467281437444	1.77567725563282\\
-3.11477591089001	1.77572242111809\\
-3.11487898701366	1.77576758758909\\
-3.12955787385621	1.78230000528241\\
-3.14383451446854	1.78884114691129\\
-3.18533301076605	1.80886061241473\\
-3.2363683952824	1.83540203353982\\
-3.29801300845091	1.87002938698027\\
-3.37111799671745	1.91434308330117\\
-3.45629541263606	1.96986177608149\\
-3.55395606203566	2.03792731170025\\
-3.66438479682309	2.11965788135532\\
-3.7878256102048	2.21595334267563\\
-3.92455452562459	2.32753893105085\\
-4.07493059403701	2.45502726021708\\
-4.23942558458848	2.59898182420575\\
-4.41863789267797	2.75997221768887\\
-4.61329688069298	2.93861752491266\\
-4.82426255606251	3.13561821106619\\
-5.05252378567797	3.35177861306715\\
-5.29919685578166	3.58802254739917\\
-5.56552526159123	3.84540431518027\\
-5.85288107189405	4.12511691924863\\
-6.16276793440798	4.42849883255713\\
-6.49682566295659	4.75704025841028\\
-6.856836306824	5.11238952067469\\
-7.2447316036385	5.49636000692364\\
-7.66260173629013	5.91093794148167\\
-8.11270533960169	6.35829117037366\\
-8.59748072801077	6.84077908118869\\
-9.11955833902618	7.36096374643808\\
-167.50374464114	165.740537665457\\
inf	0\\
};
\addplot [color=darkgray,solid,forget plot]
  table[row sep=crcr]{%
-1.65289318632125e-18	0\\
-2.20017862728807e-16	0\\
-2.20108783189928e-16	0\\
-2.19842449896917e-16	0\\
-2.19779182177895e-16	0\\
-2.19779182177796e-16	0\\
-2.19779182177968e-16	0\\
-2.19779182178119e-16	0\\
-2.19779182177949e-16	0\\
-2.19779182178109e-16	0\\
-2.1977918217816e-16	0\\
-2.19779182178174e-16	0\\
-2.19779182178154e-16	0\\
-2.19779182178129e-16	0\\
-2.19779182178212e-16	0\\
-2.19779182178118e-16	0\\
-2.1977918217807e-16	0\\
-2.19779182178092e-16	0\\
-2.19779182178128e-16	0\\
-2.19779182178124e-16	0\\
-2.19779182178118e-16	0\\
-2.19779182178158e-16	0\\
-2.1977918217815e-16	0\\
-2.19779182178233e-16	0\\
-2.19779182178209e-16	0\\
-2.19779182178189e-16	0\\
-2.19779182178154e-16	0\\
-2.19779182178156e-16	0\\
-2.19779182178181e-16	0\\
-2.19779182178167e-16	0\\
-2.1977918217818e-16	0\\
-2.19779182178181e-16	0\\
-2.19779182178179e-16	0\\
-2.19779182178179e-16	0\\
-2.19779182178186e-16	0\\
-2.19779182178175e-16	0\\
-2.19779182178191e-16	0\\
-2.19779182178185e-16	0\\
-2.1977918217819e-16	0\\
-2.19779182178197e-16	0\\
-2.19779182178192e-16	0\\
-2.19779182178191e-16	0\\
-2.19779182178193e-16	0\\
-2.19779182178186e-16	0\\
-2.19779182178189e-16	0\\
-2.19779182178188e-16	0\\
-2.19779182178191e-16	0\\
-2.1977918217819e-16	0\\
-2.19779182178191e-16	0\\
-2.19779182178189e-16	0\\
-2.19779182178191e-16	0\\
-2.19779182178189e-16	0\\
-2.1977918217819e-16	0\\
-2.1977918217819e-16	0\\
-2.1977918217819e-16	0\\
-2.19779182178192e-16	0\\
-2.19779182178189e-16	0\\
-2.19779182178191e-16	0\\
-2.1977918217819e-16	0\\
-2.1977918217819e-16	0\\
-2.19779182178193e-16	0\\
-2.19779182178191e-16	0\\
-2.19779182178193e-16	0\\
-2.1977918217819e-16	0\\
-2.1977918217819e-16	0\\
-2.1977918217819e-16	0\\
-2.19779182178189e-16	0\\
};
\addplot [color=mycolor2,solid,forget plot]
  table[row sep=crcr]{%
4.8806657811541e-17	0\\
-1.12523922035296e-12	0\\
-1.12590277412615e-12	0\\
-1.12838707433109e-12	0\\
-0.00166082730712354	0\\
-0.00225180971162835	0\\
-0.00306408247812187	0\\
-0.00419138973183814	0\\
-0.00577965690512844	0\\
-0.00807412702544666	0\\
-0.0115480702936874	0\\
-0.0174389045256782	0\\
-0.0315800751047079	0\\
-0.0326381436951767	2.0035159080951e-08\\
-0.0326360143750463	0.00106007432692908\\
-0.0325222010726573	0.00792363802264028\\
-0.031927376276678	0.0212404397374045\\
-0.0314220921049791	0.030652579550166\\
-0.0310745051612175	0.0394437582051263\\
-0.0309807290306922	0.0483272061963577\\
-0.031275969575853	0.0576580061846414\\
-0.0321509516873781	0.0676823977873882\\
-0.0338769893694386	0.0786030766850838\\
-0.0368459721111171	0.0905969461201523\\
-0.0416379398104174	0.10380730055516\\
-0.0491444323196527	0.118301468212372\\
-0.0608186867821561	0.133943512103664\\
-0.0792651060725459	0.149977348430113\\
-0.109961693076483	0.163158310022309\\
-0.134340811494869	0.164275517609474\\
-0.161454212033936	0.152245669411022\\
-0.161651019566681	0.152055912265094\\
-0.161846579877063	0.151864440859541\\
-0.163803722478871	0.149767227711246\\
-0.169449739122379	0.140163783810233\\
-0.169542907387284	0.139901298657482\\
-0.169633301197232	0.139638741840658\\
-0.166873558838923	0.112821005599766\\
-0.160361645034247	0.102269708978243\\
-0.150398223655247	0.0907135830573359\\
-0.145111969835922	0.0854735046907944\\
-0.141920434232436	0.0824661648284569\\
-0.13984760729342	0.0805565715817262\\
-0.138441329560915	0.0792761364164901\\
-0.137460287305012	0.0783888385078409\\
-0.136763009033116	0.0777607526981368\\
-0.136260995524241	0.0773097302627165\\
-0.135896270394896	0.0769826158114825\\
-0.135629562551154	0.0767436917864527\\
-0.135433612638044	0.0765682980073349\\
-0.135289155042162	0.0764390694744112\\
-0.135182391085725	0.0763436003729852\\
-0.135103339569485	0.0762729331033424\\
-0.135044727494537	0.0762205488858227\\
-0.135001226280639	0.0761816761279487\\
-0.134968916076855	0.0761528070676185\\
-0.134944904627433	0.0761313547761771\\
-0.13492705309505	0.0761154068951782\\
-0.134913777157607	0.0761035472397231\\
-0.134903901786412	0.0760947256872687\\
-0.134896554703729	0.0760881627965108\\
-0.134891087932271	0.0760832796164374\\
-0.134887019871364	0.0760796458834948\\
-0.134883992441188	0.0760769417069688\\
-0.134881739326665	0.0760749291846393\\
-0.134875180997266	0.0760690712449868\\
-0.134875180971928	0.0760690712223533\\
};
\addplot [color=mycolor3,solid,forget plot]
  table[row sep=crcr]{%
-1.15912180370184	0\\
-1.15912180370133	0\\
-1.15912180370133	0\\
-1.15912180370133	0\\
-1.15837755310719	0\\
-1.15812187999556	0\\
-1.15777835166931	0\\
-1.15731675994645	0\\
-1.15669649433297	0\\
-1.15586294724616	0\\
-1.15474266645881	0\\
-1.15323680929649	0\\
-1.15164149998898	0\\
-1.15163400078818	0\\
-1.15162650156425	0\\
-1.15121228700693	0\\
-1.1484897526908	0\\
-1.14482724704539	0\\
-1.13989781332376	0\\
-1.13325861905764	0\\
-1.12430787036064	0\\
-1.11222364502067	0\\
-1.09587473390263	0\\
-1.07368527885509	0\\
-1.04341591383222	0\\
-1.00177406642727	0\\
-0.94361079124688	0\\
-0.859847785001545	0\\
-0.729563162458321	0\\
-0.62279283251851	0\\
-0.407945108017524	0\\
-0.388828371233524	0\\
-0.388533949451722	-0.0188523190339899\\
-0.385547006491493	-0.064297528438528\\
-0.375877446503183	-0.142323858148687\\
-0.375681181722702	-0.143859327676446\\
-0.375487711789106	-0.14538209155385\\
-0.36356856730487	-0.294257300730666\\
-0.355803840497215	-0.383853899025846\\
-0.324268765578707	-0.56183489725048\\
-0.27851963488168	-0.720036519658047\\
-0.220066557316655	-0.873726387539305\\
-0.149034395989124	-1.02798551741661\\
-0.0652632578030296	-1.18512067945298\\
0.0314163493406752	-1.34645510295588\\
0.141147805856202	-1.51297533616862\\
0.264086605729042	-1.68560182422243\\
0.400450796019479	-1.86529915451281\\
0.55056015658816	-2.05311242785714\\
0.714859197226519	-2.25017196438932\\
0.893927047720134	-2.45768750018207\\
1.0884792717787	-2.67694155406908\\
1.29936589563199	-2.9092855825112\\
1.52756851317251	-3.15613967944948\\
1.7741980820623	-3.41899545290353\\
2.04049417766808	-3.69942141573024\\
2.32782597652148	-3.99907027181541\\
2.63769498750303	-4.31968762511556\\
2.97173944011421	-4.66312178970166\\
3.33174020861042	-5.03133450231319\\
3.71962815834223	-5.42641243041062\\
4.13749282422239	-5.85057943328586\\
4.58759235947305	-6.30620957833135\\
5.07236472045195	-6.79584094517458\\
5.59444007835284	-7.32219027171108\\
163.978619822137	-165.740461428786\\
inf	0\\
};
\addplot [color=mycolor4,solid,forget plot]
  table[row sep=crcr]{%
-0.0804391018657927	0\\
-0.0804392904054883	0\\
-0.0804392905847375	0\\
-0.0804392904603088	0\\
-0.087194941005693	0\\
-0.0881857638465612	0\\
-0.089310285828975	0\\
-0.0905837184529872	0\\
-0.0920224782264052	0\\
-0.0936442628470515	0\\
-0.0954681844854813	0\\
-0.0975150014520303	0\\
-0.099355034741433	0\\
-0.0993630773738298	0\\
-0.0993711150165938	0\\
-0.0998075099547562	0\\
-0.102371188786128	0\\
-0.105235239566858	0\\
-0.108434249010061	0\\
-0.112010848456188	0\\
-0.116020024801773	0\\
-0.120536291580873	0\\
-0.125666109780082	0\\
-0.13157067652676	0\\
-0.138511185783053	0\\
-0.146949115296747	0\\
-0.157806107734789	0\\
-0.17332285458748	0\\
-0.201513773588797	0\\
-0.2343064423519	0\\
-0.370302923137532	0\\
-0.388828327828783	0\\
-0.388533949451722	0.0188523190339899\\
-0.385547006491493	0.064297528438528\\
-0.375877446503183	0.142323858148687\\
-0.375681181722702	0.143859327676446\\
-0.375487711789106	0.14538209155385\\
-0.36356856730487	0.294257300730666\\
-0.355803840497215	0.383853899025846\\
-0.324268765578707	0.56183489725048\\
-0.27851963488168	0.720036519658047\\
-0.220066557316655	0.873726387539305\\
-0.149034395989124	1.02798551741661\\
-0.0652632578030296	1.18512067945298\\
0.0314163493406752	1.34645510295588\\
0.141147805856202	1.51297533616862\\
0.264086605729042	1.68560182422243\\
0.400450796019479	1.86529915451281\\
0.55056015658816	2.05311242785714\\
0.714859197226519	2.25017196438932\\
0.893927047720134	2.45768750018207\\
1.0884792717787	2.67694155406908\\
1.29936589563199	2.9092855825112\\
1.52756851317251	3.15613967944948\\
1.7741980820623	3.41899545290353\\
2.04049417766808	3.69942141573024\\
2.32782597652148	3.99907027181541\\
2.63769498750303	4.31968762511556\\
2.97173944011421	4.66312178970166\\
3.33174020861042	5.03133450231319\\
3.71962815834223	5.42641243041062\\
4.13749282422239	5.85057943328586\\
4.58759235947305	6.30620957833135\\
5.07236472045195	6.79584094517458\\
5.59444007835284	7.32219027171108\\
163.978619822137	165.740461428786\\
inf	0\\
};
\addplot [color=red,solid,forget plot]
  table[row sep=crcr]{%
-0.0804390944323632	0\\
-0.0804389058918015	0\\
-0.0804389057125516	0\\
-0.0804389058369783	0\\
-0.0723951916792066	0\\
-0.0709414970904845	0\\
-0.0691768789280074	0\\
-0.0670075687860854	0\\
-0.064291670408685	0\\
-0.0607937875615021	0\\
-0.0560586815651727	0\\
-0.0488783294079573	0\\
-0.0337004759514588	0\\
-0.0326381436951767	-2.0035159080951e-08\\
-0.0326360143750463	-0.00106007432692908\\
-0.0325222010726573	-0.00792363802264028\\
-0.031927376276678	-0.0212404397374045\\
-0.0314220921049791	-0.030652579550166\\
-0.0310745051612175	-0.0394437582051263\\
-0.0309807290306922	-0.0483272061963577\\
-0.031275969575853	-0.0576580061846414\\
-0.0321509516873781	-0.0676823977873882\\
-0.0338769893694386	-0.0786030766850838\\
-0.0368459721111171	-0.0905969461201523\\
-0.0416379398104174	-0.10380730055516\\
-0.0491444323196527	-0.118301468212372\\
-0.0608186867821561	-0.133943512103664\\
-0.0792651060725459	-0.149977348430113\\
-0.109961693076483	-0.163158310022309\\
-0.134340811494869	-0.164275517609474\\
-0.161454212033936	-0.152245669411022\\
-0.161651019566681	-0.152055912265094\\
-0.161846579877063	-0.151864440859541\\
-0.163803722478871	-0.149767227711246\\
-0.169449739122379	-0.140163783810233\\
-0.169542907387284	-0.139901298657482\\
-0.169633301197232	-0.139638741840658\\
-0.166873558838923	-0.112821005599766\\
-0.160361645034247	-0.102269708978243\\
-0.150398223655247	-0.0907135830573359\\
-0.145111969835922	-0.0854735046907944\\
-0.141920434232436	-0.0824661648284569\\
-0.13984760729342	-0.0805565715817262\\
-0.138441329560915	-0.0792761364164901\\
-0.137460287305012	-0.0783888385078409\\
-0.136763009033116	-0.0777607526981368\\
-0.136260995524241	-0.0773097302627165\\
-0.135896270394896	-0.0769826158114825\\
-0.135629562551154	-0.0767436917864527\\
-0.135433612638044	-0.0765682980073349\\
-0.135289155042162	-0.0764390694744112\\
-0.135182391085725	-0.0763436003729852\\
-0.135103339569485	-0.0762729331033424\\
-0.135044727494537	-0.0762205488858227\\
-0.135001226280639	-0.0761816761279487\\
-0.134968916076855	-0.0761528070676185\\
-0.134944904627433	-0.0761313547761771\\
-0.13492705309505	-0.0761154068951782\\
-0.134913777157607	-0.0761035472397231\\
-0.134903901786412	-0.0760947256872687\\
-0.134896554703729	-0.0760881627965108\\
-0.134891087932271	-0.0760832796164374\\
-0.134887019871364	-0.0760796458834948\\
-0.134883992441188	-0.0760769417069688\\
-0.134881739326665	-0.0760749291846393\\
-0.134875180997266	-0.0760690712449868\\
-0.134875180971928	-0.0760690712223533\\
};
\addplot [color=black!50!green,solid,forget plot]
  table[row sep=crcr]{%
-3	-1.73205080756888\\
-3.00000000000013	-1.73205080756892\\
-3.00000000000013	-1.73205080756892\\
-3.00000000000013	-1.73205080756892\\
-3.00018574345039	-1.73210999549279\\
-3.00024952467788	-1.73213032886482\\
-3.00033520054779	-1.73215764963554\\
-3.00045028154132	-1.73219436064315\\
-3.00060485006341	-1.73224369235797\\
-3.00081243765992	-1.73230998901812\\
-3.00109119859842	-1.73239909453141\\
-3.00146547765892	-1.73251887357957\\
-3.00186145710671	-1.73264577354041\\
-3.00186331722382	-1.7326463700815\\
-3.00186517733453	-1.73264696662454\\
-3.0019679004465	-1.73267991643415\\
-3.00264215298486	-1.73289649422645\\
-3.0035466645889	-1.73318785704113\\
-3.00475946367187	-1.73358000349137\\
-3.00638453721239	-1.73410810210982\\
-3.00856008284294	-1.73481981859991\\
-3.01146908001185	-1.73577990748758\\
-3.01535258878921	-1.73707657195051\\
-3.02052605019796	-1.73883028872727\\
-3.02739851038195	-1.74120602903249\\
-3.03649397681834	-1.74443003413943\\
-3.04847286372701	-1.74881238821826\\
-3.06414957413294	-1.75477626804255\\
-3.08449983889996	-1.76289340972457\\
-3.09710955106993	-1.76812942496817\\
-3.10942177238854	-1.77338999769741\\
-3.10952063090217	-1.7734328193247\\
-3.10961947067122	-1.77347564204358\\
-3.11064927102963	-1.77392235455516\\
-3.11467281437444	-1.77567725563282\\
-3.11477591089001	-1.77572242111809\\
-3.11487898701366	-1.77576758758909\\
-3.12955787385621	-1.78230000528241\\
-3.14383451446854	-1.78884114691129\\
-3.18533301076605	-1.80886061241473\\
-3.2363683952824	-1.83540203353982\\
-3.29801300845091	-1.87002938698027\\
-3.37111799671745	-1.91434308330117\\
-3.45629541263606	-1.96986177608149\\
-3.55395606203566	-2.03792731170025\\
-3.66438479682309	-2.11965788135532\\
-3.7878256102048	-2.21595334267563\\
-3.92455452562459	-2.32753893105085\\
-4.07493059403701	-2.45502726021708\\
-4.23942558458848	-2.59898182420575\\
-4.41863789267797	-2.75997221768887\\
-4.61329688069298	-2.93861752491266\\
-4.82426255606251	-3.13561821106619\\
-5.05252378567797	-3.35177861306715\\
-5.29919685578166	-3.58802254739917\\
-5.56552526159123	-3.84540431518027\\
-5.85288107189405	-4.12511691924863\\
-6.16276793440798	-4.42849883255713\\
-6.49682566295659	-4.75704025841028\\
-6.856836306824	-5.11238952067469\\
-7.2447316036385	-5.49636000692364\\
-7.66260173629013	-5.91093794148167\\
-8.11270533960169	-6.35829117037366\\
-8.59748072801077	-6.84077908118869\\
-9.11955833902618	-7.36096374643808\\
-167.50374464114	-165.740537665457\\
inf	0\\
};
\addplot [color=white!40!black,dotted,forget plot]
  table[row sep=crcr]{%
0	0\\
-0	14\\
nan	nan\\
0	0\\
-2.24	13.8196382007634\\
nan	nan\\
0	0\\
-4.76	13.1659560989698\\
nan	nan\\
0	0\\
-7	12.1243556529821\\
nan	nan\\
0	0\\
-8.96	10.7572487188872\\
nan	nan\\
0	0\\
-10.64	9.09892301319228\\
nan	nan\\
0	0\\
-12.04	7.14411646041692\\
nan	nan\\
0	0\\
-13.16	4.77644219058495\\
nan	nan\\
0	0\\
-13.79	2.41576074974324\\
nan	nan\\
0	0\\
-14	0\\
nan	nan\\
0	-0\\
-0	-14\\
nan	nan\\
0	-0\\
-2.24	-13.8196382007634\\
nan	nan\\
0	-0\\
-4.76	-13.1659560989698\\
nan	nan\\
0	-0\\
-7	-12.1243556529821\\
nan	nan\\
0	-0\\
-8.96	-10.7572487188872\\
nan	nan\\
0	-0\\
-10.64	-9.09892301319228\\
nan	nan\\
0	-0\\
-12.04	-7.14411646041692\\
nan	nan\\
0	-0\\
-13.16	-4.77644219058495\\
nan	nan\\
0	-0\\
-13.79	-2.41576074974324\\
nan	nan\\
0	-0\\
-14	-0\\
nan	nan\\
};
\addplot [color=white!40!black,dotted,forget plot]
  table[row sep=crcr]{%
-0	0\\
-0	0\\
-0	0\\
-0	0\\
-0	0\\
-0	0\\
-0	0\\
-0	0\\
-0	0\\
-0	0\\
-0	0\\
-0	0\\
-0	0\\
-0	0\\
-0	0\\
-0	0\\
-0	0\\
-0	0\\
-0	0\\
-0	0\\
-0	0\\
-0	0\\
-0	0\\
-0	0\\
-0	0\\
-0	0\\
-0	0\\
-0	0\\
-0	0\\
-0	0\\
-0	0\\
-0	0\\
-0	0\\
-0	0\\
-0	0\\
-0	0\\
-0	0\\
-0	0\\
-0	0\\
-0	0\\
-0	0\\
nan	nan\\
-0	2\\
-0.0785196315181373	1.99845807248145\\
-0.15691819145569	1.99383466746626\\
-0.235074794915675	1.98613691390985\\
-0.312868930080462	1.97537668119028\\
-0.390180644032257	1.96157056080646\\
-0.466890727711811	1.94473984079535\\
-0.542880899730149	1.92491047290729\\
-0.618033988749895	1.90211303259031\\
-0.692234114154986	1.87638267184497\\
-0.76536686473018	1.84775906502257\\
-0.837319475074856	1.81628634765016\\
-0.907980999479094	1.78201304837674\\
-0.97724248299391	1.74499201414559\\
-1.0449971294319	1.70528032870818\\
-1.1111404660392	1.66293922460509\\
-1.17557050458495	1.61803398874989\\
-1.23818789861967	1.57063386176149\\
-1.29889609666037	1.52081193120006\\
-1.35760149106588	1.46864501887137\\
-1.4142135623731	1.41421356237309\\
-1.46864501887137	1.35760149106588\\
-1.52081193120006	1.29889609666037\\
-1.57063386176149	1.23818789861967\\
-1.61803398874989	1.17557050458495\\
-1.66293922460509	1.1111404660392\\
-1.70528032870818	1.0449971294319\\
-1.74499201414559	0.97724248299391\\
-1.78201304837674	0.907980999479093\\
-1.81628634765016	0.837319475074855\\
-1.84775906502257	0.76536686473018\\
-1.87638267184497	0.692234114154986\\
-1.90211303259031	0.618033988749894\\
-1.92491047290729	0.542880899730149\\
-1.94473984079535	0.466890727711811\\
-1.96157056080646	0.390180644032257\\
-1.97537668119028	0.312868930080462\\
-1.98613691390985	0.235074794915674\\
-1.99383466746626	0.156918191455692\\
-1.99845807248145	0.0785196315181388\\
-2	0\\
nan	nan\\
-0	4\\
-0.157039263036275	3.99691614496289\\
-0.31383638291138	3.98766933493251\\
-0.470149589831351	3.97227382781971\\
-0.625737860160924	3.95075336238055\\
-0.780361288064513	3.92314112161292\\
-0.933781455423622	3.88947968159071\\
-1.0857617994603	3.84982094581459\\
-1.23606797749979	3.80422606518061\\
-1.38446822830997	3.75276534368994\\
-1.53073372946036	3.69551813004515\\
-1.67463895014971	3.63257269530033\\
-1.81596199895819	3.56402609675347\\
-1.95448496598782	3.48998402829119\\
-2.0899942588638	3.41056065741637\\
-2.22228093207841	3.32587844921018\\
-2.35114100916989	3.23606797749979\\
-2.47637579723934	3.14126772352298\\
-2.59779219332073	3.04162386240012\\
-2.71520298213177	2.93729003774274\\
-2.82842712474619	2.82842712474619\\
-2.93729003774274	2.71520298213177\\
-3.04162386240012	2.59779219332073\\
-3.14126772352298	2.47637579723934\\
-3.23606797749979	2.35114100916989\\
-3.32587844921018	2.22228093207841\\
-3.41056065741637	2.0899942588638\\
-3.48998402829119	1.95448496598782\\
-3.56402609675347	1.81596199895819\\
-3.63257269530033	1.67463895014971\\
-3.69551813004515	1.53073372946036\\
-3.75276534368994	1.38446822830997\\
-3.80422606518061	1.23606797749979\\
-3.84982094581459	1.0857617994603\\
-3.88947968159071	0.933781455423622\\
-3.92314112161292	0.780361288064513\\
-3.95075336238055	0.625737860160923\\
-3.97227382781971	0.470149589831347\\
-3.98766933493251	0.313836382911385\\
-3.99691614496289	0.157039263036278\\
-4	0\\
nan	nan\\
-0	6\\
-0.235558894554412	5.99537421744434\\
-0.47075457436707	5.98150400239877\\
-0.705224384747026	5.95841074172956\\
-0.938606790241385	5.92613004357083\\
-1.17054193209677	5.88471168241938\\
-1.40067218313543	5.83421952238606\\
-1.62864269919045	5.77473141872188\\
-1.85410196624969	5.70633909777092\\
-2.07670234246496	5.6291480155349\\
-2.29610059419054	5.54327719506772\\
-2.51195842522457	5.44885904295049\\
-2.72394299843728	5.34603914513021\\
-2.93172744898173	5.23497604243678\\
-3.13499138829569	5.11584098612455\\
-3.33342139811761	4.98881767381527\\
-3.52671151375484	4.85410196624968\\
-3.71456369585901	4.71190158528447\\
-3.8966882899811	4.56243579360019\\
-4.07280447319765	4.40593505661411\\
-4.24264068711929	4.24264068711928\\
-4.40593505661411	4.07280447319765\\
-4.56243579360019	3.8966882899811\\
-4.71190158528447	3.714563695859\\
-4.85410196624968	3.52671151375484\\
-4.98881767381527	3.33342139811761\\
-5.11584098612455	3.13499138829569\\
-5.23497604243678	2.93172744898173\\
-5.34603914513021	2.72394299843728\\
-5.44885904295049	2.51195842522457\\
-5.54327719506772	2.29610059419054\\
-5.6291480155349	2.07670234246496\\
-5.70633909777092	1.85410196624968\\
-5.77473141872188	1.62864269919045\\
-5.83421952238606	1.40067218313543\\
-5.88471168241938	1.17054193209677\\
-5.92613004357083	0.938606790241385\\
-5.95841074172956	0.705224384747021\\
-5.98150400239877	0.470754574367077\\
-5.99537421744434	0.235558894554416\\
-6	0\\
nan	nan\\
-0	8\\
-0.314078526072549	7.99383228992578\\
-0.62767276582276	7.97533866986502\\
-0.940299179662701	7.94454765563941\\
-1.25147572032185	7.9015067247611\\
-1.56072257612903	7.84628224322584\\
-1.86756291084724	7.77895936318141\\
-2.17152359892059	7.69964189162918\\
-2.47213595499958	7.60845213036123\\
-2.76893645661994	7.50553068737987\\
-3.06146745892072	7.39103626009029\\
-3.34927790029942	7.26514539060065\\
-3.63192399791637	7.12805219350694\\
-3.90896993197564	6.97996805658238\\
-4.17998851772759	6.82112131483274\\
-4.44456186415682	6.65175689842036\\
-4.70228201833979	6.47213595499958\\
-4.95275159447867	6.28253544704596\\
-5.19558438664147	6.08324772480025\\
-5.43040596426353	5.87458007548549\\
-5.65685424949238	5.65685424949238\\
-5.87458007548549	5.43040596426353\\
-6.08324772480025	5.19558438664147\\
-6.28253544704596	4.95275159447867\\
-6.47213595499958	4.70228201833979\\
-6.65175689842036	4.44456186415682\\
-6.82112131483274	4.17998851772759\\
-6.97996805658238	3.90896993197564\\
-7.12805219350694	3.63192399791637\\
-7.26514539060065	3.34927790029942\\
-7.39103626009029	3.06146745892072\\
-7.50553068737987	2.76893645661994\\
-7.60845213036123	2.47213595499958\\
-7.69964189162918	2.17152359892059\\
-7.77895936318141	1.86756291084724\\
-7.84628224322584	1.56072257612903\\
-7.9015067247611	1.25147572032185\\
-7.94454765563941	0.940299179662694\\
-7.97533866986502	0.62767276582277\\
-7.99383228992578	0.314078526072555\\
-8	0\\
nan	nan\\
-0	10\\
-0.392598157590687	9.99229036240723\\
-0.78459095727845	9.96917333733128\\
-1.17537397457838	9.93068456954926\\
-1.56434465040231	9.87688340595138\\
-1.95090322016128	9.8078528040323\\
-2.33445363855905	9.72369920397677\\
-2.71440449865074	9.62455236453647\\
-3.09016994374947	9.51056516295153\\
-3.46117057077493	9.38191335922484\\
-3.8268343236509	9.23879532511287\\
-4.18659737537428	9.08143173825081\\
-4.53990499739547	8.91006524188368\\
-4.88621241496955	8.72496007072797\\
-5.22498564715949	8.52640164354092\\
-5.55570233019602	8.31469612302545\\
-5.87785252292473	8.09016994374947\\
-6.19093949309834	7.85316930880745\\
-6.49448048330184	7.60405965600031\\
-6.78800745532942	7.34322509435686\\
-7.07106781186548	7.07106781186547\\
-7.34322509435686	6.78800745532942\\
-7.60405965600031	6.49448048330184\\
-7.85316930880745	6.19093949309834\\
-8.09016994374947	5.87785252292473\\
-8.31469612302545	5.55570233019602\\
-8.52640164354092	5.22498564715949\\
-8.72496007072797	4.88621241496955\\
-8.91006524188368	4.53990499739547\\
-9.08143173825082	4.18659737537428\\
-9.23879532511287	3.8268343236509\\
-9.38191335922484	3.46117057077493\\
-9.51056516295154	3.09016994374947\\
-9.62455236453647	2.71440449865074\\
-9.72369920397677	2.33445363855905\\
-9.8078528040323	1.95090322016128\\
-9.87688340595138	1.56434465040231\\
-9.93068456954926	1.17537397457837\\
-9.96917333733128	0.784590957278462\\
-9.99229036240723	0.392598157590694\\
-10	0\\
nan	nan\\
-0	12\\
-0.471117789108824	11.9907484348887\\
-0.94150914873414	11.9630080047975\\
-1.41044876949405	11.9168214834591\\
-1.87721358048277	11.8522600871417\\
-2.34108386419354	11.7694233648388\\
-2.80134436627087	11.6684390447721\\
-3.25728539838089	11.5494628374438\\
-3.70820393249937	11.4126781955418\\
-4.15340468492992	11.2582960310698\\
-4.59220118838108	11.0865543901354\\
-5.02391685044913	10.897718085901\\
-5.44788599687456	10.6920782902604\\
-5.86345489796346	10.4699520848736\\
-6.26998277659139	10.2316819722491\\
-6.66684279623522	9.97763534763054\\
-7.05342302750968	9.70820393249937\\
-7.42912739171801	9.42380317056894\\
-7.7933765799622	9.12487158720037\\
-8.1456089463953	8.81187011322823\\
-8.48528137423857	8.48528137423857\\
-8.81187011322823	8.1456089463953\\
-9.12487158720037	7.7933765799622\\
-9.42380317056894	7.42912739171801\\
-9.70820393249937	7.05342302750968\\
-9.97763534763054	6.66684279623522\\
-10.2316819722491	6.26998277659139\\
-10.4699520848736	5.86345489796346\\
-10.6920782902604	5.44788599687456\\
-10.897718085901	5.02391685044913\\
-11.0865543901354	4.59220118838108\\
-11.2582960310698	4.15340468492991\\
-11.4126781955418	3.70820393249937\\
-11.5494628374438	3.25728539838089\\
-11.6684390447721	2.80134436627087\\
-11.7694233648388	2.34108386419354\\
-11.8522600871417	1.87721358048277\\
-11.9168214834591	1.41044876949404\\
-11.9630080047975	0.941509148734155\\
-11.9907484348887	0.471117789108833\\
-12	0\\
nan	nan\\
-0	14\\
-0.549637420626961	13.9892065073701\\
-1.09842734018983	13.9568426722638\\
-1.64552356440973	13.902958397369\\
-2.19008251056323	13.8276367683319\\
-2.7312645082258	13.7309939256452\\
-3.26823509398268	13.6131788855675\\
-3.80016629811104	13.4743733103511\\
-4.32623792124927	13.3147912281321\\
-4.8456387990849	13.1346787029148\\
-5.35756805311126	12.934313455158\\
-5.86123632552399	12.7140044335511\\
-6.35586699635366	12.4740913386371\\
-6.84069738095737	12.2149440990192\\
-7.31497990602328	11.9369623009573\\
-7.77798326227443	11.6405745722356\\
-8.22899353209462	11.3262379212493\\
-8.66731529033768	10.9944370323304\\
-9.09227267662257	10.6456835184004\\
-9.50321043746118	10.2805151320996\\
-9.89949493661167	9.89949493661166\\
-10.2805151320996	9.50321043746118\\
-10.6456835184004	9.09227267662257\\
-10.9944370323304	8.66731529033768\\
-11.3262379212493	8.22899353209462\\
-11.6405745722356	7.77798326227443\\
-11.9369623009573	7.31497990602328\\
-12.2149440990192	6.84069738095737\\
-12.4740913386371	6.35586699635365\\
-12.7140044335511	5.86123632552399\\
-12.934313455158	5.35756805311126\\
-13.1346787029148	4.8456387990849\\
-13.3147912281322	4.32623792124926\\
-13.4743733103511	3.80016629811104\\
-13.6131788855675	3.26823509398268\\
-13.7309939256452	2.7312645082258\\
-13.8276367683319	2.19008251056323\\
-13.902958397369	1.64552356440972\\
-13.9568426722638	1.09842734018985\\
-13.9892065073701	0.549637420626972\\
-14	0\\
nan	nan\\
-0	-0\\
-0	-0\\
-0	-0\\
-0	-0\\
-0	-0\\
-0	-0\\
-0	-0\\
-0	-0\\
-0	-0\\
-0	-0\\
-0	-0\\
-0	-0\\
-0	-0\\
-0	-0\\
-0	-0\\
-0	-0\\
-0	-0\\
-0	-0\\
-0	-0\\
-0	-0\\
-0	-0\\
-0	-0\\
-0	-0\\
-0	-0\\
-0	-0\\
-0	-0\\
-0	-0\\
-0	-0\\
-0	-0\\
-0	-0\\
-0	-0\\
-0	-0\\
-0	-0\\
-0	-0\\
-0	-0\\
-0	-0\\
-0	-0\\
-0	-0\\
-0	-0\\
-0	-0\\
-0	-0\\
nan	nan\\
-0	-2\\
-0.0785196315181373	-1.99845807248145\\
-0.15691819145569	-1.99383466746626\\
-0.235074794915675	-1.98613691390985\\
-0.312868930080462	-1.97537668119028\\
-0.390180644032257	-1.96157056080646\\
-0.466890727711811	-1.94473984079535\\
-0.542880899730149	-1.92491047290729\\
-0.618033988749895	-1.90211303259031\\
-0.692234114154986	-1.87638267184497\\
-0.76536686473018	-1.84775906502257\\
-0.837319475074856	-1.81628634765016\\
-0.907980999479094	-1.78201304837674\\
-0.97724248299391	-1.74499201414559\\
-1.0449971294319	-1.70528032870818\\
-1.1111404660392	-1.66293922460509\\
-1.17557050458495	-1.61803398874989\\
-1.23818789861967	-1.57063386176149\\
-1.29889609666037	-1.52081193120006\\
-1.35760149106588	-1.46864501887137\\
-1.4142135623731	-1.41421356237309\\
-1.46864501887137	-1.35760149106588\\
-1.52081193120006	-1.29889609666037\\
-1.57063386176149	-1.23818789861967\\
-1.61803398874989	-1.17557050458495\\
-1.66293922460509	-1.1111404660392\\
-1.70528032870818	-1.0449971294319\\
-1.74499201414559	-0.97724248299391\\
-1.78201304837674	-0.907980999479093\\
-1.81628634765016	-0.837319475074855\\
-1.84775906502257	-0.76536686473018\\
-1.87638267184497	-0.692234114154986\\
-1.90211303259031	-0.618033988749894\\
-1.92491047290729	-0.542880899730149\\
-1.94473984079535	-0.466890727711811\\
-1.96157056080646	-0.390180644032257\\
-1.97537668119028	-0.312868930080462\\
-1.98613691390985	-0.235074794915674\\
-1.99383466746626	-0.156918191455692\\
-1.99845807248145	-0.0785196315181388\\
-2	-0\\
nan	nan\\
-0	-4\\
-0.157039263036275	-3.99691614496289\\
-0.31383638291138	-3.98766933493251\\
-0.470149589831351	-3.97227382781971\\
-0.625737860160924	-3.95075336238055\\
-0.780361288064513	-3.92314112161292\\
-0.933781455423622	-3.88947968159071\\
-1.0857617994603	-3.84982094581459\\
-1.23606797749979	-3.80422606518061\\
-1.38446822830997	-3.75276534368994\\
-1.53073372946036	-3.69551813004515\\
-1.67463895014971	-3.63257269530033\\
-1.81596199895819	-3.56402609675347\\
-1.95448496598782	-3.48998402829119\\
-2.0899942588638	-3.41056065741637\\
-2.22228093207841	-3.32587844921018\\
-2.35114100916989	-3.23606797749979\\
-2.47637579723934	-3.14126772352298\\
-2.59779219332073	-3.04162386240012\\
-2.71520298213177	-2.93729003774274\\
-2.82842712474619	-2.82842712474619\\
-2.93729003774274	-2.71520298213177\\
-3.04162386240012	-2.59779219332073\\
-3.14126772352298	-2.47637579723934\\
-3.23606797749979	-2.35114100916989\\
-3.32587844921018	-2.22228093207841\\
-3.41056065741637	-2.0899942588638\\
-3.48998402829119	-1.95448496598782\\
-3.56402609675347	-1.81596199895819\\
-3.63257269530033	-1.67463895014971\\
-3.69551813004515	-1.53073372946036\\
-3.75276534368994	-1.38446822830997\\
-3.80422606518061	-1.23606797749979\\
-3.84982094581459	-1.0857617994603\\
-3.88947968159071	-0.933781455423622\\
-3.92314112161292	-0.780361288064513\\
-3.95075336238055	-0.625737860160923\\
-3.97227382781971	-0.470149589831347\\
-3.98766933493251	-0.313836382911385\\
-3.99691614496289	-0.157039263036278\\
-4	-0\\
nan	nan\\
-0	-6\\
-0.235558894554412	-5.99537421744434\\
-0.47075457436707	-5.98150400239877\\
-0.705224384747026	-5.95841074172956\\
-0.938606790241385	-5.92613004357083\\
-1.17054193209677	-5.88471168241938\\
-1.40067218313543	-5.83421952238606\\
-1.62864269919045	-5.77473141872188\\
-1.85410196624969	-5.70633909777092\\
-2.07670234246496	-5.6291480155349\\
-2.29610059419054	-5.54327719506772\\
-2.51195842522457	-5.44885904295049\\
-2.72394299843728	-5.34603914513021\\
-2.93172744898173	-5.23497604243678\\
-3.13499138829569	-5.11584098612455\\
-3.33342139811761	-4.98881767381527\\
-3.52671151375484	-4.85410196624968\\
-3.71456369585901	-4.71190158528447\\
-3.8966882899811	-4.56243579360019\\
-4.07280447319765	-4.40593505661411\\
-4.24264068711929	-4.24264068711928\\
-4.40593505661411	-4.07280447319765\\
-4.56243579360019	-3.8966882899811\\
-4.71190158528447	-3.714563695859\\
-4.85410196624968	-3.52671151375484\\
-4.98881767381527	-3.33342139811761\\
-5.11584098612455	-3.13499138829569\\
-5.23497604243678	-2.93172744898173\\
-5.34603914513021	-2.72394299843728\\
-5.44885904295049	-2.51195842522457\\
-5.54327719506772	-2.29610059419054\\
-5.6291480155349	-2.07670234246496\\
-5.70633909777092	-1.85410196624968\\
-5.77473141872188	-1.62864269919045\\
-5.83421952238606	-1.40067218313543\\
-5.88471168241938	-1.17054193209677\\
-5.92613004357083	-0.938606790241385\\
-5.95841074172956	-0.705224384747021\\
-5.98150400239877	-0.470754574367077\\
-5.99537421744434	-0.235558894554416\\
-6	-0\\
nan	nan\\
-0	-8\\
-0.314078526072549	-7.99383228992578\\
-0.62767276582276	-7.97533866986502\\
-0.940299179662701	-7.94454765563941\\
-1.25147572032185	-7.9015067247611\\
-1.56072257612903	-7.84628224322584\\
-1.86756291084724	-7.77895936318141\\
-2.17152359892059	-7.69964189162918\\
-2.47213595499958	-7.60845213036123\\
-2.76893645661994	-7.50553068737987\\
-3.06146745892072	-7.39103626009029\\
-3.34927790029942	-7.26514539060065\\
-3.63192399791637	-7.12805219350694\\
-3.90896993197564	-6.97996805658238\\
-4.17998851772759	-6.82112131483274\\
-4.44456186415682	-6.65175689842036\\
-4.70228201833979	-6.47213595499958\\
-4.95275159447867	-6.28253544704596\\
-5.19558438664147	-6.08324772480025\\
-5.43040596426353	-5.87458007548549\\
-5.65685424949238	-5.65685424949238\\
-5.87458007548549	-5.43040596426353\\
-6.08324772480025	-5.19558438664147\\
-6.28253544704596	-4.95275159447867\\
-6.47213595499958	-4.70228201833979\\
-6.65175689842036	-4.44456186415682\\
-6.82112131483274	-4.17998851772759\\
-6.97996805658238	-3.90896993197564\\
-7.12805219350694	-3.63192399791637\\
-7.26514539060065	-3.34927790029942\\
-7.39103626009029	-3.06146745892072\\
-7.50553068737987	-2.76893645661994\\
-7.60845213036123	-2.47213595499958\\
-7.69964189162918	-2.17152359892059\\
-7.77895936318141	-1.86756291084724\\
-7.84628224322584	-1.56072257612903\\
-7.9015067247611	-1.25147572032185\\
-7.94454765563941	-0.940299179662694\\
-7.97533866986502	-0.62767276582277\\
-7.99383228992578	-0.314078526072555\\
-8	-0\\
nan	nan\\
-0	-10\\
-0.392598157590687	-9.99229036240723\\
-0.78459095727845	-9.96917333733128\\
-1.17537397457838	-9.93068456954926\\
-1.56434465040231	-9.87688340595138\\
-1.95090322016128	-9.8078528040323\\
-2.33445363855905	-9.72369920397677\\
-2.71440449865074	-9.62455236453647\\
-3.09016994374947	-9.51056516295153\\
-3.46117057077493	-9.38191335922484\\
-3.8268343236509	-9.23879532511287\\
-4.18659737537428	-9.08143173825081\\
-4.53990499739547	-8.91006524188368\\
-4.88621241496955	-8.72496007072797\\
-5.22498564715949	-8.52640164354092\\
-5.55570233019602	-8.31469612302545\\
-5.87785252292473	-8.09016994374947\\
-6.19093949309834	-7.85316930880745\\
-6.49448048330184	-7.60405965600031\\
-6.78800745532942	-7.34322509435686\\
-7.07106781186548	-7.07106781186547\\
-7.34322509435686	-6.78800745532942\\
-7.60405965600031	-6.49448048330184\\
-7.85316930880745	-6.19093949309834\\
-8.09016994374947	-5.87785252292473\\
-8.31469612302545	-5.55570233019602\\
-8.52640164354092	-5.22498564715949\\
-8.72496007072797	-4.88621241496955\\
-8.91006524188368	-4.53990499739547\\
-9.08143173825082	-4.18659737537428\\
-9.23879532511287	-3.8268343236509\\
-9.38191335922484	-3.46117057077493\\
-9.51056516295154	-3.09016994374947\\
-9.62455236453647	-2.71440449865074\\
-9.72369920397677	-2.33445363855905\\
-9.8078528040323	-1.95090322016128\\
-9.87688340595138	-1.56434465040231\\
-9.93068456954926	-1.17537397457837\\
-9.96917333733128	-0.784590957278462\\
-9.99229036240723	-0.392598157590694\\
-10	-0\\
nan	nan\\
-0	-12\\
-0.471117789108824	-11.9907484348887\\
-0.94150914873414	-11.9630080047975\\
-1.41044876949405	-11.9168214834591\\
-1.87721358048277	-11.8522600871417\\
-2.34108386419354	-11.7694233648388\\
-2.80134436627087	-11.6684390447721\\
-3.25728539838089	-11.5494628374438\\
-3.70820393249937	-11.4126781955418\\
-4.15340468492992	-11.2582960310698\\
-4.59220118838108	-11.0865543901354\\
-5.02391685044913	-10.897718085901\\
-5.44788599687456	-10.6920782902604\\
-5.86345489796346	-10.4699520848736\\
-6.26998277659139	-10.2316819722491\\
-6.66684279623522	-9.97763534763054\\
-7.05342302750968	-9.70820393249937\\
-7.42912739171801	-9.42380317056894\\
-7.7933765799622	-9.12487158720037\\
-8.1456089463953	-8.81187011322823\\
-8.48528137423857	-8.48528137423857\\
-8.81187011322823	-8.1456089463953\\
-9.12487158720037	-7.7933765799622\\
-9.42380317056894	-7.42912739171801\\
-9.70820393249937	-7.05342302750968\\
-9.97763534763054	-6.66684279623522\\
-10.2316819722491	-6.26998277659139\\
-10.4699520848736	-5.86345489796346\\
-10.6920782902604	-5.44788599687456\\
-10.897718085901	-5.02391685044913\\
-11.0865543901354	-4.59220118838108\\
-11.2582960310698	-4.15340468492991\\
-11.4126781955418	-3.70820393249937\\
-11.5494628374438	-3.25728539838089\\
-11.6684390447721	-2.80134436627087\\
-11.7694233648388	-2.34108386419354\\
-11.8522600871417	-1.87721358048277\\
-11.9168214834591	-1.41044876949404\\
-11.9630080047975	-0.941509148734155\\
-11.9907484348887	-0.471117789108833\\
-12	-0\\
nan	nan\\
-0	-14\\
-0.549637420626961	-13.9892065073701\\
-1.09842734018983	-13.9568426722638\\
-1.64552356440973	-13.902958397369\\
-2.19008251056323	-13.8276367683319\\
-2.7312645082258	-13.7309939256452\\
-3.26823509398268	-13.6131788855675\\
-3.80016629811104	-13.4743733103511\\
-4.32623792124927	-13.3147912281321\\
-4.8456387990849	-13.1346787029148\\
-5.35756805311126	-12.934313455158\\
-5.86123632552399	-12.7140044335511\\
-6.35586699635366	-12.4740913386371\\
-6.84069738095737	-12.2149440990192\\
-7.31497990602328	-11.9369623009573\\
-7.77798326227443	-11.6405745722356\\
-8.22899353209462	-11.3262379212493\\
-8.66731529033768	-10.9944370323304\\
-9.09227267662257	-10.6456835184004\\
-9.50321043746118	-10.2805151320996\\
-9.89949493661167	-9.89949493661166\\
-10.2805151320996	-9.50321043746118\\
-10.6456835184004	-9.09227267662257\\
-10.9944370323304	-8.66731529033768\\
-11.3262379212493	-8.22899353209462\\
-11.6405745722356	-7.77798326227443\\
-11.9369623009573	-7.31497990602328\\
-12.2149440990192	-6.84069738095737\\
-12.4740913386371	-6.35586699635365\\
-12.7140044335511	-5.86123632552399\\
-12.934313455158	-5.35756805311126\\
-13.1346787029148	-4.8456387990849\\
-13.3147912281322	-4.32623792124926\\
-13.4743733103511	-3.80016629811104\\
-13.6131788855675	-3.26823509398268\\
-13.7309939256452	-2.7312645082258\\
-13.8276367683319	-2.19008251056323\\
-13.902958397369	-1.64552356440972\\
-13.9568426722638	-1.09842734018985\\
-13.9892065073701	-0.549637420626972\\
-14	-0\\
nan	nan\\
};
\end{axis}
\end{tikzpicture}%}
  \caption{Root locus of the open-loop system with the use of a zero-order hold.}
  \label{fig:Q7.rloc_F_ZOH_G}
\end{figure}

Figures \ref{fig:Q7.1}-\ref{fig:Q7.8} illustrate the step response when between
the continuous controller and the plant a zero-order hold has been inserted, for
sampling time varying between $1$ and $8$ seconds. The values of the $\chi$,
$\zeta$ and $\omega_0$ parameters were chosen to be the ones giving the best
performance among the three sets,
hence $(\chi, \zeta, \omega_0) \equiv (0.5, 0.8, 0.2)$.

Here, up until $h = 5$ sec, as the sampling time increases, so do the rise
time, settling time and overshoot. However, increasing $h$ beyond $7$ sec
make the system critically stable, since two conjugate poles are approaching
$0$.

\begin{figure}[H]\centering
	\centering
	\scalebox{1}{% This file was created by matlab2tikz.
%
%The latest updates can be retrieved from
%  http://www.mathworks.com/matlabcentral/fileexchange/22022-matlab2tikz-matlab2tikz
%where you can also make suggestions and rate matlab2tikz.
%
\definecolor{mycolor1}{rgb}{0.00000,0.44700,0.74100}%
%
\begin{tikzpicture}

\begin{axis}[%
width=4.133in,
height=3.26in,
at={(0.693in,0.44in)},
scale only axis,
xmin=0,
xmax=200,
xmajorgrids,
ymin=34,
ymax=54,
ymajorgrids,
axis background/.style={fill=white}
]
\addplot [color=mycolor1,solid,forget plot]
  table[row sep=crcr]{%
0	40\\
0.0670825919132015	39.9988385752381\\
0.134165183826403	39.9953626337708\\
0.201247775739604	39.9895846264077\\
0.268330367652806	39.9815169394076\\
0.335412959566007	39.9711718946171\\
0.402495551479209	39.9585617525181\\
0.477501758366411	39.9417949287895\\
0.552507965253614	39.9222285268844\\
0.627514172140816	39.8998794219546\\
0.705588818114703	39.8736783197704\\
0.78366346408859	39.8444992018444\\
0.861738110062477	39.8123608029845\\
0.907825406708318	39.7920081211627\\
0.953912703354159	39.770634588136\\
1	39.7482440111656\\
1.08628811745454	39.7039295742489\\
1.17257623490908	39.6567445984562\\
1.25886435236362	39.6067133696279\\
1.35835823925136	39.5455238123996\\
1.45785212613911	39.4806192776484\\
1.55734601302685	39.4120365944279\\
1.66075710056335	39.3368949100915\\
1.76416818809986	39.2578604524095\\
1.86757927563636	39.1749741449114\\
1.91171951709091	39.1384318452781\\
1.95585975854545	39.1011983588226\\
2	39.0632768467846\\
2.09120259149091	38.9835649390049\\
2.18240518298182	38.9025662449768\\
2.27360777447273	38.8203000247858\\
2.36481036596365	38.7367854528032\\
2.45601295745456	38.6520416144975\\
2.54721554894547	38.5660875103945\\
2.64534117737806	38.4722788735721\\
2.74346680581065	38.3771145989071\\
2.84159243424324	38.2806180184845\\
2.89439495616216	38.2281488819011\\
2.94719747808108	38.1753042869428\\
3	38.1220878281964\\
3.08537191920714	38.0360365409306\\
3.17074383841428	37.9505603167535\\
3.25611575762143	37.8656560307849\\
3.34148767682857	37.7813205668404\\
3.42685959603571	37.6975508176525\\
3.51223151524285	37.6143436847962\\
3.60532541949373	37.5242479986754\\
3.69841932374461	37.4348136431963\\
3.79151322799549	37.3460366374596\\
3.86100881866366	37.280189811428\\
3.93050440933183	37.2147054544706\\
4	37.149581924632\\
4.11108402780871	37.047231504434\\
4.22216805561743	36.9477608925404\\
4.33325208342614	36.8511175405079\\
4.42794224446606	36.7709302983874\\
4.52263240550599	36.6927284489292\\
4.61732256654591	36.6164810331743\\
4.72233976063703	36.5341668196466\\
4.82735695472815	36.4541781690356\\
4.93237414881926	36.3764748288743\\
4.95491609921284	36.3600898964094\\
4.97745804960642	36.3438080451803\\
5	36.3276288890659\\
5.11270975196789	36.248845746864\\
5.22541950393579	36.1737224664249\\
5.33812925590368	36.1021805242655\\
5.446603627054	36.0366373464052\\
5.55507799820431	35.9742732956905\\
5.66355236935463	35.9150227911334\\
5.77570157956975	35.8569703169124\\
5.88785078978488	35.8021078299113\\
6	35.7503673378685\\
6.13144100486509	35.6939069047617\\
6.26288200973019	35.6421181208389\\
6.39432301459528	35.5948797842423\\
6.52375464133709	35.5526954664153\\
6.65318626807889	35.5146992908606\\
6.78261789482069	35.4807842395802\\
6.85507859654712	35.4635402785782\\
6.92753929827356	35.4475247359181\\
7	35.4327199218714\\
7.16457588784749	35.4034882407397\\
7.32915177569498	35.3801891213667\\
7.49372766354247	35.3626300207972\\
7.66248510902831	35.3503900972095\\
7.83124255451416	35.3437946782391\\
8	35.3426550648257\\
8.20742294650158	35.3480897038036\\
8.41484589300315	35.3604469792504\\
8.62226883950473	35.3794449630615\\
8.74817922633648	35.3941006581158\\
8.87408961316824	35.4110441327959\\
9	35.4302177737987\\
9.16091180611379	35.4575919587937\\
9.32182361222757	35.4878424431611\\
9.48273541834136	35.5208767960452\\
9.65515694556091	35.5592615944973\\
9.82757847278045	35.6006320795458\\
10	35.6448838775606\\
10.1795818795046	35.6935836137047\\
10.3591637590093	35.744516517821\\
10.5387456385139	35.7975986030859\\
10.6924970923426	35.8446915262933\\
10.8462485461713	35.8932502767595\\
11	35.9432263360222\\
11.1806205947633	36.0033817551058\\
11.3612411895266	36.0647650507104\\
11.5418617842898	36.1273236756191\\
11.6945745228599	36.1810954686871\\
11.8472872614299	36.2356409503858\\
12	36.290930854059\\
12.1846849401062	36.3584908268905\\
12.3693698802124	36.4265415673965\\
12.5540548203186	36.4950521699232\\
12.7027032135457	36.5505083185195\\
12.8513516067729	36.6062277563371\\
13	36.6621956641521\\
13.1921248483663	36.7346506470072\\
13.3842496967327	36.8070223307294\\
13.576374545099	36.8792960761916\\
13.717583030066	36.9323453776143\\
13.858791515033	36.9853287135197\\
14	37.0382408034469\\
14.201821249267	37.1135293619717\\
14.4036424985341	37.1882528434436\\
14.6054637478011	37.2624106639627\\
14.7369758318674	37.3104292710988\\
14.8684879159337	37.3582074356561\\
15	37.4057451114208\\
15.2129644047453	37.4820339237975\\
15.4259288094907	37.5573378463602\\
15.638893214236	37.6316695251297\\
15.759262142824	37.6732569640421\\
15.879631071412	37.71454002644\\
16	37.7555209229447\\
16.2250935082959	37.8311895915911\\
16.4501870165918	37.9055042235078\\
16.6752805248877	37.9784903958166\\
16.7835203499251	38.0131213647583\\
16.8917601749626	38.0474536907559\\
17	38.0814900876962\\
17.2380615901245	38.1551745175648\\
17.4761231802489	38.2271821425647\\
17.7141847703734	38.2975513399227\\
17.8094565135823	38.3252623192407\\
17.9047282567911	38.3527192601107\\
18	38.3799245002212\\
18.252026091304	38.4505647504476\\
18.5040521826079	38.5192478329826\\
18.7560782739119	38.5860249089952\\
18.8373855159412	38.6071697799916\\
18.9186927579706	38.6281231130622\\
19	38.6488865380231\\
19.2674437758336	38.7157501231683\\
19.5348875516672	38.780411493582\\
19.8023313275008	38.8429348479447\\
19.8682208850005	38.8580175516427\\
19.9341104425003	38.8729751955376\\
20	38.8878086896527\\
20.2850850877728	38.9504832474696\\
20.5701701755455	39.0107342014931\\
20.8552552633183	39.0686396222405\\
20.9035035088789	39.0782130059521\\
20.9517517544394	39.0877217414587\\
21	39.0971661873196\\
21.2412412278028	39.1433868788165\\
21.4824824556057	39.1879562937511\\
21.7237236834085	39.2309208496473\\
21.815815788939	39.2469085097937\\
21.9079078944695	39.2626713699055\\
22	39.2782118758794\\
22.2664844577584	39.3219105138801\\
22.5329689155167	39.3637471466299\\
22.7994533732751	39.4037809220672\\
22.8663022488501	39.4135475306999\\
22.933151124425	39.4232052032538\\
23	39.4327548238569\\
23.3333333333333	39.4787541812901\\
23.6666666666667	39.5221249817807\\
24	39.5629735579325\\
24.292141752894	39.5967694234001\\
24.584283505788	39.6287504318333\\
24.876425258682	39.6589822255406\\
24.9176168391213	39.6631078355195\\
24.9588084195607	39.6672001120186\\
25	39.6712592291757\\
25.2059579021967	39.6910636486322\\
25.4119158043933	39.7100651125171\\
25.61787370659	39.7282845088497\\
25.7452491377266	39.7391699008002\\
25.8726245688633	39.7497687242706\\
26	39.7600856549505\\
26.3333333333333	39.7857957212109\\
26.6666666666667	39.8097103879323\\
27	39.8319060941284\\
27.3333333333333	39.8524763762326\\
27.6666666666667	39.8715106544135\\
28	39.8890753241677\\
28.3333333333333	39.905259029482\\
28.6666666666667	39.9201463089155\\
29	39.9337940420656\\
29.3333333333333	39.9462841139504\\
29.6666666666667	39.9576944630473\\
30	39.9680731733187\\
30.3333333333333	39.9774942814516\\
30.6666666666667	39.9860281375646\\
31	39.9937148639681\\
31.3333333333333	40.0006201804822\\
31.6666666666667	40.006806442233\\
32	40.0123066939301\\
32.3333333333333	40.0171783392735\\
32.6666666666667	40.0214757699242\\
33	40.0252258384477\\
33.3333333333333	40.0284779660328\\
33.6666666666667	40.0312789217942\\
34	40.0336502180421\\
34.3333333333333	40.0356338446133\\
34.6666666666667	40.0372694898393\\
35	40.0385741208683\\
35.3333333333333	40.0395829728116\\
35.6666666666667	40.0403293093783\\
36	40.0408262771402\\
36.3333333333333	40.0411030889515\\
36.6666666666667	40.0411872892772\\
37	40.0410888517796\\
37.3333333333333	40.0408317056022\\
37.6666666666667	40.040438384133\\
38	40.0399162577539\\
38.3333333333333	40.0392846831515\\
38.6666666666667	40.0385618625955\\
39	40.0377530572705\\
39.3333333333333	40.036873714753\\
39.6666666666667	40.0359383389296\\
40	40.0349505050949\\
40.3333333333333	40.0339223559536\\
40.6666666666667	40.0328652723279\\
41	40.0317815022028\\
41.3333333333333	40.0306804239406\\
41.6666666666667	40.0295708091743\\
42	40.0284538790552\\
42.3333333333333	40.0273367242065\\
42.6666666666667	40.0262259595633\\
43	40.025122028527\\
43.3333333333333	40.0240301491854\\
43.6666666666667	40.0229551734827\\
44	40.0218969712612\\
44.3333333333333	40.0208592459845\\
44.6666666666667	40.019845425584\\
45	40.0188549717435\\
45.3333333333333	40.0178903788275\\
45.6666666666667	40.0169539397829\\
46	40.0160448403542\\
46.3333333333333	40.015164624153\\
46.6666666666667	40.0143146932531\\
47	40.0134940615711\\
47.3333333333333	40.0127035383269\\
47.6666666666667	40.0119438389184\\
48	40.011213885969\\
48.3333333333333	40.0105139336854\\
48.6666666666667	40.0098441798736\\
49	40.0092035168248\\
49.3333333333333	40.0085917908372\\
49.6666666666667	40.0080088206044\\
50	40.0074535129747\\
50.3333333333333	40.0069254255304\\
50.6666666666667	40.0064241098964\\
51	40.0059485193302\\
51.3333333333333	40.0054980179707\\
51.6666666666667	40.005071979789\\
52	40.0046694258729\\
52.3333333333333	40.004289601884\\
52.6666666666667	40.003931774369\\
53	40.0035950454491\\
53.3333333333333	40.00327860021\\
53.6666666666667	40.0029816518995\\
54	40.0027033905208\\
54.3333333333333	40.0024429842983\\
54.6666666666667	40.0021996339342\\
55	40.0019726193591\\
55.3333333333333	40.0017611239929\\
55.6666666666667	40.0015643657744\\
56	40.0013817130773\\
56.3333333333333	40.0012123871192\\
56.6666666666667	40.0010556439708\\
57	40.0009109364704\\
57.3333333333333	40.000777538677\\
57.6666666666667	40.0006547585959\\
58	40.0005421278854\\
58.3333333333333	40.0004389825456\\
58.6666666666667	40.0003446907522\\
59	40.0002588563247\\
59.3333333333333	40.0001808817207\\
59.6666666666667	40.000110199239\\
60	40.0000464776897\\
60.3333333333333	39.9999891870527\\
60.6666666666667	39.9999378244855\\
61	39.9998921164861\\
61.3333333333333	39.9998515990998\\
61.6666666666667	39.9998158327365\\
62	39.9997845944251\\
62.3333333333333	39.9997574830505\\
62.6666666666667	39.9997341190349\\
63	39.9997143231213\\
63.3333333333333	39.9996977526417\\
63.6666666666667	39.9996840837201\\
64	39.9996731744669\\
64.3333333333333	39.9996647355727\\
64.6666666666667	39.9996584939251\\
65	39.9996543392025\\
65.3333333333333	39.9996520300326\\
65.6666666666667	39.9996513388387\\
66	39.9996521816529\\
66.3333333333333	39.999654359558\\
66.6666666666667	39.9996576852491\\
67	39.9996620964911\\
67.3333333333333	39.9996674314763\\
67.6666666666667	39.9996735380561\\
68	39.9996803716856\\
68.3333333333333	39.9996878025971\\
68.6666666666667	39.9996957089566\\
69	39.9997040604184\\
69.3333333333333	39.9997127545489\\
69.6666666666667	39.9997216953461\\
70	39.999730863682\\
70.3333333333333	39.9997401801704\\
70.6666666666667	39.999749570563\\
71	39.999759024435\\
71.3333333333333	39.999768481602\\
71.6666666666667	39.9997778859181\\
72	39.9997872335646\\
72.3333333333333	39.9997964801593\\
72.6666666666667	39.9998055844337\\
73	39.9998145474486\\
73.3333333333333	39.9998233376564\\
73.6666666666667	39.999831925855\\
74	39.9998403165814\\
74.3333333333333	39.9998484885638\\
74.6666666666667	39.9998564222457\\
75	39.9998641245162\\
75.3333333333333	39.999871582199\\
75.6666666666667	39.9998787833233\\
76	39.9998857362436\\
76.3333333333333	39.9998924340435\\
76.6666666666667	39.9998988706056\\
77	39.9999050550634\\
77.3333333333333	39.9999109852331\\
77.6666666666667	39.9999166594108\\
78	39.9999220869894\\
78.3333333333333	39.9999272692629\\
78.6666666666667	39.9999322077594\\
79	39.9999369117489\\
79.3333333333333	39.999941384985\\
79.6666666666667	39.9999456312714\\
80	39.9999496594838\\
80.3333333333333	39.9999534750221\\
80.6666666666667	39.9999570832019\\
81	39.9999604923228\\
81.3333333333333	39.9999637087911\\
81.6666666666667	39.9999667388353\\
82	39.9999695900665\\
82.3333333333333	39.9999722694043\\
82.6666666666667	39.9999747835284\\
83	39.9999771393039\\
83.3333333333333	39.9999793437912\\
83.6666666666667	39.9999814037749\\
84	39.9999833253565\\
84.3333333333333	39.9999851154654\\
84.6666666666667	39.9999867807375\\
85	39.999988326524\\
85.3333333333333	39.9999897594314\\
85.6666666666667	39.99999108577\\
86	39.9999923101745\\
86.3333333333333	39.9999934388011\\
86.6666666666667	39.999994477518\\
87	39.9999954302914\\
87.3333333333333	39.9999963027507\\
87.6666666666667	39.9999971002521\\
88	39.9999978261503\\
88.3333333333333	39.9999984855098\\
88.6666666666667	39.999999083142\\
89	39.9999996218508\\
89.3333333333333	40.0000001061274\\
89.6666666666667	40.0000005402323\\
90	40.0000009264808\\
90.3333333333333	40.0000012688033\\
90.6666666666667	40.0000015709238\\
91	40.0000018347292\\
91.3333333333333	40.000002063617\\
91.6666666666667	40.0000022608021\\
92	40.0000024278014\\
92.3333333333333	40.0000025675166\\
92.6666666666667	40.0000026826905\\
93	40.0000027745233\\
93.3333333333333	40.0000028454647\\
93.6666666666667	40.0000028978271\\
94	40.0000029325434\\
94.3333333333333	40.0000029516572\\
94.6666666666667	40.0000029570948\\
95	40.0000029495662\\
95.3333333333333	40.0000029307553\\
95.6666666666667	40.0000029022472\\
96	40.000002864568\\
96.3333333333333	40.0000028190871\\
96.6666666666667	40.000002767092\\
97	40.0000027089591\\
97.3333333333333	40.0000026457864\\
97.6666666666667	40.0000025786047\\
98	40.0000025076703\\
98.3333333333333	40.0000024338496\\
98.6666666666667	40.0000023579548\\
99	40.0000022801472\\
99.3333333333333	40.0000022010982\\
99.6666666666667	40.0000021214358\\
99.9999999999991	40.0000020412478\\
100	40.0000020412478\\
100.000000000001	40.0000020412478\\
100.051963134241	40.0012071941124\\
100.103926268481	40.0047979054622\\
100.155889402721	40.0107376371509\\
100.207582010479	40.0189417720223\\
100.259274618237	40.0294007855491\\
100.310967225995	40.0420810183344\\
100.364545577105	40.0575328864236\\
100.418123928215	40.0752997115387\\
100.471702279325	40.0953463648469\\
100.527190248097	40.1184740094981\\
100.582678216869	40.1439726452724\\
100.638166185641	40.171805691924\\
100.697156973014	40.2039158030199\\
100.756147760387	40.2385818085934\\
100.81513854776	40.2757625334927\\
100.87675903184	40.3172424335724\\
100.93837951592	40.3613774074291\\
101	40.4081235394367\\
101.067110212809	40.4619544254583\\
101.134220425618	40.5187772874578\\
101.201330638428	40.578539147783\\
101.272378103611	40.6449523583619\\
101.343425568794	40.7145415732072\\
101.414473033977	40.7872482389985\\
101.4896356513	40.8674962765185\\
101.564798268624	40.9511033170341\\
101.639960885947	41.0380048889285\\
101.719366714915	41.1333215416946\\
101.798772543883	41.2321726397746\\
101.878178372851	41.3344876104923\\
101.918785581901	41.3881265776738\\
101.95939279095	41.4426443498277\\
102	41.4980320056366\\
102.089473927374	41.6230997458632\\
102.178947854748	41.7522557469137\\
102.268421782122	41.8854100702741\\
102.359159714698	42.0244391016579\\
102.449897647273	42.1674012978546\\
102.540635579849	42.3142100916463\\
102.636922367988	42.4741092200228\\
102.733209156126	42.6381468478399\\
102.829495944265	42.8062275777354\\
102.88633062951	42.9072995546425\\
102.943165314755	43.0097291317724\\
103	43.1134979213554\\
103.109896663079	43.316297966749\\
103.219793326159	43.5208025541632\\
103.329689989238	43.7269664246864\\
103.430663309574	43.9178150211604\\
103.53163662991	44.1099935392874\\
103.632609950246	44.3034691719732\\
103.741771153587	44.5140557316533\\
103.850932356927	44.7260811048192\\
103.960093560268	44.9395065106615\\
103.973395706845	44.9656079217598\\
103.986697853423	44.991729490874\\
104	45.0178711504889\\
104.066510732887	45.1484183775866\\
104.133021465774	45.2785446059912\\
104.199532198661	45.4082518787554\\
104.311106968414	45.6249063631952\\
104.422681738167	45.8403969770318\\
104.53425650792	46.0547330283892\\
104.649748073758	46.2753869801891\\
104.765239639595	46.4948237352648\\
104.880731205433	46.7130531686734\\
104.920487470288	46.7878978555899\\
104.960243735144	46.8626010245585\\
105	46.9371630665723\\
105.138086328167	47.1937522729162\\
105.276172656335	47.4460943217876\\
105.414258984502	47.6942501941458\\
105.533996305177	47.9060885883234\\
105.653733625852	48.1148629324407\\
105.773470946527	48.320611108592\\
105.848980631018	48.4488225494591\\
105.924490315509	48.5758546156523\\
106	48.7017164738036\\
106.145869071484	48.9406175887658\\
106.291738142968	49.1733536351734\\
106.437607214451	49.4000163819098\\
106.567893677114	49.5974045496235\\
106.698180139776	49.790083140744\\
106.828466602438	49.9781146755886\\
106.885644401626	50.0591832399142\\
106.942822200813	50.1393738291211\\
107	50.2186915735267\\
107.161315303939	50.4370604927872\\
107.322630607877	50.6471520966493\\
107.483945911816	50.8490984185137\\
107.625320439701	51.0194867449029\\
107.766694967587	51.183805207383\\
107.908069495472	51.3421389515227\\
107.938712996982	51.37567744828\\
107.969356498491	51.4089395710241\\
108	51.4419261666324\\
108.153217507546	51.6023737537493\\
108.306435015092	51.7553009264248\\
108.459652522638	51.9008199211689\\
108.61529485509	52.0411709652706\\
108.770937187542	52.1741064516232\\
108.926579519995	52.2997395780725\\
108.95105301333	52.3188373063359\\
108.975526506665	52.3377576817126\\
109	52.3565011348233\\
109.122367466676	52.4474890465803\\
109.244734933351	52.5339380274203\\
109.367102400027	52.615902596561\\
109.544788176568	52.7270476737764\\
109.722473953108	52.8290146768853\\
109.900159729648	52.9219650017972\\
109.933439819766	52.9383847553071\\
109.966719909883	52.9544948344093\\
110	52.9702962755152\\
110.166400450586	53.0447765910527\\
110.332800901172	53.1118270708501\\
110.499201351758	53.1715729486856\\
110.666134234505	53.2242948297214\\
110.833067117253	53.2699134440728\\
111	53.3085509824856\\
111.234867785976	53.3518228519423\\
111.469735571953	53.3828321061869\\
111.704603357929	53.4018874036236\\
111.803068905286	53.4063928969054\\
111.901534452643	53.4088729712652\\
112	53.4093496328592\\
112.179485671192	53.40557855597\\
112.358971342385	53.396230761328\\
112.538457013577	53.3814220854685\\
112.692304675718	53.3644689584944\\
112.846152337859	53.3436589530597\\
113	53.3190625816225\\
113.17196082367	53.2876580219303\\
113.34392164734	53.2526618709667\\
113.51588247101	53.2141530347773\\
113.677254980674	53.174889859353\\
113.838627490337	53.1326654757092\\
114	53.0875427259343\\
114.198208114676	53.0289092661974\\
114.396416229352	52.9674191069537\\
114.594624344028	52.9031551327104\\
114.729749562685	52.8577959471382\\
114.864874781343	52.8112109902538\\
115	52.7634254472215\\
115.196808560005	52.6923425160877\\
115.39361712001	52.6200471366407\\
115.590425680016	52.5465846724208\\
115.726950453344	52.494962160307\\
115.863475226672	52.4428140823709\\
116	52.3901548585875\\
116.195603361127	52.314366161237\\
116.391206722254	52.2386211706076\\
116.586810083381	52.1629336199074\\
116.724540055588	52.1096815426631\\
116.862270027794	52.0564691505467\\
117	52.0033009253253\\
117.198623631288	51.9271392141941\\
117.397247262576	51.8519291106972\\
117.595870893864	51.7776597070607\\
117.730580595909	51.7278188724189\\
117.865290297955	51.6784024213198\\
118	51.6294070310734\\
118.206129159229	51.555582161134\\
118.412258318458	51.4833880971594\\
118.618387477688	51.4127931756319\\
118.745591651792	51.3700127630368\\
118.872795825896	51.3278222783187\\
119	51.2862145938467\\
119.218331905276	51.2163963527129\\
119.436663810552	51.1487448602131\\
119.654995715829	51.0832091113537\\
119.769997143886	51.049523377478\\
119.884998571943	51.0164035818065\\
120	50.9838426204225\\
120.23607336248	50.9189086617034\\
120.47214672496	50.8566022417263\\
120.70822008744	50.7968519963945\\
120.805480058293	50.7729626091593\\
120.902740029147	50.7494904589936\\
121	50.7264308016748\\
121.261351857623	50.6665918128682\\
121.522703715245	50.609838728905\\
121.784055572868	50.5560752077286\\
121.856037048579	50.5417801965931\\
121.928018524289	50.5277029098029\\
122	50.5138414366912\\
122.298877498108	50.4586077000813\\
122.597754996216	50.4070340485279\\
122.896632494324	50.3589866422946\\
122.931088329549	50.353668310549\\
122.965544164775	50.3483949209929\\
123	50.3431662791411\\
123.172279176127	50.3176833340002\\
123.344558352254	50.2932820737232\\
123.516837528382	50.2699391873494\\
123.677891685588	50.2490541534639\\
123.838945842794	50.2290556907784\\
124	50.2099257400467\\
124.333333333333	50.1729493692338\\
124.666666666667	50.1393368753542\\
125	50.1089467048245\\
125.333333333333	50.0815491540078\\
125.666666666667	50.0569255535634\\
126	50.0349576966831\\
126.300563509414	50.0172469076872\\
126.601127018828	50.0013571103649\\
126.901690528242	49.9872174086448\\
126.934460352161	49.9857786532544\\
126.967230176081	49.9843597959495\\
127	49.9829607496714\\
127.163849119596	49.9762340981721\\
127.327698239193	49.9699395431447\\
127.491547358789	49.9640677597506\\
127.661031572526	49.9584291157092\\
127.830515786263	49.9532230608176\\
128	49.9484398142544\\
128.333333333333	49.9401297642134\\
128.666666666667	49.9331550927635\\
129	49.9274565683217\\
129.285127234098	49.9234856802966\\
129.570254468197	49.9202439106073\\
129.855381702296	49.9177030575764\\
129.90358780153	49.9173407811452\\
129.951793900765	49.9169976265762\\
130	49.9166734647856\\
130.241030496174	49.9152915922272\\
130.482060992348	49.9142845241949\\
130.723091488522	49.9136397037854\\
130.815394325682	49.9134861093265\\
130.907697162841	49.9133831673092\\
131	49.9133302099546\\
131.333333333333	49.9134773259392\\
131.666666666667	49.9141064835112\\
132	49.9151946505352\\
132.333333333333	49.9166630984322\\
132.666666666667	49.9184375146639\\
133	49.9205025260823\\
133.333333333333	49.9227970072683\\
133.666666666667	49.9252633274626\\
134	49.927892036307\\
134.333333333333	49.9306368907076\\
134.666666666667	49.9334543508134\\
135	49.9363394303689\\
135.333333333333	49.9392582049348\\
135.666666666667	49.9421787878334\\
136	49.9450994557379\\
136.333333333333	49.9479962730723\\
136.666666666667	49.9508467934788\\
137	49.9536515867508\\
137.333333333333	49.9563946471231\\
137.666666666667	49.9590610147793\\
138	49.9616527855397\\
138.333333333333	49.9641601058557\\
138.666666666667	49.9665738173116\\
139	49.9688969485051\\
139.333333333333	49.9711242957247\\
139.666666666667	49.9732510790313\\
140	49.9752808127291\\
140.333333333333	49.9772116987911\\
140.666666666667	49.9790421585241\\
141	49.9807758654416\\
141.333333333333	49.9824134178611\\
141.666666666667	49.9839554842547\\
142	49.9854056683604\\
142.333333333333	49.9867661635284\\
142.666666666667	49.9880391287962\\
143	49.9892279463589\\
143.333333333333	49.990335783358\\
143.666666666667	49.991365703555\\
144	49.9923207755708\\
144.333333333333	49.9932046716009\\
144.666666666667	49.9940209187602\\
145	49.9947722250422\\
145.333333333333	49.995462425388\\
145.666666666667	49.9960951885377\\
146	49.9966728475794\\
146.333333333333	49.9971991604521\\
146.666666666667	49.9976777127413\\
147	49.9981104712453\\
147.333333333333	49.9985009584733\\
147.666666666667	49.9988525288556\\
148	49.9991668065869\\
148.333333333333	49.9994469832084\\
148.666666666667	49.9996960932947\\
149	49.9999154511516\\
149.333333333333	50.0001078690876\\
149.666666666667	50.0002760176276\\
150	50.0004209383968\\
150.333333333333	50.0005450505494\\
150.666666666667	50.0006506486372\\
151	50.0007385400182\\
151.333333333333	50.0008107606514\\
151.666666666667	50.0008692395745\\
152	50.0009145872155\\
152.333333333333	50.0009484819142\\
152.666666666667	50.0009725122353\\
153	50.0009871264503\\
153.333333333333	50.0009936801149\\
153.666666666667	50.0009934549474\\
154	50.0009867683996\\
154.333333333333	50.0009746925836\\
154.666666666667	50.0009582401141\\
155	50.0009376251019\\
155.333333333333	50.0009136766378\\
155.666666666667	50.0008871768717\\
156	50.0008582601052\\
156.333333333333	50.0008275515515\\
156.666666666667	50.0007956402215\\
157	50.0007626003256\\
157.333333333333	50.0007288895429\\
157.666666666667	50.0006949383372\\
158	50.0006607770128\\
158.333333333333	50.0006267283746\\
158.666666666667	50.0005930953799\\
159	50.0005598774416\\
159.333333333333	50.0005272910587\\
159.666666666667	50.0004955387997\\
160	50.0004645994314\\
160.333333333333	50.0004346075536\\
160.666666666667	50.0004056884876\\
161	50.0003778082365\\
161.333333333333	50.0003510399065\\
161.666666666667	50.0003254508998\\
162	50.0003010003645\\
162.333333333333	50.0002777166259\\
162.666666666667	50.0002556249802\\
163	50.0002346820153\\
163.333333333333	50.0002148846844\\
163.666666666667	50.000196228852\\
164	50.0001786715467\\
164.333333333333	50.000162188881\\
164.666666666667	50.0001467572306\\
165	50.0001323360513\\
165.333333333333	50.0001188886962\\
165.666666666667	50.0001063796739\\
166	50.0000947720754\\
166.333333333333	50.000084022536\\
166.666666666667	50.0000740893857\\
167	50.0000649399787\\
167.333333333333	50.0000565286081\\
167.666666666667	50.0000488115379\\
168	50.0000417605946\\
168.333333333333	50.0000353307722\\
168.666666666667	50.0000294791234\\
169	50.0000241818643\\
169.333333333333	50.0000193966917\\
169.666666666667	50.0000150833168\\
170	50.0000112220729\\
170.333333333333	50.0000077745638\\
170.666666666667	50.0000047042791\\
171	50.0000019952818\\
171.333333333333	49.9999996136953\\
171.666666666667	49.9999975273484\\
172	49.9999957235901\\
172.333333333333	49.9999941732484\\
172.666666666667	49.9999928486513\\
173	49.999991739974\\
173.333333333333	49.9999908226389\\
173.666666666667	49.9999900733575\\
174	49.9999894846827\\
174.333333333333	49.9999890363305\\
174.666666666667	49.9999887091\\
175	49.9999884975044\\
175.333333333333	49.9999883851384\\
175.666666666667	49.9999883564884\\
176	49.9999884076512\\
176.333333333333	49.9999885256302\\
176.666666666667	49.9999886981482\\
177	49.9999889225553\\
177.333333333333	49.9999891887797\\
177.666666666667	49.9999894873186\\
178	49.9999898164918\\
178.333333333333	49.9999901686842\\
178.666666666667	49.9999905367203\\
179	49.9999909196519\\
179.333333333333	49.9999913118851\\
179.666666666667	49.9999917081574\\
180	49.9999921080574\\
180.333333333333	49.9999925076207\\
180.666666666667	49.9999929031257\\
181	49.9999932945405\\
181.333333333333	49.9999936791876\\
181.666666666667	49.9999940545606\\
182	49.9999944208832\\
182.333333333333	49.9999947764716\\
182.666666666667	49.9999951197568\\
183	49.9999954511223\\
183.333333333333	49.9999957696325\\
183.666666666667	49.9999960744234\\
184	49.9999963659663\\
184.333333333333	49.9999966438728\\
184.666666666667	49.9999969077932\\
185	49.9999971582342\\
185.333333333333	49.9999973951925\\
185.666666666667	49.99999761868\\
186	49.999997829202\\
186.333333333333	49.9999980270131\\
186.666666666667	49.9999982123663\\
187	49.9999983857409\\
187.333333333333	49.9999985475513\\
187.666666666667	49.9999986981994\\
188	49.9999988381233\\
188.333333333333	49.9999989678238\\
188.666666666667	49.9999990877824\\
189	49.999999198388\\
189.333333333333	49.9999993001743\\
189.666666666667	49.9999993936521\\
190	49.9999994791581\\
190.333333333333	49.9999995572215\\
190.666666666667	49.9999996283476\\
191	49.9999996928212\\
191.333333333333	49.9999997511424\\
191.666666666667	49.9999998037877\\
192	49.9999998509931\\
192.333333333333	49.9999998932145\\
192.666666666667	49.9999999308852\\
193	49.9999999641971\\
193.333333333333	49.9999999935538\\
193.666666666667	50.0000000193383\\
194	50.0000000417035\\
194.333333333333	50.0000000609978\\
194.666666666667	50.0000000775519\\
195	50.0000000914846\\
195.333333333333	50.0000001030905\\
195.666666666667	50.0000001126484\\
196	50.000000120249\\
196.333333333333	50.0000001261358\\
196.666666666667	50.0000001305394\\
197	50.0000001335269\\
197.333333333333	50.0000001352958\\
197.666666666667	50.0000001360329\\
198	50.0000001357862\\
198.333333333333	50.0000001347125\\
198.666666666667	50.0000001329603\\
199	50.0000001305622\\
199.333333333333	50.0000001276403\\
199.666666666667	50.0000001243096\\
200	50.0000001205913\\
};
\end{axis}
\end{tikzpicture}%}
  \caption{Step response using a zero-order hold of sample time $1$ sec.}
  \label{fig:Q7.1}
\end{figure}

\begin{figure}[H]\centering
	\centering
	\scalebox{1}{% This file was created by matlab2tikz.
%
%The latest updates can be retrieved from
%  http://www.mathworks.com/matlabcentral/fileexchange/22022-matlab2tikz-matlab2tikz
%where you can also make suggestions and rate matlab2tikz.
%
\definecolor{mycolor1}{rgb}{0.00000,0.44700,0.74100}%
%
\begin{tikzpicture}

\begin{axis}[%
width=4.133in,
height=3.26in,
at={(0.693in,0.44in)},
scale only axis,
xmin=0,
xmax=200,
xmajorgrids,
ymin=30,
ymax=55,
ymajorgrids,
axis background/.style={fill=white}
]
\addplot [color=mycolor1,solid,forget plot]
  table[row sep=crcr]{%
0	40\\
0.0666666666666667	39.9988529199354\\
0.133333333333333	39.9954198587329\\
0.2	39.9897130376372\\
0.266666666666667	39.9817446149091\\
0.333333333333333	39.9715266859614\\
0.4	39.959071286313\\
0.47499490925035	39.942400621861\\
0.5499898185007	39.9229306571585\\
0.62498472775105	39.9006782627291\\
0.703019200761635	39.8745881554063\\
0.78105367377222	39.8455224785902\\
0.859088146782806	39.8134999414055\\
0.940721928520819	39.7768560227102\\
1.02235571025883	39.7370177534163\\
1.10398949199685	39.6940062251002\\
1.18937445628243	39.6456457248368\\
1.27475942056802	39.5938604847577\\
1.36014438485361	39.5386742795039\\
1.44941284679236	39.4773672565188\\
1.53868130873111	39.4123956293716\\
1.62794977066986	39.3437861740105\\
1.72115687139107	39.2682963135632\\
1.81436397211228	39.1888997791019\\
1.90757107283349	39.1056266274356\\
1.93838071522233	39.0772532766076\\
1.96919035761116	39.0484607009771\\
2	39.0192499766143\\
2.10280158530272	38.9204150008745\\
2.20560317060544	38.8202527136271\\
2.30840475590817	38.7187881496068\\
2.39217645422804	38.6351602646607\\
2.47594815254792	38.5506975512143\\
2.55971985086779	38.4654133506129\\
2.65035248755473	38.3722345382189\\
2.74098512424168	38.2781264478673\\
2.83161776092862	38.1831057321341\\
2.92774506709594	38.0813454925111\\
3.02387237326326	37.9785969186242\\
3.11999967943058	37.8748795694164\\
3.22554318749687	37.7599098975372\\
3.33108669556316	37.6438214083989\\
3.43663020362945	37.5266395469625\\
3.55442732437873	37.3945927247898\\
3.67222444512802	37.2612503430385\\
3.7900215658773	37.1266470934711\\
3.86001437725153	37.0460853530923\\
3.93000718862577	36.9650978161219\\
4	36.8836916387702\\
4.084026152496	36.7870247096103\\
4.168052304992	36.6929479594034\\
4.25207845748801	36.6014191408825\\
4.33610460998401	36.5123967818476\\
4.42013076248001	36.4258401647491\\
4.50415691497601	36.341709300342\\
4.59533018726636	36.253120763005\\
4.68650345955671	36.1672927168842\\
4.77767673184706	36.0841768491583\\
4.87628569769649	35.9972806184676\\
4.97489466354593	35.9134427675073\\
5.07350362939537	35.8326056757701\\
5.18063700929633	35.7481157495863\\
5.28777038919728	35.6670302663071\\
5.39490376909824	35.5892798109623\\
5.51185402760203	35.5081324277794\\
5.62880428610583	35.4307918945455\\
5.74575454460962	35.3571737298582\\
5.83050302973975	35.3061051583374\\
5.91525151486987	35.2569170868271\\
6	35.2095792060385\\
6.11052917961399	35.1515428985431\\
6.22105835922798	35.0984402854427\\
6.33158753884198	35.0501493727963\\
6.44211671845597	35.0065515787593\\
6.55264589806996	34.967531594126\\
6.66317507768395	34.932977223695\\
6.78544037940584	34.8998242881971\\
6.90770568112773	34.8718585229317\\
7.02997098284962	34.8489404668414\\
7.16439929713635	34.8294070437655\\
7.29882761142308	34.8156373499137\\
7.43325592570981	34.8074621554466\\
7.58318995160709	34.8047429072671\\
7.73312397750436	34.8085590544253\\
7.88305800340164	34.8186976907307\\
7.92203866893443	34.8223435936097\\
7.96101933446721	34.8263993631492\\
8	34.8308614896486\\
8.1491425417522	34.8510070946668\\
8.29828508350441	34.8756573539332\\
8.44742762525661	34.9046715098583\\
8.59657016700882	34.9379129590438\\
8.74571270876102	34.9752491247576\\
8.89485525051322	35.0165512690264\\
9.0594247179955	35.0665782033574\\
9.22399418547779	35.1211191184216\\
9.38856365296007	35.1800163541979\\
9.56617752048635	35.2482942530195\\
9.74379138801264	35.3212831019091\\
9.92140525553892	35.3988035353563\\
9.94760350369262	35.4106108159973\\
9.97380175184631	35.422512362722\\
10	35.4345076330473\\
10.1309912407685	35.495057241796\\
10.2619824815369	35.5562711808732\\
10.3929737223054	35.6181267798147\\
10.5231695337911	35.680220674964\\
10.6533653452769	35.7429052874002\\
10.7835611567627	35.8061598095524\\
10.9260985340574	35.8760398257142\\
11.0686359113522	35.9465524683477\\
11.211173288647	36.0176722931975\\
11.3665561224047	36.095864040564\\
11.5219389561624	36.1747165104869\\
11.6773217899201	36.2541991888802\\
11.7848811932801	36.3095723069258\\
11.89244059664	36.3652235238133\\
12	36.4211433869249\\
12.1370411138001	36.4921631776317\\
12.2740822276002	36.562399870204\\
12.4111233414003	36.6318628823105\\
12.5481644552004	36.7005614996648\\
12.6852055690005	36.7685048776944\\
12.8222466828006	36.8357020451163\\
12.975201860551	36.9098322575142\\
13.1281570383013	36.9830561733408\\
13.2811122160516	37.055385804101\\
13.4487700363402	37.1336546967418\\
13.6164278566288	37.2108786513634\\
13.7840856769174	37.2870727416355\\
13.8560571179449	37.3194687716989\\
13.9280285589725	37.351678914167\\
14	37.3837043195419\\
14.1590988325837	37.4535029540272\\
14.3181976651673	37.5217354882392\\
14.477296497751	37.5884308806629\\
14.6363953303347	37.6536175746264\\
14.7954941629183	37.717323503011\\
14.954592995502	37.7795761067536\\
15.134927877162	37.8484147807504\\
15.3152627588219	37.9154593919385\\
15.4955976404819	37.9807476849207\\
15.6637317603213	38.0400693079075\\
15.8318658801606	38.0979257867839\\
16	38.1543459790778\\
16.2282853176513	38.2285910742229\\
16.4565706353026	38.3000919572608\\
16.6848559529539	38.3689238324263\\
16.8993876723249	38.4312416309378\\
17.1139193916959	38.4913264665264\\
17.3284511110669	38.5492363183925\\
17.5523007407112	38.6074038978111\\
17.7761503703556	38.6633278574768\\
18	38.7170694443995\\
18.2928624601938	38.7843752913728\\
18.5857249203877	38.8485498757785\\
18.8785873805815	38.9097024002744\\
19.1217935840583	38.9582652681366\\
19.3649997875352	39.0048757878162\\
19.608205991012	39.0495912194539\\
19.738803994008	39.0728401956449\\
19.869401997004	39.0955672763524\\
20	39.1177807984113\\
20.2260668109429	39.1551598066438\\
20.4521336218859	39.1912947311899\\
20.6782004328288	39.2262183152966\\
20.9042672437718	39.2599625158377\\
21.1303340547147	39.2925585081959\\
21.3564008656576	39.3240367243378\\
21.5709339104384	39.3529022632402\\
21.7854669552192	39.380812702355\\
22	39.407792257176\\
22.3202318024385	39.4464765450369\\
22.640463604877	39.4833952183368\\
22.9606954073156	39.518613118539\\
23.226643482588	39.5466133527644\\
23.4925915578605	39.5735185777666\\
23.7585396331329	39.5993625985648\\
23.8390264220886	39.6069798741802\\
23.9195132110443	39.6145038618878\\
24	39.6219354478601\\
24.4024339447784	39.6577131310881\\
24.8048678895569	39.6912619658454\\
25.2073018343353	39.7226874759396\\
25.4715345562236	39.7422151344367\\
25.7357672781118	39.7608986354856\\
26	39.7787647836766\\
26.2991886251125	39.7980015011387\\
26.5983772502251	39.81618460871\\
26.8975658753376	39.8333531818913\\
27.1967545004502	39.849545041174\\
27.4959431255627	39.8647967578398\\
27.7951317506753	39.8791437353967\\
27.8634211671169	39.882295184037\\
27.9317105835584	39.8854016861226\\
28	39.8884636376839\\
28.3414470822079	39.9030955023857\\
28.6828941644157	39.9166263733936\\
29.0243412466236	39.9291049240536\\
29.3495608310824	39.9400548373682\\
29.6747804155412	39.9501315788277\\
30	39.9593727486098\\
30.4607574449338	39.9711370224952\\
30.9215148898676	39.9814557978763\\
31.3822723348014	39.9904182791666\\
31.5881815565343	39.9940075482554\\
31.7940907782671	39.9973501789157\\
32	40.0004532715921\\
32.3135635903446	40.0047702967245\\
32.6271271806893	40.0086414476255\\
32.9406907710339	40.0120861829015\\
33.2542543613786	40.0151232762012\\
33.5678179517232	40.017770818754\\
33.8813815420679	40.0200462671524\\
33.9209210280452	40.0203076304559\\
33.9604605140226	40.0205633787407\\
34	40.0208135449063\\
34.1976974298869	40.0219975040755\\
34.3953948597738	40.0230775389986\\
34.5930922896607	40.0240566256708\\
34.9257236660814	40.0254841818704\\
35.2583550425021	40.0266475950156\\
35.5909864189228	40.0275597834539\\
35.7273242792819	40.0278639804722\\
35.8636621396409	40.0281288961617\\
36	40.0283553489978\\
36.3233466415018	40.0287762392971\\
36.6466932830036	40.0290593654254\\
36.9700399245054	40.0292115112489\\
37.3133599496703	40.0292369780929\\
37.6566799748351	40.0291296424187\\
38	40.0288967073387\\
38.4179047967019	40.0284885862398\\
38.8358095934037	40.0279832655716\\
39.2537143901056	40.027387609267\\
39.5024762600704	40.0269928398617\\
39.7512381300352	40.0265696946308\\
40	40.026119451176\\
40.3922815228351	40.0253744632188\\
40.7845630456702	40.0246025813638\\
41.1768445685053	40.0238061929149\\
41.4512297123369	40.0232358023216\\
41.7256148561684	40.0226552731911\\
42	40.0220653192403\\
42.5404692055779	40.0208957548615\\
43.0809384111559	40.0197301005514\\
43.6214076167338	40.0185701393012\\
43.7476050778226	40.0183002902946\\
43.8738025389113	40.0180308603351\\
44	40.0177618684613\\
44.4703868427218	40.0167715340475\\
44.9407736854437	40.0158039144348\\
45.4111605281655	40.014858688047\\
45.6074403521103	40.0144708138758\\
45.8037201760552	40.0140867572762\\
46	40.0137064938111\\
46.5020044119357	40.0127580544558\\
47.0040088238714	40.0118471765723\\
47.506013235807	40.0109723727806\\
47.670675490538	40.0106930425443\\
47.835337745269	40.0104173903291\\
48	40.0101453677169\\
48.529852964495	40.0093001168072\\
49.05970592899	40.0085017387463\\
49.589558893485	40.0077477937477\\
49.7263725956567	40.0075600485917\\
49.8631862978283	40.007375072169\\
50	40.0071928263741\\
50.6666666666667	40.0063490716525\\
51.3333333333333	40.0055770883808\\
52	40.0048716814963\\
52.5691423420476	40.0043209209872\\
53.1382846840952	40.0038169021933\\
53.7074270261428	40.0033565446988\\
53.8049513507619	40.003281829344\\
53.9024756753809	40.0032082964054\\
54	40.0031359319704\\
54.4876216230953	40.002791902113\\
54.9752432461907	40.0024764888706\\
55.462864869286	40.0021879999288\\
55.6419099128574	40.0020885098789\\
55.8209549564287	40.0019923561518\\
56	40.0018994636999\\
56.6666666666667	40.0015807819101\\
57.3333333333333	40.0013022099768\\
58	40.0010604277779\\
58.6666666666667	40.0008513053059\\
59.3333333333333	40.0006710913097\\
60	40.0005173368297\\
60.6666666666667	40.0003866131534\\
61.3333333333333	40.000275826128\\
62	40.0001832500542\\
62.6666666666667	40.0001062732968\\
63.3333333333333	40.000042548459\\
64	39.9999909014454\\
64.6666666666667	39.9999494485743\\
65.3333333333333	39.999916501732\\
66	39.9998912856168\\
66.6666666666667	39.9998724811295\\
67.3333333333333	39.9998589095072\\
68	39.9998500755359\\
68.6666666666667	39.9998450681474\\
69.3333333333333	39.9998430760385\\
70	39.9998437979724\\
70.6666666666667	39.9998466121402\\
71.3333333333333	39.9998509673229\\
72	39.9998566949224\\
72.6666666666667	39.9998633809624\\
73.3333333333333	39.9998706605847\\
74	39.9998784535777\\
74.6666666666667	39.9998864977804\\
75.3333333333333	39.9998945638837\\
76	39.9999026276748\\
76.6666666666667	39.9999105371136\\
77.3333333333333	39.9999181606837\\
78	39.9999255066795\\
78.6666666666667	39.9999324997395\\
79.3333333333333	39.9999390760242\\
80	39.9999452599963\\
80.6666666666667	39.9999510258853\\
81.3333333333333	39.9999563533067\\
82	39.9999612723334\\
82.6666666666667	39.9999657859688\\
83.3333333333333	39.9999698987943\\
84	39.9999736403634\\
84.6666666666667	39.9999770278249\\
85.3333333333333	39.9999800778144\\
86	39.9999828163406\\
86.6666666666667	39.9999852654974\\
87.3333333333333	39.9999874459078\\
88	39.9999893789612\\
88.6666666666667	39.9999910866241\\
89.3333333333333	39.9999925890996\\
90	39.9999939031606\\
90.6666666666667	39.9999950482957\\
91.3333333333333	39.9999960422791\\
92	39.9999968977741\\
92.6666666666667	39.9999976309962\\
93.3333333333333	39.9999982566466\\
94	39.9999987839604\\
94.6666666666667	39.9999992258562\\
95.3333333333333	39.9999995939882\\
96	39.9999998948551\\
96.6666666666667	40.0000001383848\\
97.3333333333333	40.000000333493\\
98	40.0000004845728\\
98.6666666666667	40.000000598987\\
99.3333333333333	40.0000006833196\\
99.9999999999991	40.0000007404001\\
100	40.0000007404001\\
100.000000000001	40.0000007404001\\
100.033333333334	40.0004975857709\\
100.066666666668	40.0019815376542\\
100.100000000001	40.0044428335458\\
100.133333333334	40.0078718527244\\
100.166666666668	40.0122591125174\\
100.200000000001	40.0175952645405\\
100.254274164709	40.0282902398129\\
100.308548329416	40.0414373957124\\
100.362822494124	40.0569985428575\\
100.41803748384	40.0752680217827\\
100.473252473556	40.095958804653\\
100.528467463273	40.1190332694758\\
100.585766063412	40.145459387482\\
100.643064663551	40.1743726118562\\
100.700363263691	40.2057335748845\\
100.76081717281	40.2414326459742\\
100.82127108193	40.2797693510467\\
100.881724991049	40.3207004212347\\
100.946273824899	40.367220177493\\
101.010822658749	40.4165990814328\\
101.075371492599	40.4687879214533\\
101.143924281103	40.5272370869229\\
101.212477069608	40.5887450823957\\
101.281029858113	40.6532569351475\\
101.353634408266	40.72479744793\\
101.426238958419	40.7995848977796\\
101.498843508573	40.8775584885077\\
101.575602947023	40.9633933247244\\
101.652362385473	41.0526546439006\\
101.729121823924	41.1452756799471\\
101.810187716988	41.2466687789199\\
101.891253610052	41.3516616864428\\
101.972319503116	41.4601814390083\\
101.98154633541	41.472753400652\\
101.990773167705	41.4853700247536\\
102	41.4980312081635\\
102.046134161474	41.5620019248484\\
102.092268322948	41.6270713181892\\
102.138402484421	41.6932268164349\\
102.230246616295	41.8281139963288\\
102.322090748169	41.9671609691358\\
102.413934880042	42.1102745009525\\
102.507484608774	42.2601324325319\\
102.601034337507	42.4140219321834\\
102.694584066239	42.5718521696631\\
102.793972553828	42.7437516405134\\
102.893361041418	42.9198961925135\\
102.992749529007	43.1001855045723\\
103.098276007634	43.2960367249455\\
103.203802486261	43.4963369884492\\
103.309328964888	43.7009760001321\\
103.421697878597	43.9235335547472\\
103.534066792305	44.1507636408656\\
103.646435706014	44.3825443042402\\
103.764290470676	44.6304005426399\\
103.882145235338	44.8829998586948\\
104	45.1402137331725\\
104.101469251519	45.3628130736932\\
104.202938503038	45.583814619065\\
104.304407754556	45.8032340464065\\
104.405877006075	46.0210867999223\\
104.507346257594	46.2373880961723\\
104.608815509113	46.4521529303088\\
104.718528428149	46.6826542052029\\
104.828241347185	46.911394894137\\
104.937954266221	47.1383931128812\\
105.055499926632	47.3796841769273\\
105.173045587043	47.6190171504455\\
105.290591247454	47.8564132760171\\
105.417301859542	48.1101737612527\\
105.544012471631	48.3617336228118\\
105.670723083719	48.6111181707124\\
105.780482055813	48.8254004468978\\
105.890241027906	49.0380851067573\\
106	49.2491878391875\\
106.114259224274	49.4654990008019\\
106.228518448549	49.6765908688701\\
106.342777672823	49.8825210517944\\
106.457036897098	50.0833464330912\\
106.571296121372	50.2791231878583\\
106.685555345647	50.4699068106986\\
106.810316963616	50.6725886011237\\
106.935078581585	50.8694530088901\\
107.059840199554	51.0605696313335\\
107.194301523506	51.2601882554572\\
107.328762847457	51.453295352062\\
107.463224171409	51.6399747394876\\
107.609037538345	51.8352462377171\\
107.754850905281	52.0231602682223\\
107.900664272217	52.2038195447009\\
107.933776181478	52.243844144323\\
107.966888090739	52.2835010449697\\
108	52.322791420263\\
108.165559546304	52.5122894554199\\
108.331119092609	52.689824327722\\
108.496678638913	52.8555679757899\\
108.630842727954	52.9813486427892\\
108.765006816994	53.0995866533566\\
108.899170906035	53.2103702526823\\
109.049157837651	53.3255017460703\\
109.199144769267	53.431546718846\\
109.349131700883	53.5286246010183\\
109.510557860468	53.6232227672134\\
109.671984020053	53.7077164413944\\
109.833410179638	53.78224977285\\
109.888940119759	53.8056122623193\\
109.944470059879	53.8278187674797\\
110	53.8488750331026\\
110.192543987691	53.913399334135\\
110.385087975381	53.9650797862923\\
110.577631963072	54.0041440614322\\
110.822248729758	54.0359312729087\\
111.066865496443	54.0481744443582\\
111.311482263128	54.0413233326328\\
111.540988175419	54.017929493365\\
111.770494087709	53.9784805300522\\
112	53.9233345488446\\
112.142803465281	53.8823710225739\\
112.285606930562	53.8377674695639\\
112.428410395842	53.7895855437377\\
112.571213861123	53.7378863814822\\
112.714017326404	53.6827305925928\\
112.856820791685	53.624178282685\\
113.018722337529	53.55376240709\\
113.180623883374	53.4791434488655\\
113.342525429219	53.400406668928\\
113.522231350578	53.308291701189\\
113.701937271938	53.2113218004926\\
113.881643193297	53.1096101064313\\
113.921095462198	53.0866568707625\\
113.960547731099	53.0634816726758\\
114	53.040085686395\\
114.133349880282	52.9605999604485\\
114.266699760565	52.8810504649274\\
114.400049640847	52.8014457247983\\
114.53339952113	52.7217941337212\\
114.666749401412	52.6421039531797\\
114.800099281695	52.5623833160781\\
114.948836958371	52.4734372554374\\
115.097574635048	52.3844741795348\\
115.246312311724	52.2955048147834\\
115.408860634146	52.1982796567856\\
115.571408956569	52.1010730521126\\
115.733957278991	52.0038982200703\\
115.822638185994	51.9509009350274\\
115.911319092997	51.897919037229\\
116	51.8449545738399\\
116.139563397847	51.7625909226542\\
116.279126795694	51.6821569294565\\
116.418690193542	51.6036209936437\\
116.558253591389	51.5269519703344\\
116.697816989236	51.4521191685854\\
116.837380387083	51.3790923391437\\
116.993818258117	51.299345769519\\
117.150256129151	51.221789510435\\
117.306694000185	51.1463828953579\\
117.478405328439	51.0660412937804\\
117.650116656693	50.9881894004937\\
117.821827984947	50.9127761481275\\
117.881218656631	50.8872510268095\\
117.940609328316	50.862009581853\\
118	50.8370497762349\\
118.164690007043	50.7697923383388\\
118.329380014085	50.705623137672\\
118.494070021128	50.6444734945263\\
118.65876002817	50.5862760710435\\
118.823450035213	50.5309648577043\\
118.988140042255	50.4784751278107\\
119.176020110948	50.4219586191469\\
119.363900179642	50.3689393423409\\
119.551780248335	50.3193271095808\\
119.701186832223	50.2822475551558\\
119.850593416112	50.2472233062244\\
120	50.2142116885825\\
120.244413009701	50.1642877418166\\
120.488826019402	50.1191657383097\\
120.733239029102	50.0786834157326\\
121.085644372993	50.0281644696925\\
121.438049716883	49.9865084235577\\
121.790455060774	49.9532780963349\\
121.860303373849	49.9476554269789\\
121.930151686925	49.9423440639636\\
122	49.9373408532308\\
122.188408908929	49.9249970183925\\
122.376817817858	49.9141053866112\\
122.565226726787	49.9046277417478\\
122.753635635716	49.8965266805363\\
122.942044544644	49.8897656067465\\
123.130453453573	49.8843087005069\\
123.353496748022	49.8794864445646\\
123.576540042472	49.8763860313422\\
123.799583336921	49.8749521090853\\
123.866388891281	49.8748388633172\\
123.93319444564	49.8748688469558\\
124	49.8750406458624\\
124.190107212195	49.8759522599024\\
124.380214424391	49.8773001971165\\
124.570321636586	49.8790722864698\\
124.760428848782	49.8812566126782\\
124.950536060977	49.8838415149332\\
125.140643273173	49.8868155771081\\
125.3637442663	49.8907870663061\\
125.586845259428	49.8952614977556\\
125.809946252555	49.9002216683039\\
125.873297501703	49.9017163562577\\
125.936648750852	49.9032484776\\
126	49.9048176572858\\
126.20756003604	49.9099676615454\\
126.41512007208	49.9150160236589\\
126.622680108121	49.9199646277236\\
126.830240144161	49.9248153242801\\
127.037800180201	49.9295699304326\\
127.245360216241	49.9342302310724\\
127.496906810828	49.9397542644465\\
127.748453405414	49.9451454173458\\
128	49.9504066823066\\
128.28078119446	49.9559592372164\\
128.56156238892	49.9610259060065\\
128.84234358338	49.9656238203714\\
129.090904130448	49.9693163236762\\
129.339464677516	49.9726657652231\\
129.588025224583	49.9756830950609\\
129.725350149722	49.9772115884651\\
129.862675074861	49.9786437354566\\
130	49.9799812747432\\
130.372357210488	49.9831491484512\\
130.744714420976	49.9856743893093\\
131.117071631465	49.9875886269729\\
131.411381087643	49.9886893722913\\
131.705690543822	49.9894419255151\\
132	49.9898603387024\\
132.300347805088	49.990034354802\\
132.600695610176	49.9900389419751\\
132.901043415265	49.9898812498705\\
133.201391220353	49.9895682019766\\
133.501739025441	49.9891064964652\\
133.802086830529	49.9885026199359\\
133.868057887019	49.9883515619892\\
133.93402894351	49.9881940137605\\
134	49.9880300399789\\
134.329855282451	49.9872051840036\\
134.659710564903	49.9864018883091\\
134.989565847354	49.9856195954302\\
135.326377231569	49.9848419127234\\
135.663188615785	49.9840849985226\\
136	49.983348304252\\
136.387145869272	49.9825922849973\\
136.774291738544	49.9819895799365\\
137.161437607815	49.981532331031\\
137.44095840521	49.981288351061\\
137.720479202605	49.9811135199806\\
138	49.9810051795132\\
138.452950298499	49.9809813806334\\
138.905900596998	49.9811477588245\\
139.358850895497	49.9814921342498\\
139.572567263664	49.981713087647\\
139.786283631832	49.9819699052696\\
140	49.9822614406545\\
140.439229850775	49.9829417458365\\
140.87845970155	49.9837102794753\\
141.317689552326	49.9845608925962\\
141.545126368217	49.9850316595129\\
141.772563184109	49.9855220786947\\
142	49.9860313882083\\
142.394571764462	49.9869309978142\\
142.789143528923	49.9878302778749\\
143.183715293385	49.9887285426891\\
143.45581019559	49.9893470509826\\
143.727905097795	49.9899645662606\\
144	49.9905808948056\\
144.419048513668	49.9915043198229\\
144.838097027336	49.9923796236225\\
145.257145541004	49.9932088497029\\
145.504763694003	49.9936779867298\\
145.752381847001	49.994132114873\\
146	49.9945716177962\\
146.472114281842	49.9953561229957\\
146.944228563683	49.9960640358864\\
147.416342845525	49.9966995957194\\
147.61089523035	49.996941403169\\
147.805447615175	49.9971718857061\\
148	49.9973913112911\\
148.666666666667	49.9980586937726\\
149.333333333333	49.998601466483\\
150	49.9990298349003\\
150.666666666667	49.9993619660413\\
151.333333333333	49.9996142785255\\
152	49.9997935188618\\
152.666666666667	49.9999170519692\\
153.333333333333	50.0000004130571\\
154	50.0000470934354\\
154.571808103044	50.0000669626872\\
155.143616206088	50.0000766100828\\
155.715424309132	50.0000768271711\\
155.810282872755	50.0000760039003\\
155.905141436377	50.0000749449635\\
156	50.0000736536031\\
156.474292818113	50.0000665835793\\
156.948585636227	50.0000594471508\\
157.42287845434	50.0000522567238\\
157.615252302894	50.0000493275729\\
157.807626151447	50.0000463922056\\
158	50.0000434513457\\
158.666666666667	50.0000351044108\\
159.333333333333	50.0000302478209\\
160	50.0000285729773\\
160.666666666667	50.0000295279998\\
161.333333333333	50.0000326148809\\
162	50.0000376318614\\
162.666666666667	50.0000436110357\\
163.333333333333	50.0000496957797\\
164	50.0000558630967\\
164.666666666667	50.000061445667\\
165.333333333333	50.0000658586372\\
166	50.0000691993093\\
166.666666666667	50.0000712692611\\
167.333333333333	50.0000718996888\\
168	50.000071220876\\
168.666666666667	50.000069376318\\
169.333333333333	50.0000664978436\\
170	50.0000626863307\\
170.666666666667	50.0000582211172\\
171.333333333333	50.0000533513363\\
172	50.0000481246686\\
172.666666666667	50.0000427958376\\
173.333333333333	50.0000375897888\\
174	50.000032506706\\
174.666666666667	50.0000277010469\\
175.333333333333	50.0000233077171\\
176	50.0000192986352\\
176.666666666667	50.0000157233011\\
177.333333333333	50.0000126234543\\
178	50.0000099624006\\
178.666666666667	50.0000077177453\\
179.333333333333	50.0000058679637\\
180	50.0000043809941\\
180.666666666667	50.000003203271\\
181.333333333333	50.0000022862779\\
182	50.0000016081774\\
182.666666666667	50.000001114981\\
183.333333333333	50.0000007583413\\
184	50.00000052655\\
184.666666666667	50.000000380699\\
185.333333333333	50.0000002861547\\
186	50.0000002384316\\
186.666666666667	50.0000002165815\\
187.333333333333	50.0000002020494\\
188	50.000000194198\\
188.666666666667	50.000000185845\\
189.333333333333	50.0000001706727\\
190	50.000000149338\\
190.666666666667	50.0000001222208\\
191.333333333333	50.0000000896838\\
192	50.000000052294\\
192.666666666667	50.0000000128743\\
193.333333333333	49.9999999739207\\
194	49.9999999354758\\
194.666666666667	49.9999998998716\\
195.333333333333	49.9999998691488\\
196	49.9999998429283\\
196.666666666667	49.999999822078\\
197.333333333333	49.999999807338\\
198	49.9999997981817\\
198.666666666667	49.9999997942318\\
199.333333333333	49.9999997951322\\
200	49.9999998004375\\
};
\end{axis}
\end{tikzpicture}%}
  \caption{Step response using a zero-order hold of sample time $2$ sec.}
  \label{fig:Q7.2}
\end{figure}

\begin{figure}[H]\centering
	\centering
	\scalebox{1}{% This file was created by matlab2tikz.
%
%The latest updates can be retrieved from
%  http://www.mathworks.com/matlabcentral/fileexchange/22022-matlab2tikz-matlab2tikz
%where you can also make suggestions and rate matlab2tikz.
%
\definecolor{mycolor1}{rgb}{0.00000,0.44700,0.74100}%
%
\begin{tikzpicture}

\begin{axis}[%
width=4.133in,
height=3.26in,
at={(0.693in,0.44in)},
scale only axis,
xmin=0,
xmax=200,
xmajorgrids,
ymin=30,
ymax=60,
ymajorgrids,
axis background/.style={fill=white}
]
\addplot [color=mycolor1,solid,forget plot]
  table[row sep=crcr]{%
0	40\\
0.05	39.9993544793656\\
0.1	39.9974213715054\\
0.15	39.994205842355\\
0.2	39.9897130376822\\
0.25	39.9839480831286\\
0.3	39.9769160849107\\
0.374542556630099	39.9640908323915\\
0.449085113260198	39.9484774815311\\
0.523627669890297	39.9300927193473\\
0.600058766825396	39.9083819359557\\
0.676489863760495	39.8837927796124\\
0.752920960695595	39.8563429532007\\
0.832933948898952	39.8245609661365\\
0.912946937102309	39.7896832996327\\
0.992959925305666	39.7517299505148\\
1.07664458525826	39.7087659641457\\
1.16032924521086	39.6624819503386\\
1.24401390516346	39.6129004415291\\
1.33153125762246	39.5575450020043\\
1.41904861008146	39.4986331443407\\
1.50656596254046	39.4361902626919\\
1.5980086873395	39.3672019258635\\
1.68945141213853	39.2944149917764\\
1.78089413693756	39.2178580178033\\
1.87626097895818	39.1340302531138\\
1.9716278209788	39.0461649529627\\
2.06699466299941	38.9542940754375\\
2.16615474351188	38.8545554758567\\
2.26531482402435	38.7505563351347\\
2.36447490453682	38.642332125169\\
2.46714118586662	38.5258670825041\\
2.56980746719642	38.4049498296676\\
2.67247374852621	38.2796192789922\\
2.78164916568414	38.1415442185022\\
2.89082458284207	37.9985688039618\\
3	37.8507393338439\\
3.07337598747675	37.7504423657042\\
3.14675197495351	37.6514934952636\\
3.22012796243026	37.5538781102825\\
3.29350394990701	37.4575817581931\\
3.36687993738376	37.3625901444868\\
3.44025592486052	37.2688891290938\\
3.51940098185695	37.1692517280438\\
3.59854603885339	37.0710822200267\\
3.67769109584983	36.9743634501445\\
3.76331605826453	36.8713401023489\\
3.84894102067924	36.7699737146233\\
3.93456598309394	36.670243335216\\
4.02743257329495	36.5639037388332\\
4.12029916349596	36.4594381846559\\
4.21316575369697	36.3568209597736\\
4.31418076134619	36.2472688345966\\
4.41519576899541	36.1398414136184\\
4.51621077664464	36.0345069578449\\
4.62646876614056	35.9218864890417\\
4.73672675563648	35.8116821917338\\
4.84698474513241	35.7038546143571\\
4.96783173163062	35.5883554917566\\
5.08867871812883	35.4756145703804\\
5.20952570462705	35.3655823894292\\
5.34264912958553	35.2474497149381\\
5.47577255454401	35.1324812404805\\
5.6088959795025	35.0206142921626\\
5.739263986335	34.9140094708621\\
5.8696319931675	34.8102633953907\\
6	34.7093203654032\\
6.0954059172298	34.6391357627401\\
6.1908118344596	34.5742137998863\\
6.2862177516894	34.5144257230779\\
6.3816236689192	34.4596468375502\\
6.477029586149	34.4097562896636\\
6.57243550337881	34.3646368485625\\
6.67725772592733	34.3204295949473\\
6.78207994847586	34.2816982973685\\
6.88690217102438	34.2483013278706\\
7.00068999510666	34.217926451406\\
7.11447781918894	34.1935059195241\\
7.22826564327122	34.1748753225619\\
7.35298859330369	34.1609171300783\\
7.47771154333615	34.1535215933872\\
7.60243449336861	34.1524930263044\\
7.74069986077879	34.1585655123197\\
7.87896522818896	34.1719819566027\\
8.01723059559913	34.1925027621003\\
8.17343985863127	34.2239443395376\\
8.32964912166342	34.2638388283139\\
8.48585838469556	34.3118781936409\\
8.65723892313037	34.3736018456694\\
8.82861946156519	34.4443980953924\\
9	34.5239072369972\\
9.11999983747504	34.5835403441867\\
9.23999967495008	34.6451222741973\\
9.35999951242511	34.7086007719313\\
9.47999934990015	34.7739248001054\\
9.59999918737519	34.8410445121164\\
9.71999902485023	34.909911208966\\
9.85457347820286	34.9891614322544\\
9.9891479315555	35.0704835241761\\
10.1237223849081	35.1538137125282\\
10.2689015122406	35.2458906800475\\
10.4140806395731	35.3401564983503\\
10.5592597669056	35.4365373320721\\
10.7173250892777	35.5437939276389\\
10.8753904116497	35.6533825932087\\
11.0334557340218	35.765215997226\\
11.2062190400506	35.8899161172493\\
11.3789823460794	36.017089497745\\
11.5517456521082	36.1466323969662\\
11.7011637680721	36.2605038744066\\
11.8505818840361	36.3760096399986\\
12	36.4930884557089\\
12.1153457792865	36.5832577045911\\
12.2306915585731	36.6718348901178\\
12.3460373378596	36.7588419456583\\
12.4613831171461	36.8443005026134\\
12.5767288964326	36.9282318937394\\
12.6920746757192	37.0106571618656\\
12.8201789581694	37.1004595527977\\
12.9482832406196	37.1884578263037\\
13.0763875230699	37.2746795889315\\
13.2147517209261	37.3658426058856\\
13.3531159187823	37.4549986032352\\
13.4914801166386	37.5421807754055\\
13.6423689575685	37.6350430530534\\
13.7932577984985	37.7256384840943\\
13.9441466394284	37.8140080079689\\
14.1097897358203	37.9085032959172\\
14.2754328322123	38.0004168264795\\
14.4410759286042	38.0897998933126\\
14.6273839524028	38.1873729345816\\
14.8136919762014	38.281878598413\\
15	38.3733856296828\\
15.1445792930755	38.4417709929517\\
15.289158586151	38.5072322793826\\
15.4337378792266	38.5698233410258\\
15.5783171723021	38.6295972263196\\
15.7228964653776	38.6866061837895\\
15.8674757584531	38.7409016917672\\
16.0295237716375	38.7985950747701\\
16.1915717848218	38.8530136690934\\
16.3536197980062	38.9042266060449\\
16.5314259120402	38.9568119978193\\
16.7092320260742	39.0057080637922\\
16.8870381401081	39.051001494075\\
17.08407078375	39.0970875422445\\
17.2811034273918	39.1389678696579\\
17.4781360710336	39.1767538943841\\
17.6520907140224	39.2067983651785\\
17.8260453570112	39.2338103928115\\
18	39.2578624528573\\
18.1562802877711	39.2775373994902\\
18.3125605755421	39.2959801354667\\
18.4688408633132	39.3132174720572\\
18.6251211510842	39.3292757671162\\
18.7814014388553	39.3441809255784\\
18.9376817266264	39.3579584155174\\
19.1160706820416	39.3723387571139\\
19.2944596374569	39.3853189593196\\
19.4728485928721	39.3969348654315\\
19.6707778837855	39.4082691215052\\
19.8687071746989	39.418014088692\\
20.0666364656123	39.4262157650165\\
20.2900087158756	39.4336744451838\\
20.5133809661389	39.4392880199171\\
20.7367532164022	39.4431181163783\\
20.8245021442681	39.4441479654844\\
20.9122510721341	39.4449154585608\\
21	39.4454241409906\\
21.146575811971	39.4462507916776\\
21.2931516239421	39.4474535578281\\
21.4397274359131	39.449023299454\\
21.5863032478842	39.450951045335\\
21.7328790598552	39.4532279921614\\
21.8794548718263	39.4558454989317\\
22.048586720368	39.4592779369797\\
22.2177185689098	39.463139763788\\
22.3868504174516	39.4674184866974\\
22.5730896034314	39.4725976645788\\
22.7593287894112	39.4782515774005\\
22.9455679753911	39.4843646554592\\
23.1542312906527	39.4917404356219\\
23.3628946059143	39.499652712286\\
23.571557921176	39.5080812123859\\
23.7143719474506	39.5141372427888\\
23.8571859737253	39.5204196597139\\
24	39.5269223984336\\
24.2139628379571	39.5372140469104\\
24.4279256759142	39.5482601831026\\
24.6418885138713	39.5600314298334\\
24.8558513518283	39.5724992288994\\
25.0698141897854	39.5856358345141\\
25.2837770277425	39.5994142727323\\
25.5365258036673	39.616481514667\\
25.789274579592	39.6343659639929\\
26.0420233555168	39.6530271418315\\
26.3613489036779	39.6776537253226\\
26.6806744518389	39.7033819235423\\
27	39.7301390670505\\
27.1783569713014	39.7452059835771\\
27.3567139426028	39.7599719434803\\
27.5350709139043	39.7744424871543\\
27.7134278852057	39.7886230570835\\
27.8917848565071	39.8025189982905\\
28.0701418278085	39.8161355618384\\
28.2804801820014	39.8318417332736\\
28.4908185361944	39.8471748186895\\
28.7011568903873	39.8621429418237\\
28.9373814044267	39.8785279456908\\
29.1736059184661	39.8944736723553\\
29.4098304325055	39.9099908816182\\
29.6065536216703	39.9225938683025\\
29.8032768108352	39.9349128971109\\
30	39.9469537899732\\
30.1921999392937	39.9582220570938\\
30.3843998785875	39.968779576073\\
30.5765998178812	39.9786452257174\\
30.768799757175	39.9878374817368\\
30.9609996964687	39.9963744178226\\
31.1531996357625	40.0042737227158\\
31.3800603456029	40.0128006718071\\
31.6069210554434	40.0204911599053\\
31.8337817652838	40.0273722715542\\
32.0907512796465	40.0342222711594\\
32.3477207940092	40.0401049433671\\
32.6046903083719	40.0450565160412\\
32.7364602055813	40.0472459314534\\
32.8682301027906	40.0492044178076\\
33	40.0509365417246\\
33.2819588720653	40.053992544103\\
33.5639177441307	40.0562489447623\\
33.845876616196	40.0577397272601\\
34.1278354882614	40.0584977785863\\
34.4097943603267	40.0585548918281\\
34.6917532323921	40.0579418358999\\
35.0419754438312	40.0562918773221\\
35.3921976552704	40.0537087185235\\
35.7424198667096	40.0502450408396\\
35.8282799111397	40.0492673132893\\
35.9141399555699	40.0482404343921\\
36	40.0471651213844\\
36.2094201593553	40.0445084477552\\
36.4188403187105	40.0419003744356\\
36.6282604780658	40.0393400204323\\
36.8376806374211	40.03682652052\\
37.0471007967764	40.0343590251958\\
37.2565209561316	40.0319367000852\\
37.5117897943528	40.0290439502147\\
37.7670586325739	40.026215643168\\
38.022327470795	40.0234503562852\\
38.3142691644072	40.0203632629837\\
38.6062108580193	40.0173547463859\\
38.8981525516315	40.0144228244209\\
38.9321017010877	40.0140867611408\\
38.9660508505438	40.0137517044499\\
39	40.0134176513611\\
39.1697457472808	40.0118114278963\\
39.3394914945617	40.0103262530715\\
39.5092372418425	40.0089591303742\\
39.7673926387505	40.0070996623031\\
40.0255480356584	40.0054963898279\\
40.2837034325663	40.0041395405815\\
40.5678863202123	40.0029195144246\\
40.8520692078583	40.0019743889957\\
41.1362520955043	40.001292331324\\
41.4241680636695	40.0008578730406\\
41.7120840318348	40.0006702139635\\
42	40.0007182558469\\
42.3030383234165	40.0009510545448\\
42.606076646833	40.0013037766809\\
42.9091149702495	40.0017705065246\\
43.212153293666	40.0023455429004\\
43.5151916170825	40.0030233982212\\
43.8182299404989	40.0037987836362\\
44.212153293666	40.0049441543306\\
44.606076646833	40.0062350339204\\
45	40.0076612266084\\
45.2680970840108	40.0086296932812\\
45.5361941680216	40.0095084843475\\
45.8042912520324	40.0103009806109\\
46.0723883360432	40.0110104601496\\
46.340485420054	40.0116400987227\\
46.6085825040649	40.0121929758649\\
46.9466014626388	40.0127852611455\\
47.2846204212128	40.0132660325496\\
47.6226393797867	40.0136408391063\\
47.7484262531911	40.0137543457532\\
47.8742131265956	40.0138541859939\\
48	40.0139406231169\\
48.3535654164697	40.0140875704256\\
48.7071308329394	40.01408534007\\
49.0606962494091	40.0139421691183\\
49.4142616658788	40.0136659535099\\
49.7678270823485	40.0132642493219\\
50.1213924988182	40.0127443003369\\
50.4142616658788	40.0122290346785\\
50.7071308329394	40.0116411895981\\
51	40.0109844189557\\
51.3201965105477	40.0102303427578\\
51.6403930210954	40.00947728684\\
51.9605895316432	40.0087256865666\\
52.2807860421909	40.0079759547027\\
52.6009825527386	40.0072284815106\\
52.9211790632863	40.0064836366315\\
53.2807860421909	40.005650681269\\
53.6403930210954	40.004821946619\\
54	40.0039978689114\\
54.3355514215375	40.003268146352\\
54.6711028430749	40.0026100192413\\
55.0066542646124	40.0020201914195\\
55.3422056861499	40.0014954906859\\
55.6777571076873	40.0010328682873\\
56.0133085292248	40.0006293896859\\
56.3422056861499	40.0002885865415\\
56.6711028430749	39.9999993374555\\
57	39.9997591825216\\
57.5594351227124	39.9994562128217\\
58.1188702454248	39.9992763297752\\
58.6783053681372	39.9992092852787\\
59.1188702454248	39.9992296129478\\
59.5594351227124	39.9993095626469\\
60	39.9994449577876\\
60.3614610690658	39.9995764561926\\
60.7229221381315	39.9997055089009\\
61.0843832071973	39.999832148137\\
61.445844276263	39.9999564062773\\
61.8073053453288	40.0000783158499\\
62.1687664143945	40.000197909455\\
62.445844276263	40.0002880361989\\
62.7229221381315	40.0003768360008\\
63	40.0004643235869\\
63.4288795620133	40.0005835263247\\
63.8577591240266	40.0006735832304\\
64.2866386860398	40.0007363332082\\
64.7155182480531	40.0007735248801\\
65.1443978100664	40.0007868168886\\
65.5732773720796	40.0007777867404\\
65.7155182480531	40.0007701219565\\
65.8577591240265	40.000760219969\\
66	40.0007481326416\\
66.5667904455866	40.000686168327\\
67.1335808911732	40.0006063753299\\
67.7003713367598	40.0005104655824\\
68.1335808911732	40.0004272888049\\
68.5667904455866	40.0003362984801\\
69	40.0002381271015\\
69.4657091023098	40.0001347582874\\
69.9314182046196	40.0000425383379\\
70.3971273069294	39.9999607872394\\
70.8628364092392	39.9998888595431\\
71.3285455115489	39.9998261442632\\
71.7942546138587	39.9997720612751\\
71.8628364092392	39.9997647924016\\
71.9314182046196	39.9997576970538\\
72	39.9997507735431\\
72.3429089769021	39.9997202011688\\
72.6858179538042	39.9996966200237\\
73.0287269307064	39.9996796724243\\
73.6858179538042	39.9996644888183\\
74.3429089769021	39.9996701387972\\
75	39.9996945173255\\
75.6076858978826	39.9997270709932\\
76.2153717957651	39.9997632500365\\
76.8230576936477	39.9998026345032\\
77.2153717957651	39.9998295799989\\
77.6076858978826	39.9998576014125\\
78	39.999886605934\\
78.5733758502283	39.9999262706991\\
79.1467517004567	39.9999594833891\\
79.720127550685	39.9999867450264\\
80.1467517004567	40.0000034476655\\
80.5733758502283	40.0000172980332\\
81	40.0000284695209\\
82	40.0000437108707\\
83	40.0000446137897\\
84	40.0000333048513\\
84.8465649388734	40.0000196800031\\
85.6931298777468	40.000007129948\\
86.5396948166201	39.9999955721933\\
86.6931298777468	39.999993577594\\
86.8465649388734	39.9999916126371\\
87	39.9999896768954\\
87.7671753056331	39.999981439858\\
88.5343506112662	39.9999757466396\\
89.3015259168993	39.999972322216\\
89.5343506112662	39.9999716928747\\
89.7671753056331	39.999971242599\\
90	39.9999709649635\\
91	39.9999715047707\\
92	39.999974508382\\
93	39.9999795914969\\
94	39.999985279338\\
95	39.9999903734021\\
96	39.9999949315289\\
97	39.999998832123\\
98	40.0000019839186\\
99	40.0000044794156\\
99.9999999999991	40.0000064341427\\
100	40.0000064341427\\
100.000000000001	40.0000064341427\\
100.033333333334	40.0000064913615\\
100.066666666668	40.0000065480926\\
100.100000000001	40.0000066043383\\
100.133333333334	40.0000066601006\\
100.166666666668	40.0000067153818\\
100.200000000001	40.0000067701841\\
100.254639830188	40.0000068589854\\
100.309279660375	40.0000069465151\\
100.363919490562	40.0000070327827\\
100.419515942674	40.0000071192749\\
100.475112394786	40.0000072044799\\
100.530708846897	40.0000072884075\\
100.588443629653	40.0000073742215\\
100.646178412409	40.0000074586791\\
100.703913195165	40.0000075417912\\
100.763889207983	40.0000076267161\\
100.823865220801	40.0000077102123\\
100.883841233618	40.0000077922915\\
100.947725188373	40.0000078781735\\
101.011609143128	40.0000079624753\\
101.075493097883	40.0000080452107\\
101.143449990742	40.0000081315171\\
101.211406883602	40.0000082160832\\
101.279363776462	40.0000082989254\\
101.351496064725	40.0000083849898\\
101.423628352987	40.0000084691496\\
101.49576064125	40.000008551424\\
101.572258121524	40.0000086366385\\
101.648755601798	40.0000087197762\\
101.725253082072	40.0000088008595\\
101.806380889453	40.0000088846306\\
101.887508696834	40.0000089661419\\
101.968636504215	40.0000090454192\\
101.97909100281	40.0000090554741\\
101.989545501405	40.0000090654924\\
102	40.0000090754741\\
102.052272492974	40.001228644944\\
102.104544985949	40.0048620389402\\
102.156817478923	40.0108720703165\\
102.28478026108	40.0353891084462\\
102.412743043237	40.0734193375565\\
102.540705825394	40.1244754849534\\
102.669000068876	40.1882756296829\\
102.797294312357	40.2642606623795\\
102.925588555839	40.3520077945964\\
103.06031976021	40.4563787124718\\
103.195050964581	40.5728320301003\\
103.329782168953	40.7009428441123\\
103.470541561347	40.8467977829388\\
103.611300953741	41.0044927279319\\
103.752060346135	41.1736059167767\\
103.899212745636	41.3621707254311\\
104.046365145137	41.5623345029865\\
104.193517544638	41.7736768398022\\
104.347313100845	42.0060714053622\\
104.501108657051	42.2497993420083\\
104.654904213258	42.5044413808059\\
104.769936142172	42.7017956142941\\
104.884968071086	42.9048695974883\\
105	43.1135070697727\\
105.123444988254	43.3418917268233\\
105.246889976508	43.5733319071952\\
105.370334964761	43.8077403006181\\
105.493779953015	44.0450320111847\\
105.617224941269	44.2851244753024\\
105.740669929523	44.5279373644923\\
105.876290340453	44.7977421201053\\
106.011910751383	45.0706354339807\\
106.147531162313	45.3465195577534\\
106.294059107357	45.6478440612873\\
106.440587052401	45.9524330498174\\
106.587114997445	46.2601740722896\\
106.746437410452	46.5982360389574\\
106.905759823459	46.9397602260864\\
107.065082236465	47.2846155537522\\
107.238931601736	47.6645663547747\\
107.412780967006	48.0481736371848\\
107.586630332277	48.4352840057993\\
107.724420221518	48.7444912605882\\
107.862210110759	49.0557347251733\\
108	49.3689450297541\\
108.107687062615	49.612373677142\\
108.21537412523	49.8516211803924\\
108.323061187844	50.0867327061641\\
108.430748250459	50.317752819296\\
108.538435313074	50.5447254966626\\
108.646122375689	50.7676941474566\\
108.765331564851	51.0099018064573\\
108.884540754013	51.2473126175667\\
109.003749943175	51.4799831421548\\
109.131821916212	51.7247335655644\\
109.259893889248	51.9641452355663\\
109.387965862285	52.1982853847114\\
109.526734669045	52.4461079352237\\
109.665503475806	52.6879024949914\\
109.804272282567	52.9237509141972\\
109.955423523148	53.1739730238316\\
110.10657476373	53.4173387718639\\
110.257726004312	53.6539491914676\\
110.423375064608	53.9056138692891\\
110.589024124905	54.1494134607031\\
110.754673185202	54.3854747027831\\
110.836448790134	54.4991957867073\\
110.918224395067	54.6110762094711\\
111	54.7211306746179\\
111.119457381859	54.8767205116594\\
111.238914763718	55.0246561987354\\
111.358372145577	55.1650077451953\\
111.477829527436	55.297844407816\\
111.597286909295	55.4232347073629\\
111.716744291154	55.5412464676208\\
111.848757504672	55.6631603266406\\
111.98077071819	55.7762347118678\\
112.112783931709	55.880558356728\\
112.254411600357	55.9828499235877\\
112.396039269005	56.0752780715245\\
112.537666937654	56.1579489868232\\
112.690108654494	56.2361494972588\\
112.842550371334	56.3032973114564\\
112.994992088174	56.3595208146897\\
113.158549160868	56.4078404623084\\
113.322106233562	56.4438863030379\\
113.485663306256	56.4678121939516\\
113.657108870838	56.4800473674674\\
113.828554435419	56.4793065033518\\
114	56.4657618029942\\
114.183616793328	56.4380759288806\\
114.367233586655	56.3977543526043\\
114.550850379983	56.3449951787301\\
114.754639079022	56.2721162278793\\
114.958427778062	56.1844253053602\\
115.162216477102	56.0821867524052\\
115.400426330521	55.9445897556279\\
115.63863618394	55.7878896030562\\
115.876846037359	55.6124986010191\\
116.115055890778	55.4188260716738\\
116.353265744197	55.2072781669283\\
116.591475597615	54.9782585695467\\
116.72765039841	54.8396433602888\\
116.863825199205	54.6955242882266\\
117	54.5459757425918\\
117.115735881883	54.4169362677138\\
117.231471763766	54.2885940270161\\
117.347207645649	54.1609475969403\\
117.462943527533	54.0339955266111\\
117.578679409416	53.9077363379828\\
117.694415291299	53.7821685266304\\
117.823190496129	53.6432644214819\\
117.951965700959	53.5052122153083\\
118.080740905789	53.3680097235755\\
118.219799949801	53.2208020458584\\
118.358858993813	53.0745797756442\\
118.497918037826	52.9293400153675\\
118.649538396668	52.7720968160294\\
118.80115875551	52.6160142572245\\
118.952779114353	52.4610883887089\\
119.11914132735	52.2924253576034\\
119.285503540346	52.1251446518172\\
119.451865753343	51.9592407909405\\
119.634577168895	51.7786126018094\\
119.817288584448	51.5996309620182\\
120	51.4222882590327\\
120.121035123288	51.3075342330783\\
120.242070246576	51.1970913109501\\
120.363105369864	51.0908768702449\\
120.484140493152	50.9888099325378\\
120.60517561644	50.8908111237999\\
120.726210739728	50.7968026170941\\
120.861091957828	50.6966476425067\\
120.995973175927	50.6012499864873\\
121.130854394027	50.5105084180147\\
121.277590736403	50.4169634672259\\
121.42432707878	50.3286871159638\\
121.571063421157	50.2455574821873\\
121.732705359254	50.1597985071792\\
121.894347297352	50.0799851071092\\
122.055989235449	50.0059657135957\\
122.236131822705	49.9301204628218\\
122.416274409961	49.8610886488983\\
122.596416997217	49.7986763867862\\
122.730944664811	49.7562725577283\\
122.865472332406	49.7173783822473\\
123	49.6819187072571\\
123.243491908806	49.6258296053121\\
123.486983817611	49.579542250076\\
123.730475726417	49.5426834198171\\
123.961698845791	49.5160816423092\\
124.192921965166	49.49735922952\\
124.42414508454	49.4862278619229\\
124.682098409783	49.4824262920252\\
124.940051735027	49.4873517109328\\
125.19800506027	49.500643786099\\
125.465336706847	49.5228767990928\\
125.732668353423	49.5533507786059\\
126	49.5917072712477\\
126.142755468999	49.6142721406293\\
126.285510937998	49.6369139986497\\
126.428266406997	49.6596289261523\\
126.571021875997	49.6824130791761\\
126.713777344996	49.7052626885318\\
126.856532813995	49.7281740574342\\
127.020302802273	49.7545295116235\\
127.184072790552	49.7809561196875\\
127.34784277883	49.8074486425622\\
127.527719222734	49.8366165181986\\
127.707595666638	49.8658511109811\\
127.887472110541	49.8951459600124\\
128.088204010397	49.9279007691186\\
128.288935910252	49.960714254669\\
128.489667810107	49.9935781330501\\
128.659778540072	50.0214621658362\\
128.829889270036	50.0493718347416\\
129	50.0773024847409\\
129.158464873768	50.1025762834609\\
129.316929747536	50.1263634968001\\
129.475394621305	50.1486930998848\\
129.633859495073	50.169593632669\\
129.792324368841	50.1890932002693\\
129.950789242609	50.2072194868989\\
130.132304791911	50.226330126421\\
130.313820341213	50.2437149797743\\
130.495335890515	50.2594136646968\\
130.696639571532	50.2748990067708\\
130.897943252549	50.2884104927271\\
131.099246933566	50.29999924871\\
131.326123137291	50.3108181639581\\
131.552999341016	50.3193287691263\\
131.77987554474	50.325599835092\\
131.85325036316	50.3271599017618\\
131.92662518158	50.3284950105604\\
132	50.3296073876763\\
132.312980091479	50.3319769634516\\
132.625960182958	50.3306469814809\\
132.938940274437	50.325774694784\\
133.334740622932	50.3147779113717\\
133.730540971427	50.2986570804826\\
134.126341319922	50.2776983523371\\
134.417560879948	50.2593487444346\\
134.708780439974	50.2386349268302\\
135	50.2156595963861\\
135.174031986158	50.2013432192647\\
135.348063972317	50.1871647263957\\
135.522095958475	50.1731229296054\\
135.696127944633	50.1592166485461\\
135.870159930791	50.1454447107136\\
136.04419191695	50.1318059514142\\
136.2570322214	50.1153051608392\\
136.469872525851	50.098999737484\\
136.682712830302	50.0828875984018\\
136.920577464546	50.0651072709994\\
137.15844209879	50.0475629130323\\
137.396306733034	50.0302516940308\\
137.597537822022	50.0157865445507\\
137.798768911011	50.0014845630127\\
138	49.9873440806902\\
138.195076907751	49.9741659940473\\
138.390153815502	49.9618769767143\\
138.585230723254	49.9504543909315\\
138.780307631005	49.9398760683886\\
138.975384538756	49.9301203079449\\
139.170461446507	49.9211658576387\\
139.401289538732	49.9115769048586\\
139.632117630957	49.9030470909793\\
139.862945723181	49.8955435665136\\
140.125176853577	49.8882238451038\\
140.387407983973	49.8821416270547\\
140.649639114369	49.8772524904669\\
140.766426076246	49.8754476964901\\
140.883213038123	49.8738672343134\\
141	49.8725074186333\\
141.383153123329	49.8695023881274\\
141.766306246657	49.8686151821693\\
142.149459369986	49.8697312664816\\
142.766306246657	49.8754608577583\\
143.383153123329	49.8856670513216\\
144	49.8999456557251\\
144.211019776053	49.9054764899633\\
144.422039552105	49.9109935917562\\
144.633059328158	49.9164961235414\\
144.84407910421	49.9219832761256\\
145.055098880263	49.9274542685574\\
145.266118656316	49.9329083467174\\
145.535416387292	49.9398429760251\\
145.804714118269	49.9467474248582\\
146.074011849246	49.9536203083521\\
146.382674566164	49.9614573210365\\
146.691337283082	49.9692492468114\\
147	49.9769943022113\\
147.23930928451	49.982770912105\\
147.47861856902	49.9881366568901\\
147.717927853529	49.9931038100985\\
147.957237138039	49.9976843493297\\
148.196546422549	50.0018899572348\\
148.435855707059	50.00573203545\\
148.727291049211	50.0099361552425\\
149.018726391364	50.0136371555559\\
149.310161733516	50.0168540032061\\
149.540107822344	50.0190625264369\\
149.770053911172	50.0209899048609\\
150	50.0226447249508\\
150.422465657496	50.0249793031475\\
150.844931314991	50.0264277740025\\
151.267396972487	50.0270408494023\\
151.672178864512	50.0268894770763\\
152.076960756538	50.026056499411\\
152.481742648564	50.0245809939015\\
152.654495099042	50.0237649290105\\
152.827247549521	50.0228414848281\\
153	50.0218134267259\\
153.261466414277	50.0201675911965\\
153.522932828555	50.0184986448515\\
153.784399242832	50.0168080192692\\
154.04586565711	50.0150971010414\\
154.307332071387	50.0133672319456\\
154.568798485664	50.0116197114597\\
154.918452658305	50.009257380452\\
155.268106830946	50.0068686212344\\
155.617761003587	50.0044562033772\\
155.745174002391	50.0035717665028\\
155.872587001196	50.0026846651343\\
156	50.0017950214823\\
156.28623897976	49.9998805295321\\
156.57247795952	49.9981356584071\\
156.85871693928	49.9965542810618\\
157.14495591904	49.995130449205\\
157.4311948988	49.9938583925553\\
157.71743387856	49.9927325087327\\
158.088441956264	49.9914818410593\\
158.459450033968	49.990456191915\\
158.830458111672	49.9896445627399\\
158.886972074448	49.9895390426877\\
158.943486037224	49.989438205905\\
159	49.9893420160877\\
159.282569813881	49.9889370017569\\
159.565139627761	49.9886574100116\\
159.847709441642	49.9884983512156\\
160.320693240643	49.988488902507\\
160.793677039643	49.9887825862582\\
161.266660838644	49.9893591073701\\
161.511107225763	49.9897615445704\\
161.755553612881	49.9902318076134\\
162	49.9907674074755\\
162.32820002623	49.9915350544743\\
162.65640005246	49.9923135997133\\
162.984600078689	49.9931021169544\\
163.312800104919	49.9938997170298\\
163.641000131149	49.9947055476787\\
163.969200157379	49.9955187909074\\
164.312800104919	49.9963772825619\\
164.65640005246	49.997242178088\\
165	49.9981126535048\\
165.356388666433	49.9989734584026\\
165.712777332867	49.9997475148658\\
166.0691659993	50.0004386602053\\
166.425554665734	50.001050594067\\
166.781943332167	50.001586878989\\
167.138331998601	50.00205095015\\
167.425554665734	50.0023746627457\\
167.712777332867	50.0026552980747\\
168	50.0028944814432\\
168.622664852516	50.0032614346825\\
169.245329705031	50.0034247787576\\
169.867994557547	50.0034016392996\\
170.245329705031	50.0033036905993\\
170.622664852516	50.0031465337503\\
171	50.0029334079639\\
171.417033129112	50.0026617289773\\
171.834066258224	50.0023779860929\\
172.251099387335	50.0020831293678\\
172.668132516447	50.0017780636514\\
173.085165645559	50.0014636487599\\
173.502198774671	50.0011407035053\\
173.668132516447	50.0010100086072\\
173.834066258224	50.0008781342065\\
174	50.0007451266724\\
174.421397441605	50.0004245471384\\
174.842794883209	50.000139889378\\
175.264192324814	49.9998892940306\\
175.685589766419	49.9996709801181\\
176.106987208023	49.9994832447383\\
176.528384649628	49.9993244564672\\
176.685589766419	49.9992723146314\\
176.842794883209	49.9992239048482\\
177	49.99917914979\\
177.591812132621	49.9990492230768\\
178.183624265242	49.9989795037528\\
178.775436397863	49.9989651980211\\
179.183624265242	49.9989852583339\\
179.591812132621	49.9990281370277\\
180	49.9990924968113\\
180.509033169787	49.9991896165351\\
181.018066339574	49.9992942787231\\
181.52709950936	49.9994058155058\\
182.018066339574	49.9995193165268\\
182.509033169787	49.9996380945665\\
183	49.999761646782\\
183.531210669748	49.9998887899922\\
184.062421339497	49.9999988082149\\
184.593632009245	50.0000928067413\\
185.062421339497	50.0001632900603\\
185.531210669748	50.0002227956222\\
186	50.0002719738816\\
186.798726572154	50.000328548883\\
187.597453144308	50.0003502077216\\
188.396179716463	50.0003406739945\\
188.597453144308	50.0003337629431\\
188.798726572154	50.0003251412102\\
189	50.0003148593852\\
189.659107758387	50.0002760066611\\
190.318215516775	50.0002324783837\\
190.977323275162	50.0001847909025\\
191.318215516775	50.0001586540978\\
191.659107758387	50.0001315956849\\
192	50.0001036756324\\
192.642632296611	50.0000540669371\\
193.285264593221	50.0000116744896\\
193.927896889832	49.9999759474873\\
194.285264593221	49.9999587720036\\
194.642632296611	49.9999434135392\\
195	49.99992979094\\
196	49.9999034935015\\
197	49.9998945007543\\
198	49.9999005359235\\
198.666666666667	49.9999097191874\\
199.333333333333	49.9999200515411\\
200	49.9999314052289\\
};
\end{axis}
\end{tikzpicture}%}
  \caption{Step response using a zero-order hold of sample time $3$ sec.}
  \label{fig:Q7.3}
\end{figure}

\begin{figure}[H]\centering
	\centering
	\scalebox{1}{% This file was created by matlab2tikz.
%
%The latest updates can be retrieved from
%  http://www.mathworks.com/matlabcentral/fileexchange/22022-matlab2tikz-matlab2tikz
%where you can also make suggestions and rate matlab2tikz.
%
\definecolor{mycolor1}{rgb}{0.00000,0.44700,0.74100}%
%
\begin{tikzpicture}

\begin{axis}[%
width=4.133in,
height=3.26in,
at={(0.693in,0.44in)},
scale only axis,
xmin=0,
xmax=200,
xmajorgrids,
ymin=30,
ymax=60,
ymajorgrids,
axis background/.style={fill=white}
]
\addplot [color=mycolor1,solid,forget plot]
  table[row sep=crcr]{%
0	40\\
0.0666666666666667	39.9988529199354\\
0.133333333333333	39.9954198587329\\
0.2	39.9897130376372\\
0.266666666666667	39.9817446149091\\
0.333333333333333	39.9715266859614\\
0.4	39.959071286313\\
0.47499490925035	39.942400621861\\
0.5499898185007	39.9229306571585\\
0.62498472775105	39.9006782627291\\
0.703019200761635	39.8745881554063\\
0.78105367377222	39.8455224785902\\
0.859088146782806	39.8134999414055\\
0.940721928520819	39.7768560227102\\
1.02235571025883	39.7370177534163\\
1.10398949199685	39.6940062251002\\
1.18937445628243	39.6456457248368\\
1.27475942056802	39.5938604847577\\
1.36014438485361	39.5386742795039\\
1.44941284679236	39.4773672565188\\
1.53868130873111	39.4123956293716\\
1.62794977066986	39.3437861740105\\
1.72115687139107	39.2682963135632\\
1.81436397211228	39.1888997791019\\
1.90757107283349	39.1056266274356\\
2.00466164289194	39.0147939058355\\
2.10175221295039	38.9198209377759\\
2.19884278300884	38.820741250687\\
2.29961827190097	38.7135934398924\\
2.4003937607931	38.6020944165113\\
2.50116924968523	38.4862812151995\\
2.60526965685247	38.3621560855701\\
2.70937006401971	38.2335072833433\\
2.81347047118694	38.1003751782987\\
2.92038536533207	37.9590191789915\\
3.02730025947719	37.8130200499409\\
3.13421515362232	37.6624210955585\\
3.24774120796246	37.4975227201725\\
3.36126726230261	37.3275382844275\\
3.47479331664276	37.1525191924883\\
3.59549874757794	36.9609670376133\\
3.71620417851312	36.7638427524235\\
3.8369096094483	36.5612076866583\\
3.89127307296553	36.4681611053372\\
3.94563653648277	36.3740147074874\\
4	36.2787740753648\\
4.07306942194238	36.152095843641\\
4.14613884388476	36.0294931183831\\
4.21920826582714	35.9108811838989\\
4.29227768776952	35.796178044701\\
4.3653471097119	35.6853042573769\\
4.43841653165428	35.5781827711582\\
4.5171075185368	35.46693067802\\
4.59579850541931	35.3598532595318\\
4.67448949230183	35.2568627152937\\
4.7592944849153	35.1503479371381\\
4.84409947752878	35.0483775603095\\
4.92890447014226	34.9508513932065\\
5.02063315931593	34.8502539384762\\
5.1123618484896	34.7546232236021\\
5.20409053766328	34.6638437873081\\
5.30372507114198	34.570606453934\\
5.40335960462069	34.4828218084415\\
5.50299413809939	34.4003553458356\\
5.61179781360876	34.3162212594037\\
5.72060148911813	34.238109789708\\
5.8294051646275	34.1658621089286\\
5.94908455034606	34.0929817836071\\
6.06876393606462	34.0268125816024\\
6.18844332178317	33.9671635161121\\
6.3214736675571	33.9082875860319\\
6.45450401333103	33.8569952407483\\
6.58753435910497	33.8130506489221\\
6.7379330059141	33.7719325624768\\
6.88833165272323	33.7395992897998\\
7.03873029953236	33.71574659807\\
7.21451342374266	33.6982254123178\\
7.39029654795296	33.691441932337\\
7.56607967216326	33.6949672827354\\
7.71071978144218	33.7053088774931\\
7.85535989072109	33.7221319822755\\
8	33.74522419756\\
8.15252700317569	33.7767894410732\\
8.30505400635138	33.8161272006975\\
8.45758100952708	33.8629608681523\\
8.61010801270277	33.9170242882754\\
8.76263501587846	33.9780612407922\\
8.91516201905415	34.0458248474517\\
9.09931712462248	34.1362652848691\\
9.28347223019081	34.2357631553153\\
9.46762733575914	34.3439336416887\\
9.69963552431586	34.4919899943087\\
9.93164371287258	34.6525267448353\\
10.1636519014293	34.824878560811\\
10.4391405218495	35.0440063327187\\
10.7146291422698	35.2779081358734\\
10.99011776269	35.52564605868\\
11.2142019674259	35.7367503696152\\
11.4382861721618	35.9559754962579\\
11.6623703768977	36.1828934821622\\
11.7749135845985	36.2996329541861\\
11.8874567922992	36.4181599860009\\
12	36.538425967994\\
12.1046532338433	36.6499871442254\\
12.2093064676866	36.7594320430574\\
12.3139597015299	36.866787413311\\
12.4186129353732	36.9720796627864\\
12.5232661692165	37.0753348627797\\
12.6279194030599	37.1765787581565\\
12.7443083172095	37.2868447107008\\
12.8606972313592	37.3946889206864\\
12.9770861455089	37.5001454248574\\
13.101868051356	37.6105925589471\\
13.2266499572031	37.7183745757524\\
13.3514318630503	37.8235316747092\\
13.4864175167662	37.9343779025481\\
13.6214031704822	38.0422481322269\\
13.7563888241981	38.1471910175533\\
13.903174783484	38.2580414566287\\
14.04996074277	38.3655476354312\\
14.1967467020559	38.4697691913331\\
14.3573830489624	38.5801300352673\\
14.5180193958688	38.686702967507\\
14.6786557427753	38.7895622395453\\
14.8556869209316	38.8987053126733\\
15.0327180990878	39.0035224114959\\
15.2097492772441	39.1041075609045\\
15.406333661228	39.2109555185282\\
15.6029180452118	39.3128225199338\\
15.7995024291957	39.4098297673651\\
15.8663349527971	39.4417234237903\\
15.9331674763986	39.4730737322289\\
16	39.5038852582858\\
16.1327262155386	39.5626547181104\\
16.2654524310771	39.6176875327205\\
16.3981786466157	39.6690445179591\\
16.5309048621542	39.7167857686254\\
16.6636310776928	39.760970660097\\
16.7963572932313	39.8016578775067\\
16.9460517610879	39.8434222604604\\
17.0957462289444	39.8808934293317\\
17.245440696801	39.9141525042314\\
17.4078633004524	39.9455681024085\\
17.5702859041037	39.9722200303759\\
17.7327085077551	39.9942078126419\\
17.9104071135667	40.0130367284705\\
18.0881057193784	40.026527054908\\
18.26580432519	40.034803605116\\
18.4603184206402	40.0380315578811\\
18.6548325160904	40.035318368683\\
18.8493466115407	40.0268202732182\\
19.0616820126806	40.0111202096076\\
19.2740174138205	39.9889064142977\\
19.4863528149604	39.9603720992146\\
19.6575685433069	39.9328900069572\\
19.8287842716535	39.9015198267338\\
20	39.8663581567543\\
20.1208933589023	39.8404873279588\\
20.2417867178047	39.8151559384308\\
20.362680076707	39.7903558348309\\
20.4835734356094	39.7660789713245\\
20.6044667945117	39.7423174093784\\
20.7253601534141	39.7190633150461\\
20.8612333973674	39.6935238882434\\
20.9971066413207	39.6686049006851\\
21.132979885274	39.6442956945198\\
21.2801979593893	39.6186328762818\\
21.4274160335047	39.5936604523864\\
21.57463410762	39.5693654912071\\
21.7359761404339	39.543502695946\\
21.8973181732478	39.5184218568609\\
22.0586602060616	39.4941068095691\\
22.2367662909783	39.4681355495683\\
22.414872375895	39.4430571979824\\
22.5929784608116	39.4188512186743\\
22.7914222337681	39.3928840773718\\
22.9898660067246	39.3679476850679\\
23.1883097796811	39.3440153925612\\
23.411858463321	39.3182256223527\\
23.6354071469609	39.2936406682986\\
23.8589558306008	39.2702250791665\\
23.9059705537339	39.2654460946238\\
23.9529852768669	39.260716983882\\
24	39.2560374321674\\
24.1385694113088	39.2432962426967\\
24.2771388226177	39.2324812136223\\
24.4157082339265	39.2235470180477\\
24.5542776452353	39.2164492395379\\
24.6928470565442	39.2111443608442\\
24.831416467853	39.2075897336139\\
24.9880803400675	39.2056264474203\\
25.144744212282	39.2057878676585\\
25.3014080844964	39.2080162342612\\
25.4736056180221	39.2127828340913\\
25.6458031515477	39.2199052557249\\
25.8180006850733	39.2293121500175\\
26.0102280851084	39.2424264604957\\
26.2024554851435	39.2582067533857\\
26.3946828851787	39.2765614264422\\
26.6122894085667	39.3003339577143\\
26.8298959319547	39.3271661564961\\
27.0475024553426	39.3569366215433\\
27.2994121284508	39.3949152379287\\
27.5513218015589	39.4364979428166\\
27.803231474667	39.4815149615515\\
27.8688209831113	39.4937791475988\\
27.9344104915557	39.5062622145298\\
28	39.5189613701785\\
28.1392395133098	39.5459325702102\\
28.2784790266195	39.572477640315\\
28.4177185399293	39.5986027835891\\
28.556958053239	39.6243141151023\\
28.6961975665488	39.6496176623588\\
28.8354370798585	39.6745193677976\\
28.9962072431167	39.7027794390204\\
29.1569774063748	39.7305204722606\\
29.317747569633	39.7577512021688\\
29.4938952930581	39.787010855333\\
29.6700430164833	39.8156793337178\\
29.8461907399085	39.8437675460977\\
30.0424314412404	39.8743895139419\\
30.2386721425724	39.9043191108213\\
30.4349128439043	39.9335705828379\\
30.6556582436586	39.9656816481366\\
30.8764036434129	39.9969717720502\\
31.0971490431671	40.0274599763527\\
31.348749521522	40.0612555543731\\
31.6003499998768	40.0940603893223\\
31.8519504782316	40.1259006815833\\
31.9013003188211	40.1320350091484\\
31.9506501594105	40.1381334026344\\
32	40.1441960508241\\
32.1538964440721	40.1623176487676\\
32.3077928881442	40.1790018873996\\
32.4616893322163	40.194279663293\\
32.6155857762884	40.208181372883\\
32.7694822203606	40.2207369124273\\
32.9233786644327	40.2319756954817\\
33.0985928776949	40.2432051237268\\
33.2738070909571	40.2528071774334\\
33.4490213042194	40.2608230330602\\
33.6427017183607	40.2678865853065\\
33.836382132502	40.2731148379764\\
34.0300625466434	40.2765601963564\\
34.2472583056547	40.2783666155159\\
34.464454064666	40.278066212083\\
34.6816498236773	40.2757281241652\\
34.9274613097975	40.2707089006609\\
35.1732727959177	40.2632615547545\\
35.4190842820379	40.2534789946514\\
35.6127228546919	40.2441876775045\\
35.806361427346	40.2335465390153\\
36	40.2215978087901\\
36.1641111483948	40.2109068518111\\
36.3282222967896	40.2001740895774\\
36.4923334451844	40.189402454527\\
36.6564445935791	40.1785948111575\\
36.8205557419739	40.1677539561493\\
36.9846668903687	40.156882621057\\
37.176337987329	40.144150586511\\
37.3680090842892	40.1313847493276\\
37.5596801812495	40.1185891338382\\
37.7730131918708	40.1043171611969\\
37.9863462024922	40.0900183729214\\
38.1996792131136	40.0756978357886\\
38.4420904314972	40.0594051687731\\
38.6845016498809	40.0430976965517\\
38.9269128682646	40.0267821135723\\
39.206029494205	40.0079943896947\\
39.4851461201455	39.9892138891339\\
39.7642627460859	39.9704496437213\\
39.8428418307239	39.9651711140647\\
39.921420915362	39.9598947463623\\
40	39.9546207252474\\
40.1617428209242	39.9441764447705\\
40.3234856418483	39.9345367957106\\
40.4852284627725	39.9256822216263\\
40.6469712836966	39.917593551581\\
40.8087141046208	39.9102519980419\\
40.9704569255449	39.9036391430925\\
41.1620194431999	39.8967249474144\\
41.3535819608549	39.8907782710226\\
41.5451444785098	39.8857704654969\\
41.7580682553436	39.8812719609\\
41.9709920321773	39.8778614025494\\
42.183915809011	39.8755023244266\\
42.4262993247523	39.8740514466391\\
42.6686828404936	39.8738662810997\\
42.911066356235	39.8748974692764\\
43.1911032531322	39.8775406943499\\
43.4711401500294	39.8816723121499\\
43.7511770469267	39.8872234155529\\
43.8341180312845	39.889130174913\\
43.9170590156422	39.8911539198827\\
44	39.8932929812234\\
44.1942676071621	39.8985117271784\\
44.3885352143242	39.9038761287946\\
44.5828028214863	39.9093803893127\\
44.7770704286484	39.9150188584413\\
44.9713380358105	39.9207860316221\\
45.1656056429726	39.9266765433991\\
45.4000078624982	39.9339407645848\\
45.6344100820239	39.9413680250569\\
45.8688123015495	39.9489496781365\\
46.1343665285908	39.9577149057466\\
46.3999207556321	39.9666557771843\\
46.6654749826733	39.975760938339\\
46.9754659175433	39.9865829429793\\
47.2854568524132	39.9975971907879\\
47.5954477872831	40.0087876520462\\
47.7302985248554	40.013706759434\\
47.8651492624277	40.0186550763682\\
48	40.0236314018587\\
48.1791023649896	40.030010312875\\
48.3582047299792	40.0359007160372\\
48.5373070949688	40.041315114199\\
48.7164094599584	40.0462657581187\\
48.895511824948	40.0507646470745\\
49.0746141899376	40.0548235388299\\
49.2922357639058	40.0591798471408\\
49.509857337874	40.0629237341656\\
49.7274789118421	40.0660748969728\\
49.9715415704278	40.0689275640875\\
50.2156042290134	40.0710852438324\\
50.459666887599	40.0725735770596\\
50.7416478314561	40.0734923367446\\
51.0236287753133	40.0735881109836\\
51.3056097191704	40.0728969565762\\
51.5370731461136	40.0717660854348\\
51.7685365730568	40.070147216931\\
52	40.0680585546281\\
52.2416113218349	40.065521268362\\
52.4832226436699	40.0627544824453\\
52.7248339655048	40.0597679205184\\
52.9664452873398	40.0565710217197\\
53.2080566091747	40.0531729415119\\
53.4496679310097	40.0495825674365\\
53.746869756334	40.0449150780345\\
54.0440715816584	40.039985170366\\
54.3412734069828	40.0348077895345\\
54.6880687188549	40.0284727901701\\
55.034864030727	40.0218423444718\\
55.3816593425991	40.0149374413684\\
55.5877728950661	40.0107118238305\\
55.793886447533	40.006400424998\\
56	40.0020071995535\\
56.1976456098025	39.997907289364\\
56.3952912196051	39.9941033445803\\
56.5929368294076	39.9905870021667\\
56.7905824392101	39.9873500901113\\
56.9882280490127	39.9843846266481\\
57.1858736588152	39.9816828119665\\
57.4321542136346	39.9786735806126\\
57.678434768454	39.9760475385348\\
57.9247153232733	39.9737907384756\\
58.2039701754422	39.971661381916\\
58.483225027611	39.9699702290155\\
58.7624798797798	39.9686987732249\\
59.0915172085873	39.967714951067\\
59.4205545373948	39.967260774192\\
59.7495918662023	39.9673091242332\\
59.8330612441348	39.9673981624197\\
59.9165306220674	39.9675174430507\\
60	39.9676665557172\\
60.3241863499502	39.968467779083\\
60.6483726999005	39.9695867415144\\
60.9725590498507	39.9710065061803\\
61.296745399801	39.9727107975248\\
61.6209317497512	39.9746839979357\\
61.9451180997015	39.9769110985624\\
62.3591935902716	39.9801021405801\\
62.7732690808418	39.9836552946801\\
63.187344571412	39.9875433597474\\
63.4582297142746	39.9902557058266\\
63.7291148571373	39.9930934076444\\
64	39.9960497933183\\
64.2185746848966	39.998396866915\\
64.4371493697932	40.0005782858387\\
64.6557240546898	40.0025990962676\\
64.8742987395863	40.0044642197744\\
65.0928734244829	40.0061784537705\\
65.3114481093795	40.0077464774894\\
65.5937631881889	40.0095627783448\\
65.8760782669982	40.0111523083222\\
66.1583933458076	40.0125243092919\\
66.4819663213369	40.0138408023033\\
66.8055392968661	40.0148962681404\\
67.1291122723954	40.0157032173664\\
67.419408181597	40.0162255927292\\
67.7097040907985	40.0165660194411\\
68	40.0167325977132\\
68.4925710113153	40.0166245789218\\
68.9851420226306	40.0160467168743\\
69.4777130339459	40.0150355988888\\
69.9245294540649	40.0137725903838\\
70.3713458741839	40.0122057245573\\
70.8181622943029	40.0103579861579\\
71.212108196202	40.008513262866\\
71.606054098101	40.00648125564\\
72	40.0042754210235\\
72.2402078460706	40.002925944353\\
72.4804156921412	40.0016662381764\\
72.7206235382119	40.0004933587746\\
72.9608313842825	39.999404441931\\
73.2010392303531	39.9983967026153\\
73.4412470764237	39.9974674308049\\
73.7665271812036	39.9963294336038\\
74.0918072859835	39.9953241683563\\
74.4170873907634	39.994445533977\\
74.7937240519241	39.9935786598981\\
75.1703607130848	39.9928649432311\\
75.5469973742455	39.992296009304\\
75.697998249497	39.9921067588306\\
75.8489991247485	39.9919389885465\\
76	39.9917922071436\\
76.3621152547958	39.9915486878758\\
76.7242305095916	39.9914663319195\\
77.0863457643875	39.9915360785927\\
77.4484610191833	39.9917492523827\\
77.8105762739791	39.992097561291\\
78.1726915287749	39.9925730649719\\
78.6678116539695	39.9934153230999\\
79.162931779164	39.994462815525\\
79.6580519043585	39.9956982207785\\
79.7720346029057	39.9960074998632\\
79.8860173014528	39.9963256800456\\
80	39.996652573054\\
80.2932950780083	39.9974764289363\\
80.5865901560166	39.9982450560415\\
80.8798852340248	39.9989605805011\\
81.1731803120331	39.9996250597638\\
81.4664753900414	40.000240482874\\
81.7597704680497	40.0008087748744\\
82.1425089145372	40.0014825692436\\
82.5252473610248	40.0020832479359\\
82.9079858075124	40.0026146226689\\
83.2719905383416	40.0030590219016\\
83.6359952691708	40.003447040549\\
84	40.00378156052\\
84.3714348664166	40.0040449898533\\
84.7428697328331	40.0042091335041\\
85.1143045992497	40.0042796027259\\
85.4857394656663	40.0042617667842\\
85.8571743320829	40.0041607538633\\
86.2286091984994	40.0039814715641\\
86.7339256393048	40.0036203056519\\
87.2392420801101	40.003134122031\\
87.7445585209154	40.0025334456604\\
87.8297056806103	40.0024216718682\\
87.9148528403051	40.0023069748745\\
88	40.0021893997484\\
88.4257357984743	40.0016165787972\\
88.8514715969487	40.0010865696839\\
89.277207395423	40.0005971392053\\
89.6274766144983	40.0002233852011\\
89.9777458335736	39.9998745175628\\
90.3280150526488	39.9995494421308\\
90.8049953568355	39.999143137772\\
91.2819756610221	39.9987764688765\\
91.7589559652087	39.9984470273529\\
91.8393039768058	39.9983950318269\\
91.9196519884029	39.9983440170392\\
92	39.9982939725144\\
92.4017400579855	39.9980799763812\\
92.803480115971	39.9979309167238\\
93.2052201739564	39.997842879746\\
93.6185534012421	39.9978120603774\\
94.0318866285278	39.9978380219193\\
94.4452198558134	39.99791708354\\
94.9634799038756	39.99808589386\\
95.4817399519378	39.9983261802131\\
96	39.9986318237591\\
96.3852185921925	39.9988763774502\\
96.770437184385	39.9991102766698\\
97.1556557765776	39.9993339438858\\
97.5408743687701	39.9995477858224\\
97.9260929609626	39.9997521935216\\
98.3113115531552	39.9999475435788\\
98.8742077021034	40.0002174579237\\
99.4371038510517	40.000469881138\\
99.9999999999991	40.0007058329336\\
100	40.0007058329336\\
100.000000000001	40.0007058329336\\
100.056289614896	40.0021420234399\\
100.112579229791	40.0063735495015\\
100.168868844685	40.0133541055031\\
100.223663179069	40.0227466664\\
100.278457513453	40.034660616682\\
100.333251847836	40.0490561688067\\
100.390233326462	40.0666166964019\\
100.447214805087	40.0867760855073\\
100.504196283712	40.1094925486376\\
100.563329561955	40.1357268731215\\
100.622462840197	40.164626433793\\
100.681596118439	40.1961476085594\\
100.743052471355	40.2316394581531\\
100.804508824272	40.269870104653\\
100.865965177188	40.3107938083631\\
100.930092035253	40.3563185847931\\
100.994218893318	40.4046772244906\\
101.058345751384	40.4558212008061\\
101.126558721887	40.5132280951283\\
101.194771692389	40.5736771660847\\
101.262984662892	40.6371139360197\\
101.335237819715	40.7075072211167\\
101.407490976537	40.7811311660149\\
101.47974413336	40.8579254916164\\
101.556141462588	40.9425076051432\\
101.632538791816	41.0305005522234\\
101.708936121044	41.1218381001056\\
101.789625003677	41.2218666957692\\
101.87031388631	41.3254798828487\\
101.951002768943	41.4326052548281\\
102.036172059064	41.5494111192955\\
102.121341349185	41.6699704635271\\
102.206510639307	41.7942044171415\\
102.296383365436	41.929198893435\\
102.386256091566	42.0681115074527\\
102.476128817695	42.2108565817301\\
102.571144709898	42.365845842281\\
102.666160602101	42.5249295346105\\
102.761176494304	42.6880142483091\\
102.86188394078	42.8651344350987\\
102.962591387256	43.0465419581533\\
103.063298833732	43.2321344444255\\
103.170315104109	43.4338284897854\\
103.277331374487	43.6400198112844\\
103.384347644865	43.8505956467482\\
103.498368889654	44.0796564711345\\
103.612390134443	44.3134416706109\\
103.726411379232	44.5518264369992\\
103.817607586155	44.7457199167985\\
103.908803793077	44.9424179187508\\
104	45.1418615226792\\
104.101372757585	45.364261975954\\
104.20274551517	45.5850659187867\\
104.304118272755	45.8042890071934\\
104.40549103034	46.0219466648174\\
104.506863787925	46.2380540881801\\
104.60823654551	46.4526262528993\\
104.717928020219	46.6830940244043\\
104.827619494929	46.9117999232302\\
104.937310969638	47.1387620830928\\
105.054823333207	47.3799920244127\\
105.172335696776	47.6192628029653\\
105.289848060346	47.8565956770188\\
105.416521191209	48.1102836466395\\
105.543194322072	48.3617698570411\\
105.669867452935	48.6110796363428\\
105.807170250762	48.8788808851525\\
105.944473048589	49.1441855104142\\
106.081775846417	49.407024107773\\
106.231598949303	49.6910520352359\\
106.381422052189	49.9722177717028\\
106.531245155075	50.2505589045418\\
106.696061303514	50.5535353457014\\
106.860877451952	50.8531862940785\\
107.025693600391	51.1495588862648\\
107.208876051663	51.4751672532267\\
107.392058502935	51.7968449778652\\
107.575240954207	52.1146527490234\\
107.716827302805	52.3576804750599\\
107.858413651402	52.5984586032473\\
108	52.8370135665447\\
108.103944338247	53.0081945502034\\
108.207888676494	53.1731245233163\\
108.311833014741	53.3318535076323\\
108.415777352989	53.4844309811913\\
108.519721691236	53.6309058907536\\
108.623666029483	53.7713266757858\\
108.739071094181	53.920196029503\\
108.85447615888	54.0617266232072\\
108.969881223578	54.1959826670418\\
109.092978799717	54.3312412690399\\
109.216076375856	54.4583717239816\\
109.339173951995	54.5774494618966\\
109.471266741673	54.6963568008403\\
109.603359531351	54.8061686687541\\
109.735452321029	54.906975296179\\
109.877261216949	55.0052753842589\\
110.01907011287	55.0934088937176\\
110.160879008791	55.1714839322835\\
110.312924241998	55.2441426573011\\
110.464969475205	55.3054918015733\\
110.617014708413	55.3556604663421\\
110.779274521898	55.3970112271523\\
110.941534335383	55.4259280329553\\
111.103794148868	55.4425630523621\\
111.275412522636	55.4469598616497\\
111.447030896404	55.437962369565\\
111.618649270171	55.4157454471365\\
111.745766180114	55.3908695209754\\
111.872883090057	55.3589059776666\\
112	55.3199242437082\\
112.187430766437	55.2513168465795\\
112.374861532875	55.1708660851557\\
112.562292299312	55.0787748920327\\
112.721696736736	54.991445000871\\
112.881101174159	54.8959644580365\\
113.040505611583	54.792455712049\\
113.233491997705	54.656571703545\\
113.426478383827	54.5093138613237\\
113.61946476995	54.350895755505\\
113.864481102243	54.1340217050855\\
114.109497434535	53.8999324665019\\
114.354513766828	53.6490571185635\\
114.619217866961	53.3596519100087\\
114.883921967093	53.0516867951783\\
115.148626067225	52.7256933307935\\
115.367489858817	52.442914949849\\
115.586353650409	52.1484707651387\\
115.805217442001	51.8426572214179\\
115.870144961334	51.7497922863971\\
115.935072480667	51.6559604880012\\
116	51.561169514966\\
116.098585572433	51.4184856159539\\
116.197171144866	51.2796985420527\\
116.295756717299	51.1447430421525\\
116.394342289732	51.0135551366924\\
116.492927862166	50.8860720799188\\
116.591513434599	50.7622323137589\\
116.700616904414	50.6293545285979\\
116.80972037423	50.500784705563\\
116.918823844045	50.3764442300188\\
117.035583568023	50.2479759233426\\
117.152343292002	50.1241707588591\\
117.26910301598	50.0049382597204\\
117.395190415579	49.881213680235\\
117.521277815178	49.7626091954096\\
117.647365214777	49.649018123999\\
117.784386520113	49.5311395393886\\
117.921407825448	49.4189276216757\\
118.058429130784	49.3122546486754\\
118.208596659293	49.2015609897603\\
118.358764187801	49.0972102114757\\
118.50893171631	48.9990461784627\\
118.675375310337	48.8972873007301\\
118.841818904364	48.8027402568067\\
119.00826249839	48.7152087258118\\
119.195854814092	48.6247260753492\\
119.383447129793	48.5426470360439\\
119.571039445494	48.4687140095769\\
119.714026296996	48.4176767755943\\
119.857013148498	48.3711191890112\\
120	48.3289359552343\\
120.14499286046	48.2913949136359\\
120.289985720921	48.2598505365815\\
120.434978581381	48.2341475769488\\
120.579971441842	48.2141348902271\\
120.724964302302	48.1996653033494\\
120.869957162762	48.1905954415211\\
121.037927468915	48.1866556916935\\
121.205897775068	48.1895642863599\\
121.37386808122	48.199116293481\\
121.562885150616	48.2175617996534\\
121.751902220012	48.2438935034054\\
121.940919289408	48.2778456150564\\
122.162912323423	48.3271045732955\\
122.384905357438	48.3861183805533\\
122.606898391453	48.454499900788\\
122.897707441356	48.5576323670576\\
123.188516491258	48.6754003564846\\
123.47932554116	48.8070397797807\\
123.652883694107	48.8919118429555\\
123.826441847053	48.9813161449582\\
124	49.075108133676\\
124.119169276447	49.1404102982759\\
124.238338552895	49.2045787395029\\
124.357507829342	49.2676277922734\\
124.476677105789	49.3295716230919\\
124.595846382236	49.3904242310708\\
124.715015658684	49.4501994520246\\
124.849984211697	49.5166157020167\\
124.98495276471	49.5816871030345\\
125.119921317723	49.6454330138568\\
125.265949083627	49.712931455945\\
125.411976849532	49.7789245090303\\
125.558004615436	49.8434356968754\\
125.717944851942	49.9124201273604\\
125.877885088447	49.9796847058604\\
126.037825324953	50.0452589838576\\
126.214226658327	50.1156568683257\\
126.390627991701	50.1840723918255\\
126.567029325075	50.2505432999515\\
126.763346212026	50.3222776337343\\
126.959663098977	50.3916996352625\\
127.155979985928	50.4588585837064\\
127.376764490746	50.5317441985622\\
127.597548995565	50.6018960252023\\
127.818333500384	50.6693800415552\\
127.878889000256	50.6874315570406\\
127.939444500128	50.7052885635689\\
128	50.722952367912\\
128.136833190379	50.7611513252548\\
128.273666380758	50.796386799483\\
128.410499571137	50.828705531918\\
128.547332761516	50.858153749981\\
128.684165951895	50.8847771663346\\
128.820999142274	50.9086209958047\\
128.977565441666	50.9325511192653\\
129.134131741057	50.9529668783336\\
129.290698040448	50.9699336549158\\
129.461430743874	50.9845802098282\\
129.632163447299	50.995284808828\\
129.802896150724	51.0021289774934\\
129.991403489444	51.0052981121571\\
130.179910828164	51.0039658980726\\
130.368418166885	50.9982374644121\\
130.577390697177	50.9868744771434\\
130.786363227468	50.970375022017\\
130.99533575776	50.9488756993753\\
131.227543695069	50.9192846040348\\
131.459751632377	50.8838669175088\\
131.691959569686	50.8428004532423\\
131.794639713124	50.8228855082267\\
131.897319856562	50.8019151074305\\
132	50.7799040196196\\
132.144987737073	50.7479787440406\\
132.289975474146	50.7158247929175\\
132.434963211219	50.6834502436576\\
132.579950948293	50.6508630426241\\
132.724938685366	50.6180710048799\\
132.869926422439	50.5850818181198\\
133.036404447837	50.546969560501\\
133.202882473235	50.5086185850486\\
133.369360498633	50.470039919975\\
133.552500343897	50.4273500344868\\
133.73564018916	50.3844117960375\\
133.918780034424	50.3412390133612\\
134.123561261042	50.2927034315644\\
134.328342487659	50.243909945282\\
134.533123714277	50.194876568623\\
134.764518733143	50.1392040829785\\
134.99591375201	50.0832724194004\\
135.227308770876	50.0271055834334\\
135.484872513918	49.9643386607768\\
135.742436256959	49.9013404612775\\
136	49.8381412054251\\
136.139434821275	49.8048034429062\\
136.27886964255	49.7732913190431\\
136.418304463825	49.7435690063821\\
136.5577392851	49.7156012809997\\
136.697174106375	49.6893535173048\\
136.83660892765	49.664791670648\\
136.997053755871	49.6385714887434\\
137.157498584092	49.6144892092718\\
137.317943412313	49.5924957362575\\
137.494016660313	49.5707066835169\\
137.670089908314	49.5513129064653\\
137.846163156315	49.534253407806\\
138.042744661984	49.5178916103191\\
138.239326167653	49.5042839681876\\
138.435907673323	49.4933512189405\\
138.658019376387	49.4841201079623\\
138.880131079452	49.4780963313568\\
139.102242782517	49.475174064887\\
139.357795606742	49.4755152803325\\
139.613348430967	49.4796738067481\\
139.868901255192	49.4875021142089\\
139.912600836795	49.4891977526081\\
139.956300418397	49.4909958082278\\
140	49.4928955834719\\
140.218497908013	49.5034663741792\\
140.436995816025	49.515637699441\\
140.655493724038	49.5293551024147\\
140.856785777021	49.5433148292113\\
141.058077830003	49.5585012689867\\
141.259369882986	49.5748749621875\\
141.494156127914	49.5954218895921\\
141.728942372841	49.6174721994005\\
141.963728617769	49.6409680560913\\
142.230472823262	49.6693449028932\\
142.497217028755	49.6994351964952\\
142.763961234248	49.7311615768346\\
143.072856046146	49.7698472754185\\
143.381750858043	49.810514507583\\
143.690645669941	49.8530552679238\\
143.793763779961	49.8676556107174\\
143.89688188998	49.8824493602306\\
144	49.8974327965723\\
144.151796356991	49.9191973166272\\
144.303592713982	49.9401125659099\\
144.455389070973	49.9601940172169\\
144.607185427964	49.9794569142501\\
144.758981784955	49.9979162721328\\
144.910778141946	50.015586884266\\
145.092033104518	50.0356738812103\\
145.27328806709	50.0546814599989\\
145.454543029663	50.0726335929768\\
145.654418096119	50.0912344967879\\
145.854293162575	50.1086115684767\\
146.05416822903	50.124795180994\\
146.279726867829	50.1416622803327\\
146.505285506627	50.1570892898294\\
146.730844145425	50.1711171708073\\
146.988136969041	50.185461565623\\
147.245429792656	50.1980955363494\\
147.502722616272	50.2090756230638\\
147.668481744181	50.2152992763871\\
147.834240872091	50.22087379957\\
148	50.2258133721318\\
148.217856233162	50.231140940793\\
148.435712466324	50.2349911035592\\
148.653568699486	50.2374071366105\\
148.871424932648	50.2384313850157\\
149.08928116581	50.2381052626444\\
149.307137398971	50.2364692929603\\
149.569226283693	50.2328214289301\\
149.831315168414	50.2274025055355\\
150.093404053136	50.2202781143348\\
150.393910415926	50.2100930026224\\
150.694416778717	50.1978431966913\\
150.994923141508	50.1836193751048\\
151.329948761005	50.1655420160385\\
151.664974380503	50.1452380962416\\
152	50.1228214223994\\
152.175472627205	50.1106986453673\\
152.350945254409	50.0988762240071\\
152.526417881614	50.0873485182431\\
152.701890508818	50.0761099817304\\
152.877363136023	50.0651551616142\\
153.052835763228	50.0544786954697\\
153.263224213894	50.0420373483378\\
153.473612664561	50.0299796556019\\
153.684001115227	50.0182968753822\\
153.919610692202	50.0056480307031\\
154.155220269177	49.9934468635985\\
154.390829846151	49.9816818655926\\
154.661871097701	49.9686725022973\\
154.93291234925	49.9562087753569\\
155.2039536008	49.9442744122648\\
155.469302400533	49.9330882524143\\
155.734651200267	49.9223797824887\\
156	49.9121348900671\\
156.200535306064	49.9049813921875\\
156.401070612129	49.898647339247\\
156.601605918193	49.8931103201141\\
156.802141224257	49.8883484106242\\
157.002676530322	49.8843401709817\\
157.203211836386	49.8810646263581\\
157.444292651083	49.8780677370386\\
157.68537346578	49.876065374069\\
157.926454280478	49.8750236726226\\
158.201155512446	49.8749656393331\\
158.475856744414	49.8760647377202\\
158.750557976382	49.8782750126861\\
159.073679079109	49.882236794974\\
159.396800181836	49.8876043089297\\
159.719921284564	49.8943099497139\\
159.813280856376	49.8964872606987\\
159.906640428188	49.8987692899034\\
160	49.9011545189018\\
160.212244781889	49.9067242535161\\
160.424489563778	49.9123538044012\\
160.636734345667	49.9180400769628\\
160.848979127556	49.9237800570945\\
161.061223909445	49.9295708108131\\
161.273468691334	49.9354094806076\\
161.534665509191	49.9426564397654\\
161.795862327047	49.9499667277904\\
162.057059144903	49.9573354779653\\
162.356112023017	49.9658379378066\\
162.655164901131	49.9744040599459\\
162.954217779244	49.9830272918345\\
163.30281185283	49.9931427549124\\
163.651405926415	50.0033177081999\\
164	50.0135430238471\\
164.209454497015	50.0194645554228\\
164.41890899403	50.0249229646432\\
164.628363491045	50.0299306656242\\
164.83781798806	50.0344998090999\\
165.047272485075	50.0386422831573\\
165.25672698209	50.0423697241207\\
165.512181750454	50.0463705408107\\
165.767636518818	50.0497910939611\\
166.023091287182	50.0526510222405\\
166.31520741254	50.0552588439669\\
166.607323537899	50.0571863495976\\
166.899439663258	50.0584604899379\\
167.266293108838	50.0591752339946\\
167.633146554419	50.0589506614714\\
168	50.0578348664867\\
168.306630626399	50.0563365007402\\
168.613261252798	50.0544378902782\\
168.919891879198	50.0521571222429\\
169.226522505597	50.0495116945681\\
169.533153131996	50.0465185177726\\
169.839783758395	50.0431939518395\\
170.228887365157	50.038522883298\\
170.617990971919	50.0333747052933\\
171.00709457868	50.0277792061269\\
171.338063052454	50.0226890237434\\
171.669031526227	50.0173127023491\\
172	50.0116664754093\\
172.222327405684	50.0078957678188\\
172.444654811368	50.004339584226\\
172.666982217051	50.000992050082\\
172.889309622735	49.9978474232132\\
173.111637028419	49.9949000933253\\
173.333964434103	49.992144576267\\
173.618310255854	49.988891416411\\
173.902656077605	49.9859323327634\\
174.187001899356	49.9832567220861\\
174.513926910676	49.980517434178\\
174.840851921996	49.9781241283637\\
175.167776933316	49.9760621630424\\
175.445184622211	49.9745618233693\\
175.722592311105	49.9732814916936\\
176	49.972213040173\\
176.31659713458	49.9713046034212\\
176.633194269159	49.9707710524441\\
176.949791403739	49.9705959086619\\
177.266388538319	49.9707632459907\\
177.582985672899	49.9712576881689\\
177.899582807478	49.97206437342\\
178.311417473493	49.9735574019187\\
178.723252139508	49.9755241953193\\
179.135086805523	49.9779357431701\\
179.423391203682	49.9798737030556\\
179.711695601841	49.982006878395\\
180	49.9843264193407\\
180.253968662071	49.9864132462242\\
180.507937324143	49.9884293153049\\
180.761905986214	49.9903765039056\\
181.015874648285	49.9922566447357\\
181.269843310356	49.9940715260611\\
181.523811972428	49.9958228938284\\
181.857959315754	49.9980332821757\\
182.192106659081	50.0001404424018\\
182.526254002408	50.0021480295681\\
182.917314027394	50.0043758419304\\
183.30837405238	50.0064775947733\\
183.699434077366	50.0084585651529\\
183.799622718244	50.0089472684404\\
183.899811359122	50.0094284617914\\
184	50.0099022278155\\
184.297575091091	50.0111889460766\\
184.595150182181	50.0122624405542\\
184.892725273272	50.0131311452604\\
185.190300364362	50.0138032349745\\
185.487875455453	50.0142866262126\\
185.785450546543	50.0145889926728\\
186.168835071391	50.0147236865247\\
186.552219596239	50.0145855454012\\
186.935604121088	50.0141892827359\\
187.290402747392	50.0136048739091\\
187.645201373696	50.0128222420861\\
188	50.0118517293133\\
188.31722660585	50.0108919821634\\
188.6344532117	50.009911197877\\
188.95167981755	50.008910813385\\
189.2689064234	50.0078922118986\\
189.58613302925	50.0068567231282\\
189.903359635101	50.0058056269251\\
190.324943908743	50.0043867070685\\
190.746528182385	50.0029451484932\\
191.168112456028	50.0014835690827\\
191.445408304019	50.0005125088914\\
191.722704152009	49.9995345384083\\
192	49.9985503091617\\
192.300862356292	49.9975370942553\\
192.601724712583	49.9966365563887\\
192.902587068875	49.9958443316361\\
193.203449425166	49.9951561907486\\
193.504311781458	49.9945680385896\\
193.805174137749	49.9940759060792\\
194.210385236881	49.9935580727159\\
194.615596336013	49.9931984103255\\
195.020807435145	49.992988251941\\
195.347204956763	49.9929220265847\\
195.673602478382	49.992943244549\\
196	49.9930478806904\\
196.456850935189	49.9933073664397\\
196.913701870379	49.9936747981873\\
197.370552805568	49.9941429220798\\
197.827403740758	49.9947048375623\\
198.284254675947	49.9953539960359\\
198.741105611137	49.9960841673222\\
199.160737074091	49.9968211528602\\
199.580368537046	49.9976171307471\\
200	49.9984679404915\\
};
\end{axis}
\end{tikzpicture}%}
  \caption{Step response using a zero-order hold of sample time $4$ sec.}
  \label{fig:Q7.4}
\end{figure}

\begin{figure}[H]\centering
	\centering
	\scalebox{1}{% This file was created by matlab2tikz.
%
%The latest updates can be retrieved from
%  http://www.mathworks.com/matlabcentral/fileexchange/22022-matlab2tikz-matlab2tikz
%where you can also make suggestions and rate matlab2tikz.
%
\definecolor{mycolor1}{rgb}{0.00000,0.44700,0.74100}%
%
\begin{tikzpicture}

\begin{axis}[%
width=4.133in,
height=3.26in,
at={(0.693in,0.44in)},
scale only axis,
xmin=0,
xmax=200,
xmajorgrids,
ymin=30,
ymax=60,
ymajorgrids,
axis background/.style={fill=white}
]
\addplot [color=mycolor1,solid,forget plot]
  table[row sep=crcr]{%
0	40\\
0.0666666666666667	39.9988529199354\\
0.133333333333333	39.9954198587329\\
0.2	39.9897130376372\\
0.266666666666667	39.9817446149091\\
0.333333333333333	39.9715266859614\\
0.4	39.959071286313\\
0.47499490925035	39.942400621861\\
0.5499898185007	39.9229306571585\\
0.62498472775105	39.9006782627291\\
0.703019200761635	39.8745881554063\\
0.78105367377222	39.8455224785902\\
0.859088146782806	39.8134999414055\\
0.940721928520819	39.7768560227102\\
1.02235571025883	39.7370177534163\\
1.10398949199685	39.6940062251002\\
1.18937445628243	39.6456457248368\\
1.27475942056802	39.5938604847577\\
1.36014438485361	39.5386742795039\\
1.44941284679236	39.4773672565188\\
1.53868130873111	39.4123956293716\\
1.62794977066986	39.3437861740105\\
1.72115687139107	39.2682963135632\\
1.81436397211228	39.1888997791019\\
1.90757107283349	39.1056266274356\\
2.00466164289194	39.0147939058355\\
2.10175221295039	38.9198209377759\\
2.19884278300884	38.820741250687\\
2.29961827190097	38.7135934398924\\
2.4003937607931	38.6020944165113\\
2.50116924968523	38.4862812151995\\
2.60526965685247	38.3621560855701\\
2.70937006401971	38.2335072833433\\
2.81347047118694	38.1003751782987\\
2.92038536533207	37.9590191789915\\
3.02730025947719	37.8130200499409\\
3.13421515362232	37.6624210955585\\
3.24774120796246	37.4975227201725\\
3.36126726230261	37.3275382844275\\
3.47479331664276	37.1525191924883\\
3.59549874757794	36.9609670376133\\
3.71620417851312	36.7638427524235\\
3.8369096094483	36.5612076866583\\
3.96415943740082	36.3416866294139\\
4.09140926535334	36.1161806774553\\
4.21865909330586	35.8847613318055\\
4.35177080227781	35.6364307829988\\
4.48488251124977	35.3817892660967\\
4.61799422022172	35.1209183765544\\
4.74532948014781	34.8656136149151\\
4.87266474007391	34.6047544140075\\
5	34.3384121678119\\
5.07188113054193	34.1903691697724\\
5.14376226108385	34.0498331984597\\
5.21564339162578	33.9165514362346\\
5.28752452216771	33.7902878085081\\
5.35940565270964	33.6708208155608\\
5.43128678325156	33.5579418209755\\
5.50914238624018	33.4428850845902\\
5.5869979892288	33.3350904691311\\
5.66485359221742	33.2343349910315\\
5.74897099681835	33.1331439079066\\
5.83308840141929	33.039671941439\\
5.91720580602023	32.9536804310779\\
6.00855085440045	32.8685090390744\\
6.09989590278066	32.7916171023606\\
6.19124095116088	32.7227434860242\\
6.29103683903454	32.6563713455716\\
6.3908327269082	32.5989664603365\\
6.49062861478185	32.5502365813833\\
6.60060408068319	32.5062510586337\\
6.71057954658453	32.4721096035373\\
6.82055501248586	32.4474758798045\\
6.94339405264031	32.430809244692\\
7.06623309279477	32.4251819892494\\
7.18907213294922	32.4301913916045\\
7.32968689000925	32.4484908317833\\
7.47030164706929	32.4796822115521\\
7.61091640412932	32.5232490091797\\
7.78204326417996	32.592257645568\\
7.95317012423059	32.6780372049536\\
8.12429698428122	32.7798020258653\\
8.30329663811621	32.902550215193\\
8.48229629195121	33.0411800984961\\
8.6612959457862	33.1949378202119\\
8.84029559962119	33.3631119178271\\
9.01929525345618	33.5450298006921\\
9.19829490729117	33.7400543973555\\
9.37448400845108	33.9442302431058\\
9.55067310961098	34.159974048609\\
9.72686221077089	34.3867670120123\\
9.81790814051393	34.508129576293\\
9.90895407025696	34.6322450891075\\
10	34.7590497035086\\
10.100965610559	34.9003230090415\\
10.2019312211181	35.0399793325062\\
10.3028968316771	35.1780376305257\\
10.4038624422361	35.3145165854478\\
10.5048280527951	35.4494346104649\\
10.6057936633542	35.5828098573935\\
10.7165033063666	35.7273048629442\\
10.8272129493791	35.8699896903132\\
10.9379225923916	36.0108872265656\\
11.0564262820374	36.159749153854\\
11.1749299716833	36.3066163029055\\
11.2934336613291	36.4515154447425\\
11.4212323889045	36.6056046751385\\
11.54903111648	36.7574681863896\\
11.6768298440554	36.9071379226312\\
11.8153841863954	37.0669619246992\\
11.9539385287354	37.2242835754991\\
12.0924928710754	37.3791414873085\\
12.2436890960447	37.5453622200439\\
12.3948853210139	37.7087426101596\\
12.5460815459831	37.8693300863317\\
12.7123478573059	38.0427548106517\\
12.8786141686286	38.2129189828815\\
13.0448804799514	38.3798819789414\\
13.2294445742505	38.5615409928804\\
13.4140086685495	38.7394049230526\\
13.5985727628486	38.9135498522321\\
13.8059069883337	39.1048362378041\\
14.0132412138187	39.2916261490397\\
14.2205754393038	39.4740200205282\\
14.4803836262026	39.6965197653482\\
14.7401918131013	39.9124578437094\\
15	40.1220161338783\\
15.1101083787649	40.2071691858439\\
15.2202167575298	40.2876932152547\\
15.3303251362948	40.363641252395\\
15.4404335150597	40.4350658235818\\
15.5505418938246	40.5020189579566\\
15.6606502725895	40.5645522123229\\
15.784233382258	40.6295371047098\\
15.9078164919265	40.6890900476065\\
16.031399601595	40.7432819074987\\
16.1636701389102	40.795423204673\\
16.2959406762255	40.8415888160919\\
16.4282112135407	40.8818631361842\\
16.5708242884589	40.9187823908301\\
16.7134373633771	40.9490539219628\\
16.8560504382952	40.9727804530081\\
17.0099039846572	40.9911549945522\\
17.1637575310191	41.0021570004276\\
17.317611077381	41.0059116921195\\
17.4833950661884	41.0019878313946\\
17.6491790549958	40.989946669606\\
17.8149630438032	40.9699402930213\\
17.992737388701	40.9398081406824\\
18.1705117335989	40.9008739488331\\
18.3482860784967	40.8533197656177\\
18.537155458002	40.7935555626352\\
18.7260248375073	40.724479353208\\
18.9148942170125	40.6463031169913\\
19.1129674050544	40.5547704860171\\
19.3110405930963	40.4537000209034\\
19.5091137811381	40.343329270198\\
19.6727425207588	40.2453040768572\\
19.8363712603794	40.1412241332017\\
20	40.0312200146416\\
20.1051823355641	39.959675780011\\
20.2103646711283	39.8901962284788\\
20.3155470066924	39.8227472575399\\
20.4207293422566	39.757295292597\\
20.5259116778207	39.693807280653\\
20.6310940133848	39.6322506758833\\
20.7474851546155	39.566346708149\\
20.8638762958461	39.5027257832607\\
20.9802674370767	39.4413459089974\\
21.1052101514479	39.3779033200145\\
21.230152865819	39.3169454683443\\
21.3550955801901	39.2584230944299\\
21.4904154797828	39.1977313605068\\
21.6257353793754	39.1397794594377\\
21.7610552789681	39.0845082785208\\
21.9085060597632	39.0272662976186\\
22.0559568405582	38.9730649694427\\
22.2034076213533	38.9218323952667\\
22.3653329733179	38.8689070676208\\
22.5272583252824	38.8193860054387\\
22.689183677247	38.7731801942131\\
22.8687352245699	38.7257156270492\\
23.0482867718928	38.6821048619115\\
23.2278383192156	38.6422352198651\\
23.429511458774	38.6017781720439\\
23.6311845983324	38.5657511541226\\
23.8328577378907	38.5340071842891\\
24.0636906704691	38.5027466498531\\
24.2945236030474	38.4767029899376\\
24.5253565356257	38.4556755591307\\
24.6835710237505	38.4440599710701\\
24.8417855118752	38.4346489083691\\
25	38.4273830128923\\
25.1365009205864	38.4236187776103\\
25.2730018411728	38.4229917458523\\
25.4095027617592	38.4254215102017\\
25.5460036823456	38.4308295453276\\
25.682504602932	38.439139168056\\
25.8190055235184	38.4502754676665\\
25.9784714540263	38.4667675970269\\
26.1379373845341	38.4869040251027\\
26.297403315042	38.5105740751483\\
26.4731203194609	38.5406195905751\\
26.6488373238798	38.5746849335552\\
26.8245543282988	38.6126341899038\\
27.0230298242675	38.6600042539881\\
27.2215053202362	38.7119771526109\\
27.4199808162049	38.7683747928518\\
27.6496864541305	38.838944488599\\
27.879392092056	38.9149530375908\\
28.1090977299816	38.9961526586368\\
28.3906193484811	39.1023996919989\\
28.6721409669807	39.2156660870595\\
28.9536625854803	39.3355506629725\\
29.3024417236535	39.4926836885314\\
29.6512208618268	39.6586984798938\\
30	39.8329432612507\\
30.1148297731958	39.8905530983148\\
30.2296595463915	39.9460954149461\\
30.3444893195873	39.9995997824343\\
30.459319092783	40.0510954333265\\
30.5741488659788	40.100611262742\\
30.6889786391746	40.1481758384891\\
30.8201638802741	40.2001634085703\\
30.9513491213736	40.2496827462355\\
31.0825343624731	40.2967749159651\\
31.2238956310208	40.3448496767556\\
31.3652568995685	40.3902032174601\\
31.5066181681162	40.4328848940184\\
31.660844800246	40.4764605658393\\
31.8150714323758	40.5169761412984\\
31.9692980645056	40.5544930016799\\
32.138466310753	40.5922677281299\\
32.3076345570003	40.626585844308\\
32.4768028032476	40.6575246342046\\
32.6635747138672	40.6878491581869\\
32.8503466244869	40.7142476767608\\
33.0371185351065	40.7368189755075\\
33.2445896485527	40.7575227129699\\
33.4520607619989	40.7737540670407\\
33.6595318754452	40.7856409498585\\
33.8910268401859	40.7939297180505\\
34.1225218049267	40.7971361885746\\
34.3540167696674	40.7954271358657\\
34.5693445131116	40.789567941563\\
34.7846722565558	40.7797254026994\\
35	40.7660253128628\\
35.1982372832795	40.7504380286565\\
35.396474566559	40.7324276384672\\
35.5947118498385	40.7120690236701\\
35.792949133118	40.689435570237\\
35.9911864163976	40.6645991569471\\
36.1894236996771	40.6376302244588\\
36.423788016571	40.6030899111156\\
36.6581523334649	40.5657768573254\\
36.8925166503589	40.5257998645191\\
37.1630271252339	40.4764839454803\\
37.4335376001088	40.4239192449421\\
37.7040480749838	40.3682602078537\\
38.0319596417309	40.2968568232073\\
38.3598712084779	40.2213854954991\\
38.687782775225	40.1420959542394\\
39.1251885168167	40.0308028284866\\
39.5625942584083	39.913691324497\\
40	39.7912833389742\\
40.1198430000446	39.7579830092828\\
40.2396860000892	39.7264935480144\\
40.3595290001337	39.6967797264838\\
40.4793720001783	39.6688069212108\\
40.5992150002229	39.6425411072653\\
40.7190580002675	39.6179488407619\\
40.8606689216672	39.5910019321726\\
41.0022798430669	39.5662926606062\\
41.1438907644667	39.5437688861183\\
41.2971617148276	39.52179424619\\
41.4504326651886	39.5022565976103\\
41.6037036155495	39.4850939217673\\
41.7727615732928	39.4688452545014\\
41.9418195310361	39.4553327144337\\
42.1108774887795	39.4444786890845\\
42.2990103144664	39.4354330494072\\
42.4871431401534	39.4294842392316\\
42.6752759658404	39.4265332652787\\
42.8875469683191	39.4266826134357\\
43.0998179707979	39.4303914480898\\
43.3120889732766	39.4375294026405\\
43.5564480309463	39.449827976035\\
43.8008070886161	39.466316839105\\
44.0451661462858	39.486815899909\\
44.3634440975239	39.5192430562543\\
44.6817220487619	39.5578106985149\\
45	39.6021667483612\\
45.1467706608634	39.6238290866102\\
45.2935413217268	39.6453430747839\\
45.4403119825902	39.6667092856885\\
45.5870826434536	39.687928298608\\
45.733853304317	39.7090006992816\\
45.8806239651804	39.7299270793839\\
46.0529233072653	39.7543077875796\\
46.2252226493503	39.7784890613915\\
46.3975219914352	39.8024718874334\\
46.5870935978425	39.8286307926542\\
46.7766652042499	39.8545520150707\\
46.9662368106572	39.8802369072683\\
47.1790188252207	39.908786744952\\
47.3918008397843	39.9370425057346\\
47.6045828543478	39.9650061533121\\
47.8459708378111	39.9963780196986\\
48.0873588212744	40.027379412246\\
48.3287468047377	40.0580132657506\\
48.6067406713301	40.09284114318\\
48.8847345379226	40.1271900211807\\
49.1627284045151	40.1610644720852\\
49.4418189363434	40.1945999470931\\
49.7209094681717	40.2276665538201\\
50	40.2602689856696\\
50.1454093206976	40.2764116743324\\
50.2908186413953	40.2911271009145\\
50.4362279620929	40.304444374181\\
50.5816372827905	40.3163921640482\\
50.7270466034881	40.3269987012603\\
50.8724559241858	40.3362917919776\\
51.0433923802553	40.3455738063179\\
51.2143288363249	40.3531223073375\\
51.3852652923945	40.3589802656948\\
51.5729143011139	40.3635145680845\\
51.7605633098334	40.3661172140283\\
51.9482123185529	40.3668419052565\\
52.158142333258	40.365491802757\\
52.368072347963	40.3619293130058\\
52.5780023626681	40.3562250001499\\
52.8147607109956	40.3473080687757\\
53.051519059323	40.335851220557\\
53.2882774076505	40.3219486456029\\
53.5579010105794	40.3032545671582\\
53.8275246135084	40.2816395080586\\
54.0971482164374	40.2572315542582\\
54.3980988109582	40.2268416577474\\
54.6990494054791	40.1932945198013\\
55	40.156752206773\\
55.1495217564851	40.1381740880411\\
55.2990435129701	40.1201784987169\\
55.4485652694552	40.1027540136861\\
55.5980870259403	40.0858893992382\\
55.7476087824253	40.0695736124822\\
55.8971305389104	40.0537957952647\\
56.0716632594711	40.0360450052711\\
56.2461959800318	40.0189960099419\\
56.4207287005925	40.0026324709434\\
56.613173520992	39.9853649559881\\
56.8056183413914	39.9688903280459\\
56.9980631617909	39.953188024121\\
57.2144189131648	39.9364319659608\\
57.4307746645387	39.9205987569528\\
57.6471304159126	39.9056611732189\\
57.8931283620344	39.8897314277126\\
58.1391263081562	39.8748872463658\\
58.385124254278	39.8610917103923\\
58.6693275220312	39.84641227983\\
58.9535307897844	39.8330310576259\\
59.2377340575377	39.8208961767319\\
59.4918227050251	39.8110613585428\\
59.7459113525126	39.802147923018\\
60	39.7941221816722\\
60.180578966958	39.7892746133748\\
60.361157933916	39.7855032865941\\
60.5417369008739	39.782777197621\\
60.7223158678319	39.7810660505299\\
60.9028948347899	39.7803402520058\\
61.0834738017479	39.7805708825602\\
61.3003701814629	39.7820723892906\\
61.5172665611779	39.7848652809257\\
61.7341629408929	39.7889031593632\\
61.9783847111357	39.7948836664045\\
62.2226064813784	39.8023229803874\\
62.4668282516212	39.8111605657048\\
62.7503048338896	39.8230947348166\\
63.0337814161579	39.8367453333958\\
63.3172579984263	39.8520270302744\\
63.6539331884958	39.8721822468764\\
63.9906083785653	39.8943903159321\\
64.3272835686347	39.9185248661399\\
64.5515223790898	39.9356084765964\\
64.7757611895449	39.9534586333718\\
65	39.9720419921983\\
65.1510362653211	39.9845007374574\\
65.3020725306423	39.9963722702592\\
65.4531087959634	40.007668718091\\
65.6041450612845	40.0184020063363\\
65.7551813266057	40.028583858744\\
65.9062175919268	40.038225804131\\
66.0938249097195	40.0494676082784\\
66.2814322275122	40.0599151809733\\
66.4690395453049	40.0695893937308\\
66.6754505192976	40.0793641160058\\
66.8818614932903	40.0882540266655\\
67.088272467283	40.0962851311168\\
67.322264382137	40.1043828290083\\
67.556256296991	40.1114455973008\\
67.7902482118451	40.1175085852357\\
68.058333068321	40.1232701364589\\
68.326417924797	40.1278142370836\\
68.594502781273	40.1311893852632\\
68.906861643371	40.1337099809504\\
69.219220505469	40.1347793282513\\
69.531579367567	40.13446680095\\
69.687719578378	40.1338135429108\\
69.843859789189	40.1328398694596\\
70	40.1315537612325\\
70.3643731346958	40.1274282070377\\
70.7287462693916	40.1218310709277\\
71.0931194040873	40.1148497024594\\
71.4574925387831	40.1065677646408\\
71.8218656734789	40.0970652341448\\
72.1862388081747	40.0864187123092\\
72.6713624407524	40.0705944637989\\
73.1564860733301	40.0530359863793\\
73.6416097059077	40.0338979628199\\
74.0944064706052	40.0147387392171\\
74.5472032353026	39.9944419035786\\
75	39.9731128208222\\
75.1583994425534	39.9657317535093\\
75.3167988851068	39.9588412614302\\
75.4751983276601	39.9524299252571\\
75.6335977702135	39.9464865406606\\
75.7919972127669	39.9410001173711\\
75.9503966553203	39.9359598716984\\
76.1581417302515	39.9300081016246\\
76.3658868051827	39.9247822864997\\
76.5736318801139	39.9202596645919\\
76.802775747573	39.9160601527981\\
77.0319196150322	39.9126601785863\\
77.2610634824914	39.9100315449951\\
77.5246846301962	39.9079259794306\\
77.7883057779011	39.9067645830209\\
78.051926925606	39.906508152608\\
78.359165276377	39.9073007795247\\
78.666403627148	39.9092131634598\\
78.973641977919	39.912189508259\\
79.3157613186126	39.9166902491209\\
79.6578806593063	39.9223725917141\\
80	39.9291683768526\\
80.2145361780731	39.9337762318863\\
80.4290723561463	39.9384047684591\\
80.6436085342194	39.9430519320989\\
80.8581447122925	39.9477157327067\\
81.0726808903657	39.9523942442519\\
81.2872170684388	39.9570856013708\\
81.5510183797032	39.9628692097468\\
81.8148196909676	39.9686662793802\\
82.0786210022321	39.9744737031797\\
82.3811228870729	39.9811419953661\\
82.6836247719138	39.9878156874026\\
82.9861266567546	39.9944906808673\\
83.346031820323	40.0024286311046\\
83.7059369838913	40.010356682107\\
84.0658421474597	40.0182689641453\\
84.3772280983065	40.0250975907027\\
84.6886140491532	40.0319068957574\\
85	40.0386936781251\\
85.195979677238	40.042729317171\\
85.3919593544759	40.0463175094974\\
85.5879390317139	40.049471123127\\
85.7839187089518	40.0522027389943\\
85.9798983861898	40.0545246516777\\
86.1758780634277	40.0564488818858\\
86.4205952570319	40.0583111556546\\
86.6653124506362	40.0595939485038\\
86.9100296442404	40.0603189274162\\
87.1870412153706	40.0604929903845\\
87.4640527865008	40.0600088635815\\
87.7410643576311	40.0588952059474\\
88.0667498759333	40.0568181969076\\
88.3924353942356	40.0539527488031\\
88.7181209125379	40.0503406802844\\
89.1454139416919	40.0445365221336\\
89.572706970846	40.0376029897958\\
90	40.0296224656756\\
90.2115353607014	40.0255050521854\\
90.4230707214029	40.0215438540035\\
90.6346060821043	40.0177347676409\\
90.8461414428058	40.0140737832378\\
91.0576768035072	40.0105569842343\\
91.2692121642087	40.0071805431607\\
91.5300060885409	40.0032055067807\\
91.7908000128731	39.9994313035265\\
92.0515939372053	39.9958513001214\\
92.3500741985235	39.9919840038027\\
92.6485544598417	39.988353290073\\
92.9470347211599	39.9849500987987\\
93.3016654661477	39.9811903561893\\
93.6562962111355	39.9777252698369\\
94.0109269561233	39.9745412217727\\
94.3406179707489	39.9718217091303\\
94.6703089853744	39.9693236720633\\
95	39.967037367567\\
95.2409971437027	39.965615173855\\
95.4819942874054	39.9645323572032\\
95.7229914311081	39.9637763448865\\
95.9639885748109	39.9633349236259\\
96.2049857185136	39.9631962378556\\
96.4459828622163	39.9633487703031\\
96.75240606611	39.963945672172\\
97.0588292700038	39.9649730977395\\
97.3652524738975	39.9664096358225\\
97.7222192276045	39.9685716337793\\
98.0791859813114	39.971229623516\\
98.4361527350184	39.9743536823076\\
98.8738775267365	39.9787787487895\\
99.3116023184547	39.9838115175905\\
99.7493271101728	39.9894045029758\\
99.8328847401152	39.9905320572985\\
99.9164423700576	39.9916780802721\\
99.9999999999991	39.9928422732308\\
100	39.9928422732308\\
100.000000000001	39.9928422732308\\
100.056236911462	39.9950364132209\\
100.112473822924	40.0000085098849\\
100.168710734385	40.0077126758093\\
100.224947645847	40.0181041265853\\
100.281184557308	40.0311391331538\\
100.33742146877	40.046774975574\\
100.396311001445	40.0658906584773\\
100.45520053412	40.0877657424729\\
100.514090066795	40.1123544938449\\
100.575256569304	40.140719153416\\
100.636423071812	40.1719142903635\\
100.697589574321	40.2058921309778\\
100.761240358095	40.24415429892\\
100.824891141868	40.2853278917182\\
100.888541925642	40.3293627220133\\
100.954873601699	40.3782438993089\\
101.021205277756	40.4301249772785\\
101.087536953813	40.4849530158164\\
101.156771092151	40.5452674152035\\
101.22600523049	40.6086785632682\\
101.295239368829	40.6751303775964\\
101.368085828718	40.7482719803735\\
101.440932288607	40.8246569120813\\
101.513778748496	40.9042243675103\\
101.590813307363	40.9917627753127\\
101.66784786623	41.0827253025674\\
101.744882425097	41.1770451188438\\
101.826228229124	41.2802158613351\\
101.907574033151	41.3869826280348\\
101.988919837178	41.4972724110555\\
102.074773085315	41.6174160835316\\
102.160626333452	41.7413231700034\\
102.246479581588	41.8689141820407\\
102.337067096342	42.0074497844008\\
102.427654611095	42.1499119565875\\
102.518242125849	42.2962143831112\\
102.614051893091	42.4550364821604\\
102.709861660334	42.6179635655683\\
102.805671427576	42.7849014333058\\
102.907259217564	42.9661856657335\\
103.008847007551	43.1517697273903\\
103.110434797539	43.3415502876066\\
103.218429869846	43.5477900009344\\
103.326424942154	43.7585414852771\\
103.434420014462	43.9736908226598\\
103.549531142516	44.2077345081355\\
103.66464227057	44.446518983954\\
103.779753398625	44.6899180558563\\
103.902770816994	44.9549972860448\\
104.025788235363	45.2250621227222\\
104.148805653732	45.499972168968\\
104.28059527136	45.7997070090989\\
104.412384888987	46.1046846742263\\
104.544174506614	46.4147484308555\\
104.696116337743	46.7783392218429\\
104.848058168871	47.1482646417033\\
105	47.5243083693922\\
105.097908001569	47.766604879338\\
105.195816003137	48.0050393776166\\
105.293724004706	48.2396502737532\\
105.391632006274	48.4704754778998\\
105.489540007843	48.6975524128009\\
105.587448009412	48.9209180303025\\
105.693742873987	49.1592579671153\\
105.800037738562	49.3933126158972\\
105.906332603137	49.6231274600354\\
106.019855048745	49.8639373432864\\
106.133377494353	50.1000163868928\\
106.246899939961	50.3314178132608\\
106.368872631017	50.5748911142725\\
106.490845322072	50.813089343579\\
106.612818013127	51.0460758383795\\
106.744455411305	51.2917477783253\\
106.876092809483	51.5314994767581\\
107.007730207661	51.7654072169237\\
107.150521843691	52.0126147058555\\
107.293313479721	52.2531289146455\\
107.436105115751	52.4870429819937\\
107.591835315923	52.7347357703063\\
107.747565516096	52.9748044328978\\
107.903295716269	53.2073643352375\\
108.074066003152	53.4538875388407\\
108.244836290036	53.6916654804741\\
108.41560657692	53.9208431086732\\
108.603764514783	54.1635676080431\\
108.791922452647	54.3962118620144\\
108.98008039051	54.6189601677395\\
109.187903464307	54.8537051826228\\
109.395726538103	55.076838306495\\
109.6035496119	55.2885949600534\\
109.735699741266	55.4174351802255\\
109.867849870633	55.5418275589474\\
110	55.6618300167724\\
110.111168673869	55.757387614605\\
110.222337347737	55.8458579944016\\
110.333506021606	55.9272951153926\\
110.444674695475	56.0017524089839\\
110.555843369344	56.0692827869097\\
110.667012043212	56.1299386684396\\
110.790827618219	56.1894663157416\\
110.914643193227	56.2406018441475\\
111.038458768234	56.2834156254993\\
111.170401037757	56.3199589124581\\
111.30234330728	56.347214850904\\
111.434285576803	56.3652661513138\\
111.575392965266	56.3744781807141\\
111.716500353728	56.3733553260975\\
111.857607742191	56.3619958855208\\
112.007938729062	56.3387412490913\\
112.158269715933	56.3040951919417\\
112.308600702804	56.2581732692053\\
112.467636715917	56.1974450756724\\
112.62667272903	56.1243598056791\\
112.785708742144	56.0390506046842\\
112.95213007485	55.9368347432166\\
113.118551407556	55.8215277523638\\
113.284972740262	55.693278342292\\
113.456815103584	55.547426152739\\
113.628657466905	55.3880930303307\\
113.800499830227	55.2154388950118\\
113.975625872785	55.0259447183916\\
114.150751915344	54.8229473122821\\
114.325877957902	54.6066124473062\\
114.502471988134	54.375125849128\\
114.679066018366	54.1304126392856\\
114.855660048597	53.8726397849977\\
114.903773365732	53.8001672388642\\
114.951886682866	53.7267409599579\\
115	53.652364292188\\
115.103839777783	53.4915197907766\\
115.207679555567	53.3320700087154\\
115.31151933335	53.174004373039\\
115.415359111133	53.0173123726914\\
115.519198888917	52.8619835591759\\
115.6230386667	52.7080075454059\\
115.736980048749	52.5405969780345\\
115.850921430798	52.3747892276693\\
115.964862812848	52.2105708261439\\
116.086978312163	52.0363207101449\\
116.209093811479	51.8638644352365\\
116.331209310794	51.6931857578473\\
116.463057950364	51.5108804222643\\
116.594906589933	51.3306083907922\\
116.726755229503	51.1523496676543\\
116.869835607118	50.9611604792651\\
117.012915984732	50.7722937085686\\
117.155996362347	50.5857244159368\\
117.312202228208	50.3846343658987\\
117.46840809407	50.1862212859401\\
117.624613959932	49.9904535813275\\
117.796292363493	49.7783087600753\\
117.967970767054	49.5692803866932\\
118.139649170616	49.3633277445434\\
118.329724005578	49.1388450648171\\
118.519798840539	48.9180291405014\\
118.709873675501	48.7008265188518\\
118.921955887778	48.4626759024115\\
119.134038100054	48.2288855190762\\
119.34612031233	47.9993838503413\\
119.56408020822	47.7679161683332\\
119.78204010411	47.5408281136725\\
120	47.3180451588421\\
120.098663427973	47.2213514448451\\
120.197326855947	47.1309396582622\\
120.29599028392	47.0466669411524\\
120.394653711894	46.9683948224444\\
120.493317139867	46.8959889795552\\
120.591980567841	46.8293190044099\\
120.703009380427	46.7609949644355\\
120.814038193012	46.6995993689627\\
120.925067005598	46.6449626388814\\
121.044599100609	46.5935083524786\\
121.16413119562	46.5494988382671\\
121.283663290631	46.51274177234\\
121.414139833689	46.4806791720113\\
121.544616376747	46.4568017091154\\
121.675092919805	46.4408827037468\\
121.819220722073	46.4322856580776\\
121.963348524341	46.4328466759308\\
122.10747632661	46.4422901570197\\
122.269880529391	46.4632426271005\\
122.432284732173	46.4947686484512\\
122.594688934955	46.5365149895463\\
122.785014721021	46.5979888477861\\
122.975340507087	46.6725073183668\\
123.165666293153	46.7595667368681\\
123.422057951511	46.8958242527122\\
123.678449609869	47.0528436528287\\
123.934841268226	47.2295616571858\\
124.184654282143	47.4197323426791\\
124.434467296059	47.6267659462262\\
124.684280309975	47.8498242370462\\
124.78952020665	47.9483966798048\\
124.894760103325	48.0496139202275\\
125	48.1534197670231\\
125.108669263588	48.2610863227072\\
125.217338527175	48.3670869205512\\
125.326007790763	48.4714405804368\\
125.434677054351	48.5741661057265\\
125.543346317938	48.6752820853935\\
125.652015581526	48.7748068993536\\
125.772920561589	48.8836899640751\\
125.893825541651	48.9906505947259\\
126.014730521714	49.0957132163607\\
126.144713151852	49.2065744899127\\
126.27469578199	49.3152995123725\\
126.404678412127	49.4219174840453\\
126.545734564837	49.5352680549428\\
126.686790717547	49.6462075229304\\
126.827846870256	49.7547716971323\\
126.981810281254	49.8706002218783\\
127.135773692252	49.9836861469554\\
127.28973710325	50.0940740163712\\
127.459005500859	50.2123737631517\\
127.628273898468	50.3275227490129\\
127.797542296077	50.4395773860159\\
127.985185842916	50.5602469687079\\
128.172829389755	50.6772563303544\\
128.360472936594	50.7906783869986\\
128.570498553336	50.9134629766871\\
128.780524170079	51.0319415521447\\
128.990549786821	51.1462106012143\\
129.228247900523	51.2705764550429\\
129.465946014225	51.3898056044542\\
129.703644127926	51.5040291707605\\
129.802429418618	51.5500575505687\\
129.901214709309	51.5952526083954\\
130	51.6396233119967\\
130.119698102094	51.6908109949343\\
130.239396204187	51.7378711780864\\
130.359094306281	51.7808536435441\\
130.478792408375	51.8198077697317\\
130.598490510469	51.8547825323607\\
130.718188612562	51.8858265215235\\
130.854906563818	51.9165375241201\\
130.991624515075	51.9422545621919\\
131.128342466331	51.9630482483606\\
131.275521638964	51.9800103846284\\
131.422700811598	51.9914345688251\\
131.569879984231	51.9974065681129\\
131.729907757983	51.9978122448731\\
131.889935531734	51.991980950522\\
132.049963305486	51.9800198958124\\
132.224130610029	51.9601601350747\\
132.398297914572	51.9333008954601\\
132.572465219115	51.8995763973196\\
132.761828654519	51.8552747059716\\
132.951192089923	51.803183609686\\
133.140555525326	51.7434703787097\\
133.345260226979	51.6705372758838\\
133.549964928632	51.5890972427939\\
133.754669630285	51.4993548509304\\
133.97344530869	51.3944945440271\\
134.192220987096	51.2806265404217\\
134.410996665501	51.1579922378572\\
134.607331110334	51.0406700546161\\
134.803665555167	50.9166505926876\\
135	50.7861027241774\\
135.114610423719	50.7088597999268\\
135.229220847438	50.6332757046605\\
135.343831271157	50.5593267531556\\
135.458441694875	50.4869895623559\\
135.573052118594	50.4162410497453\\
135.687662542313	50.3470584263003\\
135.815879162191	50.2714907174463\\
135.944095782069	50.1978235417733\\
136.072312401947	50.1260262705589\\
136.210711430587	50.0505909515168\\
136.349110459227	49.977261757505\\
136.487509487866	49.9060018610142\\
136.63848608085	49.8305832734603\\
136.789462673833	49.7575377422069\\
136.940439266816	49.6868197393617\\
137.106257217006	49.6117787204739\\
137.272075167196	49.5394331049138\\
137.437893117386	49.4697257240763\\
137.621613203429	49.395505563072\\
137.805333289471	49.3243803878059\\
137.989053375514	49.2562769207109\\
138.194796175029	49.1835080221558\\
138.400538974543	49.1143401415687\\
138.606281774057	49.0486769680702\\
138.839895047797	48.978248756031\\
139.073508321538	48.9120835479189\\
139.307121595278	48.8500505540362\\
139.538081063519	48.7926595890179\\
139.769040531759	48.7390632237932\\
140	48.6891447741767\\
140.131751144218	48.6633745373943\\
140.263502288436	48.6409128155127\\
140.395253432654	48.6216878965717\\
140.527004576873	48.6056295225304\\
140.658755721091	48.5926688613875\\
140.790506865309	48.5827384597065\\
140.942567513751	48.5749578970358\\
141.094628162194	48.5710264203171\\
141.246688810636	48.5708468910001\\
141.413337740906	48.5748468572167\\
141.579986671176	48.5831173576975\\
141.746635601447	48.5955395186329\\
141.932930617867	48.6141982899139\\
142.119225634286	48.6377429103761\\
142.305520650706	48.6660201065156\\
142.517346337201	48.7037332562745\\
142.729172023696	48.7471599084973\\
142.94099771019	48.796094100212\\
143.189731110356	48.8603158318311\\
143.438464510521	48.9315446695761\\
143.687197910686	49.0094798561807\\
144.005519828806	49.1185412109033\\
144.323841746926	49.2375304207845\\
144.642163665045	49.3658918745764\\
144.761442443364	49.4162954219394\\
144.880721221682	49.4679132820955\\
145	49.520718941323\\
145.123467387709	49.5752940716204\\
145.246934775419	49.6285007886173\\
145.370402163128	49.6803580638023\\
145.493869550838	49.7308846446726\\
145.617336938547	49.7800990556275\\
145.740804326257	49.8280196037077\\
145.883163964294	49.881690283122\\
146.025523602331	49.9336922697574\\
146.167883240368	49.9840525728568\\
146.322212299189	50.0368235088475\\
146.476541358011	50.0877298649703\\
146.630870416832	50.1368046178651\\
146.800670261596	50.1887214557197\\
146.97047010636	50.2385030009417\\
147.140269951125	50.2861911833125\\
147.328429313023	50.336640416335\\
147.516588674921	50.3846249543397\\
147.704748036819	50.4301990056668\\
147.915226076424	50.4783880919594\\
148.125704116028	50.5237006536344\\
148.336182155632	50.5662083670802\\
148.574100526651	50.6109688965\\
148.812018897669	50.6523345940646\\
149.049937268687	50.6904025559777\\
149.321902289739	50.7300004825645\\
149.59386731079	50.7655516089104\\
149.865832331842	50.7971907482787\\
149.910554887895	50.8020281445573\\
149.955277443947	50.8067638900378\\
150	50.8113985574329\\
150.157267334689	50.8261657555666\\
150.314534669378	50.8382608105017\\
150.471802004067	50.8477358100926\\
150.629069338756	50.8546421529232\\
150.786336673445	50.8590305453278\\
150.943604008134	50.8609510251047\\
151.121575758892	50.8602098238707\\
151.29954750965	50.8564418699384\\
151.477519260408	50.8497166381683\\
151.674150126545	50.838930025669\\
151.870780992682	50.824707844694\\
152.067411858819	50.8071393984201\\
152.28708810562	50.7836622480208\\
152.506764352422	50.7562383159876\\
152.726440599223	50.7249856194801\\
152.972971498591	50.685496393158\\
153.21950239796	50.6414912341226\\
153.466033297329	50.5931271332466\\
153.742837114793	50.533819952097\\
154.019640932257	50.4694237031603\\
154.296444749721	50.4001458058938\\
154.530963166481	50.3377814905795\\
154.76548158324	50.2721787308074\\
155	50.203455326255\\
155.131662472678	50.1646487826411\\
155.263324945357	50.1270884124395\\
155.394987418035	50.0907531475293\\
155.526649890714	50.0556222247774\\
155.658312363392	50.0216751845344\\
155.789974836071	49.9888918627387\\
155.942073511953	49.952442701586\\
156.094172187836	49.9174898929965\\
156.246270863718	49.8840037899696\\
156.412139705371	49.8491237231171\\
156.578008547024	49.8159161695819\\
156.743877388678	49.784344675662\\
156.927738459716	49.7512171183311\\
157.111599530755	49.7200080732667\\
157.295460601794	49.6906707301892\\
157.501135738782	49.6600142032766\\
157.70681087577	49.6315797258676\\
157.912486012758	49.6053059501974\\
158.145543942081	49.578069939121\\
158.378601871405	49.5534463000306\\
158.611659800729	49.5313521924447\\
158.880210269259	49.5089250408651\\
159.14876073779	49.4896295786319\\
159.417311206321	49.473349484492\\
159.611540804214	49.4633900261107\\
159.805770402107	49.4549072702359\\
160	49.4478605856084\\
160.176488344389	49.4430881930509\\
160.352976688778	49.4402615449249\\
160.529465033166	49.4393292210006\\
160.705953377555	49.4402409072236\\
160.882441721944	49.4429473843247\\
161.058930066333	49.4474004874052\\
161.270329924362	49.4549674700541\\
161.481729782391	49.4648938198793\\
161.69312964042	49.4771025176177\\
161.931255100232	49.4934944946289\\
162.169380560045	49.5125829919695\\
162.407506019857	49.5342667858512\\
162.684690394913	49.5626456758753\\
162.96187476997	49.5942580343841\\
163.239059145026	49.6289589082428\\
163.572339932659	49.6745743122908\\
163.905620720291	49.7242185794719\\
164.238901507923	49.7776671266168\\
164.492601005282	49.8207701042817\\
164.746300502641	49.8658610784026\\
165	49.9128514559354\\
165.137774802833	49.938407060704\\
165.275549605665	49.963046361224\\
165.413324408498	49.9867844177265\\
165.55109921133	50.0096360900415\\
165.688874014163	50.0316160380107\\
165.826648816995	50.0527387269739\\
165.995045034918	50.0774126554931\\
166.163441252841	50.1008528957952\\
166.331837470764	50.1230847025443\\
166.51566114658	50.1460029027482\\
166.699484822395	50.1675424616788\\
166.883308498211	50.1877345742257\\
167.089058731003	50.2087748557734\\
167.294808963795	50.2282075810201\\
167.500559196587	50.2460740516829\\
167.732852813979	50.2644137647936\\
167.96514643137	50.2808655949677\\
168.197440048761	50.2954852476835\\
168.46315790236	50.3100318093565\\
168.72887575596	50.3223316715603\\
168.994593609559	50.3324622521204\\
169.329729073039	50.3422627166709\\
169.66486453652	50.3488789385032\\
170	50.3524528499262\\
170.214262476156	50.3529656083878\\
170.428524952311	50.3518524522638\\
170.642787428467	50.3491611577944\\
170.857049904622	50.3449384993287\\
171.071312380778	50.3392302480579\\
171.285574856933	50.3320812153906\\
171.542617386535	50.3216654529273\\
171.799659916137	50.3093131477356\\
172.056702445739	50.2950963597275\\
172.350437079601	50.2766574409048\\
172.644171713464	50.2559772238217\\
172.937906347326	50.2331545732173\\
173.281883881167	50.2038334394435\\
173.625861415009	50.1718547196162\\
173.96983894885	50.1373622530192\\
174.3132259659	50.1005600924384\\
174.65661298295	50.061524195505\\
175	50.0203829529216\\
175.152127548178	50.0021232338671\\
175.304255096357	49.9846818724355\\
175.456382644535	49.9680427931804\\
175.608510192714	49.9521901824357\\
175.760637740892	49.9371084871814\\
175.912765289071	49.9227824072708\\
176.095693689949	49.90653509199\\
176.278622090826	49.8913329032493\\
176.461550491704	49.8771506541585\\
176.66337403319	49.862657274657\\
176.865197574676	49.8493429166677\\
177.067021116162	49.8371758235487\\
177.295319979278	49.8247564772996\\
177.523618842394	49.8137215120726\\
177.75191770551	49.8040281121368\\
178.01332274583	49.7945228309229\\
178.27472778615	49.7866607304968\\
178.53613282647	49.7803825596885\\
178.841181560107	49.7749821431729\\
179.146230293744	49.7715718158724\\
179.451279027381	49.7700659405483\\
179.634186018254	49.7700418077117\\
179.817093009127	49.7706552093522\\
180	49.771889148619\\
180.302079205992	49.7753138988305\\
180.604158411983	49.7804534168566\\
180.906237617975	49.7872309048854\\
181.208316823966	49.7955721650208\\
181.510396029958	49.8054055772509\\
181.812475235949	49.8166619391018\\
182.20558228822	49.8333311428537\\
182.59868934049	49.8521557464379\\
182.99179639276	49.8730006221503\\
183.483893632403	49.901745301147\\
183.975990872046	49.9332139118685\\
184.468088111689	49.9671797312912\\
184.645392074459	49.9799880478945\\
184.82269603723	49.9930830808641\\
185	50.0064554322043\\
185.158548259263	50.0182356321033\\
185.317096518525	50.0294260412503\\
185.475644777788	50.0400383011489\\
185.634193037051	50.0500838700141\\
185.792741296313	50.0595740231542\\
185.951289555576	50.0685198586806\\
186.153423496462	50.0791528813849\\
186.355557437347	50.0889411411922\\
186.557691378233	50.0979064142018\\
186.780852735715	50.1068740869926\\
187.004014093196	50.1148927293582\\
187.227175450678	50.1219898081848\\
187.482562484599	50.1290154591548\\
187.737949518519	50.1349085787983\\
187.99333655244	50.1397074239865\\
188.288957081652	50.1439447722484\\
188.584577610865	50.1468219730514\\
188.880198140078	50.1483935209203\\
189.253465426719	50.1485947107601\\
189.626732713359	50.1469011462171\\
190	50.1434116543018\\
190.310401934771	50.1392921831158\\
190.620803869542	50.1342077227212\\
190.931205804313	50.1282023924773\\
191.241607739084	50.1213188781612\\
191.552009673855	50.1135984340799\\
191.862411608626	50.1050809740628\\
192.257919872426	50.0931343280376\\
192.653428136225	50.0800330337406\\
193.048936400025	50.0658506544116\\
193.532682212967	50.0471370878606\\
194.01642802591	50.0270340335274\\
194.500173838852	50.0056577306562\\
194.666782559235	49.9980207596778\\
194.833391279617	49.9902501898262\\
195	49.9823502695222\\
195.170274874815	49.9744491894099\\
195.340549749631	49.9670158093304\\
195.510824624446	49.9600397800677\\
195.681099499262	49.953510938005\\
195.851374374077	49.9474193043949\\
196.021649248892	49.94175507917\\
196.243350696976	49.9350048178514\\
196.465052145059	49.9289420315128\\
196.686753593142	49.9235464108205\\
196.933184436476	49.9183077633547\\
197.179615279811	49.9138423809245\\
197.426046123145	49.9101244091931\\
197.712034109695	49.9067134150132\\
197.998022096244	49.90423695249\\
198.284010082794	49.9026579763242\\
198.620842061915	49.9019005762197\\
198.957674041037	49.9022817258152\\
199.294506020159	49.9037468060161\\
199.529670680106	49.9053840933412\\
199.764835340053	49.9075066990342\\
200	49.9100977281918\\
};
\end{axis}
\end{tikzpicture}%}
  \caption{Step response using a zero-order hold of sample time $5$ sec.}
  \label{fig:Q7.5}
\end{figure}

\begin{figure}[H]\centering
	\centering
	\scalebox{1}{% This file was created by matlab2tikz.
%
%The latest updates can be retrieved from
%  http://www.mathworks.com/matlabcentral/fileexchange/22022-matlab2tikz-matlab2tikz
%where you can also make suggestions and rate matlab2tikz.
%
\definecolor{mycolor1}{rgb}{0.00000,0.44700,0.74100}%
%
\begin{tikzpicture}

\begin{axis}[%
width=4.133in,
height=3.26in,
at={(0.693in,0.44in)},
scale only axis,
xmin=0,
xmax=200,
xmajorgrids,
ymin=30,
ymax=60,
ymajorgrids,
axis background/.style={fill=white}
]
\addplot [color=mycolor1,solid,forget plot]
  table[row sep=crcr]{%
0	40\\
0.0666666666666667	39.9988529199354\\
0.133333333333333	39.9954198587329\\
0.2	39.9897130376372\\
0.266666666666667	39.9817446149091\\
0.333333333333333	39.9715266859614\\
0.4	39.959071286313\\
0.47499490925035	39.942400621861\\
0.5499898185007	39.9229306571585\\
0.62498472775105	39.9006782627291\\
0.703019200761635	39.8745881554063\\
0.78105367377222	39.8455224785902\\
0.859088146782806	39.8134999414055\\
0.940721928520819	39.7768560227102\\
1.02235571025883	39.7370177534163\\
1.10398949199685	39.6940062251002\\
1.18937445628243	39.6456457248368\\
1.27475942056802	39.5938604847577\\
1.36014438485361	39.5386742795039\\
1.44941284679236	39.4773672565188\\
1.53868130873111	39.4123956293716\\
1.62794977066986	39.3437861740105\\
1.72115687139107	39.2682963135632\\
1.81436397211228	39.1888997791019\\
1.90757107283349	39.1056266274356\\
2.00466164289194	39.0147939058355\\
2.10175221295039	38.9198209377759\\
2.19884278300884	38.820741250687\\
2.29961827190097	38.7135934398924\\
2.4003937607931	38.6020944165113\\
2.50116924968523	38.4862812151995\\
2.60526965685247	38.3621560855701\\
2.70937006401971	38.2335072833433\\
2.81347047118694	38.1003751782987\\
2.92038536533207	37.9590191789915\\
3.02730025947719	37.8130200499409\\
3.13421515362232	37.6624210955585\\
3.24774120796246	37.4975227201725\\
3.36126726230261	37.3275382844275\\
3.47479331664276	37.1525191924883\\
3.59549874757794	36.9609670376133\\
3.71620417851312	36.7638427524235\\
3.8369096094483	36.5612076866583\\
3.96415943740082	36.3416866294139\\
4.09140926535334	36.1161806774553\\
4.21865909330586	35.8847613318055\\
4.35177080227781	35.6364307829988\\
4.48488251124977	35.3817892660967\\
4.61799422022172	35.1209183765544\\
4.75625532436033	34.8434477559509\\
4.89451642849894	34.5594358361441\\
5.03277753263755	34.2689740153455\\
5.17549817685085	33.9624751879954\\
5.31821882106414	33.649301764491\\
5.46093946527744	33.3295545382605\\
5.60747261901361	32.9945323623854\\
5.75400577274979	32.6527963614143\\
5.90053892648596	32.3044562823032\\
5.93369261765731	32.2247381961632\\
5.96684630882865	32.1446889525368\\
6	32.0643098293956\\
6.07019285234842	31.899665124437\\
6.14038570469683	31.7463623462781\\
6.21057855704525	31.6038213737188\\
6.28077140939366	31.4715280627677\\
6.35096426174208	31.3490200915252\\
6.4211571140905	31.2358777134702\\
6.4980152583905	31.1222823912678\\
6.5748734026905	31.0189997159486\\
6.6517315469905	30.9256086410873\\
6.73542725974784	30.8347072093451\\
6.81912297250518	30.7546155160441\\
6.90281868526252	30.6849063576352\\
6.99445884128422	30.6200201677796\\
7.08609899730592	30.5666238235043\\
7.17773915332761	30.5242675771202\\
7.27886172611253	30.4898388107422\\
7.37998429889745	30.4678173718824\\
7.48110687168236	30.4577118200149\\
7.59415466874327	30.4599564956984\\
7.70720246580417	30.4759182391891\\
7.82025026286508	30.505034172882\\
7.94975065175232	30.5538745452904\\
8.07925104063957	30.6185431154867\\
8.20875142952682	30.698345140758\\
8.36661479449466	30.8151524988717\\
8.5244781594625	30.9523840469232\\
8.68234152443034	31.1090160203407\\
8.84660313328062	31.2915671471859\\
9.0108647421309	31.4930892843559\\
9.17512635098119	31.7126452413801\\
9.33938795983147	31.9493585507123\\
9.50364956868175	32.2024072471281\\
9.66791117753204	32.4710184658202\\
9.82983527957722	32.7503311526219\\
9.9917593816224	33.0434042625242\\
10.1536834836676	33.3496178266014\\
10.3156947820054	33.6685610443607\\
10.4777060803433	33.9995173390028\\
10.6397173786811	34.341960463297\\
10.804822042684	34.7022436889036\\
10.9699267066868	35.0734380375626\\
11.1350313706896	35.4550681328161\\
11.3046986419779	35.857642439747\\
11.4743659132661	36.2702957165603\\
11.6440331845544	36.6925859415574\\
11.7626887897029	36.9934126335565\\
11.8813443948515	37.2986090899629\\
12	37.6080423091366\\
12.0925652804967	37.8482068998788\\
12.1851305609935	38.082709173226\\
12.2776958414902	38.3116004881\\
12.370261121987	38.5349313966488\\
12.4628264024837	38.7527516714793\\
12.5553916829805	38.9651103410625\\
12.6564289218538	39.1907274725003\\
12.7574661607271	39.409956906577\\
12.8585033996005	39.6228595534904\\
12.965869943006	39.8422325203017\\
13.0732364864115	40.0545998061723\\
13.180603029817	40.2600311980348\\
13.2953341638031	40.4719655814028\\
13.4100652977892	40.6761409803687\\
13.5247964317752	40.8726387149037\\
13.6477321407303	41.0747583925657\\
13.7706678496854	41.2682520325019\\
13.8936035586405	41.4532151873454\\
14.0256673351603	41.6425153315472\\
14.1577311116802	41.8221946706642\\
14.2897948882001	41.9923663757719\\
14.4318814974649	42.1649633205611\\
14.5739681067296	42.3268205097641\\
14.7160547159944	42.4780726607434\\
14.8688603253054	42.6290518129816\\
15.0216659346164	42.7680816444906\\
15.1744715439274	42.8953224429866\\
15.3381969624404	43.0187527732787\\
15.5019223809534	43.129021815622\\
15.6656477994664	43.2263184092694\\
15.8396276198157	43.3157005647257\\
16.0136074401651	43.390865051601\\
16.1875872605144	43.4520293070418\\
16.370084937732	43.5013773105442\\
16.5525826149495	43.5358012553999\\
16.735080292167	43.5555426266202\\
16.9235840429855	43.5607716196195\\
17.112087793804	43.5508501045056\\
17.3005915446224	43.5260350688794\\
17.4925622122974	43.485719667835\\
17.6845328799724	43.4304873090214\\
17.8765035476474	43.3606012219298\\
17.9176690317649	43.3437321670454\\
17.9588345158825	43.3262038498944\\
18	43.3080188236021\\
18.2037678819762	43.2097118996809\\
18.4075357639523	43.0982561938614\\
18.6113036459285	42.9739383598353\\
18.7875545590609	42.8562536635071\\
18.9638054721933	42.7293417532183\\
19.1400563853257	42.5933843498775\\
19.3717503281998	42.4012014816611\\
19.6034442710739	42.1941071338207\\
19.8351382139481	41.9725064965664\\
20.0546043802455	41.7495878238988\\
20.274070546543	41.5143551640778\\
20.4935367128404	41.267147065316\\
20.7130028791379	41.0083006017246\\
20.9324690454354	40.7381510395175\\
21.1519352117328	40.4570324495607\\
21.3652788287268	40.1735582566311\\
21.5786224457207	39.8803376414707\\
21.7919660627147	39.5776743000378\\
22.0015535491963	39.2714381034868\\
22.2111410356779	38.9566672954612\\
22.4207285221596	38.6336474978812\\
22.6292033028293	38.304441551594\\
22.8376780834991	37.967635769841\\
23.0461528641689	37.6235094675312\\
23.2545683862596	37.2724423868637\\
23.4629839083503	36.914615652707\\
23.671399430441	36.5503068857424\\
23.7809329536273	36.3563382034512\\
23.8904664768137	36.1606959065506\\
24	35.9634201506837\\
24.0818137053378	35.8204213418212\\
24.1636274106755	35.6870829812403\\
24.2454411160133	35.5630578900034\\
24.327254821351	35.4480225652027\\
24.4090685266888	35.3416740380536\\
24.4908822320265	35.243727413901\\
24.5816158472107	35.1446033388896\\
24.6723494623949	35.0551391412121\\
24.7630830775791	34.9750101653096\\
24.860404856259	34.8990913535133\\
24.9577266349389	34.8332037350864\\
25.0550484136188	34.7770098646248\\
25.1605705574302	34.7266639213746\\
25.2660927012417	34.6869536840856\\
25.3716148450532	34.6575140144242\\
25.487189762717	34.6366452443316\\
25.6027646803809	34.6272510207153\\
25.7183395980448	34.6289224012083\\
25.8469900788313	34.6433242988569\\
25.9756405596178	34.6704495482398\\
26.1042910404044	34.7098171427806\\
26.2517731479522	34.7694213671974\\
26.3992552554999	34.8438650734474\\
26.5467373630477	34.9325330756272\\
26.7294574715463	35.0612462479815\\
26.9121775800448	35.2098597124006\\
27.0948976885433	35.3773952832765\\
27.2777497306299	35.5630750869763\\
27.4606017727164	35.7659192211806\\
27.6434538148029	35.9851113588722\\
27.8263058568894	36.2198817570808\\
28.0091578989759	36.4695033267822\\
28.1920099410624	36.7332879352777\\
28.3720714314094	37.0062541278557\\
28.5521329217563	37.2917334684425\\
28.7321944121033	37.5891661304521\\
28.9133573755164	37.8999439159691\\
29.0945203389296	38.2217689790269\\
29.2756833023427	38.5541515704504\\
29.4607371574146	38.904088526909\\
29.6457910124865	39.264082582064\\
29.8308448675584	39.6336834831602\\
29.8872299117056	39.7481426685611\\
29.9436149558528	39.8634420507244\\
30	39.9795700479073\\
30.0942436032545	40.1716850980348\\
30.188487206509	40.3584900565903\\
30.2827308097635	40.5400316205717\\
30.3769744130181	40.7163558733\\
30.4712180162726	40.8875083019209\\
30.5654616195271	41.0535338239489\\
30.6689013836684	41.2299108682588\\
30.7723411478097	41.4002229699206\\
30.875780911951	41.5645272837955\\
30.9857554201144	41.7326855127089\\
31.0957299282779	41.8941835961715\\
31.2057044364413	42.0490875988874\\
31.3233288160139	42.2075425683162\\
31.4409531955865	42.3586075498265\\
31.558577575159	42.5023603024093\\
31.6847003751541	42.6484628112829\\
31.8108231751492	42.7863399757317\\
31.9369459751443	42.9160840283778\\
32.0724815248282	43.0465488737883\\
32.2080170745122	43.167838066134\\
32.3435526241962	43.2800617733223\\
32.489337353499	43.3907766105189\\
32.6351220828019	43.4912626047699\\
32.7809068121048	43.5816517850847\\
32.9374972785344	43.6676424340718\\
33.094087744964	43.7422941051467\\
33.2506782113936	43.8057645097736\\
33.4180322653792	43.8614133751995\\
33.5853863193648	43.9046569899755\\
33.7527403733504	43.9356811863124\\
33.9298865207475	43.9554119371268\\
34.1070326681445	43.9618728317474\\
34.2841788155416	43.9552770493345\\
34.469154323391	43.9346831983714\\
34.6541298312405	43.9003207488088\\
34.8391053390899	43.8524251202743\\
35.0294062079763	43.7892731399814\\
35.2197070768627	43.7122959898208\\
35.4100079457491	43.6217431777901\\
35.6066719638327	43.5141613491873\\
35.8033359819164	43.3926161123002\\
36	43.257376525167\\
36.1331972237772	43.159726988291\\
36.2663944475545	43.0591616810474\\
36.3995916713317	42.9557377145599\\
36.5327888951089	42.8495118128838\\
36.6659861188861	42.7405402969257\\
36.7991833426634	42.6288791072094\\
36.9483429165909	42.500712531175\\
37.0975024905185	42.3693200984149\\
37.2466620644461	42.2347786603676\\
37.4109317117542	42.0830555375825\\
37.5752013590623	41.9277065998218\\
37.7394710063705	41.7688320103623\\
37.9237288387271	41.5865533990421\\
38.1079866710838	41.400102701714\\
38.2922445034404	41.2096174164385\\
38.5041570010797	40.9857312061531\\
38.7160694987189	40.7568933691848\\
38.9279819963582	40.5233064457619\\
39.1868671712059	40.2317893563415\\
39.4457523460537	39.9338432513987\\
39.7046375209014	39.6298230590599\\
39.9713100236367	39.3106763811911\\
40.237982526372	38.9858319086072\\
40.5046550291072	38.655659881529\\
40.7713275318425	38.320524727508\\
41.0380000345778	37.980784303306\\
41.3046725373131	37.6367905556612\\
41.5364483582087	37.3346257937811\\
41.7682241791044	37.0297325246605\\
42	36.72233032058\\
42.0876689468228	36.6095987599817\\
42.1753378936455	36.5046884202547\\
42.2630068404683	36.4073693124944\\
42.3506757872911	36.3174225134051\\
42.4383447341138	36.2346391420457\\
42.5260136809366	36.1588194766308\\
42.6246284013628	36.0816180971326\\
42.723243121789	36.0127264414042\\
42.8218578422153	35.9518962203235\\
42.9275676365777	35.8953721042921\\
43.0332774309402	35.8475592488543\\
43.1389872253027	35.8081904764397\\
43.2537615056109	35.7747097145195\\
43.3685357859192	35.7505692897014\\
43.4833100662274	35.7354699701803\\
43.6092332063486	35.7289654088937\\
43.7351563464699	35.7326379308685\\
43.8610794865911	35.7461420525683\\
44.0015812892818	35.7724100722022\\
44.1420830919724	35.8100690298352\\
44.2825848946631	35.8587008784684\\
44.4442374500164	35.927711142633\\
44.6058900053698	36.0101404806335\\
44.7675425607232	36.1054386492818\\
44.9696463069985	36.2418908425059\\
45.1717500532738	36.3966742937569\\
45.3738537995491	36.5688765172689\\
45.5733379005901	36.7550881991881\\
45.7728220016311	36.9566505864888\\
45.972306102672	37.172824004708\\
46.171790203713	37.4029083515252\\
46.3712743047539	37.6462401940317\\
46.5707584057949	37.9021898176364\\
46.7680398696279	38.1671373057675\\
46.9653213334609	38.4432919138119\\
47.162602797294	38.7301310867912\\
47.3617267310523	39.0299782004491\\
47.5608506648107	39.3397173472953\\
47.759974598569	39.6588848554082\\
47.8399830657127	39.7896820834598\\
47.9199915328564	39.9219022416954\\
48	40.0555181535137\\
48.0973187588596	40.2164819988884\\
48.1946375177191	40.3726791672627\\
48.2919562765787	40.5241551671366\\
48.3892750354382	40.6709549755557\\
48.4865937942978	40.8131230511591\\
48.5839125531573	40.9507033569549\\
48.6912246927041	41.0971448844048\\
48.7985368322509	41.2381185374771\\
48.9058489717976	41.3736812742593\\
49.0201654691133	41.5122033788996\\
49.134481966429	41.6447163223373\\
49.2487984637447	41.7712865953943\\
49.3713878806023	41.9004880343967\\
49.4939772974599	42.0230112398381\\
49.6165667143175	42.1389353892644\\
49.7483973920285	42.256323866192\\
49.8802280697396	42.3662675600271\\
50.0120587474506	42.4688615491212\\
50.1541969250218	42.5713509684792\\
50.296335102593	42.6655221106496\\
50.4384732801643	42.751490049067\\
50.5919255072433	42.8352233727039\\
50.7453777343223	42.9096696307613\\
50.8988299614014	42.9749686719758\\
51.0643195977311	43.0352973463065\\
51.2298092340608	43.0853196406126\\
51.3952988703905	43.1252050415039\\
51.5728981175953	43.1569233376555\\
51.7504973648002	43.177364762939\\
51.9280966120051	43.1867318860178\\
52.1168317807189	43.1847729100619\\
52.3055669494328	43.1707714798105\\
52.4943021181467	43.1449629831544\\
52.6920716231783	43.1055046977674\\
52.8898411282099	43.0536030305355\\
53.0876106332414	42.9895206883335\\
53.2917162510513	42.9108887731912\\
53.4958218688611	42.8198436088575\\
53.699927486671	42.7166659536872\\
53.7999516577806	42.6617573768636\\
53.8999758288903	42.6040344844837\\
54	42.543529690679\\
54.1325064704736	42.4606434577058\\
54.2650129409472	42.3759755158035\\
54.3975194114208	42.2895663862164\\
54.5300258818944	42.2014562014444\\
54.662532352368	42.1116846951745\\
54.7950388228416	42.0202912169058\\
54.943834436811	41.9157776753615\\
55.0926300507804	41.8093225805586\\
55.2414256647497	41.7009799156241\\
55.4044329638116	41.5801858887289\\
55.5674402628735	41.4572598967422\\
55.7304475619354	41.3322704224859\\
55.9116209482173	41.1910127372278\\
56.0927943344991	41.0473811304386\\
56.273967720781	40.9014659029092\\
56.4781357163185	40.7344054314667\\
56.682303711856	40.5646832838415\\
56.8864717073935	40.3924228094141\\
57.1218178398582	40.1908654768878\\
57.3571639723228	39.9862788492434\\
57.5925101047874	39.7788417251149\\
57.8763947614618	39.5250781882878\\
58.1602794181361	39.2677224571778\\
58.4441640748104	39.0070676679711\\
58.8597793232206	38.6201048864103\\
59.2753945716308	38.227549272437\\
59.691009820041	37.8302349658354\\
59.794006546694	37.7311314144089\\
59.897003273347	37.6317968386449\\
60	37.5322429834959\\
60.0929356785526	37.445390493287\\
60.1858713571052	37.3645789003899\\
60.2788070356578	37.2896574054469\\
60.3717427142104	37.2204804578151\\
60.464678392763	37.156907425817\\
60.5576140713156	37.0988022828436\\
60.6645599730185	37.038533319227\\
60.7715058747214	36.9851372842514\\
60.8784517764243	36.9384273150812\\
60.9930343973086	36.8955964737385\\
61.1076170181928	36.8600199044208\\
61.222199639077	36.8314913311334\\
61.3468453488521	36.8082285378406\\
61.4714910586272	36.7928238650237\\
61.5961367684023	36.7850394450934\\
61.7330995222826	36.7850000052364\\
61.8700622761628	36.7935919554144\\
62.007025030043	36.8105336986076\\
62.1600110763649	36.8389972316073\\
62.3129971226867	36.877178102622\\
62.4659831690086	36.9247282517611\\
62.6419054005757	36.9905559917706\\
62.8178276321428	37.0678454469809\\
62.9937498637098	37.1561317831556\\
63.2116925732656	37.2800932033973\\
63.4296352828214	37.4194480210498\\
63.6475779923772	37.5734342607956\\
63.8695533475392	37.7445630182483\\
64.0915287027012	37.9293971803194\\
64.3135040578632	38.1272500157763\\
64.5354794130252	38.3374698876434\\
64.7574547681871	38.5594380297916\\
64.9794301233491	38.7925660263669\\
65.2000649367474	39.0347918913399\\
65.4206997501456	39.2869681304471\\
65.6413345635439	39.5485970182034\\
65.7608897090292	39.6941457152905\\
65.8804448545146	39.8422577473803\\
66	39.9928623368825\\
66.1018702080585	40.1201992857639\\
66.203740416117	40.2435014388457\\
66.3056106241755	40.3628122338187\\
66.407480832234	40.4781746440073\\
66.5093510402925	40.5896311867531\\
66.6112212483509	40.6972239420711\\
66.7242823860316	40.8121630282262\\
66.8373435237123	40.9224504130517\\
66.950404661393	41.0281418146804\\
67.071214402117	41.136059777284\\
67.1920241428411	41.2388596218265\\
67.3128338835651	41.3366071107087\\
67.4429197454932	41.4362853547263\\
67.5730056074212	41.5302612747054\\
67.7030914693493	41.6186142481797\\
67.8436412315826	41.7078451651795\\
67.9841909938159	41.7907016077218\\
68.1247407560493	41.8672802907625\\
68.2771138320987	41.9433198156392\\
68.429486908148	42.0122136864566\\
68.5818599841975	42.0740808498647\\
68.7474188965413	42.1334737286502\\
68.9129778088851	42.1848588414748\\
69.0785367212289	42.2283833208267\\
69.2583996585596	42.2669290067899\\
69.4382625958903	42.2965526724236\\
69.618125533221	42.3174360731044\\
69.8127097748468	42.3303937652464\\
70.0072940164725	42.3335563435502\\
70.2018782580983	42.3271453261961\\
70.410345598134	42.3099027906143\\
70.6188129381697	42.2821882657035\\
70.8272802782054	42.2442637612378\\
71.0473346417962	42.1934401602261\\
71.2673890053871	42.1318284221334\\
71.487443368978	42.0597248508794\\
71.6582955793187	41.9966945357817\\
71.8291477896593	41.9276513190475\\
72	41.8527291899308\\
72.1385610856749	41.7889781040168\\
72.2771221713499	41.7239662546407\\
72.4156832570248	41.6577261845302\\
72.5542443426997	41.5902900303775\\
72.6928054283746	41.5216895163453\\
72.8313665140496	41.4519559676666\\
72.9880477094088	41.3717770117498\\
73.1447289047681	41.2902329348949\\
73.3014101001274	41.2073671970329\\
73.4734635377281	41.1148997734925\\
73.6455169753288	41.020946071657\\
73.8175704129295	40.9255610046313\\
74.0093285594543	40.8176312075342\\
74.2010867059791	40.708063936709\\
74.392844852504	40.5969312877899\\
74.6092441436839	40.4697264986805\\
74.8256434348637	40.340717838206\\
75.0420427260436	40.2100027755862\\
75.2908623911417	40.0577189593917\\
75.5396820562398	39.9034466084917\\
75.7885017213378	39.7473236047789\\
76.0835864663935	39.5599572940274\\
76.3786712114491	39.3703943995655\\
76.6737559565048	39.1788456356073\\
77.0504466687522	38.9317614528256\\
77.4271373809997	38.6821813011804\\
77.8038280932472	38.4304964265004\\
77.8692187288315	38.3866193708865\\
77.9346093644157	38.3426921195059\\
78	38.2987165638147\\
78.1026294992103	38.2321136846702\\
78.2052589984207	38.1703413967664\\
78.307888497631	38.1132890559053\\
78.4105179968413	38.0608490165612\\
78.5131474960517	38.012916505275\\
78.615776995262	37.969389482803\\
78.7328105564889	37.9250029435699\\
78.8498441177158	37.8860728266834\\
78.9668776789428	37.85245991628\\
79.0928210037588	37.8220730531536\\
79.2187643285749	37.7975236770688\\
79.3447076533909	37.7786530586763\\
79.4822299130235	37.764352244951\\
79.6197521726561	37.7564444675787\\
79.7572744322887	37.7547414569794\\
79.9089919181363	37.7598425047901\\
80.0607094039839	37.7720366587545\\
80.2124268898314	37.7910949707856\\
80.382584886203	37.8203624100875\\
80.5527428825745	37.8576847318085\\
80.7229008789461	37.9027717228883\\
80.9193154948111	37.9641052471444\\
81.1157301106761	38.0349990925837\\
81.3121447265411	38.1150567580764\\
81.5558992816423	38.2265830261639\\
81.7996538367435	38.3509442847323\\
82.0434083918447	38.4874805547407\\
82.2952374024005	38.6406602594131\\
82.5470664129563	38.8055078535354\\
82.7988954235121	38.9813981241515\\
83.050724434068	39.1677368692576\\
83.3025534446238	39.3639592573409\\
83.5543824551796	39.5695276190483\\
83.7029216367864	39.694954387016\\
83.8514608183932	39.8233534213299\\
84	39.9546266822087\\
84.1079213731045	40.0493914736913\\
84.2158427462091	40.1409172646191\\
84.3237641193136	40.2292439531797\\
84.4316854924181	40.314411015767\\
84.5396068655227	40.396457511419\\
84.6475282386272	40.4754220968482\\
84.768482389272	40.5603072239265\\
84.8894365399169	40.6414224662917\\
85.0103906905617	40.7188205029569\\
85.1401544378009	40.7977819027089\\
85.2699181850401	40.8725884231651\\
85.3996819322793	40.9433029007529\\
85.5401993608317	41.0153347697662\\
85.6807167893841	41.0827189890789\\
85.8212342179364	41.1455325107853\\
85.9740381071027	41.2087393977289\\
86.1268419962689	41.2667277111379\\
86.2796458854352	41.3195927001083\\
86.4465733258858	41.3715991744354\\
86.6135007663365	41.4177245614657\\
86.7804282067872	41.4580881160289\\
86.9634417747853	41.4958593642129\\
87.1464553427834	41.5269989017709\\
87.3294689107815	41.5516572567553\\
87.5303830924801	41.5714406925405\\
87.7312972741788	41.5837850064455\\
87.9322114558774	41.5888803660021\\
88.1520871675179	41.5863703749\\
88.3719628791583	41.5756439741672\\
88.5918385907987	41.5569383443825\\
88.8301139483729	41.5279291879832\\
89.068389305947	41.4901146976622\\
89.3066646635212	41.443781061587\\
89.5377764423475	41.3909674488672\\
89.7688882211737	41.3306553784688\\
90	41.2630922228855\\
90.1478741407485	41.2171303199818\\
90.2957482814969	41.1702567337953\\
90.4436224222454	41.1224978451884\\
90.5914965629938	41.0738796310558\\
90.7393707037423	41.0244276605427\\
90.8872448444907	40.9741671088072\\
91.0562695931994	40.9157591868583\\
91.2252943419081	40.8563635865906\\
91.3943190906167	40.7960160390389\\
91.5808566041663	40.72835295382\\
91.7673941177158	40.6596194022106\\
91.9539316312654	40.5898607003342\\
92.1632574096297	40.5104142906246\\
92.3725831879941	40.4297941914091\\
92.5819089663584	40.3480603449565\\
92.8198312252879	40.2538825580179\\
93.0577534842173	40.1584253056773\\
93.2956757431467	40.0617701249141\\
93.5712022378252	39.9484456954079\\
93.8467287325037	39.8337405265116\\
94.1222552271822	39.7177703013652\\
94.4505146433795	39.5781084159173\\
94.7787740595768	39.4369919562631\\
95.1070334757742	39.2945961292775\\
95.4046889838494	39.1645110314595\\
95.7023444919247	39.0336327779699\\
96	38.9020772967463\\
96.1047202583965	38.8572896738846\\
96.2094405167929	38.8156620899579\\
96.3141607751894	38.7771303794465\\
96.4188810335858	38.7416317133643\\
96.5236012919823	38.7091045655108\\
96.6283215503787	38.6794886663609\\
96.7569478042314	38.6470090792623\\
96.885574058084	38.6187257341405\\
97.0142003119367	38.5945343428907\\
97.1521806874929	38.5730170487896\\
97.2901610630492	38.5559691344977\\
97.4281414386054	38.5432714058147\\
97.5796102194387	38.5342033113599\\
97.731079000272	38.5300882402549\\
97.8825477811054	38.5307809614423\\
98.0503628265851	38.5369918736475\\
98.2181778720649	38.5487447955544\\
98.3859929175447	38.5658590978909\\
98.5749396328578	38.5913268840967\\
98.7638863481709	38.6231276601319\\
98.952833063484	38.6610277658311\\
99.1713530695347	38.7121700794232\\
99.3898730755855	38.7708323750174\\
99.6083930816362	38.8366911328448\\
99.8774169462933	38.9271954731462\\
99.9999999999991	38.9717440786966\\
100	38.9717440786969\\
100.000000000001	38.9717440786973\\
100.055167754864	38.9924486423792\\
100.110335509726	39.0135538180255\\
100.165503264589	39.0350550980157\\
100.220671019451	39.056948018748\\
100.275838774314	39.079228160246\\
100.331006529177	39.1018911454879\\
100.391578415145	39.127209935537\\
100.452150301113	39.1529793457146\\
100.512722187081	39.1791937580749\\
100.5755700677	39.2068576292025\\
100.638417948319	39.2349884487465\\
100.701265828938	39.2635801433921\\
100.766776208929	39.293867194614\\
100.83228658892	39.3246417460875\\
100.897796968911	39.3558971490194\\
100.966183436588	39.3890306185045\\
101.034569904265	39.422673536999\\
101.102956371942	39.4568186034719\\
101.174473071899	39.4930558989668\\
101.245989771855	39.5298263256314\\
101.317506471812	39.5671218350696\\
101.392440863241	39.6067543094915\\
101.46737525467	39.6469455005696\\
101.5423096461	39.6876864991346\\
101.62098947295	39.7310459455517\\
101.6996692998	39.7749917989744\\
101.778349126651	39.819514152967\\
101.8522327511	39.8618386327097\\
101.92611637555	39.9046548683848\\
102	39.9479549775239\\
102.088295135527	40.0028171952424\\
102.176590271053	40.0632489631292\\
102.26488540658	40.1291175080114\\
102.374635836895	40.2183773906831\\
102.48438626721	40.3156006474586\\
102.594136697525	40.4205578863362\\
102.709102156482	40.5385558948214\\
102.824067615439	40.6645537532832\\
102.939033074396	40.7983161864686\\
103.061218172611	40.9487365855056\\
103.183403270827	41.1074098206128\\
103.305588369042	41.2740839496754\\
103.435428540137	41.459697823433\\
103.565268711231	41.6537929150844\\
103.695108882326	41.8560999796726\\
103.83316406463	42.0799249392715\\
103.971219246934	42.3124419672454\\
104.109274429238	42.5533634141909\\
104.256033178614	42.8183801065935\\
104.40279192799	43.0922630252216\\
104.549550677367	43.3747058043453\\
104.705408866228	43.6837046512068\\
104.861267055089	44.0016879135924\\
105.017125243951	44.3283311298844\\
105.182356147105	44.6837276871478\\
105.347587050259	45.0481518086178\\
105.512817953412	45.4212626429529\\
105.687566351168	45.8249547215096\\
105.862314748924	46.2376276620691\\
106.03706314668	46.6589269566218\\
106.221371679473	47.1122475493024\\
106.405680212267	47.5744055807502\\
106.58998874506	48.0450363342089\\
106.783855506566	48.5488343490819\\
106.977722268072	49.0612324748253\\
107.171589029577	49.5818594404597\\
107.37504229661	50.1366923667642\\
107.578495563643	50.6998053883344\\
107.781948830675	51.2708239952923\\
107.854632553784	51.4766685028004\\
107.927316276892	51.6834603638614\\
107.945921471444	51.7365442482351\\
107.945921471445	51.7365442482392\\
107.96394764763	51.7879728142651\\
107.981973823815	51.8393359177223\\
108	51.8906336743139\\
108.018026176185	51.9417885057021\\
108.03605235237	51.9927229719421\\
108.054078528555	52.0434373954499\\
108.090130880924	52.144207399699\\
108.126183233294	52.244101080878\\
108.162235585664	52.3431209881125\\
108.234340290403	52.5385496116179\\
108.306444995143	52.7305134016969\\
108.378549699882	52.919032287174\\
108.482268662972	53.1842051157084\\
108.585987626062	53.4423487703674\\
108.689706589152	53.6935204668291\\
108.79771748143	53.9477358643916\\
108.905728373708	54.1945144141527\\
109.013739265986	54.4339181211409\\
109.129430317206	54.6822340966701\\
109.245121368427	54.9222334690832\\
109.360812419647	55.1539893633562\\
109.484823216181	55.3933396116863\\
109.608834012715	55.6233885993109\\
109.732844809249	55.8442227345685\\
109.866155537314	56.0714379407333\\
109.99946626538	56.2882072622093\\
110.132776993445	56.4946336666679\\
110.276323791638	56.7054620044762\\
110.419870589831	56.9045408867482\\
110.563417388023	57.0919936210556\\
110.717949959056	57.2809368387491\\
110.872482530088	57.4566968663503\\
111.027015101121	57.619421292437\\
111.192787003699	57.7796543898638\\
111.358558906277	57.9252315313525\\
111.524330808854	58.0563276047697\\
111.700701731397	58.1800996194562\\
111.877072653939	58.2878800039362\\
112.053443576481	58.3798711759248\\
112.23863519137	58.4596882480783\\
112.423826806259	58.5225460466178\\
112.609018421148	58.5686702099026\\
112.800397041067	58.5989906325802\\
112.991775660986	58.6119218818891\\
113.183154280906	58.6077044021171\\
113.378022489854	58.5860352912963\\
113.572890698802	58.5470813649259\\
113.76775890775	58.4910885785171\\
113.845172605167	58.4641699346343\\
113.922586302583	58.4346159092039\\
114	58.4024416203847\\
114.155229710647	58.3294456221449\\
114.310459421294	58.2448039365235\\
114.46568913194	58.1486369835383\\
114.620918842587	58.0410642368055\\
114.776148553234	57.9222042075348\\
114.931378263881	57.7921745570562\\
115.098048409788	57.6402602125924\\
115.264718555695	57.4757474012257\\
115.431388701602	57.2987786946849\\
115.604453743179	57.1019898243008\\
115.777518784757	56.8920810776129\\
115.950583826334	56.6692089855708\\
116.128102381912	56.4272967449488\\
116.305620937491	56.1720760553769\\
116.48313949307	55.9037129883239\\
116.662947713136	55.6186600055477\\
116.842755933203	55.3204638339998\\
117.02256415327	55.0092946258438\\
117.203038587341	54.6840979465445\\
117.383513021412	54.3461733503532\\
117.563987455483	53.995691002458\\
117.744070857499	53.6336201314341\\
117.924154259514	53.2593829036701\\
118.10423766153	52.8731470788066\\
118.2832978503	52.4773750612154\\
118.46235803907	52.0700703118617\\
118.64141822784	51.6513970245148\\
118.819080940256	51.2249173019684\\
118.996743652672	50.7875675098054\\
119.174406365088	50.3395078609922\\
119.350435061286	49.885164272632\\
119.526463757484	49.4206202584379\\
119.702492453682	48.9460320128951\\
119.801661635788	48.6742991501352\\
119.900830817894	48.3994560575263\\
120	48.121530778369\\
120.080990442911	47.8986251209761\\
120.161980885822	47.6861085809401\\
120.242971328733	47.4836138205808\\
120.323961771645	47.2908002639143\\
120.404952214556	47.1073502989653\\
120.485942657467	46.9329663608206\\
120.574848559761	46.7516463804351\\
120.663754462055	46.5805684156609\\
120.752660364349	46.4194064442028\\
120.847914600247	46.2573917963547\\
120.943168836145	46.1060520332039\\
121.038423072043	45.9650512101979\\
121.141410095661	45.8238694147143\\
121.24439711928	45.6940229497418\\
121.347384142898	45.5751524002374\\
121.459720245523	45.4576069451604\\
121.572056348148	45.3522966710884\\
121.684392450772	45.2588240540021\\
121.808594541635	45.1688031989395\\
121.932796632498	45.0923080360874\\
122.05699872336	45.0288791321496\\
122.197491003921	44.9723322191412\\
122.337983284482	44.9313469839363\\
122.478475565042	44.9053551844071\\
122.645566529845	44.8932181588517\\
122.812657494648	44.9006696466026\\
122.979748459451	44.9268989035078\\
123.171339860064	44.979096882967\\
123.362931260677	45.0538878143009\\
123.554522661291	45.1502450742451\\
123.746114061904	45.2672011170601\\
123.937705462517	45.403842029208\\
124.129296863131	45.5593026110775\\
124.311291749405	45.7236593239236\\
124.493286635679	45.903591591164\\
124.675281521953	46.0984645244818\\
124.856662951741	46.3069448992262\\
125.038044381529	46.5290941616389\\
125.219425811317	46.7643673928638\\
125.403727574535	47.0163356129454\\
125.588029337754	47.2807953515325\\
125.772331100972	47.5572481249408\\
125.848220733981	47.6744564772187\\
125.924110366991	47.7935850179411\\
126	47.9146020858\\
126.101530809005	48.0761983740156\\
126.20306161801	48.2346988861994\\
126.304592427015	48.3901359250301\\
126.406123236019	48.5425414367923\\
126.507654045024	48.6919470170474\\
126.609184854029	48.8383839213314\\
126.720972863026	48.9962179508878\\
126.832760872023	49.1505314653725\\
126.944548881021	49.3013648021553\\
127.064154718707	49.4589379906663\\
127.183760556393	49.6126211143492\\
127.30336639408	49.7624618576007\\
127.432309562542	49.919750396927\\
127.561252731004	50.0726862653963\\
127.690195899466	50.2213270397651\\
127.829857597146	50.3775441477781\\
127.969519294826	50.5288604216958\\
128.109180992506	50.6753462380066\\
128.261284026059	50.8294675463119\\
128.413387059612	50.9780303079211\\
128.565490093165	51.121121787473\\
128.732105721506	51.2716886846208\\
128.898721349847	51.4159051446168\\
129.065336978188	51.5538810002516\\
129.248911045569	51.6988040941635\\
129.43248511295	51.8364254108743\\
129.616059180331	51.9668852077312\\
129.819314831106	52.1031443753597\\
130.022570481882	52.2309775551353\\
130.225826132658	52.3505659068216\\
130.451355076202	52.4738254676919\\
130.676884019747	52.5873923608442\\
130.902412963291	52.6915011725747\\
131.151647871042	52.7958287341579\\
131.400882778794	52.8891913960258\\
131.650117686545	52.9718880032395\\
131.766745124363	53.0070048862199\\
131.883372562182	53.0398800381692\\
132	53.0705427694746\\
132.121099241253	53.0983512105932\\
132.242198482507	53.1204174442235\\
132.36329772376	53.1368021396995\\
132.484396965013	53.147565553393\\
132.605496206267	53.1527675306439\\
132.72659544752	53.1524675329888\\
132.863681982387	53.1455628933684\\
133.000768517254	53.1317685060917\\
133.13785505212	53.1111689355109\\
133.284901376858	53.0816036037176\\
133.431947701596	53.0444079551068\\
133.578994026334	52.9996842898567\\
133.737677552155	52.9430929959745\\
133.896361077976	52.8779802502407\\
134.055044603797	52.804472128708\\
134.225702783174	52.7161906582669\\
134.396360962552	52.6184986792561\\
134.567019141929	52.5115500920885\\
134.749284475488	52.3872778891779\\
134.931549809048	52.2528068244702\\
135.113815142608	52.108321074337\\
135.30622400761	51.9451305434082\\
135.498632872612	51.7711979733414\\
135.691041737614	51.5867365044183\\
135.891232660229	51.3838663099374\\
136.091423582844	51.1700655149542\\
136.291614505459	50.9455706864977\\
136.49702125615	50.7043581966007\\
136.70242800684	50.4523884273414\\
136.907834757531	50.1899137081131\\
137.116392693838	49.9129229771168\\
137.324950630144	49.6256238464101\\
137.53350856645	49.3282776613776\\
137.689005710967	49.1001987967545\\
137.844502855483	48.8667874113503\\
138	48.6281510064361\\
138.093853952566	48.4857077421126\\
138.187707905132	48.3494642474387\\
138.281561857698	48.2192837563506\\
138.375415810264	48.0950338466828\\
138.46926976283	47.9765861840206\\
138.563123715396	47.8638162752769\\
138.666914567606	47.7455725030695\\
138.770705419817	47.6339663948285\\
138.874496272028	47.5288444968674\\
138.985607800577	47.4233258075132\\
139.096719329126	47.3248911681196\\
139.207830857675	47.2333693572373\\
139.327955726796	47.1420085015112\\
139.448080595918	47.058330716354\\
139.568205465039	46.9821395112253\\
139.699107195307	46.9074141659175\\
139.830008925575	46.8411150173091\\
139.960910655842	46.7830117647846\\
140.105194893226	46.7281973774499\\
140.249479130611	46.6827817374744\\
140.393763367995	46.6464866283881\\
140.555590833401	46.6162976195003\\
140.717418298808	46.5968787617563\\
140.879245764214	46.5878780826218\\
141.066650452084	46.5900277550375\\
141.254055139954	46.6051889413469\\
141.441459827824	46.6328762235683\\
141.678824177296	46.6851988124533\\
141.916188526767	46.7559673137502\\
142.153552876239	46.8443224800758\\
142.441581666607	46.9739823008175\\
142.729610456974	47.1269546789181\\
143.017639247342	47.3019355117536\\
143.248753688107	47.4574219267387\\
143.479868128871	47.6256820463338\\
143.710982569636	47.8061392441572\\
143.80732171309	47.8848320159329\\
143.903660856545	47.965509468783\\
144	48.0481336535323\\
144.116640521077	48.1487099957131\\
144.233281042154	48.2480508894761\\
144.349921563231	48.3461704421695\\
144.466562084308	48.4430825893197\\
144.583202605385	48.5388010964119\\
144.699843126462	48.6333395630611\\
144.830521975037	48.7378712224883\\
144.961200823613	48.8409572110657\\
145.091879672189	48.9426159313068\\
145.233104806869	49.0508951629916\\
145.37432994155	49.1575512558828\\
145.515555076231	49.262606483529\\
145.669834706184	49.3755689341486\\
145.824114336137	49.4866752689672\\
145.97839396609	49.595953245588\\
146.148095050668	49.7140753807504\\
146.317796135247	49.8300542039016\\
146.487497219825	49.9439248901318\\
146.675806674913	50.0678563113452\\
146.864116130001	50.1892808951778\\
147.052425585088	50.3082441952865\\
147.263582681642	50.4387693034966\\
147.474739778196	50.5663176352712\\
147.68589687475	50.690949727254\\
147.925760015535	50.8290623801909\\
148.16562315632	50.9635734727303\\
148.405486297105	51.0945660603736\\
148.682549663296	51.2416009911995\\
148.959613029487	51.3841715566219\\
149.236676395678	51.5223965079945\\
149.491117597118	51.6456066893835\\
149.745558798559	51.7653367570283\\
150	51.8816719153804\\
150.116442484231	51.9320791652339\\
150.232884968462	51.9783815014532\\
150.349327452693	52.020626234572\\
150.465769936924	52.058860308525\\
150.582212421156	52.0931303017506\\
150.698654905387	52.1234824430299\\
150.83398732109	52.1538964583462\\
150.969319736794	52.1791517797394\\
151.104652152497	52.1993192573195\\
151.25002717596	52.2153953193389\\
151.395402199423	52.2257678149118\\
151.540777222886	52.2305223333159\\
151.698856808101	52.2294155727581\\
151.856936393317	52.221874442356\\
152.015015978532	52.2080060888171\\
152.187119177633	52.1858376543304\\
152.359222376733	52.1564311858197\\
152.531325575833	52.1199211493204\\
152.718654267521	52.0722630081672\\
152.905982959208	52.0165167331337\\
153.093311650896	51.9528507496641\\
153.296250716077	51.8751356535987\\
153.499189781258	51.7885315074602\\
153.702128846439	51.6932460302471\\
153.919683044049	51.5816905061643\\
154.137237241658	51.4606450529584\\
154.354791439268	51.3303575989708\\
154.58444317169	51.1830691621563\\
154.814094904112	51.0260409721535\\
155.043746636534	50.8595552624247\\
155.28224677198	50.6769430968475\\
155.520746907425	50.4847407608329\\
155.759247042871	50.2832539265963\\
155.83949802858	50.213422661163\\
155.91974901429	50.1425858795345\\
156	50.0707549647515\\
156.108120146047	49.9749817272606\\
156.216240292093	49.8825048381943\\
156.32436043814	49.7932675409526\\
156.432480584186	49.7072140815508\\
156.540600730233	49.6242896878285\\
156.648720876279	49.5444405384178\\
156.770093367096	49.4584024361068\\
156.891465857913	49.3760995912289\\
157.01283834873	49.2974599747681\\
157.143459527516	49.2168396269022\\
157.274080706301	49.1402939544157\\
157.404701885086	49.0677384058376\\
157.546868697431	48.9932075345075\\
157.689035509776	48.9232007241651\\
157.831202322121	48.8576157277137\\
157.987080808715	48.7906687146538\\
158.14295929531	48.7287885920813\\
158.298837781904	48.6718495805001\\
158.471510963414	48.6143957518147\\
158.644184144923	48.5626917277433\\
158.816857326433	48.5165789278502\\
159.010855388831	48.4712491807852\\
159.204853451229	48.4325690165371\\
159.398851513627	48.4003314654096\\
159.621722424973	48.3709880638027\\
159.84459333632	48.3495871918417\\
160.067464247666	48.3358430249475\\
160.335333468213	48.3290656226056\\
160.60320268876	48.3324871062079\\
160.871071909307	48.3456624337355\\
161.247381272871	48.3798303692759\\
161.623690636436	48.4312753058166\\
162	48.498926227956\\
162.180417942447	48.536342517774\\
162.360835884893	48.5762291005828\\
162.54125382734	48.6185097820529\\
162.721671769787	48.6631102512178\\
162.902089712233	48.7099580498097\\
163.08250765468	48.7589824974271\\
163.293985005785	48.8191249115514\\
163.50546235689	48.8820552634672\\
163.716939707995	48.9476686402444\\
163.952657944235	49.0238404608777\\
164.188376180476	49.1030824256233\\
164.424094416716	49.1852621744978\\
164.690777879557	49.2816184818809\\
164.957461342397	49.3813922176267\\
165.224144805237	49.4844107775624\\
165.52581922096	49.6046485481098\\
165.827493636683	49.7285945733144\\
166.129168052405	49.8560269941124\\
166.468016239838	50.0030557934509\\
166.80686442727	50.1539269768652\\
167.145712614702	50.3083656734749\\
167.430475076468	50.4407199591809\\
167.715237538234	50.5752596768658\\
168	50.7118418104611\\
168.118908161684	50.767842534219\\
168.237816323368	50.8209903012708\\
168.356724485052	50.8713222308516\\
168.475632646736	50.918875090505\\
168.59454080842	50.9636852966599\\
168.713448970104	51.0057889256213\\
168.852138139649	51.0515249474757\\
168.990827309194	51.0936837186631\\
169.129516478739	51.1323207665004\\
169.279101947979	51.1701089346978\\
169.42868741722	51.2039323041598\\
169.57827288646	51.2338582452862\\
169.741991618521	51.2622232130106\\
169.905710350582	51.2860855128052\\
170.069429082643	51.3055303480623\\
170.249399987621	51.3219087286564\\
170.429370892599	51.3331610694602\\
170.609341797577	51.3393962147786\\
170.808138075032	51.3405808461905\\
171.006934352488	51.3359174203023\\
171.205730629944	51.3255465089468\\
171.425841339557	51.3075727494628\\
171.645952049169	51.2829568132344\\
171.866062758782	51.2518807680126\\
172.10914570255	51.2102701028991\\
172.352228646319	51.1612359942247\\
172.595311590088	51.105011307088\\
172.860937518437	51.0356202153872\\
173.126563446786	50.9582087536763\\
173.392189375136	50.8730639371426\\
173.594792916757	50.8030962026229\\
173.797396458379	50.7289158680534\\
174	50.6506437305695\\
174.132396701154	50.598668474043\\
174.264793402307	50.5476533718181\\
174.397190103461	50.497584696056\\
174.529586804615	50.4484488914049\\
174.661983505769	50.4002325747955\\
174.794380206922	50.3529225312448\\
174.946023975339	50.2998314989714\\
175.097667743755	50.2478929400699\\
175.249311512172	50.1970877958316\\
175.414756297765	50.1429296719107\\
175.580201083357	50.0900742853698\\
175.74564586895	50.0384979993629\\
175.928850940372	49.9828494003535\\
176.112056011794	49.9287096978599\\
176.295261083216	49.8760483887881\\
176.49990132307	49.8189368385127\\
176.704541562924	49.7635915286497\\
176.909181802777	49.7099723009268\\
177.140435096776	49.6514079939991\\
177.371688390775	49.5949420865277\\
177.602941684774	49.5405202421168\\
177.867961647982	49.4805970590454\\
178.132981611189	49.4232107698227\\
178.398001574397	49.3682854150103\\
178.707168020801	49.3072221044612\\
179.016334467206	49.249293197118\\
179.32550091361	49.1943880553821\\
179.550333942407	49.1562974330161\\
179.775166971203	49.1197090156113\\
180	49.08458356147\\
180.135549762341	49.0650984660139\\
180.271099524683	49.0480947659455\\
180.406649287024	49.0335198953457\\
180.542199049366	49.0213222738711\\
180.677748811707	49.0114512928652\\
180.813298574049	49.0038572852361\\
180.975178988826	48.9977035905032\\
181.137059403604	48.9946466219273\\
181.298939818382	48.9946068713526\\
181.476185627688	48.9979318088785\\
181.653431436994	49.0046811060671\\
181.830677246301	49.0147575110879\\
182.029473603638	49.0299001661344\\
182.228269960974	49.0489784482195\\
182.427066318311	49.0718654555256\\
182.653306195495	49.1023887107471\\
182.87954607268	49.1375084037639\\
183.105785949864	49.1770532717999\\
183.370224631094	49.2286620051413\\
183.634663312324	49.2858336648004\\
183.899101993553	49.3483212024032\\
184.22652265119	49.4327018319906\\
184.553943308828	49.5244336599129\\
184.881363966465	49.6231018124067\\
185.25424264431	49.7434118931469\\
185.627121322155	49.8716466796684\\
186	50.0072761694707\\
186.128182277987	50.0542144593129\\
186.256364555974	50.0993728465121\\
186.384546833961	50.1427777656747\\
186.512729111948	50.1844553425973\\
186.640911389935	50.224431394855\\
186.769093667922	50.2627314402733\\
186.918605016199	50.3053205602112\\
187.068116364476	50.3457033180825\\
187.217627712754	50.3839186930852\\
187.38006733472	50.4230267783261\\
187.542506956687	50.4596704371539\\
187.704946578654	50.4938975695108\\
187.884368479726	50.5289510444463\\
188.063790380798	50.5611764836008\\
188.24321228187	50.5906354871709\\
188.442782037189	50.6202265075189\\
188.642351792507	50.6465518141691\\
188.841921547826	50.6696918951851\\
189.065915578483	50.6919680855237\\
189.28990960914	50.7104408196896\\
189.513903639797	50.7252174867311\\
189.767514530035	50.7376190453778\\
190.021125420273	50.7455667042076\\
190.274736310512	50.749206424102\\
190.563731538671	50.7482843119623\\
190.85272676683	50.7421590896863\\
191.141721994989	50.7310311099358\\
191.427814663326	50.7152787846296\\
191.713907331663	50.6949990392031\\
192	50.6703712354995\\
192.214490725657	50.6494578298299\\
192.428981451314	50.6268543948243\\
192.64347217697	50.6026151054735\\
192.857962902627	50.5767930074271\\
193.072453628284	50.5494400120683\\
193.286944353941	50.5206069460077\\
193.546479110404	50.4838109692087\\
193.806013866867	50.4450068325194\\
194.06554862333	50.4042783768958\\
194.3656245721	50.354896167992\\
194.66570052087	50.3031732357701\\
194.96577646964	50.2492285352242\\
195.328520574264	50.1812162318057\\
195.691264678889	50.1103223472366\\
196.054008783513	50.0367363599814\\
196.523418967646	49.9378127823088\\
196.992829151778	49.8350635606137\\
197.462239335911	49.7288480226607\\
197.641492890607	49.6874450694704\\
197.820746445304	49.645604929044\\
198	49.6033455697246\\
198.126700138675	49.5742367946571\\
198.253400277349	49.5469156798446\\
198.380100416024	49.5213491045347\\
198.506800554699	49.4975044813955\\
198.633500693373	49.4753497512862\\
198.760200832048	49.4548533688342\\
198.916988489557	49.4317391977601\\
199.073776147066	49.4110592633359\\
199.230563804575	49.3927569579141\\
199.40070775678	49.3755210744329\\
199.570851708985	49.360950421834\\
199.740995661189	49.3489770675678\\
199.827330440793	49.3438735571922\\
199.913665220396	49.3394131150561\\
200	49.3355872940877\\
};
\end{axis}
\end{tikzpicture}%}
  \caption{Step response using a zero-order hold of sample time $6$ sec.}
  \label{fig:Q7.6}
\end{figure}

\begin{figure}[H]\centering
	\centering
	\scalebox{1}{% This file was created by matlab2tikz.
%
%The latest updates can be retrieved from
%  http://www.mathworks.com/matlabcentral/fileexchange/22022-matlab2tikz-matlab2tikz
%where you can also make suggestions and rate matlab2tikz.
%
\definecolor{mycolor1}{rgb}{0.00000,0.44700,0.74100}%
%
\begin{tikzpicture}

\begin{axis}[%
width=4.133in,
height=3.26in,
at={(0.693in,0.44in)},
scale only axis,
xmin=0,
xmax=200,
xmajorgrids,
ymin=25,
ymax=70,
ymajorgrids,
axis background/.style={fill=white}
]
\addplot [color=mycolor1,solid,forget plot]
  table[row sep=crcr]{%
0	40\\
0.0666666666666667	39.9988529199354\\
0.133333333333333	39.9954198587329\\
0.2	39.9897130376372\\
0.266666666666667	39.9817446149091\\
0.333333333333333	39.9715266859614\\
0.4	39.959071286313\\
0.47499490925035	39.942400621861\\
0.5499898185007	39.9229306571585\\
0.62498472775105	39.9006782627291\\
0.703019200761635	39.8745881554063\\
0.78105367377222	39.8455224785902\\
0.859088146782806	39.8134999414055\\
0.940721928520819	39.7768560227102\\
1.02235571025883	39.7370177534163\\
1.10398949199685	39.6940062251002\\
1.18937445628243	39.6456457248368\\
1.27475942056802	39.5938604847577\\
1.36014438485361	39.5386742795039\\
1.44941284679236	39.4773672565188\\
1.53868130873111	39.4123956293716\\
1.62794977066986	39.3437861740105\\
1.72115687139107	39.2682963135632\\
1.81436397211228	39.1888997791019\\
1.90757107283349	39.1056266274356\\
2.00466164289194	39.0147939058355\\
2.10175221295039	38.9198209377759\\
2.19884278300884	38.820741250687\\
2.29961827190097	38.7135934398924\\
2.4003937607931	38.6020944165113\\
2.50116924968523	38.4862812151995\\
2.60526965685247	38.3621560855701\\
2.70937006401971	38.2335072833433\\
2.81347047118694	38.1003751782987\\
2.92038536533207	37.9590191789915\\
3.02730025947719	37.8130200499409\\
3.13421515362232	37.6624210955585\\
3.24774120796246	37.4975227201725\\
3.36126726230261	37.3275382844275\\
3.47479331664276	37.1525191924883\\
3.59549874757794	36.9609670376133\\
3.71620417851312	36.7638427524235\\
3.8369096094483	36.5612076866583\\
3.96415943740082	36.3416866294139\\
4.09140926535334	36.1161806774553\\
4.21865909330586	35.8847613318055\\
4.35177080227781	35.6364307829988\\
4.48488251124977	35.3817892660967\\
4.61799422022172	35.1209183765544\\
4.75625532436033	34.8434477559509\\
4.89451642849894	34.5594358361441\\
5.03277753263755	34.2689740153455\\
5.17549817685085	33.9624751879954\\
5.31821882106414	33.649301764491\\
5.46093946527744	33.3295545382605\\
5.60747261901361	32.9945323623854\\
5.75400577274979	32.6527963614143\\
5.90053892648596	32.3044562823032\\
6.0502778238169	31.9417877231936\\
6.20001672114784	31.5724556873445\\
6.34975561847879	31.1965784362348\\
6.50211880765519	30.8075180951938\\
6.6544819968316	30.4119297182736\\
6.80684518600801	30.0099396450813\\
6.87123012400534	29.8381751973493\\
6.93561506200267	29.6652999113542\\
7	29.4913234403142\\
7.07098347675268	29.3062067399941\\
7.14196695350536	29.134980159296\\
7.21295043025803	28.9767162251056\\
7.28393390701071	28.8306310752864\\
7.35491738376339	28.6960437303668\\
7.42590086051607	28.5723533773937\\
7.50459138892353	28.4473264999134\\
7.583281917331	28.3343903772847\\
7.66197244573847	28.2329622074222\\
7.74845626489298	28.1341381419741\\
7.83494008404749	28.0479477264088\\
7.92142390320201	27.9738142054728\\
8.01697109076118	27.9052976063312\\
8.11251827832035	27.8502043012224\\
8.20806546587953	27.8079354989369\\
8.31463385661896	27.7752456462912\\
8.4212022473584	27.7571180365817\\
8.52777063809784	27.7528993928112\\
8.64880933805737	27.7642129705474\\
8.7698480380169	27.7918689030465\\
8.89088673797643	27.8351057374697\\
9.03400064284875	27.9053664940426\\
9.17711454772108	27.9953413643134\\
9.3202284525934	28.1040281859704\\
9.52039016676334	28.2856743939312\\
9.72055188093328	28.4997860677065\\
9.92071359510323	28.7442398904577\\
10.1006216395316	28.9882519330024\\
10.2805296839599	29.2539186034602\\
10.4604377283882	29.540039918025\\
10.623282709569	29.8157296541014\\
10.7861276907497	30.1064927267153\\
10.9489726719304	30.4116037282123\\
11.1123983709509	30.7315413789036\\
11.2758240699714	31.0645940856044\\
11.439249768992	31.4101497467984\\
11.6060614342351	31.7751597037842\\
11.7728730994781	32.1520149109506\\
11.9396847647212	32.5401672806019\\
12.1115094640346	32.9512491219264\\
12.2833341633479	33.3732324341326\\
12.4551588626613	33.8056104268142\\
12.6331256446145	34.2638916026053\\
12.8110924265678	34.7323052036028\\
12.9890592085211	35.2103734814473\\
13.174095517945	35.7171842455772\\
13.3591318273689	36.2334615474068\\
13.5441681367928	36.7587491021845\\
13.6961120911952	37.1965255325176\\
13.8480560455976	37.6398577238868\\
14	38.0885271692908\\
14.0992288972568	38.380087700881\\
14.1984577945137	38.6654008516861\\
14.2976866917705	38.9445308546093\\
14.3969155890273	39.2175408175785\\
14.4961444862842	39.4844927656753\\
14.595373383541	39.7454476924602\\
14.7045665459576	40.0257479796429\\
14.8137597083741	40.2989375697192\\
14.9229528707907	40.5650936926804\\
15.0395351277965	40.8415827441909\\
15.1561173848024	41.1102315380861\\
15.2726996418082	41.3711293202826\\
15.3980289026375	41.6430570176032\\
15.5233581634669	41.9062354625884\\
15.6486874242962	42.16076993292\\
15.7838926045926	42.4257881932151\\
15.9190977848891	42.6809946226322\\
16.0543029651855	42.9265147138648\\
16.2006718517766	43.1815308659381\\
16.3470407383676	43.4254934146057\\
16.4934096249587	43.6585535229983\\
16.6522430565285	43.8992988794334\\
16.8110764880983	44.127567423391\\
16.9699099196681	44.3435425762283\\
17.1422745582237	44.5642177324735\\
17.3146391967792	44.7708545264496\\
17.4870038353348	44.9636756365541\\
17.6732491577276	45.1567481204633\\
17.8594944801203	45.334217353191\\
18.045739802513	45.4963504486375\\
18.2448600987496	45.6530259240863\\
18.4439803949862	45.7927878453726\\
18.6431006912228	45.9159471118453\\
18.8524410019928	46.0278605585069\\
19.0617813127628	46.1221117642541\\
19.2711216235329	46.1990457252328\\
19.4870804929001	46.2606249616946\\
19.7030393622673	46.3045044183518\\
19.9189982316346	46.3310471060386\\
20.1383709262899	46.3406281165071\\
20.3577436209452	46.3330602385927\\
20.5771163156005	46.3087097050284\\
20.7180775437336	46.2843746878814\\
20.8590387718668	46.2533549446098\\
21	46.2157449519643\\
21.1348984690337	46.1721365864037\\
21.2697969380674	46.1196034248592\\
21.404695407101	46.0582354383061\\
21.5395938761347	45.9881218684865\\
21.6744923451684	45.9093512194562\\
21.8093908142021	45.8220113369651\\
21.9580305461778	45.7159560519356\\
22.1066702781535	45.5997184480433\\
22.2553100101292	45.4734134378715\\
22.4129667834173	45.328571485351\\
22.5706235567055	45.1726671220746\\
22.7282803299937	45.0058345906747\\
22.8944874547294	44.8182678578646\\
23.0606945794651	44.6188589154758\\
23.2269017042007	44.407762080717\\
23.3997782080008	44.1759587929725\\
23.5726547118008	43.931848467255\\
23.7455312156009	43.6756019367864\\
23.9227399692106	43.4005157230074\\
24.0999487228204	43.113038384013\\
24.2771574764301	42.81335154123\\
24.4564500112358	42.4979000057088\\
24.6357425460416	42.1703222382064\\
24.8150350808473	41.830804617039\\
24.9946873026146	41.4788168332788\\
25.1743395243819	41.115214701511\\
25.3539917461492	40.7401847654741\\
25.5328680181396	40.3556058314077\\
25.7117442901299	39.9600662939191\\
25.8906205621203	39.5537501277336\\
26.0680230197169	39.1403193872628\\
26.2454254773136	38.7166496955412\\
26.4228279349102	38.2829211344551\\
26.5983194759214	37.8441447029604\\
26.7738110169326	37.3958766378798\\
26.9493025579438	36.9382926275265\\
27.122579618195	36.4775148501437\\
27.2958566784463	36.0079970224312\\
27.4691337386975	35.5299103387931\\
27.6460891591317	35.0330101844592\\
27.8230445795658	34.5275371844299\\
28	34.0136767913027\\
28.0754282301454	33.8005607123865\\
28.1508564602909	33.6021180595627\\
28.2262846904363	33.4174256259284\\
28.3017129205817	33.2456942903779\\
28.3771411507271	33.0862330968152\\
28.4525693808726	32.9384284678671\\
28.5364893588803	32.7870178866831\\
28.620409336888	32.6486841815123\\
28.7043293148957	32.52281671123\\
28.7955768360928	32.3994613635652\\
28.8868243572899	32.2895447898732\\
28.978071878487	32.1924751908453\\
29.0781095452802	32.1001754868721\\
29.1781472120734	32.0220085328547\\
29.2781848788666	31.9573675921112\\
29.3890863485759	31.9008382006179\\
29.4999878182851	31.8595364540474\\
29.6108892879944	31.8328059467216\\
29.7364201620877	31.8193642385957\\
29.8619510361811	31.8229897320202\\
29.9874819102744	31.8429180547155\\
30.1362578836959	31.8866719325067\\
30.2850338571174	31.9512271712022\\
30.4338098305388	32.0355618570251\\
30.6534736430254	32.1942456757434\\
30.8731374555119	32.3911862676355\\
31.0928012679985	32.6238319878194\\
31.2709658011761	32.8371458778712\\
31.4491303343536	33.0712743806608\\
31.6272948675312	33.3251499511415\\
31.7924087668529	33.5771898236141\\
31.9575226661745	33.8445876102037\\
32.1226365654962	34.1266404754078\\
32.2889628120729	34.42490939111\\
32.4552890586496	34.7367366139859\\
32.6216153052263	35.0615163889516\\
32.7915444986792	35.4061119915209\\
32.9614736921321	35.763050511278\\
33.131402885585	36.1317832834399\\
33.306420323904	36.5233375914837\\
33.481437762223	36.9263128194452\\
33.656455200542	37.3401977152873\\
33.8375922378934	37.7795263903274\\
34.0187292752449	38.2295113193847\\
34.1998663125963	38.6896679854578\\
34.3879635876419	39.1777790338435\\
34.5760608626874	39.675873835182\\
34.7641581377329	40.1834880354692\\
34.8427720918219	40.3983565119317\\
34.921386045911	40.6147794903348\\
35	40.8327265749258\\
35.0966662329616	41.0982056978432\\
35.1933324659232	41.3568434919198\\
35.2899986988848	41.6086983217637\\
35.3866649318464	41.853827570481\\
35.483331164808	42.0922876735285\\
35.5799973977696	42.3241341626438\\
35.6863736966364	42.5716912638697\\
35.7927499955032	42.8113771508806\\
35.89912629437	43.0432622705757\\
36.0123778674399	43.2816389517159\\
36.1256294405098	43.5113346654383\\
36.2388810135797	43.7324302660768\\
36.360197336449	43.9598200314543\\
36.4815136593183	44.1775285935952\\
36.6028299821876	44.385650482037\\
36.7330861249329	44.5985425590075\\
36.8633422676781	44.8006033440093\\
36.9935984104234	44.9919441223944\\
37.1336952803998	45.1859003058544\\
37.2737921503762	45.3677160467427\\
37.4138890203526	45.5375230556106\\
37.5645501209275	45.706881639377\\
37.7152112215024	45.8626604155059\\
37.8658723220773	46.0050152672571\\
38.0273597906794	46.1428684216947\\
38.1888472592815	46.2656610407537\\
38.3503347278836	46.3735759457453\\
38.5220617155711	46.4722123717695\\
38.6937887032585	46.5544409737318\\
38.865515690946	46.6204714395189\\
39.0457802482169	46.6725844050819\\
39.2260448054879	46.7073114015265\\
39.4063093627588	46.7248842966099\\
39.5925179137513	46.7252674164492\\
39.7787264647437	46.7078388130419\\
39.9649350157362	46.6728436033968\\
40.154354613486	46.6194710191883\\
40.3437742112359	46.5484225526772\\
40.5331938089858	46.4599466789462\\
40.7236698536836	46.353652069916\\
40.9141458983815	46.2302304430325\\
41.1046219430793	46.089926230435\\
41.2947880218755	45.933250295172\\
41.4849541006717	45.7602264164364\\
41.675120179468	45.5710910526959\\
41.7834134529786	45.4562746950957\\
41.8917067264893	45.3363526139018\\
42	45.2113676574719\\
42.1362442452892	45.0490245809215\\
42.2724884905785	44.88286535829\\
42.4087327358677	44.7129617701736\\
42.5449769811569	44.5393853691119\\
42.6812212264461	44.3622074531095\\
42.8174654717354	44.18149910368\\
42.9706915048414	43.9741371210969\\
43.1239175379475	43.7624998486367\\
43.2771435710536	43.5466874032979\\
43.4472804701237	43.3022861390478\\
43.6174173691938	43.0529963740331\\
43.7875542682639	42.7989537244759\\
43.9815956060972	42.5035889921604\\
44.1756369439305	42.2024174003118\\
44.3696782817638	41.8956378232025\\
44.6029496748488	41.5196921247986\\
44.8362210679339	41.1362704016188\\
45.069492461019	40.7457135593154\\
45.3184305716921	40.3214392952537\\
45.5673686823653	39.8898357118962\\
45.8163067930385	39.4513106848896\\
46.0652449037117	39.00627010376\\
46.3141830143849	38.5551169113592\\
46.5631211250581	38.0982519324873\\
46.8113206238861	37.6374525883997\\
47.0595201227141	37.171763506488\\
47.3077196215421	36.7015742402308\\
47.5526694434609	36.2335068590858\\
47.7976192653797	35.7618020883501\\
48.0425690872986	35.2868260136591\\
48.289306066628	34.8054455645027\\
48.5360430459574	34.3214820486812\\
48.7827800252868	33.8352997637545\\
48.8551866835245	33.6922576286983\\
48.9275933417623	33.5490645677984\\
49	33.4057296051556\\
49.0809579863319	33.2518080881141\\
49.1619159726639	33.1103119920888\\
49.2428739589958	32.9806699102349\\
49.3238319453277	32.8623645900568\\
49.4047899316597	32.7549231002697\\
49.4857479179916	32.6579099284733\\
49.5771361762069	32.5604258882206\\
49.6685244344223	32.4751850956929\\
49.7599126926376	32.4016931604171\\
49.8586555625351	32.3349638865203\\
49.9573984324326	32.2808787878431\\
50.0561413023301	32.238937199134\\
50.164274003007	32.2063859108537\\
50.2724067036838	32.1872649902279\\
50.3805394043606	32.1810383595098\\
50.5005527145552	32.188616561776\\
50.6205660247498	32.2108175774198\\
50.7405793349444	32.2470387114624\\
50.8771026519276	32.3045711690362\\
51.0136259689109	32.3787357930636\\
51.1501492858942	32.4688015496291\\
51.3146508699048	32.5974782813415\\
51.4791524539154	32.7471278492741\\
51.6436540379261	32.9167052755638\\
51.8175693512783	33.1165788053048\\
51.9914846646305	33.336561637185\\
52.1653999779826	33.5756474783381\\
52.3393152913348	33.832895029533\\
52.513230604687	34.1074214328782\\
52.6871459180392	34.3983965264558\\
52.8622493387215	34.7071867738071\\
53.0373527594038	35.0311096326341\\
53.2124561800861	35.3694569386359\\
53.3872860549185	35.7210000519573\\
53.5621159297508	36.0856279140337\\
53.7369458045832	36.4627453898848\\
53.9153171229725	36.8597892563908\\
54.0936884413618	37.2686807447399\\
54.2720597597511	37.6888823330493\\
54.4556439786588	38.132637872804\\
54.6392281975664	38.5873075293615\\
54.822812416474	39.0523910753349\\
55.0127890408066	39.5441276166759\\
55.2027656651392	40.046008510696\\
55.3927422894718	40.5575596811138\\
55.5951615263146	41.1127505573934\\
55.7975807631573	41.6778927623025\\
56	42.2524963788579\\
56.0973256777096	42.5274729483106\\
56.1946513554191	42.7954909852052\\
56.2919770331287	43.056609374481\\
56.3893027108382	43.3108860144607\\
56.4866283885478	43.5583778507653\\
56.5839540662573	43.7991409200873\\
56.6909130926482	44.0560397773272\\
56.797872119039	44.3049505073276\\
56.9048311454298	44.5459439260603\\
57.0187566134287	44.7940043411703\\
57.1326820814276	45.0332443063351\\
57.2466075494266	45.2637452280349\\
57.3686896656642	45.5011401830185\\
57.4907717819019	45.7286887580471\\
57.6128538981396	45.9464862418515\\
57.7439837096928	46.1696701150747\\
57.875113521246	46.3818263255626\\
58.0062433327992	46.5830671908944\\
58.1473349660114	46.7875319433054\\
58.2884265992237	46.9796230943629\\
58.429518232436	47.1594737312523\\
58.5813012409249	47.3394356110601\\
58.7330842494138	47.50554338046\\
58.8848672579027	47.657954702816\\
59.0475926717425	47.8063169708155\\
59.2103180855822	47.9393030658505\\
59.373043499422	48.0570979914374\\
59.5460861807409	48.1658989658752\\
59.7191288620598	48.2579427232062\\
59.8921715433787	48.3334414071224\\
60.0737674910095	48.3951082572481\\
60.2553634386402	48.4390218635489\\
60.4369593862709	48.4654165532294\\
60.6244567739786	48.4745318526657\\
60.8119541616862	48.4654679543925\\
60.9994515493939	48.438472151281\\
61.1900933697463	48.3928897631099\\
61.3807351900988	48.3292745382138\\
61.5713770104513	48.2478767088199\\
61.7630193864573	48.1483785909409\\
61.9546617624633	48.0314090311591\\
62.1463041384693	47.8972138345178\\
62.3376053562736	47.7463204594888\\
62.528906574078	47.5787448469555\\
62.7202077918823	47.3947245447367\\
62.8134718612549	47.2991181218618\\
62.9067359306274	47.1996862522219\\
63	47.096455957078\\
63.1948922423029	46.8685937027409\\
63.3897844846058	46.6245063352816\\
63.5846767269086	46.3644365249875\\
63.7723463968011	46.0991249247965\\
63.9600160666935	45.8194311310528\\
64.1476857365859	45.525568088153\\
64.3339339163941	45.2201313294101\\
64.5201820962023	44.9011541579027\\
64.7064302760105	44.5688420990174\\
64.8908455841051	44.2268629280322\\
65.0752608921998	43.8722091652498\\
65.2596762002945	43.505078677217\\
65.4421840119786	43.1296556843899\\
65.6246918236627	42.7423973291985\\
65.8071996353468	42.3434946314158\\
65.9877047598009	41.9377030317369\\
66.168209884255	41.5208927171573\\
66.3487150087091	41.0932485276261\\
66.5271109788545	40.6601377531583\\
66.7055069489999	40.2168043768757\\
66.8839029191453	39.7634276278135\\
67.0600667692631	39.3060436764378\\
67.236230619381	38.8392147664583\\
67.4123944694989	38.3631149842205\\
67.5861838560359	37.8845187997565\\
67.759973242573	37.3972385164777\\
67.93376262911	36.9014434480066\\
68.1050132133237	36.4047279915888\\
68.2762637975375	35.9000737235613\\
68.4475143817513	35.387645457437\\
68.6160369228293	34.8759523855068\\
68.7845594639073	34.3570509078526\\
68.9530820049854	33.8311015110212\\
69.1186587728379	33.3076384746734\\
69.2842355406905	32.777682697904\\
69.449812308543	32.241390407908\\
69.6121924685485	31.7094501053357\\
69.774572628554	31.1717175380976\\
69.9369527885594	30.6283446253451\\
69.9579685257063	30.5576153848565\\
69.9789842628531	30.486794574893\\
70	30.4158825311729\\
70.0691498330871	30.1913047180698\\
70.1382996661741	29.9833355957559\\
70.2074494992611	29.7904684263982\\
70.2765993323482	29.6115277471792\\
70.3457491654352	29.4455421189224\\
70.4148989985223	29.2916887493867\\
70.4931815861999	29.1312765727646\\
70.5714641738774	28.9846329239494\\
70.649746761555	28.8509852778461\\
70.7361771483564	28.7177093407208\\
70.8226075351579	28.5986383384805\\
70.9090379219593	28.4930464733238\\
71.0047995854639	28.3910093859206\\
71.1005612489685	28.3039210764839\\
71.1963229124731	28.2310556268828\\
71.3035577921527	28.1655283495872\\
71.4107926718324	28.1161633337396\\
71.5180275515121	28.0821855944113\\
71.6410684354237	28.0612400439346\\
71.7641093193354	28.0586319347488\\
71.887150203247	28.073454374178\\
72.0376941451311	28.1140848747684\\
72.1882380870152	28.1781575741891\\
72.3387820288992	28.2643834933554\\
72.4965422487282	28.377225204694\\
72.6543024685573	28.5118405551524\\
72.8120626883863	28.6670683754718\\
72.9698229082153	28.8418341791814\\
73.1275831280444	29.0351393291123\\
73.2853433478734	29.2460522697442\\
73.4411907456509	29.4708466460403\\
73.5970381434283	29.7111916195098\\
73.7528855412057	29.9663503998037\\
73.9093801582009	30.2367778257596\\
74.065874775196	30.5207852288417\\
74.2223693921912	30.8177506966506\\
74.3823303674212	31.1340753426419\\
74.5422913426513	31.4627345570001\\
74.7022523178814	31.8031693998466\\
74.8670926614137	32.1657505286016\\
75.0319330049459	32.5397290945488\\
75.1967733484782	32.9245875763239\\
75.3675060970961	33.334151252949\\
75.5382388457139	33.7543461507842\\
75.7089715943318	34.1846841983316\\
75.8864258685894	34.6422179965563\\
76.0638801428471	35.1097184492548\\
76.2413344171047	35.5867195604396\\
76.4262613518009	36.0934403867986\\
76.6111882864972	36.6095260410721\\
76.7961152211935	37.1345276533544\\
76.864076814129	37.3296238092978\\
76.9320384070645	37.5258461300575\\
77	37.7231746387408\\
77.1011461570236	38.0149272286749\\
77.2022923140472	38.3012379552403\\
77.3034384710708	38.582168253351\\
77.4045846280945	38.857778484589\\
77.5057307851181	39.128127975517\\
77.6068769421417	39.3932750640415\\
77.7184853950546	39.6798797101115\\
77.8300938479675	39.9602956476621\\
77.9417023008803	40.2345973261465\\
78.0610972418612	40.5213498366792\\
78.1804921828421	40.8012769130014\\
78.299887123823	41.0744650193756\\
78.4285981948193	41.3615187885962\\
78.5573092658157	41.6409441798028\\
78.686020336812	41.9128438953016\\
78.8253916356126	42.1988935828868\\
78.9647629344133	42.4763634420326\\
79.1041342332139	42.7453770003513\\
79.2558039023073	43.0286621242488\\
79.4074735714008	43.3022298408322\\
79.5591432404942	43.5662308308145\\
79.7249804891129	43.8441179606603\\
79.8908177377316	44.1109342348584\\
80.0566549863503	44.3668658963128\\
80.2386375155089	44.6354054611418\\
80.4206200446674	44.8912948964829\\
80.6026025738259	45.1347666459854\\
80.802417673164	45.3880649528557\\
81.002232772502	45.6269636116289\\
81.20204787184	45.8517529254962\\
81.4202271636677	46.0814227214996\\
81.6384064554954	46.2949733253928\\
81.856585747323	46.4927610385283\\
82.0913279215551	46.6883603629313\\
82.3260700957871	46.866543718636\\
82.5608122700192	47.027729447325\\
82.8077157123746	47.1793747273437\\
83.05461915473	47.3131376278334\\
83.3015225970854	47.4294775036347\\
83.5343483980569	47.523628081006\\
83.7671741990285	47.6030579057636\\
84	47.6681320079664\\
84.1228525019463	47.6947162579599\\
84.2457050038926	47.7132917254439\\
84.3685575058389	47.7239305555144\\
84.4914100077851	47.7267042148338\\
84.6142625097314	47.7216834963914\\
84.7371150116777	47.7089385695485\\
84.874198446038	47.6856844062013\\
85.0112818803982	47.6529952322856\\
85.1483653147585	47.6109659411463\\
85.2946332336242	47.5559269941787\\
85.44090115249	47.4904747277244\\
85.5871690713557	47.4147213760494\\
85.7433398267914	47.3225940483047\\
85.8995105822271	47.2189847004843\\
86.0556813376627	47.104026499782\\
86.2210426999027	46.9700795396419\\
86.3864040621426	46.823712307091\\
86.5517654243825	46.6650791689883\\
86.7247570203102	46.4861648652633\\
86.8977486162379	46.2941684117509\\
87.0707402121655	46.089262788483\\
87.2490839648622	45.8646850071826\\
87.4274277175589	45.626756682201\\
87.6057714702556	45.3756638781453\\
87.7870698966052	45.1071085160919\\
87.9683683229548	44.8253335225278\\
88.1496667493044	44.5305314643423\\
88.3319598363565	44.2211713594028\\
88.5142529234085	43.8990289386568\\
88.6965460104606	43.5642978096204\\
88.8785014006365	43.2178255408097\\
89.0604567908125	42.8591950440056\\
89.2424121809884	42.488597528149\\
89.4232148099753	42.1086832190578\\
89.6040174389623	41.7173287140643\\
89.7848200679493	41.3147211056425\\
89.9639812521176	40.9048524473725\\
90.1431424362859	40.4842999092544\\
90.3223036204542	40.0532458150033\\
90.4995098153664	39.6167441594108\\
90.6767160102785	39.1703244881503\\
90.8539222051907	38.7141642469969\\
90.9026148034605	38.5871385117586\\
90.9513074017302	38.4593944634468\\
91	38.3309358050464\\
91.0833390819451	38.1162785781493\\
91.1666781638903	37.9129110585497\\
91.2500172458354	37.720359427523\\
91.3333563277806	37.538191242037\\
91.4166954097257	37.3660084520282\\
91.5000344916708	37.203442345916\\
91.591977504804	37.0348402778909\\
91.6839205179373	36.8771001058809\\
91.7758635310705	36.7298239652647\\
91.8748456307451	36.5825439855855\\
91.9738277304197	36.4465289316475\\
92.0728098300943	36.3213731460094\\
92.1804272463706	36.1971765680593\\
92.288044662647	36.0849079307417\\
92.3956620789234	35.9841351580486\\
92.5138610618486	35.886217349904\\
92.6320600447739	35.8011692759399\\
92.7502590276991	35.728512246669\\
92.8822165028504	35.6614829778041\\
93.0141739780016	35.6087422240171\\
93.1461314531528	35.5697307023754\\
93.2979205246055	35.5411512572642\\
93.4497095960582	35.5292826952446\\
93.6014986675109	35.5334114684088\\
93.7906079337293	35.5599217438479\\
93.9797171999477	35.6090016024472\\
94.1688264661661	35.6795098251137\\
94.3579263355351	35.7703738198413\\
94.5470262049041	35.8805950914655\\
94.7361260742731	36.0092313925243\\
94.9252259436421	36.1553961608283\\
95.114325813011	36.3182534991777\\
95.30342568238	36.4970135292233\\
95.4903127735263	36.6885759087719\\
95.6771998646725	36.8942600240598\\
95.8640869558187	37.1134189901675\\
96.052709502697	37.3476522802377\\
96.2413320495754	37.59438815884\\
96.4299545964537	37.8530574789455\\
96.6230843468269	38.1297076823077\\
96.8162140972001	38.4177496638855\\
97.0093438475733	38.7166583594138\\
97.2084466736041	39.0356596750384\\
97.407549499635	39.3651591680259\\
97.6066523256658	39.7046614965837\\
97.7377682171105	39.9334687226152\\
97.8688841085553	40.1662798127643\\
98	40.4029695985715\\
98.0988472812584	40.5796804944527\\
98.1976945625168	40.7501088690806\\
98.2965418437752	40.9143076978127\\
98.3953891250336	41.0723292319089\\
98.494236406292	41.2242250187355\\
98.5930836875504	41.3700459354882\\
98.7020921032349	41.5238732323078\\
98.8111005189194	41.6704400410233\\
98.9201089346039	41.8098119261927\\
99.0362138336424	41.9504168848143\\
99.1523187326808	42.0830101325825\\
99.2684236317193	42.2076677477517\\
99.392833840461	42.3325203389046\\
99.5172440492027	42.4484386797236\\
99.6416542579444	42.5555126780879\\
99.7751916996574	42.660701752025\\
99.9087291413703	42.7559116342134\\
99.9999999999991	42.8152999955976\\
100	42.8152999955982\\
100.000000000001	42.8152999955988\\
100.060000111936	42.8518449130421\\
100.120000223871	42.8864208884903\\
100.180000335806	42.9190373826783\\
100.240000447742	42.9497038036531\\
100.300000559677	42.9784295072576\\
100.360000671612	43.0052237988836\\
100.423787938025	43.0316015628511\\
100.487575204438	43.0558179222908\\
100.551362470852	43.0778838763902\\
100.617790623727	43.0985893879439\\
100.684218776602	43.116986898102\\
100.750646929477	43.1330886260508\\
100.820026196955	43.1474678148695\\
100.889405464433	43.1593697324443\\
100.95878473191	43.1688080728563\\
101.031369087066	43.1760603155398\\
101.103953442222	43.1806465232152\\
101.176537797377	43.1825821224413\\
101.252619853728	43.18178245142\\
101.328701910079	43.1781049932533\\
101.40478396643	43.1715672277184\\
101.484694633188	43.1616396684104\\
101.564605299946	43.1485959022998\\
101.644515966704	43.1324558601388\\
101.728630775305	43.1121435687541\\
101.812745583907	43.0884456036515\\
101.896860392508	43.0613848440442\\
101.985606004892	43.0292138352477\\
102.074351617276	42.993351499561\\
102.16309722966	42.9538242916145\\
102.256958097534	42.9080601215057\\
102.350818965408	42.8582568420097\\
102.444679833283	42.8044452787664\\
102.544203923273	42.7430427545811\\
102.643728013263	42.6772045371829\\
102.743252103253	42.6069668381906\\
102.849053420155	42.527514968162\\
102.954854737058	42.4431749644825\\
103.060656053961	42.3539897225011\\
103.173409384546	42.2536590584504\\
103.286162715131	42.147925443628\\
103.398916045716	42.0368401107303\\
103.519335202826	41.9123501124766\\
103.639754359937	41.7818760378814\\
103.760173517048	41.6454795351111\\
103.888957751117	41.493123780708\\
104.017741985187	41.3341392880367\\
104.146526219257	41.1686006476151\\
104.284251657404	40.9844007103662\\
104.421977095551	40.7928806293732\\
104.559702533699	40.5941308035912\\
104.706468355799	40.3744791819563\\
104.8532341779	40.146828777513\\
105	39.9112882348834\\
105.085917627657	39.7766251335058\\
105.171835255313	39.6527198268937\\
105.25775288297	39.539192271347\\
105.343670510627	39.4356874720892\\
105.429588138283	39.3418722793518\\
105.51550576594	39.2574328552884\\
105.609477400533	39.1754643436049\\
105.703449035125	39.1039927515167\\
105.797420669718	39.0426730289893\\
105.89836852338	38.9877394560082\\
105.999316377042	38.9437624462116\\
106.100264230704	38.9103797557281\\
106.209748590599	38.8857515833828\\
106.319232950493	38.8727665596807\\
106.428717310388	38.8710307916984\\
106.54867575102	38.88160161803\\
106.668634191652	38.9047619163559\\
106.788592632284	38.9400687119554\\
106.922134998938	38.9931484169026\\
107.055677365592	39.0602098149656\\
107.189219732245	39.140730554867\\
107.342180793339	39.2488600353523\\
107.495141854432	39.373292820947\\
107.648102915525	39.5133614035145\\
107.836512971618	39.7064497812628\\
108.024923027711	39.9211922236028\\
108.213333083804	40.1565437762759\\
108.405404212898	40.4166656959929\\
108.597475341991	40.6962072797365\\
108.789546471085	40.9942464137421\\
108.981617600179	41.309914073002\\
109.173688729273	41.6423898023985\\
109.365759858367	41.9908974962814\\
109.555474639717	42.3501500831859\\
109.745189421068	42.7236573763008\\
109.934904202418	43.1107832594756\\
110.125840872725	43.5135421292805\\
110.316777543032	43.9288994839505\\
110.507714213338	44.3562992513134\\
110.702958763665	44.8052441963111\\
110.898203313992	45.2656904398629\\
111.093447864319	45.7371259373445\\
111.294575413112	46.2337428508897\\
111.495702961904	46.7409970733755\\
111.696830510696	47.2584055949155\\
111.797887007131	47.5220645224482\\
111.898943503565	47.7881142712183\\
112	48.0564999173512\\
112.114433794924	48.3604950717015\\
112.228867589849	48.6620684619112\\
112.343301384773	48.9612473498447\\
112.457735179697	49.2580585182673\\
112.572168974622	49.5525282840848\\
112.686602769546	49.8446825133666\\
112.814136265913	50.1675835961768\\
112.94166976228	50.4876747896965\\
113.069203258646	50.8049900143196\\
113.207010247033	51.1447846532677\\
113.344817235421	51.48141813379\\
113.482624223808	51.814930854067\\
113.633126244369	52.1756487622764\\
113.78362826493	52.5327422381483\\
113.934130285492	52.8862607989005\\
114.099784905211	53.2712967382311\\
114.265439524929	53.6521236955538\\
114.431094144648	54.0288035647605\\
114.615364519956	54.4430149671878\\
114.799634895263	54.8522511206422\\
114.98390527057	55.2565915136772\\
115.191849400088	55.7070962057664\\
115.399793529606	56.1515739358005\\
115.607737659124	56.5901306745464\\
115.847673949706	57.0889365438554\\
116.087610240288	57.5801510414815\\
116.32754653087	58.0639240577629\\
116.616829119635	58.6375039278014\\
116.9061117084	59.2007254580723\\
117.195394297165	59.7538278590743\\
117.611589659808	60.5322964484552\\
118.027785022451	61.2909590113493\\
118.443980385095	62.0304481168311\\
118.62932025673	62.3537461818807\\
118.814660128365	62.6734138447998\\
119	62.9895017825488\\
119.108931717535	63.1705334730041\\
119.217863435071	63.3441931705369\\
119.326795152606	63.5105336942449\\
119.435726870142	63.6696072858947\\
119.544658587677	63.8214656219354\\
119.653590305213	63.966159838207\\
119.777237931473	64.1217840613487\\
119.900885557734	64.268316161529\\
120.024533183994	64.4058285179328\\
120.157850285327	64.5440714968143\\
120.29116738666	64.6720007393482\\
120.424484487993	64.7897039318338\\
120.569237374031	64.9060250901828\\
120.713990260069	65.0105024986249\\
120.858743146107	65.1032445979115\\
121.015800522929	65.1907207544989\\
121.172857899752	65.2646432617263\\
121.329915276575	65.3251459025229\\
121.499720365228	65.3756141741965\\
121.669525453881	65.4107154861404\\
121.839330542534	65.4306131556301\\
122.021317057985	65.4352430029511\\
122.203303573435	65.4227915861597\\
122.385290088886	65.393453232896\\
122.577487062107	65.3443467272613\\
122.769684035328	65.2768435109723\\
122.96188100855	65.1911651526162\\
123.161085896716	65.0834157547992\\
123.360290784883	64.9566188671796\\
123.55949567305	64.8110138287981\\
123.762504142239	64.6435238963106\\
123.965512611429	64.4569955698513\\
124.168521080618	64.2516753553528\\
124.372897068347	64.0262374111214\\
124.577273056075	63.7822487623416\\
124.781649043803	63.5199550433611\\
124.985860959206	63.2398324779946\\
125.190072874608	62.9419194082537\\
125.394284790011	62.626456050876\\
125.596189860007	62.2975362434857\\
125.798094930004	61.9519234525413\\
126	61.5898461610068\\
126.201866443143	61.2116058327991\\
126.403732886287	60.817361052737\\
126.60559932943	60.4073374048153\\
126.805722710834	59.9855005221436\\
127.005846092238	59.5485955033413\\
127.205969473643	59.0968402864169\\
127.404285542159	58.6347292608517\\
127.602601610675	58.1584599403734\\
127.800917679191	57.6682434094193\\
127.997307017194	57.1692522473671\\
128.193696355196	56.656995142413\\
128.390085693199	56.1316768722444\\
128.584439969839	55.5991444370185\\
128.778794246479	55.0542194832277\\
128.973148523119	54.4971010110403\\
129.165352810831	53.9343490365951\\
129.357557098543	53.3600599420769\\
129.549761386255	52.7744274136034\\
129.739688008725	52.1847823948018\\
129.929614631194	51.5844388854715\\
130.119541253664	50.9735856448835\\
130.307045913907	50.3603978421456\\
130.494550574149	49.7373345649766\\
130.682055234392	49.1045799524179\\
130.866974035728	48.4712405282059\\
131.051892837063	47.8288338728671\\
131.236811638399	47.1775397282911\\
131.418958140448	46.5274974115493\\
131.601104642498	45.8691817635856\\
131.783251144547	45.2027682643754\\
131.96241297628	44.5395466936087\\
132.141574808014	43.8688313091645\\
132.320736639747	43.1907933694754\\
132.496671310893	42.5180102767901\\
132.67260598204	41.8384979473959\\
132.848540653187	41.1524233530075\\
132.899027102124	40.9543584925484\\
132.949513551062	40.755771160618\\
133	40.5566653860493\\
133.075587922031	40.2675697612489\\
133.151175844061	39.9961593951551\\
133.226763766092	39.7410209575736\\
133.302351688123	39.5010160334267\\
133.377939610153	39.2751861639009\\
133.453527532184	39.0627071925041\\
133.538903795048	38.8378783860431\\
133.624280057912	38.6282751266608\\
133.709656320776	38.4331028070532\\
133.803631874227	38.2341174983943\\
133.897607427679	38.0509249846972\\
133.99158298113	37.8827692175234\\
134.09579241389	37.7130587369203\\
134.200001846649	37.5601560581548\\
134.304211279409	37.4232913937226\\
134.421315489978	37.2877574166373\\
134.538419700546	37.1706754801967\\
134.655523911115	37.071205724113\\
134.790967364881	36.9771126213189\\
134.926410818647	36.9044469585222\\
135.061854272413	36.8521935713828\\
135.231887484898	36.8140599942828\\
135.401920697383	36.804918086097\\
135.571953909868	36.8232020948374\\
135.741495370171	36.8673091047796\\
135.911036830475	36.9359343832537\\
136.080578290778	37.0278525217919\\
136.250119751081	37.1419241289597\\
136.419661211385	37.2770854748816\\
136.589202671688	37.4323400218502\\
136.756823228292	37.6046738079568\\
136.924443784895	37.7948867255005\\
137.092064341498	38.0021788263028\\
137.261600345092	38.2284426817591\\
137.431136348685	38.4706695275241\\
137.600672352279	38.7281617869075\\
137.774533483179	39.0073865076781\\
137.948394614078	39.3012972827221\\
138.122255744978	39.609254882729\\
138.301748906716	39.9412827950706\\
138.481242068453	40.2870041493251\\
138.660735230191	40.6458185753643\\
138.846788298022	41.0309576647461\\
139.032841365852	41.4289499499867\\
139.218894433683	41.8392232598681\\
139.412250650792	42.2780316310634\\
139.6056068679	42.728938265856\\
139.798963085009	43.1913928735301\\
139.865975390006	43.354266258372\\
139.932987695003	43.5184426821891\\
140	43.6839014792666\\
140.111671325243	43.9587142499396\\
140.223342650487	44.2296194391235\\
140.33501397573	44.4966648391826\\
140.446685300974	44.7598974584776\\
140.558356626217	45.0193635427434\\
140.670027951461	45.2751086027168\\
140.794133546076	45.5550190095772\\
140.918239140692	45.8304493734634\\
141.042344735308	46.1014591265088\\
141.17611233102	46.3886846736132\\
141.309879926732	46.670914240033\\
141.443647522445	46.9482184351914\\
141.58922532464	47.2444897963534\\
141.734803126835	47.5350975897441\\
141.88038092903	47.8201279652782\\
142.039846128124	48.1260480209531\\
142.199311327218	48.4254862271081\\
142.358776526312	48.7185494643377\\
142.534892659007	49.0349313424221\\
142.711008791701	49.3438035272984\\
142.887124924396	49.6453014650635\\
143.083535485756	49.973006508565\\
143.279946047117	50.2918847900729\\
143.476356608477	50.6021123257645\\
143.698011802364	50.9420387992472\\
143.919666996251	51.2714085802899\\
144.141322190138	51.5904574877427\\
144.395135609558	51.9434152736763\\
144.648949028978	52.2834786155492\\
144.902762448398	52.6109754477199\\
145.198547575502	52.9771978675866\\
145.494332702605	53.327278486329\\
145.790117829709	53.6616940154749\\
146.140992738341	54.0387175122024\\
146.491867646973	54.3950956985283\\
146.842742555605	54.7315515651327\\
146.895161703736	54.7801494005742\\
146.947580851868	54.8283205141451\\
147	54.8760671798847\\
147.119448459638	54.9802648079645\\
147.238896919276	55.076264451813\\
147.358345378913	55.1641336468848\\
147.477793838551	55.2439392118945\\
147.597242298189	55.3157472608072\\
147.716690757827	55.3796232426933\\
147.850516605891	55.4418460387825\\
147.984342453954	55.4942838614071\\
148.118168302018	55.5370259376701\\
148.261691298932	55.5721905890754\\
148.405214295845	55.5964127788913\\
148.548737292758	55.6097991545432\\
148.703395615074	55.6122158363751\\
148.85805393739	55.6023033212682\\
149.012712259705	55.5801909669096\\
149.178774726808	55.5430117723645\\
149.344837193911	55.4920717501095\\
149.510899661014	55.427526242068\\
149.68784795539	55.3439500617943\\
149.864796249767	55.2452851039644\\
150.041744544144	55.1317139145805\\
150.227873047133	54.9963628352219\\
150.414001550122	54.844927393467\\
150.600130053111	54.677614362257\\
150.79274840195	54.487968309489\\
150.985366750789	54.2817630787572\\
151.177985099629	54.0592223183988\\
151.37419014529	53.8159727771485\\
151.570395190952	53.5562369940514\\
151.766600236614	53.2802464461046\\
151.96405528144	52.9863200065459\\
152.161510326266	52.6763971448823\\
152.358965371092	52.3507097985112\\
152.55615270057	52.0099619475303\\
152.753340030047	51.6539513970458\\
152.950527359525	51.2829061684886\\
153.146577308637	50.899321256003\\
153.342627257749	50.5013218882936\\
153.53867720686	50.0891302065622\\
153.69245147124	49.7560348171143\\
153.84622573562	49.4144511924134\\
154	49.064486064532\\
154.096063840442	48.8477959717038\\
154.192127680883	48.6399992859747\\
154.288191521325	48.4408511111403\\
154.384255361766	48.2501180595472\\
154.480319202208	48.0675771917302\\
154.576383042649	47.8930150954069\\
154.682413810121	47.7093598303043\\
154.788444577593	47.5349128993394\\
154.894475345065	47.3694223703366\\
155.008459585022	47.2012319299477\\
155.122443824979	47.0428260475269\\
155.236428064936	46.8939294958946\\
155.360238348966	46.7426642760647\\
155.484048632997	46.601983264676\\
155.607858917027	46.471575696705\\
155.743613136725	46.3400424087834\\
155.879367356422	46.2201256911283\\
156.01512157612	46.1114646390968\\
156.166080118685	46.0034298953163\\
156.31703866125	45.9084301119106\\
156.467997203815	45.8260290955885\\
156.639792394145	45.7470509044685\\
156.811587584474	45.6832606583221\\
156.983382774804	45.6340963841374\\
157.188850082132	45.5937535130609\\
157.394317389459	45.5726909810385\\
157.599784696786	45.570078158765\\
157.840308909477	45.5894022292452\\
158.080833122168	45.6317299080417\\
158.321357334859	45.6959274507507\\
158.56188154755	45.7809231452382\\
158.802405760241	45.8857025768477\\
159.042929972932	46.0093037262177\\
159.273341931161	46.1445158960123\\
159.503753889389	46.29540133818\\
159.734165847618	46.4612344272105\\
159.965800194614	46.6423164172645\\
160.197434541609	46.8371378778887\\
160.429068888605	47.0450607551487\\
160.61937925907	47.2252646601932\\
160.809689629535	47.4135778484041\\
161	47.6096890276672\\
161.123004037478	47.7377494409838\\
161.246008074956	47.8635342046848\\
161.369012112434	47.9870728652158\\
161.492016149912	48.1083945844617\\
161.615020187391	48.2275281442508\\
161.738024224869	48.3445019572609\\
161.876334669749	48.4734870623248\\
162.014645114629	48.5998162514307\\
162.152955559509	48.7235282971236\\
162.303024458827	48.8548418149088\\
162.453093358144	48.9831673642357\\
162.603162257461	49.1085523107248\\
162.767887781545	49.2428531064913\\
162.932613305629	49.3737277476809\\
163.097338829712	49.5012359411376\\
163.279498628337	49.6383903963148\\
163.461658426962	49.7715779199908\\
163.643818225587	49.9008751488637\\
163.847160380754	50.0407047682563\\
164.05050253592	50.1758831704003\\
164.253844691087	50.3065110279121\\
164.483317587502	50.4485836309988\\
164.712790483918	50.5851254525222\\
164.942263380334	50.7162722864098\\
165.204509101362	50.859713702359\\
165.46675482239	50.9964759771042\\
165.729000543417	51.126747864682\\
166.03286426164	51.2698187342465\\
166.336727979862	51.4047013224793\\
166.640591698085	51.5316663998267\\
166.996956790613	51.6708348673887\\
167.35332188314	51.7998861832058\\
167.709686975668	51.9192180523498\\
167.806457983779	51.9499952203775\\
167.903228991889	51.9800916459113\\
168	52.0095147969016\\
168.12761971668	52.0454048509708\\
168.255239433361	52.0763926655274\\
168.382859150041	52.1025403142919\\
168.510478866722	52.1239093822484\\
168.638098583402	52.1405609656931\\
168.765718300083	52.1525556969823\\
168.917150611079	52.1608301178018\\
169.068582922076	52.1627321144181\\
169.220015233072	52.1583605295488\\
169.383464488588	52.1467148059753\\
169.546913744104	52.1279965962363\\
169.71036299962	52.1023269475267\\
169.889362224478	52.0663822753834\\
170.068361449337	52.0224005886382\\
170.247360674195	51.970536577109\\
170.443249457357	51.9049270375165\\
170.639138240519	51.8302599916212\\
170.835027023682	51.7467324654681\\
171.048618682487	51.6457876819817\\
171.262210341293	51.5347909716531\\
171.475802000099	51.4139902487626\\
171.706264726903	51.27293412112\\
171.936727453707	51.1210548205501\\
172.16719018051	50.9586542642952\\
172.411716312888	50.7751730913864\\
172.656242445265	50.5805367818671\\
172.900768577642	50.3750945950389\\
173.155451757876	50.1500028408645\\
173.410134938111	49.9139497258968\\
173.664818118345	49.6673170946658\\
173.926279325089	49.403511540444\\
174.187740531833	49.1293578412346\\
174.449201738576	48.8452554534887\\
174.632801159051	48.6400243184451\\
174.816400579525	48.4302178213817\\
175	48.2159702883481\\
175.105821676973	48.0946630174685\\
175.211643353947	47.9801152129558\\
175.31746503092	47.872169792528\\
175.423286707894	47.7706745198721\\
175.529108384867	47.6754817422997\\
175.63493006184	47.5864481295643\\
175.754377660801	47.4931759141602\\
175.873825259761	47.4073798991209\\
175.993272858722	47.3288721002835\\
176.122151436806	47.2521303045017\\
176.251030014891	47.183439600977\\
176.379908592975	47.1225855697272\\
176.520877476794	47.0647516874093\\
176.661846360613	47.0157828222566\\
176.802815244432	46.9754249773032\\
176.958674151862	46.9405320315837\\
177.114533059292	46.9155448358605\\
177.270391966722	46.9001541793388\\
177.445677103149	46.8939396197273\\
177.620962239576	46.8990769890415\\
177.796247376003	46.9151728790482\\
177.999418722371	46.9470437227938\\
178.20259006874	46.9925633993665\\
178.405761415108	47.0511897851834\\
178.66052805595	47.1424323194999\\
178.915294696793	47.2524922791698\\
179.170061337635	47.3804422394569\\
179.429212059738	47.5280399184829\\
179.688362781841	47.6923571617663\\
179.947513503944	47.8725568998135\\
180.206664226047	48.0678435229239\\
180.46581494815	48.2774604681986\\
180.724965670252	48.5006872841888\\
180.98484261353	48.7375175018204\\
181.244719556808	48.9866859156507\\
181.504596500086	49.247564819184\\
181.669731000058	49.4191466408964\\
181.834865500029	49.5950672810074\\
182	49.7751835479025\\
182.117663811611	49.9029767685135\\
182.235327623223	50.0267877225787\\
182.352991434834	50.1466639683891\\
182.470655246445	50.262652558773\\
182.588319058057	50.3748000470937\\
182.705982869668	50.4831525049412\\
182.838927444485	50.6010673860528\\
182.971872019303	50.7142607669999\\
183.10481659412	50.8227969574205\\
183.248344055311	50.9348182501885\\
183.391871516501	51.0415645502739\\
183.535398977692	51.1431139913482\\
183.692057397579	51.2481129464983\\
183.848715817467	51.3471117808606\\
184.005374237354	51.4402084143057\\
184.177240736541	51.5356680085015\\
184.349107235729	51.6242662014962\\
184.520973734916	51.7061273269503\\
184.710579917977	51.7887691312006\\
184.900186101038	51.8635230330142\\
185.089792284099	51.9305491507046\\
185.299834960947	51.9959683798603\\
185.509877637796	52.0523092095867\\
185.719920314645	52.0997797743727\\
185.952726840712	52.1422787589599\\
186.185533366779	52.1744069037316\\
186.418339892846	52.1964342454667\\
186.674678627632	52.2093216957206\\
186.931017362418	52.2106334003199\\
187.187356097204	52.2007115322226\\
187.464978204752	52.1776840430475\\
187.742600312301	52.1422932245943\\
188.020222419849	52.0949501357402\\
188.313941761457	52.032297552854\\
188.607661103065	51.9571858853803\\
188.901380444672	51.870073974013\\
188.934253629782	51.8595974731096\\
188.967126814891	51.8489769205006\\
189	51.8382129380718\\
189.164365925546	51.7827711369151\\
189.328731851092	51.7248400850764\\
189.493097776639	51.66447840082\\
189.694714840548	51.5872022147955\\
189.896331904457	51.5064611260493\\
190.097948968366	51.422358767923\\
190.323426413064	51.3244473634835\\
190.548903857762	51.2225998086943\\
190.77438130246	51.1169538633363\\
191.036454700395	50.9895670774507\\
191.29852809833	50.8574397785912\\
191.560601496264	50.7207758475571\\
191.877118386441	50.5499571890565\\
192.193635276618	50.3731551658957\\
192.510152166795	50.1907039894901\\
192.954194536412	49.9258418243435\\
193.398236906029	49.6513695816465\\
193.842279275646	49.3681238298689\\
194.270447187977	49.0874489434123\\
194.698615100308	48.8000574980727\\
195.126783012638	48.5066157616043\\
195.417855341759	48.3040046548246\\
195.708927670879	48.0990880042793\\
196	47.8920546101299\\
196.110921121305	47.8156564965803\\
196.221842242609	47.7449276049594\\
196.332763363914	47.6797418089624\\
196.443684485219	47.61997630496\\
196.554605606523	47.5655114755514\\
196.665526727828	47.516230741028\\
196.795431703862	47.4649540920666\\
196.925336679896	47.4204545915385\\
197.055241655931	47.3825597586068\\
197.195673741096	47.3488297874478\\
197.33610582626	47.3224171242567\\
197.476537911425	47.3031224410373\\
197.63128271452	47.2898730137429\\
197.786027517615	47.2847810085227\\
197.940772320709	47.287603650707\\
198.113400757158	47.2998042273819\\
198.286029193607	47.3212466875995\\
198.458657630056	47.351626330309\\
198.655418699804	47.3967763784013\\
198.852179769553	47.4527351941632\\
199.048940839301	47.5190987464702\\
199.282976768614	47.6110452780486\\
199.517012697928	47.7165356266562\\
199.751048627241	47.8349697543447\\
199.834032418161	47.8799633397648\\
199.91701620908	47.9264870993058\\
200	47.9745165958885\\
};
\end{axis}
\end{tikzpicture}%}
  \caption{Step response using a zero-order hold of sample time $7$ sec.}
  \label{fig:Q7.7}
\end{figure}

\begin{figure}[H]\centering
	\centering
	\scalebox{1}{% This file was created by matlab2tikz.
%
%The latest updates can be retrieved from
%  http://www.mathworks.com/matlabcentral/fileexchange/22022-matlab2tikz-matlab2tikz
%where you can also make suggestions and rate matlab2tikz.
%
\definecolor{mycolor1}{rgb}{0.00000,0.44700,0.74100}%
%
\begin{tikzpicture}

\begin{axis}[%
width=4.133in,
height=3.26in,
at={(0.693in,0.44in)},
scale only axis,
xmin=0,
xmax=200,
xmajorgrids,
ymin=20,
ymax=70,
ymajorgrids,
axis background/.style={fill=white}
]
\addplot [color=mycolor1,solid,forget plot]
  table[row sep=crcr]{%
0	40\\
0.0666666666666667	39.9988529199354\\
0.133333333333333	39.9954198587329\\
0.2	39.9897130376372\\
0.266666666666667	39.9817446149091\\
0.333333333333333	39.9715266859614\\
0.4	39.959071286313\\
0.47499490925035	39.942400621861\\
0.5499898185007	39.9229306571585\\
0.62498472775105	39.9006782627291\\
0.703019200761635	39.8745881554063\\
0.78105367377222	39.8455224785902\\
0.859088146782806	39.8134999414055\\
0.940721928520819	39.7768560227102\\
1.02235571025883	39.7370177534163\\
1.10398949199685	39.6940062251002\\
1.18937445628243	39.6456457248368\\
1.27475942056802	39.5938604847577\\
1.36014438485361	39.5386742795039\\
1.44941284679236	39.4773672565188\\
1.53868130873111	39.4123956293716\\
1.62794977066986	39.3437861740105\\
1.72115687139107	39.2682963135632\\
1.81436397211228	39.1888997791019\\
1.90757107283349	39.1056266274356\\
2.00466164289194	39.0147939058355\\
2.10175221295039	38.9198209377759\\
2.19884278300884	38.820741250687\\
2.29961827190097	38.7135934398924\\
2.4003937607931	38.6020944165113\\
2.50116924968523	38.4862812151995\\
2.60526965685247	38.3621560855701\\
2.70937006401971	38.2335072833433\\
2.81347047118694	38.1003751782987\\
2.92038536533207	37.9590191789915\\
3.02730025947719	37.8130200499409\\
3.13421515362232	37.6624210955585\\
3.24774120796246	37.4975227201725\\
3.36126726230261	37.3275382844275\\
3.47479331664276	37.1525191924883\\
3.59549874757794	36.9609670376133\\
3.71620417851312	36.7638427524235\\
3.8369096094483	36.5612076866583\\
3.96415943740082	36.3416866294139\\
4.09140926535334	36.1161806774553\\
4.21865909330586	35.8847613318055\\
4.35177080227781	35.6364307829988\\
4.48488251124977	35.3817892660967\\
4.61799422022172	35.1209183765544\\
4.75625532436033	34.8434477559509\\
4.89451642849894	34.5594358361441\\
5.03277753263755	34.2689740153455\\
5.17549817685085	33.9624751879954\\
5.31821882106414	33.649301764491\\
5.46093946527744	33.3295545382605\\
5.60747261901361	32.9945323623854\\
5.75400577274979	32.6527963614143\\
5.90053892648596	32.3044562823032\\
6.0502778238169	31.9417877231936\\
6.20001672114784	31.5724556873445\\
6.34975561847879	31.1965784362348\\
6.50211880765519	30.8075180951938\\
6.6544819968316	30.4119297182736\\
6.80684518600801	30.0099396450813\\
6.96125719563924	29.5961430172421\\
7.11566920527048	29.17603488147\\
7.27008121490171	28.7497491976507\\
7.42595285881986	28.313306002723\\
7.58182450273801	27.8708452348722\\
7.73769614665617	27.4225079557523\\
7.82513076443744	27.1684982276952\\
7.91256538221872	26.9127095192791\\
8	26.655167276524\\
8.07021749735361	26.4565156254103\\
8.14043499470723	26.2744504907982\\
8.21065249206084	26.107493449743\\
8.28086998941446	25.9544803205715\\
8.35108748676807	25.8144436597778\\
8.42130498412168	25.6865602910439\\
8.50050115388712	25.5560294382255\\
8.57969732365256	25.4391835230606\\
8.65889349341799	25.335249758188\\
8.74705029296428	25.2339139787094\\
8.83520709251057	25.1468993289621\\
8.92336389205686	25.0734569880266\\
9.02181885944438	25.0066547105857\\
9.12027382683189	24.9550967854573\\
9.2187287942194	24.9180148886632\\
9.32991597142012	24.8926688026972\\
9.44110314862085	24.8839773122665\\
9.55229032582157	24.8911006264828\\
9.68106910141477	24.9181078387253\\
9.80984787700797	24.964204262628\\
9.93862665260117	25.0283843389424\\
10.0986932267202	25.1319797244251\\
10.2587598008393	25.260466741834\\
10.4188263749584	25.4123617995571\\
10.5835672238326	25.5917068761272\\
10.7483080727068	25.7930531752056\\
10.913048921581	26.0151484333657\\
11.0777897704552	26.2568375306105\\
11.2425306193295	26.5170503384755\\
11.4072714682037	26.7947918027485\\
11.5699905543449	27.0854272483497\\
11.7327096404862	27.3914289719012\\
11.8954287266275	27.7120181348773\\
12.0598150375775	28.049961817855\\
12.2242013485276	28.4013445685868\\
12.3885876594776	28.765503981144\\
12.5573381351771	29.1519683348673\\
12.7260886108766	29.5506079540084\\
12.894839086576	29.9608265937195\\
13.0694472225126	30.3968761861181\\
13.2440553584492	30.844136072427\\
13.4186634943858	31.3020539407207\\
13.6002655131304	31.7890611103942\\
13.781867531875	32.2864819781833\\
13.9634695506197	32.7937954439466\\
14.1530518252261	33.3334245648649\\
14.3426340998325	33.8827717732512\\
14.532216374439	34.4413390541349\\
14.7307235300214	35.0355718911156\\
14.9292306856039	35.6388892318996\\
15.1277378411863	36.2508104353447\\
15.3361434532061	36.9020069719599\\
15.5445490652259	37.5616916914288\\
15.7529546772457	38.2293969188403\\
15.8353031181638	38.4953436932021\\
15.9176515590819	38.7624473846483\\
16	39.0306823973631\\
16.1053489103345	39.3712247073957\\
16.210697820669	39.7051671009895\\
16.3160467310035	40.0325849656288\\
16.421395641338	40.3535522483549\\
16.5267445516724	40.668141513204\\
16.6320934620069	40.9764240080491\\
16.7494606376887	41.3125340063763\\
16.8668278133705	41.64099795449\\
16.9841949890523	41.9619089984687\\
17.1100547515225	42.2977523968758\\
17.2359145139927	42.6251259366222\\
17.3617742764628	42.9441380703261\\
17.4979384005165	43.2799750755178\\
17.6341025245702	43.6062821711235\\
17.7702666486239	43.9231889732051\\
17.9182456752645	44.2570827642216\\
18.066224701905	44.5801843424443\\
18.2142037285456	44.89265064108\\
18.3758173370271	45.2219380202893\\
18.5374309455086	45.5389201056938\\
18.69904455399	45.8437896144306\\
18.8762614929692	46.1643943444315\\
19.0534784319483	46.4709061163677\\
19.2306953709275	46.763564316702\\
19.4253226335785	47.0692913447421\\
19.6199498962296	47.3588967216772\\
19.8145771588807	47.6326790349736\\
20.0274961379684	47.9144143588266\\
20.240415117056	48.1779384057078\\
20.4533340961437	48.4236192418477\\
20.6832891648511	48.6693316782885\\
20.9132442335584	48.8950976682908\\
21.1431993022657	49.1013540533791\\
21.3861250293677	49.2985271237671\\
21.6290507564697	49.4749010712005\\
21.8719764835717	49.6309620123378\\
22.1224453669918	49.7711047392179\\
22.3729142504119	49.890674090215\\
22.6233831338319	49.9901745695946\\
22.8771656114024	50.0710305164391\\
23.1309480889729	50.1322991093731\\
23.3847305665433	50.1744791937666\\
23.5898203776955	50.1949561594879\\
23.7949101888478	50.2035430323428\\
24	50.2004921674399\\
24.1313396313497	50.1904457132119\\
24.2626792626995	50.1716152524667\\
24.3940188940492	50.14408324604\\
24.525358525399	50.1079314439313\\
24.6566981567487	50.063240884626\\
24.7880377880984	50.0100919574274\\
24.9347577950269	49.9408149375839\\
25.0814778019553	49.8611923108874\\
25.2281978088838	49.7713334379758\\
25.3849404935956	49.664149738796\\
25.5416831783074	49.5455385288135\\
25.6984258630193	49.4156299999047\\
25.8656443595598	49.2647296171316\\
26.0328628561002	49.101274099964\\
26.2000813526407	48.9254180352027\\
26.3764978805808	48.7266158484075\\
26.5529144085208	48.5143612687049\\
26.7293309364609	48.2888321880269\\
26.9127404922769	48.0404790621285\\
27.0961500480929	47.7781667832668\\
27.2795596039088	47.5020918003731\\
27.4672528416716	47.2055242558011\\
27.6549460794344	46.894956785064\\
27.8426393171972	46.570597013348\\
28.0321638493592	46.2292883640513\\
28.2216883815213	45.8743400647399\\
28.4112129136833	45.5059637527914\\
28.6008038780228	45.1242346329656\\
28.7903948423623	44.7294902669034\\
28.9799858067018	44.3219412960601\\
29.1685459273417	43.9041163791689\\
29.3571060479815	43.4740407631228\\
29.5456661686214	43.0319213791578\\
29.7325480810476	42.582057292301\\
29.9194299934738	42.1207679465202\\
30.1063119059	41.6482554013785\\
30.2911155537318	41.1701594119364\\
30.4759192015636	40.6814826588301\\
30.6607228493955	40.1824219941867\\
30.8431652791781	39.6797451881772\\
31.0256077089606	39.1673314129465\\
31.2080501387432	38.6453723420434\\
31.3878932015637	38.1216955538124\\
31.5677362643841	37.5891161790066\\
31.7475793272046	37.047820859217\\
31.8317195514697	36.7916341914135\\
31.9158597757349	36.5336001675255\\
32	36.273738188979\\
32.0794817313781	36.0355376155764\\
32.1589634627562	35.8128748869848\\
32.2384451941343	35.6047773293509\\
32.3179269255124	35.410412595719\\
32.3974086568905	35.2290512693229\\
32.4768903882686	35.0600451529833\\
32.5656698146527	34.8851684504365\\
32.6544492410368	34.7242629016095\\
32.7432286674209	34.5766748863491\\
32.8399653171827	34.4303278084356\\
32.9367019669445	34.2983938150206\\
33.0334386167063	34.1802351772164\\
33.1398731810393	34.0654505790354\\
33.2463077453722	33.9659189349772\\
33.3527423097051	33.8809795006047\\
33.4713474341109	33.8027728112522\\
33.5899525585166	33.7411470755716\\
33.7085576829223	33.6953769059369\\
33.8439816039586	33.6616434454107\\
33.9794055249948	33.6467732725631\\
34.1148294460311	33.6499010605707\\
34.2787395038811	33.6766135923141\\
34.4426495617312	33.7271919281914\\
34.6065596195812	33.8004097856727\\
34.7795720998313	33.9009994315635\\
34.9525845800813	34.0243056092228\\
35.1255970603314	34.1691718198349\\
35.2986095405814	34.3345206985998\\
35.4716220208315	34.5193450986709\\
35.6446345010816	34.7227005893885\\
35.8197648115314	34.9465125115246\\
35.9948951219812	35.1875411945734\\
36.1700254324311	35.4449735381041\\
36.3452907615342	35.718260197063\\
36.5205560906372	36.0064847979117\\
36.6958214197403	36.3089627555394\\
36.8749682849561	36.6321964390215\\
37.0541151501718	36.9689920651153\\
37.2332620153876	37.3187303718612\\
37.4179065732951	37.6921263435457\\
37.6025511312026	38.0780404398754\\
37.7871956891102	38.4758958684147\\
37.9784708040078	38.9000472790956\\
38.1697459189055	39.3358502454052\\
38.3610210338032	39.7827580878246\\
38.5598274453743	40.2584754746116\\
38.7586338569455	40.7450774117466\\
38.9574402685166	41.2420400080308\\
39.1645835392811	41.7703282363293\\
39.3717268100456	42.3087918305715\\
39.57887008081	42.8569249137542\\
39.71924672054	43.2336354867332\\
39.85962336027	43.6144229599243\\
40	43.9991492338601\\
40.1018841777287	44.2760083266208\\
40.2037683554574	44.5453732481978\\
40.3056525331861	44.8073083639956\\
40.4075367109148	45.0618769694531\\
40.5094208886435	45.3091413264762\\
40.6113050663722	45.5491627118954\\
40.7242094807197	45.8067583329253\\
40.8371138950671	46.0556141771543\\
40.9500183094145	46.295809660954\\
41.0705733714511	46.5428089638567\\
41.1911284334876	46.780117022846\\
41.3116834955241	47.0078258179639\\
41.4412986640532	47.2420426083133\\
41.5709138325822	47.4653782333981\\
41.7005290011113	47.6779414426045\\
41.8401980592596	47.8950571870657\\
41.9798671174079	48.0999203239064\\
42.1195361755563	48.2926602867056\\
42.2702447041841	48.4871826168258\\
42.4209532328119	48.6678951018719\\
42.5716617614397	48.8349524434715\\
42.7340877198317	48.9998841607425\\
42.8965136782237	49.1493189507683\\
43.0589396366157	49.283441134213\\
43.2330519498618	49.4104146240023\\
43.4071642631079	49.5202209390079\\
43.581276576354	49.6130763762476\\
43.7658755880909	49.6932486078741\\
43.9504745998277	49.754855703277\\
44.1350736115646	49.7981436880726\\
44.3276816789576	49.8240433465607\\
44.5202897463505	49.8305349253492\\
44.7128978137434	49.8178858817336\\
44.9103809226155	49.7853194648316\\
45.1078640314877	49.733189032795\\
45.3053471403599	49.661771533713\\
45.5049112174947	49.570287391491\\
45.7044752946294	49.4596657814888\\
45.9040393717642	49.3301825027889\\
46.1037863195458	49.1819663403323\\
46.3035332673273	49.0153980079887\\
46.5032802151089	48.8307456995084\\
46.7021401231664	48.6292133896721\\
46.901000031224	48.4102807007224\\
47.0998599392815	48.1742055481809\\
47.2972863230591	47.9231273695795\\
47.4947127068366	47.6556540189233\\
47.6921390906142	47.3720326777372\\
47.7947593937428	47.2183124587836\\
47.8973796968714	47.0603301288548\\
48	46.8981199628324\\
48.199899536878	46.5701322204632\\
48.399799073756	46.2264833183751\\
48.599698610634	45.8674241001139\\
48.7922005938763	45.5073203846026\\
48.9847025771185	45.1333801253332\\
49.1772045603608	44.745825342951\\
49.3679545859632	44.3485865201104\\
49.5587046115657	43.9384126918069\\
49.7494546371681	43.5155189594483\\
49.9380460274541	43.0851168692048\\
50.1266374177402	42.6426990811422\\
50.3152288080262	42.1884735316076\\
50.50155823264	41.7283035791823\\
50.6878876572538	41.2570116435504\\
50.8742170818676	40.7747992233016\\
51.0581465444148	40.2882848801702\\
51.242076006962	39.791521269604\\
51.4260054695092	39.2847040035349\\
51.6073756836376	38.7752856173646\\
51.788745897766	38.2564715333802\\
51.9701161118944	37.7284519110671\\
52.1487424155043	37.1996091288645\\
52.3273687191142	36.6622064655681\\
52.5059950227241	36.1164289430928\\
52.6816638529543	35.5717007707059\\
52.8573326831845	35.019231535577\\
53.0330015134147	34.4592013142526\\
53.2054659793155	33.9022072177067\\
53.3779304452163	33.3382739546234\\
53.5503949111172	32.7675767169928\\
53.7193690469069	32.2020401730325\\
53.8883431826967	31.6303487787786\\
54.0573173184865	31.0526727560068\\
54.222468441289	30.4824492773556\\
54.3876195640916	29.9068362129322\\
54.5527706868941	29.3259985594271\\
54.7137090072201	28.7551112432733\\
54.8746473275462	28.179578156228\\
55.0355856478722	27.5995586137326\\
55.1918497495329	27.0322462316078\\
55.3481138511935	26.4610069688762\\
55.5043779528542	25.8859937586966\\
55.6695853019028	25.2741416973814\\
55.8347926509514	24.6584318620366\\
56	24.0390552487084\\
56.059835493522	23.8258703774041\\
56.1196709870439	23.6329905540622\\
56.1795064805658	23.4568774648364\\
56.2393419740878	23.2954686581555\\
56.2991774676097	23.1472302831417\\
56.3590129611317	23.0109857120157\\
56.4311071801391	22.8614612507871\\
56.5032013991465	22.7267011700236\\
56.5752956181539	22.6056186179913\\
56.6584942505655	22.481709321135\\
56.7416928829772	22.3736295127841\\
56.8248915153888	22.2803599901753\\
56.9204307129003	22.1903882989213\\
57.0159699104118	22.1176184375571\\
57.1115091079233	22.0610200063668\\
57.2222105562881	22.0144683399124\\
57.3329120046528	21.9871441533137\\
57.4436134530176	21.9779174834547\\
57.5769694769652	21.9894039777796\\
57.7103255009128	22.0240961929162\\
57.8436815248604	22.0805633225092\\
58.0399029446001	22.2005576306333\\
58.2361243643399	22.3612701373471\\
58.4323457840796	22.5594725661875\\
58.5982975882455	22.7542399764077\\
58.7642493924115	22.9722211341905\\
58.9302011965774	23.211992181973\\
59.0843743758402	23.4531280261074\\
59.2385475551031	23.7109795227195\\
59.3927207343659	23.9846502877446\\
59.5488745957247	24.277109867869\\
59.7050284570835	24.5841371264443\\
59.8611823184424	24.9049789121464\\
60.0216792043714	25.2484008441837\\
60.1821760903005	25.6049592840512\\
60.3426729762296	25.9739850030181\\
60.5090528499424	26.3690290166434\\
60.6754327236553	26.7761430084421\\
60.8418125973682	27.1947125257929\\
61.015220348461	27.6425321354606\\
61.1886280995539	28.1015566833494\\
61.3620358506468	28.5712103758806\\
61.543470520173	29.0733950260474\\
61.7249051896992	29.5860417334995\\
61.9063398592254	30.1086031821599\\
62.0967576313663	30.6671386509665\\
62.2871754035072	31.2354695126373\\
62.4775931756481	31.8130698896224\\
62.6779732102942	32.4303558931013\\
62.8783532449404	33.0568175924872\\
63.0787332795866	33.6919452485263\\
63.2901350435668	34.3708728333389\\
63.5015368075471	35.0583833054527\\
63.7129385715273	35.753978682058\\
63.8086257143516	36.0713582780019\\
63.9043128571758	36.3902564006246\\
64	36.7106325807697\\
64.1111695026695	37.0804197454909\\
64.222339005339	37.4437255297701\\
64.3335085080085	37.8006356492472\\
64.4446780106779	38.1512339695435\\
64.5558475133474	38.4956025854538\\
64.6670170160169	38.8338219094272\\
64.7914226138132	39.2051163529118\\
64.9158282116094	39.5689167179995\\
65.0402338094057	39.9253287440173\\
65.1742770700785	40.3012015850366\\
65.3083203307513	40.6687437213492\\
65.4423635914241	41.0280790149138\\
65.5882842704917	41.4100665042556\\
65.7342049495593	41.7826247655281\\
65.8801256286269	42.1459032843151\\
66.0399495337825	42.5333196155513\\
66.1997734389381	42.9099666943172\\
66.3595973440937	43.2760280041042\\
66.535948543565	43.6678743082504\\
66.7122997430363	44.0472859446517\\
66.8886509425077	44.4144927741476\\
67.0847479688112	44.8087574882067\\
67.2808449951147	45.1885101396616\\
67.4769420214183	45.5540448086973\\
67.696514680244	45.9468202420316\\
67.9160873390696	46.3225224705383\\
68.1356599978953	46.6815346749964\\
68.3822011131727	47.0652141499975\\
68.6287422284501	47.4288382992452\\
68.8752833437275	47.7729091624185\\
69.1496911255305	48.1334736833603\\
69.4240989073335	48.4710771467974\\
69.6985066891365	48.7863586587033\\
69.9953795793266	49.103033192744\\
70.2922524695166	49.3950760831117\\
70.5891253597067	49.6632336642093\\
70.8976354885739	49.9173746848898\\
71.2061456174412	50.1472995959497\\
71.5146557463084	50.3537818432231\\
71.6764371642056	50.4529446703506\\
71.8382185821028	50.5459751079553\\
72	50.6329783461979\\
72.1268137422163	50.694398623205\\
72.2536274844326	50.7469866839444\\
72.380441226649	50.7908223866264\\
72.5072549688653	50.8259847604091\\
72.6340687110816	50.8525520156128\\
72.7608824532979	50.8706015997351\\
72.9052702015634	50.8808801409065\\
73.0496579498289	50.8803281212923\\
73.1940456980944	50.8690564527493\\
73.3487786616842	50.8452027204104\\
73.503511625274	50.8092980291226\\
73.6582445888638	50.7614747907322\\
73.8246964567364	50.6968729476789\\
73.9911483246089	50.618792182197\\
74.1576001924814	50.5273925556596\\
74.335223389313	50.4153478056915\\
74.5128465861447	50.2885078403125\\
74.6904697829763	50.1470618200626\\
74.8776271326157	49.9824269489355\\
75.0647844822551	49.8020015728431\\
75.2519418318945	49.6060013978757\\
75.445898424355	49.386675547596\\
75.6398550168155	49.1510889474297\\
75.833811609276	48.8994763547539\\
76.0315276302191	48.6267335167611\\
76.2292436511623	48.3378251903835\\
76.4269596721055	48.0329954713998\\
76.6259398045352	47.7103881800114\\
76.8249199369649	47.3721474692322\\
77.0239000693946	47.0185186238343\\
77.2225039259002	46.650457643408\\
77.4211077824058	46.2675522672725\\
77.6197116389114	45.8700440407334\\
77.816989154025	45.4609718447123\\
78.0142666691386	45.0379647743871\\
78.2115441842522	44.6012583841086\\
78.4069573635307	44.1554047993064\\
78.6023705428092	43.696569294312\\
78.7977837220877	43.2249807354307\\
78.9910004501516	42.7463785988103\\
79.1842171782156	42.2557541603912\\
79.3774339062795	41.7533296278857\\
79.5849559375197	41.2008184307198\\
79.7924779687598	40.635228914097\\
80	40.0568390531345\\
80.096798989416	39.7893941545565\\
80.193597978832	39.5322697720761\\
80.290396968248	39.285102109734\\
80.387195957664	39.0475518945774\\
80.48399494708	38.8193011229222\\
80.580793936496	38.6000504861759\\
80.6878414589345	38.3677272646065\\
80.794888981373	38.1457066949312\\
80.9019365038115	37.9336494345128\\
81.0175306774282	37.7154807405507\\
81.133124851045	37.5081747481153\\
81.2487190246617	37.3113701853127\\
81.3749299619745	37.1080801762038\\
81.5011408992873	36.916479151558\\
81.6273518366001	36.736166501413\\
81.7666605655033	36.5497917252683\\
81.9059692944064	36.3762216169948\\
82.0452780233095	36.2149953758313\\
82.2016791857749	36.0481540026128\\
82.3580803482402	35.8957322542111\\
82.5144815107056	35.7571721766296\\
82.6954888343133	35.6134226636542\\
82.8764961579211	35.4867529134501\\
83.0575034815288	35.376426385415\\
83.2843282852907	35.260181375393\\
83.5111530890525	35.1672144472061\\
83.7379778928144	35.0963106327441\\
83.968513895268	35.0456786715122\\
84.1990498977217	35.0155314624474\\
84.4295859001753	35.004805891261\\
84.660121902629	35.0124995990074\\
84.8906579050826	35.0376661842866\\
85.1211939075363	35.0794101292044\\
85.3516992068488	35.1368756164581\\
85.5822045061613	35.2092652924738\\
85.8127098054738	35.2958185260121\\
86.0470008824719	35.3975646349406\\
86.2812919594701	35.5124798791652\\
86.5155830364682	35.6398820440961\\
86.7568878520031	35.7834678990312\\
86.998192667538	35.9389413637594\\
87.2394974830729	36.1056648737479\\
87.4929983220486	36.2922698301\\
87.7464991610243	36.4899577830885\\
88	36.698097622892\\
88.1292223648615	36.805436296557\\
88.2584447297231	36.9101592035231\\
88.3876670945846	37.012307658645\\
88.5168894594461	37.1119223552707\\
88.6461118243077	37.2090433727\\
88.7753341891692	37.3037101973887\\
88.9214470518465	37.4078433530402\\
89.0675599145237	37.5089438842577\\
89.2136727772009	37.607066227862\\
89.3728025460477	37.7106048037387\\
89.5319323148944	37.8107423700318\\
89.6910620837412	37.9075456963687\\
89.8665697427728	38.0105241678522\\
90.0420774018044	38.1096128684316\\
90.2175850608359	38.2048965021359\\
90.4126871890309	38.3064531162347\\
90.6077893172259	38.403522034579\\
90.8028914454208	38.4962127155111\\
91.0219235065082	38.5951883603542\\
91.2409555675956	38.6889290975604\\
91.459987628683	38.7775796993294\\
91.7085605523031	38.8721992842711\\
91.9571334759233	38.9606466277063\\
92.2057063995435	39.0431179743568\\
92.4909191339255	39.1306302367848\\
92.7761318683075	39.2108068584855\\
93.0613446026896	39.2839200500182\\
93.3911722963227	39.3600090430898\\
93.7209999899558	39.4274000302226\\
94.050827683589	39.4864747894729\\
94.4307640838478	39.5446966993997\\
94.8107004841065	39.5929148413111\\
95.1906368843653	39.6316505756218\\
95.4604245895769	39.6536745525275\\
95.7302122947884	39.671343307015\\
96	39.6848250939904\\
96.2242135268748	39.6925792172113\\
96.4484270537497	39.6968979995468\\
96.6726405806245	39.6978945687733\\
96.8968541074993	39.6956794598141\\
97.1210676343741	39.6903606000314\\
97.345281161249	39.6820434462514\\
97.6215340637659	39.6678254301836\\
97.8977869662829	39.6493996342026\\
98.1740398687999	39.6269483118195\\
98.4880477135953	39.5967648038238\\
98.8020555583908	39.561860548028\\
99.1160634031863	39.5224790557976\\
99.4799554319199	39.4715504393712\\
99.8438474606535	39.4152778242307\\
99.9999999999991	39.3895763162346\\
100	39.3895763162345\\
100.000000000001	39.3895763162343\\
100.061415909019	39.3792209917208\\
100.122831818037	39.3687287344571\\
100.184247727055	39.3581010992734\\
100.245663636073	39.3473396304224\\
100.307079545091	39.33644586156\\
100.368495454109	39.3254213158929\\
100.437652761371	39.3128524845141\\
100.506810068633	39.3001218920882\\
100.575967375895	39.2872316662599\\
100.647991528946	39.2736396676338\\
100.720015681997	39.2598791819252\\
100.792039835048	39.245952555838\\
100.867554238689	39.2311751808712\\
100.943068642329	39.2162203921441\\
101.018583045969	39.2010908273436\\
101.097901592349	39.1850137481136\\
101.17722013873	39.1687497290287\\
101.25653868511	39.1523017469528\\
101.340048536066	39.1347890727615\\
101.423558387022	39.1170791605112\\
101.507068237978	39.0991753883164\\
101.59521473703	39.080070952129\\
101.683361236083	39.0607581541919\\
101.771507735135	39.0412408506251\\
101.864808916012	39.0203636475232\\
101.958110096889	38.9992660755538\\
102.051411277765	38.977952564955\\
102.150473977417	38.9550914482006\\
102.249536677068	38.9319970255679\\
102.34859937672	38.9086744230078\\
102.45414019878	38.8835813027955\\
102.559681020839	38.8582409859635\\
102.665221842899	38.832659449775\\
102.778094524601	38.8050405234878\\
102.890967206303	38.7771595748977\\
103.003839888005	38.7490236341969\\
103.125071538408	38.7185279374802\\
103.246303188812	38.6877545529264\\
103.367534839216	38.656711828506\\
103.498374650411	38.6229160919692\\
103.629214461606	38.5888263908773\\
103.760054272802	38.5544527441836\\
103.840036181868	38.5333046493006\\
103.920018090934	38.5120563581178\\
104	38.4907100711507\\
104.106913431959	38.4678207578043\\
104.213826863917	38.4560924454131\\
104.320740295876	38.4551533682433\\
104.427653727834	38.4646504384657\\
104.534567159793	38.4842475366922\\
104.641480591752	38.5136240335294\\
104.759617142752	38.5570896101507\\
104.877753693753	38.611728596187\\
104.995890244754	38.6771657097198\\
105.124863798632	38.760514278249\\
105.25383735251	38.8558628864802\\
105.382810906387	38.962788582616\\
105.525880157952	39.0944562923573\\
105.668949409517	39.2393503396111\\
105.812018661083	39.3969721407414\\
105.974247961705	39.5905028412902\\
106.136477262327	39.7991320228301\\
106.298706562949	40.022236135239\\
106.491448163276	40.3053013128491\\
106.684189763603	40.6070263890649\\
106.87693136393	40.9265262986952\\
107.160300068246	41.4267892384621\\
107.443668772561	41.9612039345281\\
107.727037476877	42.5274866659388\\
107.96048171835	43.0164463338878\\
108.193925959824	43.5245036070655\\
108.427370201297	44.0506267105689\\
108.64596570242	44.5588020400848\\
108.864561203542	45.0812166523155\\
109.083156704665	45.6171600847736\\
109.305105834407	46.1744764717808\\
109.52705496415	46.7443810605591\\
109.749004093892	47.3262425432955\\
109.977231735564	47.9364033765492\\
110.205459377236	48.5579585082667\\
110.433687018908	49.1903234037189\\
110.670268872871	49.8566545423246\\
110.906850726834	50.5334207978332\\
111.06079627968	50.9791372785995\\
111.060796279681	50.9791372786022\\
111.186616330741	51.3434587638627\\
111.312436381802	51.7045418195094\\
111.438256432862	52.0624264118077\\
111.564076483922	52.4171517393262\\
111.689896534983	52.7687562563649\\
111.815716586043	53.1172776993513\\
111.877144390695	53.2863238083678\\
111.938572195348	53.4546481284826\\
112	53.6222548716873\\
112.119749090461	53.9439074063794\\
112.239498180923	54.2568418481012\\
112.359247271384	54.5611442130835\\
112.478996361846	54.8568991976392\\
112.598745452307	55.1441902220547\\
112.718494542769	55.4230994920389\\
112.849677066721	55.7191140808033\\
112.980859590674	56.0052721001092\\
113.112042114627	56.2816769176866\\
113.253747458795	56.569412914808\\
113.395452802963	56.8460134888864\\
113.537158147131	57.1116031709534\\
113.691116567208	57.3878208620163\\
113.845074987284	57.6513408728675\\
113.999033407361	57.9023158070992\\
114.166729497018	58.1615871394772\\
114.334425586675	58.4063443353473\\
114.502121676332	58.6367757940077\\
114.684827939388	58.8717437133143\\
114.867534202444	59.09016318659\\
115.0502404655	59.2922670096889\\
115.248391444112	59.4932790108478\\
115.446542422724	59.6756582230488\\
115.644693401335	59.8396885566749\\
115.85709168652	59.9954557050692\\
116.069489971704	60.1308015057094\\
116.281888256888	60.2460607324411\\
116.505346375199	60.3460014873846\\
116.72880449351	60.4244538879767\\
116.952262611821	60.4817921184885\\
117.182541137834	60.5191801849541\\
117.412819663846	60.5349369586364\\
117.643098189859	60.5294569089377\\
117.876573259756	60.5026197327924\\
118.110048329654	60.4547547812722\\
118.343523399551	60.3862598053165\\
118.577998340305	60.2971057229958\\
118.812473281059	60.1879367451909\\
119.046948221812	60.0591444541798\\
119.281395957894	59.9111354948316\\
119.515843693976	59.7442809712144\\
119.750291430058	59.5589630927978\\
119.833527620038	59.4888001282698\\
119.916763810019	59.4163746868502\\
120	59.3417036382671\\
120.151024540775	59.1989984905976\\
120.30204908155	59.0459748332849\\
120.453073622325	58.8827371806388\\
120.6040981631	58.7093894398698\\
120.755122703875	58.526034884233\\
120.90614724465	58.3327762494471\\
121.073430041185	58.1072755385561\\
121.240712837721	57.8698870894783\\
121.407995634256	57.6207483561699\\
121.585500727734	57.3436886619752\\
121.763005821213	57.0537153792496\\
121.940510914692	56.7509904367499\\
122.127068535003	56.4192600069876\\
122.313626155314	56.0738086855496\\
122.500183775625	55.7148222774016\\
122.692954518639	55.3298582132084\\
122.885725261652	54.9308439484289\\
123.078496004666	54.5179827023604\\
123.274534798641	54.08413035888\\
123.470573592617	53.6363801414494\\
123.666612386593	53.174944672095\\
123.863554007774	52.6978185568654\\
124.060495628955	52.2073104813975\\
124.257437250136	51.7036357222084\\
124.453732776179	51.1887256084049\\
124.650028302223	50.6611624152833\\
124.846323828267	50.1211598680893\\
125.041058036853	49.5733722960126\\
125.23579224544	49.0137626405405\\
125.430526454027	48.4425409738476\\
125.623156585586	47.8662735842485\\
125.815786717144	47.2790546375533\\
126.008416848702	46.6810896903815\\
126.19858442761	46.0804301415433\\
126.388752006517	45.4696984563635\\
126.578919585424	44.8490954085183\\
126.766342895141	44.2279863918922\\
126.953766204858	43.5976795917995\\
127.141189514575	42.9583709620673\\
127.325607063804	42.3207206455418\\
127.510024613032	41.6747355226839\\
127.694442162261	41.0206067341378\\
127.79629477484	40.6559151729642\\
127.89814738742	40.2888311004315\\
128	39.919387413732\\
128.079775592017	39.638503878088\\
128.159551184035	39.3754309029803\\
128.239326776052	39.1288773181427\\
128.319102368069	38.8977750234279\\
128.398877960086	38.6812093709938\\
128.478653552104	38.4783828585837\\
128.568513864832	38.2655014295002\\
128.658374177561	38.0682837941045\\
128.74823449029	37.8859422594849\\
128.846866268925	37.7020921901242\\
128.94549804756	37.5344785016908\\
129.044129826195	37.3823413779514\\
129.153439003467	37.2309818365196\\
129.262748180738	37.0969281436953\\
129.372057358009	36.979395767509\\
129.494983702106	36.8660652828269\\
129.617910046202	36.7717788429545\\
129.740836390298	36.6956689557362\\
129.883387747176	36.6291163193806\\
130.025939104054	36.5847798389319\\
130.168490460931	36.5615946413906\\
130.349255872624	36.561075017839\\
130.530021284317	36.5911388459389\\
130.71078669601	36.6500873060746\\
130.930242995552	36.7582334509037\\
131.149699295095	36.904094479432\\
131.369155594637	37.0853245082251\\
131.545908554514	37.2555383141727\\
131.722661514391	37.4462571624366\\
131.899414474268	37.6565034364947\\
132.074999826294	37.8837884993974\\
132.25058517832	38.1285914612315\\
132.426170530346	38.390113369347\\
132.605068160097	38.6729836979042\\
132.783965789849	38.9716739984061\\
132.9628634196	39.285471573865\\
133.147021312089	39.6235652420215\\
133.331179204579	39.9762572233652\\
133.515337097068	40.3428878375982\\
133.705987022867	40.7364616489974\\
133.896636948665	41.1436474698241\\
134.087286874463	41.5638221429462\\
134.285350751134	42.0134592964626\\
134.483414627806	42.4758495987697\\
134.681478504477	42.9503971421079\\
134.887736650164	43.4568909932445\\
135.093994795851	43.9753508128511\\
135.300252941538	44.5052022565384\\
135.533501961025	45.1174312182967\\
135.766750980513	45.7427868878467\\
136	46.3805649938697\\
136.127080717952	46.7301790655179\\
136.254161435904	47.0775556607196\\
136.381242153856	47.4227198792941\\
136.508322871808	47.7656962971128\\
136.63540358976	48.1065089842299\\
136.762484307712	48.4451815230001\\
136.904950380427	48.8223404008648\\
137.047416453142	49.1968704476952\\
137.189882525857	49.5688028286814\\
137.345177774176	49.9713046626292\\
137.500473022496	50.3707947162814\\
137.655768270815	50.7673106852883\\
137.827205051334	51.2016312690169\\
137.998641831852	51.6324205974817\\
138.17007861237	52.0597258407988\\
138.3613100688	52.5323148077993\\
138.55254152523	53.0006881273459\\
138.74377298166	53.464906393239\\
138.960317203572	53.9856297827361\\
139.176861425485	54.5011842547669\\
139.393405647397	55.0116507469329\\
139.644413817475	55.5970966123965\\
139.895421987553	56.1759304309417\\
140.146430157631	56.7482672535382\\
140.451589659982	57.4354873414243\\
140.756749162333	58.1134627524666\\
141.061908664685	58.782379953856\\
141.527474595902	59.7858653982977\\
141.993040527118	60.7692940606515\\
142.458606458335	61.7332452229735\\
142.833767207136	62.4962167337879\\
143.208927955937	63.2471733250476\\
143.584088704739	63.9863794946357\\
143.722725803159	64.2566238266926\\
143.86136290158	64.5253108642479\\
144	64.7924529458685\\
144.112954118906	65.0054170258134\\
144.225908237812	65.2102952311127\\
144.338862356719	65.4071472028313\\
144.451816475625	65.5960318857368\\
144.564770594531	65.7770075440983\\
144.677724713437	65.9501317916466\\
144.808324393936	66.140588976205\\
144.938924074435	66.3207132400352\\
145.069523754933	66.4905905524868\\
145.209410240046	66.6612782491999\\
145.349296725158	66.8204103960009\\
145.489183210271	66.9680888189441\\
145.640595145356	67.1151430058645\\
145.792007080442	67.2490221596496\\
145.943419015527	67.3698507116927\\
146.107238345321	67.4860241479769\\
146.271057675115	67.5872180994613\\
146.434877004909	67.6735844095598\\
146.61160657626	67.7503031572456\\
146.788336147611	67.8101256459871\\
146.965065718963	67.8532356019668\\
147.154134899082	67.8810575889802\\
147.343204079203	67.8901790533372\\
147.532273259322	67.8808169569121\\
147.731943088967	67.8511000043142\\
147.931612918612	67.8012574969614\\
148.131282748256	67.7315362653061\\
148.338309674461	67.6385161268899\\
148.545336600665	67.5246569502138\\
148.75236352687	67.3902251287988\\
148.963108036872	67.2325217568326\\
149.173852546874	67.0540489970706\\
149.384597056876	66.8550798038199\\
149.596524254302	66.6345981122956\\
149.808451451728	66.3939356451274\\
150.020378649154	66.1333630388056\\
150.231956860712	65.8536264024335\\
150.44353507227	65.5545781924437\\
150.655113283828	65.2364820233732\\
150.865544649501	64.9014761782762\\
151.075976015175	64.5481436722339\\
151.286407380848	64.1767394001223\\
151.495293182766	63.7904401871651\\
151.704178984684	63.3868305962099\\
151.913064786601	62.9661566377773\\
151.942043191068	62.9064631469264\\
151.971021595534	62.8464466148663\\
152	62.7861076945123\\
152.144892022331	62.4796001085467\\
152.289784044662	62.1651305925127\\
152.434676066994	61.8427804789447\\
152.643808852668	61.3637782062701\\
152.852941638342	60.8687709778131\\
153.062074424017	60.3580021344854\\
153.265408307757	59.8465149522136\\
153.468742191497	59.32058103742\\
153.672076075237	58.7804235099346\\
153.873529749104	58.2314533461989\\
154.074983422972	57.6689574912407\\
154.27643709684	57.0931532421239\\
154.475674032955	56.5108088170012\\
154.674910969071	55.9158708948647\\
154.874147905187	55.3085507746644\\
155.071052106411	54.6963832881682\\
155.267956307635	54.0725338298089\\
155.464860508859	53.4372082055609\\
155.659285282018	52.7988243866779\\
155.853710055177	52.1496527257064\\
156.048134828336	51.4898939062307\\
156.239913454618	50.8289394429674\\
156.4316920809	50.1580749272407\\
156.623470707182	49.4774961984093\\
156.812412845133	48.7976758318876\\
157.001354983084	48.1088075252793\\
157.190297121034	47.4110824538437\\
157.376185446109	46.7161786172204\\
157.562073771183	46.0130734031119\\
157.747962096258	45.3019533982051\\
157.930548038304	44.5958468528359\\
158.11313398035	43.8823694927849\\
158.295719922396	43.1617032761632\\
158.474718172827	42.4483970845668\\
158.653716423257	41.7285335532519\\
158.832714673688	41.0022898401361\\
159.007795956412	40.2859387548428\\
159.182877239137	39.563824909886\\
159.357958521861	38.8361203374174\\
159.528739859204	38.1210688244561\\
159.699521196547	37.4010276043304\\
159.87030253389	36.6761630738836\\
159.913535022593	36.4919221784991\\
159.956767511297	36.307385783454\\
160	36.1225566648481\\
160.070753713577	35.8311975511355\\
160.141507427154	35.5605143436575\\
160.212261140731	35.3080464506342\\
160.283014854308	35.0720545494297\\
160.353768567885	34.851169173187\\
160.424522281462	34.644276619144\\
160.506662991963	34.4204352504095\\
160.588803702464	34.2130140960457\\
160.670944412965	34.0209819468333\\
160.763310141003	33.8223307342199\\
160.85567586904	33.6409267026423\\
160.948041597077	33.4758066896504\\
161.052060727365	33.3082927582352\\
161.156079857652	33.1592747023379\\
161.26009898794	33.0277845297755\\
161.37896751273	32.8978716611293\\
161.497836037521	32.7885442705671\\
161.616704562312	32.6987415558712\\
161.758168764033	32.6159837345379\\
161.899632965755	32.5580437001058\\
162.041097167476	32.5235796546541\\
162.243192698493	32.5126659150309\\
162.445288229509	32.5439665883437\\
162.647383760526	32.614596173088\\
162.824737858767	32.7069273272103\\
163.002091957009	32.8259520170903\\
163.17944605525	32.9701973397868\\
163.342459246629	33.1238542490096\\
163.505472438008	33.2966857066843\\
163.668485629387	33.4877718333315\\
163.833095337657	33.6983761630081\\
163.997705045927	33.9258831402127\\
164.162314754198	34.1695089288997\\
164.331050184815	34.4351990010478\\
164.499785615432	34.7163023518199\\
164.66852104605	35.0121123528372\\
164.842908986491	35.3325774734644\\
165.017296926932	35.6673439121993\\
165.191684867374	36.0157545309283\\
165.372765729408	36.391311323871\\
165.553846591443	36.7802527708441\\
165.734927453478	37.1819567357664\\
165.923532513353	37.6132870920999\\
166.112137573228	38.0571952169737\\
166.300742633103	38.513085477175\\
166.497605504372	39.001107109075\\
166.69446837564	39.5009624583123\\
166.891331246909	40.0120770443475\\
167.097154423609	40.5579049026153\\
167.30297760031	41.1148578729605\\
167.508800777011	41.6823789942988\\
167.67253385134	42.1410397144283\\
167.83626692567	42.6058012629602\\
168	43.0764211722467\\
168.114521874819	43.4049509300519\\
168.229043749637	43.7282535329662\\
168.343565624456	44.0463949173407\\
168.458087499275	44.3594398278253\\
168.572609374094	44.6674518567235\\
168.687131248913	44.9704934918945\\
168.814915223793	45.3028367228439\\
168.942699198673	45.6291519209547\\
169.070483173553	45.949521308651\\
169.208467714283	46.2888810126621\\
169.346452255013	46.6215015424992\\
169.484436795743	46.9474807064635\\
169.634970853887	47.2956451038136\\
169.785504912031	47.6361414854649\\
169.936038970174	47.9690895980253\\
170.101370009196	48.3262073131064\\
170.266701048218	48.6745138125855\\
170.432032087239	49.0141583177116\\
170.615140777406	49.3803914082617\\
170.798249467572	49.7363730919746\\
170.981358157739	50.0822934865502\\
171.186102943811	50.4574018237608\\
171.390847729884	50.8204174012876\\
171.595592515956	51.1715885900226\\
171.826968125659	51.5544851559226\\
172.058343735361	51.9229074370584\\
172.289719345064	52.2771889077065\\
172.553858940146	52.6647684120896\\
172.817998535228	53.0348140491163\\
173.08213813031	53.387784918739\\
173.384970437537	53.7720276457804\\
173.687802744763	54.1350688476322\\
173.99063505199	54.4775446591896\\
174.332393208648	54.8401150733135\\
174.674151365306	55.1781416190377\\
175.015909521964	55.4924596217116\\
175.34393968131	55.7726045846687\\
175.671969840655	56.032348618308\\
176	56.2723671765875\\
176.12184749722	56.3537604283348\\
176.24369499444	56.4268388404364\\
176.36554249166	56.4916711265639\\
176.487389988881	56.5483253130067\\
176.609237486101	56.596868748257\\
176.731084983321	56.6373681434588\\
176.871546021906	56.6741592944734\\
177.01200706049	56.7004490844874\\
177.152468099075	56.7163367234204\\
177.30297753265	56.7219275368692\\
177.453486966225	56.7158070233455\\
177.6039963998	56.6980936010069\\
177.766432127349	56.6661043724516\\
177.928867854897	56.6208946963701\\
178.091303582445	56.5626090740032\\
178.265659282613	56.4856589678084\\
178.44001498278	56.3939835604574\\
178.614370682948	56.287756491326\\
178.799695497274	56.1590853530724\\
178.985020311601	56.0143730816517\\
179.170345125927	55.8538226076217\\
179.364368865716	55.6689843010797\\
179.558392605504	55.4672355294479\\
179.752416345293	55.2488034610167\\
179.952043964982	55.0068871455039\\
180.151671584671	54.7477923171121\\
180.351299204359	54.4717611059734\\
180.553574792194	54.1750401827267\\
180.755850380029	53.8614263566408\\
180.958125967863	53.5311670600325\\
181.160904841209	53.1836258120001\\
181.363683714554	52.8198504202403\\
181.566462587899	52.4400867683897\\
181.768452028834	52.0461499211671\\
181.970441469769	51.6368334848347\\
182.172430910705	51.2123782006979\\
182.372915977008	50.7763511213843\\
182.573401043311	50.3258809544367\\
182.773886109615	49.861201907676\\
182.97246380895	49.387167089695\\
183.171041508286	48.8996493887712\\
183.369619207622	48.3988762233855\\
183.579746138414	47.8547967767861\\
183.789873069207	47.2963998931992\\
184	46.7239560573584\\
184.090892668356	46.4800850997461\\
184.181785336713	46.2493312006517\\
184.272678005069	46.0311695240523\\
184.363570673426	45.8251182002267\\
184.454463341782	45.6307315056827\\
184.545356010138	45.4475948167564\\
184.646168681845	45.2571620064808\\
184.746981353551	45.0796025951228\\
184.847794025257	44.9144571825495\\
184.956758308395	44.7494228894942\\
185.065722591533	44.5978811705727\\
185.17468687467	44.459354531765\\
185.293860748272	44.3222207038349\\
185.413034621873	44.1995665371174\\
185.532208495475	44.0908718387488\\
185.664179075312	43.9862023245258\\
185.796149655149	43.8974226859324\\
185.928120234986	43.8239407283857\\
186.077360985393	43.7585725422668\\
186.2266017358	43.7112912212687\\
186.375842486207	43.6813775260368\\
186.552040967855	43.6674931617474\\
186.728239449503	43.6758214251601\\
186.904437931151	43.7053737380232\\
187.151863882925	43.780929447157\\
187.3992898347	43.8941118990789\\
187.646715786474	44.042699912636\\
187.868074897437	44.2039563749277\\
188.0894340084	44.3905306607576\\
188.310793119363	44.6011334258865\\
188.512875004083	44.8133487033086\\
188.714956888803	45.0437107557751\\
188.917038773523	45.2913959314965\\
189.120832941127	45.557931076241\\
189.324627108731	45.8405309069405\\
189.528421276335	46.1384738992985\\
189.736891431802	46.4584140459996\\
189.945361587268	46.7929954131801\\
190.153831742735	47.1415549621903\\
190.36877217463	47.5149017050001\\
190.583712606524	47.9017805831271\\
190.798653038418	48.3015662616817\\
191.02122654803	48.7285177106331\\
191.243800057641	49.1680429269341\\
191.466373567252	49.6195439582428\\
191.644249044835	49.9886001515747\\
191.822124522417	50.3646634838393\\
192	50.7474666563471\\
192.106682540346	50.9752118578563\\
192.213365080692	51.1953702140617\\
192.320047621038	51.4080026799171\\
192.426730161384	51.6131693660322\\
192.53341270173	51.8109295621587\\
192.640095242076	52.0013417748296\\
192.758889918071	52.2047960972065\\
192.877684594066	52.3992894389588\\
192.996479270061	52.5848993484075\\
193.12362280358	52.7737987155389\\
193.250766337099	52.9527021824566\\
193.377909870619	53.1217009247644\\
193.514985178452	53.2929087790639\\
193.652060486286	53.4528196672641\\
193.78913579412	53.6015431405686\\
193.937203346605	53.7497484533624\\
194.085270899091	53.8851604367943\\
194.233338451576	54.0079114688723\\
194.393381485246	54.1265028063372\\
194.553424518917	54.2306174194427\\
194.713467552587	54.3204155971831\\
194.88598810907	54.4013622122739\\
195.058508665552	54.4660520167245\\
195.231029222034	54.51467763401\\
195.415555212959	54.5491223257973\\
195.600081203884	54.565636062769\\
195.784607194808	54.5644452373844\\
195.979273589346	54.5442449531429\\
196.173939983884	54.504848481813\\
196.368606378422	54.4465116514401\\
196.570349302078	54.3663384633446\\
196.772092225734	54.2663722863057\\
196.973835149389	54.1468880130933\\
197.179352168177	54.0053806717083\\
197.384869186964	53.8441847190244\\
197.590386205752	53.6635817907996\\
197.79709941239	53.4626329906032\\
198.003812619029	53.2426141592753\\
198.210525825668	53.0038042644046\\
198.416846423593	52.7469862737798\\
198.623167021519	52.4719973412378\\
198.829487619445	52.1791087472563\\
199.034564518651	51.8705142489902\\
199.239641417857	51.5447646787896\\
199.444718317063	51.20212181717\\
199.629812211376	50.8785788080525\\
199.814906105688	50.5416773038289\\
200	50.1916074077015\\
};
\end{axis}
\end{tikzpicture}%}
  \caption{Step response using a zero-order hold of sample time $8$ sec.}
  \label{fig:Q7.8}
\end{figure}
