\section{Question 16}

Solving equation \ref{eq:15} for the calculated values of $a_i$, $b_j$ and $d_k$
gives

\begin{table}[H]\centering
  \begin{tabular}{l|c}
  $r$     & $0.4543$    \\
  $c_0$   & $8.6181$    \\
  $c_1$   & $-10.3688$  \\
  $c_2$   & $3.2976$    \\
  \end{tabular}
  \caption{The unknown coefficients $r, c_0, c_1, c_2$ resolved from solving
    equation \ref{eq:15}.}
  \label{tbl:16}
\end{table}

hence

\begin{equation}
  F_d(z) = \dfrac{8.6181z^2 -10.3688z + 3.2976}{(z-1)(z+0.4543)}
\end{equation}

The desired and actual poles of the discrete closed-loop tranfer function are
featured in tables \ref{tbl:16.1} and \ref{tbl:16.2} respectively. What we found
is that, while the two complex poles reside where they should, the double pole
at $z=0.1353$ cannot be set, and is represented by two distinct poles at either
side of it, at a distance of approximately $e^{-T_s}$.

\begin{table}
\centering
\makebox[0pt][c]{\parbox{\textwidth}{%
  \begin{minipage}[b]{0.45\hsize}\centering
    \begin{tabular}{c|c}
      $p^d_{0,1}$ & $0.1353$             \\
      $p^d_2$     & $0.4677 + 0.2435 i$  \\
      $p^d_3$     & $0.4677 - 0.2435 i$  \\
    \end{tabular}
    \caption{The desired poles of the closed-loop transfer function.}
    \label{tbl:16.1}
  \end{minipage}
  \hfill
  \begin{minipage}[b]{0.45\hsize}\centering
    \begin{tabular}{c|c}
      $p_0$ & $0.1528$            \\
      $p_1$ & $0.1197$            \\
      $p_3$ & $0.4667 + 0.2430 i$ \\
      $p_4$ & $0.4667 - 0.2430 i$ \\
    \end{tabular}
    \caption{The actual poles of the closed-loop transfer function.}
    \label{tbl:16.2}
  \end{minipage}
}}
\end{table}
